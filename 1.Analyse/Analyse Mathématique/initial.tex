\documentclass[11pt]{../../TexTemplate/elegantbook}

\title{Analyse Mathématique} % 这里放置书名
% \subtitle{Subtitle} % 这里放置副标题

\author{CatMono} % 这里放置作者名
\date{July, 2025} % 这里放置日期
\version{0.1} % 这里放置版本号
% \institute{Elegant\LaTeX{} Program} % 这里放置机构名
% \bioinfo{Custom Key}{Custom Value} % 这里放置自定义信息

% \extrainfo{extra information} % 这里放置额外信息,将显示在最下方中央

\setcounter{tocdepth}{2} % 设置目录深度
\setcounter{secnumdepth}{2} % 设置章节编号深度


% \logo{logo-blue.png} % 这里放置封面logo,默认从figure目录下寻找
% \cover{LogiqueMathematique.png} % 这里放置封面图片,默认从figure目录下寻找

% modify the color in the middle of titlepage
\definecolor{customcolor}{RGB}{32,178,170} % 自定义颜色
\colorlet{coverlinecolor}{customcolor}
\usepackage{cprotect} % 保护命令参数不被 LaTeX 解析器过早处理,允许在某些特殊环境中使用脆弱命令(fragile commands)。
\usepackage{xeCJK} % 使用 xeCJK 包支持中文
% \usepackage{unicode-math} % 使用 unicode-math 包支持 Unicode 数学符号
\usepackage{esint}

% 表格颜色
\usepackage{colortbl}
\usepackage{xcolor}





% ===== 开始文档 =====
\begin{document}

\maketitle %生成文档的标题页,根据之前定义的标题信息(如标题、作者、日期等)自动创建一个格式化的标题页

% === 前言部分 ===
\frontmatter        % 开始前言,页码为 i, ii, iii...
\tableofcontents    % 目录 (页码: i, ii)
% \listoffigures      % 图表目录 (页码: iii)
% \listoftables       % 表格目录 (页码: iv)

\chapter{Preface}   % 前言章节(无编号,页码: v, vi...)
For an interval \(I\), a open interval \((a, b)\) and a closed interval \([a, b]\),
we denote \(C(I)\), \(C(a, b)\) and \(C[a, b]\)
as the set of continuous \underline{univariate} functions on \(I\), \((a, b)\) and \([a, b]\) respectively.
Similarly, the following notations are used\footnote{
    Other notations include: \(R[a,b]\) (denoting Riemann integrable functions on \([a,b]\)),
    \(B[a,b]\) (denoting bounded functions on \([a,b]\)), etc.
}:

\[
\begin{array}{cc}
\toprule
\text{Notation} & \text{Meaning} \\
\hline
D(I) & \text{Set of derivative (differential) functions on } I \\
D(a, b) & \text{Set of derivative (differential) functions on } (a, b) \\
D[a, b] & \text{Set of derivative (differential) functions on } [a, b] \\
D^{k}(I) & \text{Set of } k\text{-th order derivative (differential) functions on } I \\
\bottomrule
\end{array}
\]

Let \(U \subset \mathbb{R}^n\) be an open set, and \(\mathbf{f}: U \to \mathbb{R}^m\) be a \(C^k\) mapping:  
\begin{itemize}
    \item \(k = 0\), \(\mathbf{f}\) is a continuous mapping;
    \item \(0 < k < +\infty\), \(f_i\) has continuous partial derivatives up to order \(k\), \(i = 1, 2, \dots, m\);
    \item \(k = +\infty\), \(f_i\) has continuous partial derivatives of all orders, \(i = 1, 2, \dots, m\);
    \item \(k = \omega\), \(f_i\) is really analytic, i.e., 
        in the neighborhood of any point \(\mathbf{x}^0 = (x_1^0, x_2^0, \dots, x_n^0) \in U\), 
        \(f_i\) can be expanded into a convergent (\(n\)-dimensional) power series, \(i = 1, 2, \dots, m\).
\end{itemize}
Let \(C^k(U, \mathbb{R}^m)\) denote the set of \(C^k\) mappings from \(U\) to \(\mathbb{R}^m\).

% 有时, 我们用下标 \(i\) 来表示对第i个变量的偏导数
Sometimes, we use subscripts \(i\) to denote the partial derivative with respect to the \(i\)-th variable,
for example, for function \(f(x^{2}+y^{2}+z^{2},xyz)\), \(f_{2}:=\frac{\partial f}{\partial (xyz)}\),
and similarly for higher-order partial derivatives, e.g., \(f_{12} := \frac{\partial^2 f(u, v)}{\partial v \partial u}\).

% \chapter{Acknowledgments}  % 致谢(无编号)
% I would like to thank...
% === 正文部分 ===
\mainmatter         % 开始正文,页码从 1 重新开始

\chapter{Integers}  % 整数
\section{Divisibility and Prime Numbers} % 整除性与素数
Let\footnote{
    Sometimes, natural numbers refer to the set of positive integers excluding zero, i.e., \(\mathbb{N}^{+}=\{1,2,3,\ldots\}\).
} 
\[
\mathbb{N}=\{0,1,2,3,\ldots\},\quad \mathbb{N}^{+}=\{1,2,3,\ldots\},\quad \mathbb{Z}=\{\ldots,-2,-1,0,1,2,\ldots\}.
\]
% 对于一个实数, 引入高斯符号

\begin{definition}{Gauß Symbols}
    For a real number \(x\), the floor function (greatest integer function) is defined as:
    \[
    \lfloor x \rfloor = \max\{n \in \mathbb{Z} \mid n \leq x\}.
    \]
    Similarly, the ceiling function (least integer function) is defined as:
    \[
    \lceil x \rceil = \min\{n \in \mathbb{Z} \mid n \geq x\}.
    \]
\end{definition}
\begin{property}
    \begin{enumerate}
        \item For any \(m \in \mathbb{N}^{+}\), there is \textbf{Hermite's identity}:
            \[
            \left\lfloor mx \right\rfloor = \left\lfloor x \right\rfloor + \left\lfloor x+ \frac{1}{m} \right\rfloor 
            + \cdots + \left\lfloor x+ \frac{m-1}{m} \right\rfloor.
            \]
            \[
            \left\lceil mx \right\rceil = \left\lceil x \right\rceil + \left\lceil x - \frac{1}{m} \right\rceil 
            + \cdots + \left\lceil x - \frac{m-1}{m} \right\rceil.
            \]
        \item 
    \end{enumerate}
\end{property}


\begin{theorem}{Euclidean Algorithm}
    For any integers \(a\) and \(b\) with \(b > 0\), there exist unique integers \(q\) and \(r\) such that
    \[
    a = bq + r, \quad 0 \leq r < b.
    \]
    \(r\) is called the remainder of \(a\) divided by \(b\), denoted as \(r = a \bmod b\).

    If \(r = 0\), then \(b\) divides \(a\), denoted as \(b \mid a\); 
    otherwise, \(b\) does not divide \(a\), denoted as \(b \nmid a\).
    In other words, \(b \mid a\) if and only if there exists an integer \(k\) such that \(a = bk\).

    If \(a=kb\) and \(b\neq a, b\neq 1\), then \(b\) is called a proper divisor of \(a\).
\end{theorem}

\begin{property}
    If \(b\neq 0, c\neq 0\), then
    \begin{enumerate}
        \item If \(b\mid a, c\mid b\), then \(c\mid a\).
        \item If \(b\mid a\), then \(bc \mid ac\).
        \item If \(c \mid d, c\mid e\), then \(c \mid (md + ne)\), for any integers \(m, n\).
    \end{enumerate}
\end{property}

\section{Carry System} % 进位制
Carry system (or positional numeral system) is a method of representing numbers using a radix (or base) \(r\) (\(r \geq 2\)).
In base \(r\), any non-negative integer \(N\) can be expressed as:
\[
N = a_k r^k + a_{k-1} r^{k-1} + \cdots + a_1 r + a_0 = \sum_{i=0}^{k} a_i r^i =: (a_k a_{k-1} \cdots a_1 a_0)_r,
\]
where \(a_i\) are the digits satisfying \(0 \leq a_i < r\) and \(a_k \neq 0\).

% 扩展到小数
This can be extended to decimal fractions as:
\[
N = a_k r^k + a_{k-1} r^{k-1} + \cdots + a_1 r + a_0 + a_{-1} r^{-1} + a_{-2} r^{-2} + \cdots = \sum_{i=-m}^{k} a_i r^i =: 
(a_k a_{k-1} \cdots a_1 a_0 . a_{-1} a_{-2} \cdots a_{-m})_r,
\]
where \(m\) is a positive integer.

\begin{leftbarTitle}{Radix Conversion}\end{leftbarTitle} % 进制转换
% 十进制与 r 进制的转换
Here are \emph{methods for converting between decimal and base \(r\)}:
\begin{itemize}
    \item Decimal to base \(r\):
        \begin{enumerate}
            \item For the integer part, repeatedly divide by \(r\) and record the remainders.
            \item For the fractional part, repeatedly multiply by \(r\) and record the integer parts.
            \item Combine the results to form the base \(r\) representation.
        \end{enumerate}
    \item Base \(r\) to decimal:
        \begin{enumerate}
            \item For the integer part, multiply each digit by \(r\) raised to its position power and sum them.
            \item For the fractional part, multiply each digit by \(r\) raised to its negative position power and sum them.
            \item Combine both sums to get the decimal representation.
        \end{enumerate}
\end{itemize}

\begin{example}
    \begin{enumerate}
        \item Convert decimal \(45.625\) to binary.
        \item Convert binary \((1101.101)_2\) to decimal.
    \end{enumerate}
\end{example}

\begin{solution}
\begin{enumerate}
    \item For integer part \(45\):
        \begin{align*}
            &45 \div 2 = 22 \text{ remainder } 1 \\
            &22 \div 2 = 11 \text{ remainder } 0 \\
            &11 \div 2 = 5 \text{ remainder } 1 \\
            &5 \div 2 = 2 \text{ remainder } 1 \\
            &2 \div 2 = 1 \text{ remainder } 0 \\
            &1 \div 2 = 0 \text{ remainder } 1 \\
        \end{align*}
        Reading remainders from bottom to top gives \(101101\).
        \newline For fractional part \(0.625\):
        \begin{align*}
            &0.625 \times 2 = 1.25 \quad (\text{integer part } 1) \\
            &0.25 \times 2 = 0.5 \quad (\text{integer part } 0) \\
            &0.5 \times 2 = 1.0 \quad (\text{integer part } 1)
        \end{align*}
        Reading integer parts gives \(101\).
        \newline Combining both parts, we get \(45.625_{10} = (101101.101)_2\).
    \item Since  
        \(
        1 \times 2^3 + 1 \times 2^2 + 0 \times 2^1 + 1 \times 2^0 + 1 \times 2^{-1} + 0 \times 2^{-2} + 1 \times 2^{-3} 
        = 8 + 4 + 0 + 1 + 0.5 + 0 + 0.125 = 13.625;
        \)
        thus, \((1101.101)_2 = 13.625_{10}\).
\end{enumerate}
\end{solution}

\begin{leftbarTitle}{Generalized Carry System}\end{leftbarTitle} % 广义进制系统

\begin{leftbarTitle}{Balanced Ternary}\end{leftbarTitle} % 平衡三进制
\textbf{Balanced ternary} (symmetric ternary) is a non-standard positional numeral system 
that uses three digits: \(-1\), \(0\), and \(1\).
Since \(-1\) is not a standard digit, it is often represented by the symbol \(Z\) or \(\bar{1}\).
The weight calculation is the same as standard ternary, with the weight of the \(i\)-th digit being \(3^{i}\).

\begin{theorem}{Uniqueness of Balanced Ternary Representation}
    Every integer can be uniquely represented in balanced ternary.
\end{theorem}

\begin{proposition}
    % 对于负数, 只需将对应整数的每一位取反即可
    For negative numbers, simply negate each digit of the corresponding positive integer's balanced ternary representation.
\end{proposition}

\vspace{0.7cm}
Here are \emph{methods for converting between decimal and balanced ternary}:
\begin{itemize}
    \item Decimal to balanced ternary:
        \begin{enumerate}
            \item Repeatedly divide the number by \(3\), recording the remainders.
            \item If a remainder is \(2\), replace it with \(-1\) (or \(Z\)) and increment the quotient by \(1\).
            \item Continue until the quotient is \(0\).
            \item Read the remainders from bottom to top to form the balanced ternary representation.
        \end{enumerate}
    \item Balanced ternary to decimal:
        \begin{enumerate}
            \item Multiply each digit by \(3\) raised to its position power and sum them.
            \item For digits equal to \(-1\) (or \(Z\)), treat them as \(-1\) in the calculation.
        \end{enumerate}
\end{itemize}


\begin{example}
    \begin{enumerate}
        \item Convert decimal \(64\) to balanced ternary. 
        \item Convert balanced ternary \(1Z0Z1\) to decimal.
    \end{enumerate}
\end{example}

\begin{solution}
\begin{enumerate}
    \item For integer part \(64\):
        \begin{align*}
            &64 \div 3 = 21 \text{ remainder } 1 \\
            &21 \div 3 = 7 \text{ remainder } 0 \\
            &7 \div 3 = 2 \text{ remainder } 1 \\
            &2 \div 3 = 0 \text{ remainder } 2 \quad (\text{replace } 2 \text{ with } Z, \text{ increment quotient to } 1) \\
            &1 \div 3 = 0 \text{ remainder } 1
        \end{align*}
        Reading remainders from bottom to top gives \(1Z0Z1\).
        Thus, \(64_{10} = (1Z0Z1)_{3b}\).
    \item Since  
        \(
        1 \times 3^4 + (-1) \times 3^3 + 0 \times 3^2 + (-1) \times 3^1 + 1 \times 3^0 
        = 81 - 27 + 0 - 3 + 1 = 52;
        \)
        thus, \((1Z0Z1)_{3b} = 52_{10}\).
\end{enumerate}
\end{solution}

\section{Greatest Common Divisor and Least Common Multiple} % 最大公约数与最小公倍数

\section{Fundamental Theorem of Arithmetic} % 算数基本定理 % 预备知识
\chapter{Permutations and Combinations} % 排列与组合 % 序列极限与实数系连续性
\chapter{Binomial Coefficients} % 二项式系数 % 函数的极限与连续性
\chapter{Fourier Transform} % 傅里叶变换

\section{Laplace Transform} % 拉普拉斯变换 % 微分学
\chapter{Recurrence Relations and Generating Functions} % 递归关系与生成函数
 % 不定积分
\chapter{Real Numbers} % 实数
\section{Construction of Real Numbers and the Cardinality of the Continuum} % 实数构造与连续统基数

\section{Point Sets in Euclidean Space} % Euclid 空间中的点集
In this section, we explore the point sets in Euclidean space.
Furthermore, these concepts can be generalized to metric spaces and topological spaces.

\begin{definition}{Diameter and Bounded Set} % 直径与有界集
    Let \(A\) be a subset of the Euclidean space \(\mathbb{R}^n\). 
    The \textbf{diameter} of set \(A\) is defined as
    \[
        \mathrm{diam}(A) = \sup \{d(x, y) \mid x, y \in A\},
    \]
    where \(d(x, y)\) denotes the Euclidean distance between points \(x\) and \(y\).
    
    A set \(A\) is called \textbf{bounded} if there exists a real number \(M > 0\) such that
    \[
        d(x, y) < M, \quad \forall x, y \in A.
    \]
    
    Let \(x_{0}\in \mathbb{R}^{n}, \delta>0\), the set
    \[
        B\left(x_{0}, \delta\right)=\left\{x \in \mathbb{R}^{n} \mid d\left(x, x_{0}\right)<\delta\right\}
    \]
    is called the \textbf{open ball} (or \textbf{neighborhood}) with center \(x_{0}\) and radius \(\delta\)\footnote{
        It can be also denoted as \(N(x_0, \delta)\) or \(U(x_0, \delta)\).
        % 当delta不需要被强调时, 也可以简写为 \(B(x_0)\).
        When \(\delta\) does not need to be emphasized, it can also be abbreviated as \(B(x_0)\).
    }.
    Similarly, the closed ball can be defined as
    \[
        \bar{B}\left(x_{0}, \delta\right)=\left\{x \in \mathbb{R}^{n} \mid d\left(x, x_{0}\right) \leq \delta\right\}.
    \]

    Let \(a_{i},b_{i}\) (\(i=1,2,\ldots,n\)) be real numbers with \(a_{i} < b_{i}\), the set
    \[
        \prod_{i=1}^n [a_i, b_i] = \{(x_1, x_2, \ldots, x_n) \in 
        \mathbb{R}^n \mid a_i \leq x_i \leq b_i \text{ for all } i=1,2,\ldots,n \}
    \]
    is called a \textbf{rectangle} (or \textbf{box}) in \(\mathbb{R}^n\).
    If all the edge lengths are equal, i.e., \(b_i - a_i = c\) for some constant \(c > 0\) and for all \(i\), 
    then the rectangle is called a \textbf{cube} with side length \(c\).
    Similarly, we can define the open rectangle (or open box) as
    \[
        \prod_{i=1}^n (a_i, b_i) = \{(x_1, x_2, \ldots, x_n) \in 
        \mathbb{R}^n \mid a_i < x_i < b_i \text{ for all } i=1,2,\ldots,n \}.
    \]
    Rectangles are often denoted by \(I, J, \ldots\) and their volumes by \(|I|, |J|, \ldots\).
\end{definition}

\begin{definition}{Limit}
    Let \(\{x_k\}\) be a sequence in \(\mathbb{R}^n\) and \(x \in \mathbb{R}^n\). 
    We say that \(\{x_k\}\) \textbf{converges} to \(x\), or \(x\) is the \textbf{limit} of the sequence \(\{x_k\}\), 
    if for every \(\varepsilon > 0\), there exists a natural number \(N\) such that
    \[
        d(x_k, x) < \varepsilon, \quad \forall k > N.
    \]
    In this case, we write
    \[
        \lim_{k \to \infty} x_k = x.
    \]
\end{definition}

\begin{leftbarTitle}{Classification of Points}\end{leftbarTitle} % 点的分类
\begin{definition}{Classification of Points} % 点的分类
    Let \(E\) be a subset of the Euclidean space \(\mathbb{R}^n\).
    Points in \(\mathbb{R}^n\) can be classified based on their relationship to set \(E\):
    \begin{description}
        \item[Interior Point] A point \(x \in E\) is called an \textbf{interior point} of set \(E\) 
            if there exists \(U(x)\) such that \(U(x) \subseteq E\).
        \item[Exterior Point] A point \(x \in \mathbb{R}^n \setminus E\) is called an \textbf{exterior point} of set \(E\) 
            if there exists \(U(x)\) such that \(U(x) \subseteq \mathbb{R}^n \setminus E \), or equivalently, 
            \(U(x) \cap E = \emptyset\).
        \item[Boundary Point] A point \(x \in \mathbb{R}^n\) is called a \textbf{boundary point} of set \(E\) 
            if for every \(U(x)\), the set \(U(x)\) contains points in both \(E\) and \(\mathbb{R}^n \setminus E\).
        \item[Accumulation Point (Limit Point)] A point \(x \in \mathbb{R}^n\) is called an 
            \textbf{accumulation point} (or \textbf{limit point}) of set \(E\) if for every \(U(x)\), 
            the set \(U(x)\) contains at least one point of \(E\) different from \(x\)\footnote{
                Obviously, only infinite sets can have accumulation points.
                % 事实上, 在这里, 邻域中包含一点(相异)与包含无穷多点是等价的.
                In fact, here, containing at least one (distinct) point in the neighborhood is equivalent to 
                containing infinitely many points.
            }.
        \item[Isolated Point] A point \(x \in E\) is called an \textbf{isolated point} of set \(E\) 
            if \(x\) is not an accumulation point of \(E\), i.e., there exists \(U(x)\) such that
            \(U(x) \cap E = \{x\}\).
    \end{description}
\end{definition}

% 任何点 \(x \in \mathbb{R}^n\) 都可以被唯一地分类为下列三类之一:
Any point \(x \in \mathbb{R}^n\) can be uniquely classified into one of the following three categories:
\[
\begin{cases}
\text{Interior Point} & \text{if } \exists U(x) \subseteq E; \\
\text{Boundary Point} & \text{if } \forall U(x) \cap (\mathbb{R}^n \setminus E) \neq \emptyset; \\
\text{Exterior Point} & \text{if } \exists U(x) \subseteq \mathbb{R}^n \setminus E; \\
\end{cases}
\]
% 或是唯一地分类为下列三类之一:
Or it can be uniquely classified into one of the following three categories:
\[
\begin{cases}
\text{Accumulation Point} & \text{if } \forall U(x) \cap (E \setminus \{x\}) \neq \emptyset; \\
\text{Isolated Point} & \text{if } \exists U(x) \cap E = \{x\}; \\
\text{Exterior Point} & \text{if } \exists U(x) \cap E = \emptyset; \\
\end{cases}
\]

\begin{definition}
    Let \(E\) be a subset of the Euclidean space \(\mathbb{R}^n\).
    \begin{description}
        \item[Derived Set] The \textbf{derived set} of \(E\), denoted by \(E'\), 
            is the set of all accumulation points of \(E\).
        \item[Interior] The \textbf{interior} of set \(E\), denoted by \(\mathrm{int}(E)\), or \(\mathring{E}\), 
            is the set of all interior points of \(E\).
        \item[Boundary] The \textbf{boundary} of set \(E\), denoted by \(\partial E\), 
            is the set of all boundary points of \(E\), or equivalently, \(\partial E = \bar{E} \setminus \mathring{E}\).
        \item[Closure] The \textbf{closure} of set \(E\), denoted by \(\bar{E}\), 
            is the union of \(E\) and its accumulation points, i.e., \(\bar{E} = E \cup E'\).
    \end{description}
\end{definition}

\begin{property}
    \begin{itemize}
        \item \(\left( \mathring{E} \right)^{c}  = \overline{E^c},\quad \left( \overline{E} \right)^{c} = \mathring{E^c}\);
        \item Let \(A \subseteq B\), then \(A' \subseteq B'\), \(\mathring{A} \subseteq \mathring{B}\) 
            and \(\overline{A} \subseteq \overline{B}\);
        \item \((A \cup B)' = A' \cup B'\).
    \end{itemize}
\end{property}



\begin{note}
    % 度量空间中, 可以给出聚点的另一定义, 即: 点 \(x\) 是集合 \(E\) 的聚点, 当且仅当它是某一\(E\)中点列的极限.
    In a metric space, an alternative definition of accumulation point can be given:
    A point \(x\) is an accumulation point of set \(E\) if and only if it is the limit of some sequence of points in \(E\).
\end{note}

\begin{remark}
    % 通过把欧氏距离替换为一般的度量 \(d\),上述所有定义都可以自然地推广到一般的度量空间 \((X, d)\) 中。
    By replacing the Euclidean distance with a general metric \(d\), 
    all the above definitions can be naturally extended to a general metric space \((X, d)\).

    % 通过把度量 \(d\) 替换为一般的拓扑结构中的开集族,上述所有定义都可以推广到一般的拓扑空间 \((X, \tau)\) 中。
    By replacing the metric \(d\) with the family of open sets in a general topological structure,
    all the above definitions can be extended to a general topological space \((X, \tau)\).
\end{remark}

\begin{leftbarTitle}{Open and Closed Sets}\end{leftbarTitle} % 开集与闭集
\begin{definition}{Classification of Point Sets} % 点集的分类
    Let \(E\) be a subset of the Euclidean space \(\mathbb{R}^n\).
    Point sets can be classified:
    \begin{description}
        \item[Closed Set] A set \(E\) is called a \textbf{closed set} if it contains all its accumulation points.
        \item[Open Set] A set \(E\) is called an \textbf{open set} if every point in \(E\) is an interior point of \(E\).
        \item[Compact Set] A set \(E\) is called a \textbf{compact set} if every open cover of \(E\) has a finite subcover,
            or equivalently, if \(E\) is closed and bounded (Heine-Borel Theorem).
        \item[Perfect Set] A set \(E\) is called a \textbf{perfect set} if it is closed and has no isolated points, 
            i.e., every point in \(E\) is an accumulation point of \(E\), or equivalently, \(E=E'\).
    \end{description}
\end{definition}

\begin{note}
    % 度量空间中, 可以给出闭集的另一定义, 即: 集合 \(E\) 是闭集, 当且仅当它包含其所有点列极限.
    In a metric space, an alternative definition of closed set can be given:
    A set \(E\) is closed if and only if it contains all its sequential limits.
    % 这是由于度量空间满足第一可数公理, 序列收敛与拓扑闭包等价.
    (This is because metric spaces satisfy the first countability axiom,
    and sequential convergence is equivalent to topological closure.)
    % 实际上, 在度量空间中, 闭集与序列闭集是等价的.
    In fact, in a metric space, closed sets and sequentially closed sets are equivalent.
    
    % 而在拓扑空间中, 开集与闭集的定义需要依赖于拓扑结构, 闭集一定是序列闭集, 反之不真.
    However, in a topological space, the definitions of open and closed sets depend on the topological structure,
    and closed sets are always sequentially closed, but the converse is not true.
\end{note}

\begin{theorem}{Open Set Construction Theorem} % 开集的构造定理
    % R^1直线上任一个非空开集可以表示成至多可数个互不相交的开区间的并集.
    % 进一步, R^n 中的任一非空开集都可以表示成可数个开矩形的并集.
\end{theorem}

 % 定积分
\chapter{Diophantine Equations} % 不定方程 % 反常积分
\chapter{Numerical Series} % 数项级数
\section{Convergence of Numerical Series}

\section{Positive Term Series and Its Convergence Tests}
\begin{definition}{Positive Term Series}
    If all terms of the series \( \sum_{n=1}^{\infty} x_n \) are non-negative real numbers, 
    i.e., \( x_n \geqslant 0 \) (\( x_n > 0 \)), \( n = 1, 2, \dots \), 
    then this series is called a \textbf{positive term series} (or strictly positive term series).
\end{definition}

\begin{note}
    The positive term series converges if and only if the partial sums of the sequence are bounded. 
    If the partial sums are unbounded, the series must diverge to \( +\infty \).
\end{note}

\begin{leftbarTitle}{Comparison Test}\end{leftbarTitle}
\begin{theorem}{Comparison Test}
    Let \( \sum_{n=1}^{\infty} a_n \) and \( \sum_{n=1}^{\infty} b_n \) be positive term series. 
    If \( \exists N \in \mathbb{N}, \text{ s.t. } \forall n > N: a_n \leqslant b_n \), then:
    \begin{enumerate}
        \item If \( \sum_{n=1}^{\infty} b_n \) converges, then \( \sum_{n=1}^{\infty} a_n \) also converges.
        \item If \( \sum_{n=1}^{\infty} a_n \) diverges, then \( \sum_{n=1}^{\infty} b_n \) also diverges.
    \end{enumerate}

    \textbf{Limit Form}
    Let \( \sum_{n=1}^{\infty} a_n \) and \( \sum_{n=1}^{\infty} b_n \) be positive term series, 
    and suppose \( \lim_{n \to \infty} \frac{a_n}{b_n} \) exists. Then:

    \begin{enumerate}
        \item If \( 0 < l < +\infty \), \( \sum_{n=1}^{\infty} a_n \) and \( \sum_{n=1}^{\infty} b_n \) 
            have the same convergence or divergence behavior.
        \item If \( l = 0 \), \( \sum_{n=1}^{\infty} b_n \) converges, 
            then \( \sum_{n=1}^{\infty} a_n \) also converges.
        \item If \( l = +\infty \), \( \sum_{n=1}^{\infty} b_n \) diverges, 
            then \( \sum_{n=1}^{\infty} a_n \) also diverges.
    \end{enumerate}
\end{theorem}

\begin{theorem}
\begin{description}
    \item[Cauchy Test] Let \( \sum_{n=1}^{\infty} a_n \) be a positive term series.
        \begin{enumerate}
            \item If \( \exists q \in [0,1), \text{ s.t. } \sqrt[n]{a_n} \leqslant 
            q < 1 \quad (n \geqslant N, N \in \mathbb{N}) \), then the series converges.
            \item If \( \sqrt[n]{a_n} \geqslant 1 \) for infinitely many \( n \), then the series diverges.
        \end{enumerate}
        \textbf{Limit Form} Let \( \sum_{n=1}^{\infty} a_n \) be a positive term series, 
        and suppose \( \varlimsup_{n \to +\infty} \sqrt[n]{a_n} = r \). Then:
        \begin{enumerate}
            \item If \( 0 \leqslant r < 1 \), the series \( \sum_{n=1}^{\infty} a_n \) converges.
            \item If \( r > 1 \), the series \( \sum_{n=1}^{\infty} a_n \) diverges.
            \item If \( r = 1 \), the test fails.
        \end{enumerate}

    \item[D'Alembert Test] Let \( \sum_{n=1}^{\infty} a_n \) be a strictly positive term series.
        \begin{enumerate}
            \item If \( \exists q \in [0,1), \text{ s.t. } \frac{a_{n+1}}{a_n} \leqslant 
            q < 1 \quad (n \geqslant N, N \in \mathbb{N}) \), then the series converges.
            \item If \( \frac{a_{n+1}}{a_n} \geqslant 1 \quad (n \geqslant N, N \in \mathbb{N}) \), 
            then the series diverges.
        \end{enumerate}
        \textbf{Limit Form}
        Let \( \sum_{n=1}^{\infty} a_n \) be a strictly positive term series. Then:
        \begin{enumerate}
            \item If \( \varlimsup_{n \to +\infty} \frac{a_{n+1}}{a_n} = r \in (0,1) \), the series converges.
            \item If \( \varliminf_{n \to +\infty} \frac{a_{n+1}}{a_n} = r' > 1 \), the series diverges.
            \item If \( r = 1 \) or \( r' = 1 \), the test fails.
        \end{enumerate}
    
    \item[Raabe Test] Let \( \sum_{n=1}^{\infty} a_n \) be a strictly positive term series.
        \begin{enumerate}
            \item If \( \exists r > 1, \exists N_0 \in \mathbb{N} \text{ s.t. } 
                \forall n > N_0: n \left( \frac{a_n}{a_{n+1}} - 1 \right) \geqslant r \), 
                then the series converges.
            \item If \( \exists N_0 \in \mathbb{N}, \text{ s.t. } \forall n > N_0: 
                n \left( \frac{a_n}{a_{n+1}} - 1 \right) \leqslant 1 \), then the series diverges.
        \end{enumerate}
        \textbf{Limit Form}
        Let \( \sum_{n=1}^{\infty} a_n \) be a strictly positive term series. Then:
        \begin{enumerate}
            \item If \( \varliminf_{n \to +\infty} n \left( \frac{a_n}{a_{n+1}} - 1 \right) = l > 1 \), the series converges.
            \item If \( \varlimsup_{n \to +\infty} n \left( \frac{a_n}{a_{n+1}} - 1 \right) = l' < 1 \), the series diverges.
            \item If \( l = 1 \) or \( l' = 1 \), the test fails.
        \end{enumerate}

    \item[Bertrand Test] Let \( \sum_{n=1}^{\infty} a_n \) be a strictly positive term series.
        \begin{enumerate}
            \item If \( \varliminf_{n \to +\infty} \ln n \left[ n \left( \frac{a_n}{a_{n+1}} - 1 \right) \right] = l > 1 \), 
                the series converges.
            \item If \( \varlimsup_{n \to +\infty} \ln n \left[ n \left( \frac{a_n}{a_{n+1}} - 1 \right) \right] = l' < 1 \), 
                the series diverges.
            \item If \( l = 1 \) or \( l' = 1 \), the test fails.
        \end{enumerate}

    \item [Gauß Test] Let \( \sum_{n=1}^{\infty} a_n \) be a strictly positive term series, and suppose:
        \[
        \frac{a_n}{a_{n+1}} = 1 + \frac{1}{n} + \frac{\delta}{n \ln n} + o\left( \frac{1}{n \ln n} \right), \quad (n \to +\infty).
        \]
        Then:
        \begin{enumerate}
            \item If \( \delta > 1 \), the series converges.
            \item If \( \delta < 1 \), the series diverges.
            \item If \( \delta = 1 \), the criterion fails.
        \end{enumerate}

        \textbf{Generalized Form}
        Let \( \sum_{n=1}^{\infty} a_n \) be a strictly positive term series, and suppose:
        \[
        \frac{a_n}{a_{n+1}} = 1 + \frac{1}{n} + \frac{\delta_n}{n \ln n} + o\left( \frac{1}{n \ln n} \right), 
        \quad (n \to +\infty).
        \]
        If \( \lim_{n \to \infty} \delta_n = \delta \in \mathbb{R} \), then:
        \begin{enumerate}
            \item If \( \delta > 1 \), the series converges.
            \item If \( \delta < 1 \), the series diverges.
            \item If \( \delta = 1 \), the criterion fails.
        \end{enumerate}
\end{description}
\end{theorem}

\begin{note}
    The Bertrand test can be refined by considering series such as:
    \[
    \sum_{n=3}^{\infty} \frac{1}{n \ln n (\ln \ln n)^p}, 
    \quad \sum_{n=9}^{\infty} \frac{1}{n \ln n \ln \ln n (\ln \ln n)^p}, 
    \cdots
    \]
    These refinements are collectively known as the Bertrand test.
\end{note}

\begin{remark}
    All the aforementioned criteria are derived from the Comparison Criterion.
    \begin{itemize}
        \item By comparing positive term series with the geometric series (or equal ratio series), 
            the Cauchy Criterion and d'Alembert Criterion are derived.
        \item By comparing positive term series with the slower-converging series 
            \( \sum_{n=1}^{\infty} \frac{1}{n^\alpha} \) (\( \alpha > 1 \)), the Raabe Criterion is derived.
        \item By comparing positive term series with the even slower-converging series 
            \( \sum_{n=1}^{\infty} \frac{1}{n \ln^\alpha n} \) (\( \alpha > 1 \)), the Gauß Criterion is derived.
    \end{itemize}
    \textbf{General Observation}
    The slower the convergence of the series used for comparison, the more precise the derived criterion.
\end{remark}






\begin{leftbarTitle}{Integral Test}\end{leftbarTitle}
\begin{theorem}{Cauchy Integral Test}
    Let \( f(x) \) be defined on \( [a, +\infty) \), 
    where \( f(x) \geqslant 0 \), and \( f(x) \) is Riemann integrable on any finite interval \( [a, A] \).

    Consider a monotonic increasing sequence \( \{ a_n \} \) such that \( a = a_1 < a_2 < \dots < a_n < \dots \), and let:
    \[
    u_n = \int_{a_n}^{a_{n+1}} f(x) \, \mathrm{d}x.
    \]

    Then the improper integral \( \int_{a}^{+\infty} f(x) \, \mathrm{d}x \) 
    and the positive term series \( \sum_{n=1}^{\infty} u_n \) converge or diverge to \( +\infty \) simultaneously. 
    Moreover:
    \[
    \int_{a}^{+\infty} f(x) \, \mathrm{d}x 
    = \sum_{n=1}^{\infty} u_n 
    = \sum_{n=1}^{\infty} \int_{a_n}^{a_{n+1}} f(x) \, \mathrm{d}x.
    \]
\end{theorem}

\begin{leftbarTitle}{Other Tests}\end{leftbarTitle}
\begin{theorem}{Cauchy Condensation Test}
    Let \( \{ a_n \} \) be a monotonically decreasing sequence of positive numbers. 
    Then the positive term series \( \sum_{n=1}^{\infty} a_n \) converges if and only if the condensed series:
    \[
    \sum_{n=0}^{\infty} 2^n a_{2^n} = a_1 + 2a_2 + 4a_4 + \dots + 2^n a_{2^n} + \dots
    \]
    converges.
\end{theorem}


\section{General Term Series and Its Convergence Tests}
\begin{leftbarTitle}{Cauchy Convergence Criterion for Series}\end{leftbarTitle}
\begin{theorem}{Cauchy Convergence Criterion for Series}
    The necessary and sufficient condition for the convergence of the series \( \sum_{n=1}^{\infty} x_n \) is:
    \[
    \forall \varepsilon > 0, \exists N \in \mathbb{N}, \forall m, n > N : 
    \left| x_{n+1} + x_{n+2} + \cdots +x_{m} \right| = \left| \sum_{k=n+1}^{m} x_k \right| < \varepsilon.
    \]
\end{theorem}


\begin{leftbarTitle}{Alternative Series}\end{leftbarTitle}
\begin{definition}{Alternative Series}
    A series of the form:
    \[
    \sum_{n=1}^{\infty}x_{n} = 
    \sum_{n=1}^{\infty} (-1)^{n-1} u_n\quad (u_{n}>0),
    \]
    is called an \textbf{alternative series}.

    Moreover, if \( u_n \) is a monotonically decreasing sequence and \( \lim_{n \to \infty} u_n = 0 \), 
    then the series is called a \textbf{Leibniz series}.
\end{definition}

\begin{theorem}{Leibniz Test}
    Leibniz series converges.
\end{theorem}


\begin{leftbarTitle}{Abel-Dirichlet Test}\end{leftbarTitle}

\begin{theorem}{Abel Transform (Discrete Integration by Parts/Summation by Parts)}\label{thm:Abel Transform}
    Let \(\{a_n\}, \{b_n\}\) be two sequences, then for any \(n\in \mathbb{N}^{+}\),
    \[
        \sum_{k=1}^{n} a_k b_k = a_n B_n + \sum_{k=1}^{n-1} (a_{k+1} - a_{k})B_k,
    \]
    where \(B_n = \sum_{k=1}^{n} b_k\).
\end{theorem}

\begin{figure}[h]
    \centering
    \includegraphics[width=0.5\textwidth, angle=180]{img/AbelTransform.jpg}
\end{figure}

\begin{lemma}{Abel Lemma (Discrete Second Integral Mean Value Theorem)}
    Let \(\{a_n\}, \{b_n\}\) be two sequences, if \(\{a_n\}\) is a monotonic sequence 
    and \(\{B_k\} = \sum_{k=1}^{n} b_k\) is a bounded sequence with bound \(M\),
    then for any \(p\in \mathbb{N}^{+}\),
    \[
        \left| \sum_{k=1}^{p} a_k b_k \right| \leqslant M \left( |a_{1}| + 2|a_{p}| \right) .
    \]
\end{lemma}

\begin{theorem}{Abel-Dirichlet Test}
    The series \(\sum_{n=1}^{\infty} a_n b_n\) converges if one of the following two conditions is satisfied:
    \begin{description}
        \item[Abel] \(\{a_n\}\) is a bounded monotonic sequence and \(\sum_{n=1}^{\infty} b_n\) converges.
        \item[Dirichlet]  \(\{a_n\}\) is a monotonic sequence, \(\lim_{n \to \infty} a_n = 0\),
            and the partial sums \(B_n = \sum_{k=1}^{n} b_k\) are bounded.       
    \end{description}
\end{theorem}

\section{Absolute and Conditional Convergence of Series}
\begin{definition}{Absolute and Conditional Convergence of Series}
    If the series \( \sum_{n=1}^{\infty} |x_n| \) converges, 
    then the series \( \sum_{n=1}^{\infty} x_n \) is said to be \textbf{absolutely convergent}.

    If the series \( \sum_{n=1}^{\infty} x_n \) converges but is not absolutely convergent, 
    then the series \( \sum_{n=1}^{\infty} x_n \) is said to be \textbf{conditionally convergent}.
\end{definition}

\section{Comparison of Convergence Speed of Series}
The series \( \sum_{n=1}^{\infty} a_n \) is said to converge faster than the series \( \sum_{n=1}^{\infty} b_n \) if:
\[
\lim_{n \to \infty} \frac{a_n}{b_n} = 0.
\]

\begin{theorem}{Du Bois-Reymond Theorem}
    For a given convergent positive term series \( \sum_{n=1}^{\infty} a_n \), there always exists a convergent strictly positive term series \( \sum_{n=1}^{\infty} b_n \) such that:
    \[
    \lim_{n \to \infty} \frac{a_n}{b_n} = 0.
    \]
\end{theorem}

\begin{theorem}{Abel Theorem}
    For a given divergent positive term series \( \sum_{n=1}^{\infty} a_n \), there always exists a divergent positive term series \( \sum_{n=1}^{\infty} b_n \) such that:
    \[
    \lim_{n \to \infty} \frac{a_n}{b_n} = 0.
    \]
\end{theorem}

\begin{remark}
    The above two theorems imply that the slowest converging positive term series \underline{does not} exist.
\end{remark}



\section{Infinite Products}
\begin{leftbarTitle}{Infinite Products}\end{leftbarTitle}


\begin{leftbarTitle}{Two Formulas}\end{leftbarTitle}
\begin{theorem}{Wallis Formula}
    \[
    \lim_{n \to \infty} \frac{1}{2n+1} \left[ \frac{(2n)!!}{(2n-1)!!} \right]^{2}  = \frac{\pi}{2}.
    \]
    Equivalently (\(n\to +\infty\)),
    \begin{gather*}
        \frac{(2n)!!}{(2n-1)!!} \sim \sqrt{\pi n}, \\
        \frac{(n!)^{2}2^{2n}}{(2n)!} \sim \sqrt{\pi n}.
    \end{gather*}
\end{theorem}


\begin{theorem}{Stirling Formula}
    \[
    n! = \sqrt{2\pi n} \left( \frac{n}{e} \right)^n 
    \left( 1 + \frac{1}{12n} - \frac{1}{288n^2} + \frac{139}{51840n^3} - \frac{571}{2488320n^4} + \cdots 
    + \frac{B_{2n}}{2k(2k-1) n^{k}} + \cdots  \right),
    \]
    where \( B_{2k} \) are Bernoulli numbers of order \( 2k \).
    Simplified form:
    \[
    n! \sim \sqrt{2\pi n} \left( \frac{n}{e} \right)^{n} \quad (n \to +\infty),
    \]
    or
    \[
    n! = \sqrt{2\pi n} \left( \frac{n}{e} \right)^{n} e^{\theta_n}, \quad \frac{1}{12n+1} < \theta_n < \frac{1}{12n}.
    \]
\end{theorem}


\section{Special Series}
\begin{description}
    \item[Geometric Series] 
        \[
        \sum_{n=0}^{\infty} q^n = \frac{1}{1-q},
        \]
        it converges when \( |q| < 1 \), diverges otherwise.
    \item[Telescoping Series]
        \[
        \sum_{n=1}^{\infty} (a_n - a_{n+1}) = a_1 - \lim_{n \to \infty} a_{n+1},
        \]
        it converges when \( \lim_{n \to \infty} a_n \) exists, diverges otherwise.
    \item[\(p\)-Series/Hyperharmonic Series]
        \[
        \sum_{n=1}^{\infty} \frac{1}{n^p},
        \]
        it converges when \( p > 1 \), diverges otherwise.
    \item[\(q\)-Series]
        \[
        \sum_{n=1}^{\infty} \frac{1}{n (\ln n)^q},
        \]
        it converges when \( q > 1 \), diverges otherwise.
    \item[Generalized \(q\)-Series]
        \[
        \sum_{n=3}^{\infty} \frac{1}{n \ln n (\ln \ln n)\cdots (\ln^{(k-1)} n) (\ln^{(k)} n)^q},
        \]
        where \( \ln^{(k)} n \) denotes the \( k \)-th iterated logarithm,
        it converges when \( q > 1 \), diverges otherwise.        
\end{description} % 数项级数
\chapter{Quadratic Forms} % 二次型
\section{Quadratic Forms and Their Standard Forms}
\begin{definition}{Quadratic Form}
    Let \(P\) be a number field, a quadratic homogeneous polynomial in \( n \) variables over \( P \)\footnote{
        That is, the coefficients of the polynomial belong to the field \( P \).
    }:
    \begin{align*}
        f( x_{1}, x_{2}, \cdots, x_{n}) &= \sum_{i=1}^{n} \sum_{j=1}^{n} a_{ij} x_{i} x_{j} \\
    &= a_{11}x_{1}^{2} + 2a_{12}x_{1}x_{2} + \cdots + 2a_{1n}x_{1}x_{n} + a_{22}x_{2}^{2} + \cdots + 2a_{2n}x_{2}x_{n} + \cdots + a_{nn}x_{n}^{2},
    \end{align*}
    is called a \textbf{quadratic form} in \( n \) variables over field \( P \).

    It can be expressed in matrix form as:
    \[
    f( x_{1}, x_{2}, \cdots, x_{n}) = X^{\mathrm{T}} A X,
    \]
    where 
    \[
    X = \begin{pmatrix}
        x_{1} \\
        x_{2} \\
        \vdots \\
        x_{n}
    \end{pmatrix}, \quad
    A = (a_{ij})_{n \times n}, \quad a_{ij} = a_{ji} \quad (1 \leqslant i, j \leqslant n).
    \]
\end{definition}
It is easy to verify that the matrix \( A \) of a quadratic form is symmetric.
\begin{note}
    In fact, for any square matrix \( B \)\footnote{
        For a skew-symmetric matrix \( S \) (\( S^{T} = -S \)),
        \[
        X^{T} S X = - (X^{T} S X)^{T} = - X^{T} S^{T} X = - X^{T} S X \implies X^{T} S X = 0.
        \]
    }, we have:
    \begin{align*}
        X^{T}BX &= X^{T}\left( \frac{B + B^{T}}{2} \right)X + X^{T}\left( \frac{B - B^{T}}{2} \right)X \\
        &= X^{T}\left( \frac{B + B^{T}}{2} \right)X + 0 \\
        &= X^{T}\left( \frac{B + B^{T}}{2} \right)X.
    \end{align*}
    It shows that any quadratic form can be represented by a symmetric matrix.
\end{note}




\section{Canonical Forms}
\section{Definite Quadratic Forms}
\begin{definition}{Positive Definite Quadratic Form}
    A real quadratic form \( f( x_{1}, x_{2}, \cdots, x_{n})=X^{\mathrm{T}}AX \) is called \textbf{positive definite} if:
    \[
    f( x_{1}, x_{2}, \cdots, x_{n}) > 0, \quad \forall X \neq 0.
    \]
    And \( A \) is called a \textbf{positive definite matrix}.
\end{definition}

\begin{theorem}{Sufficient and Necessary Condition for Positive Definiteness}
    A real quadratic form \( f( x_{1}, x_{2}, \cdots, x_{n})=X^{\mathrm{T}}AX \) is positive definite if and only if:
    \begin{enumerate}
        \item The positive inertia index of \( f \) is \( n \);
        \item \(A\) is congruent to the identity matrix \( E \);
        \item All eigenvalues of \( A \) are positive;
        \item All leading principal minors\footnote{
            The leading principal minors of a matrix are the determinants of 
            the top-left \( k \times k \) submatrices for \( k = 1, 2, \ldots, n \).
        } of \( A \) are positive.
    \end{enumerate}
    
\end{theorem} % 函数项级数
\chapter{Inner Product Spaces} % 内积空间
\section{Bilinear Forms}
\begin{definition}{Bilinear Form}
    Let \( V \) be a linear space over field \( F \).
    A function \( f: V \times V \to F \) is called a \textbf{bilinear form} on \( V \) if:
    \begin{enumerate}
        \item For any fixed \( \beta \in V \), the function \( f(\cdot, \beta): V \to F \) defined by
            \( f(\alpha, \beta) \) is a linear function on \( V \);
        \item For any fixed \( \alpha \in V \), the function \( f(\alpha, \cdot): V \to F \) defined by
            \( f(\alpha, \beta) \) is a linear function on \( V \).
    \end{enumerate}
    
\end{definition}

\section{Real Inner Product Spaces}
\begin{definition}{Real Inner Product Space}
    A \textbf{real inner product space} is a real linear space \( V \) 
    equipped with a function \( (\cdot, \cdot): V \times V \to \mathbb{R} \) 
    satisfying the following properties:
    \begin{description}
        \item [Positivity] \( (\alpha, \alpha) \geq 0, \quad \forall \alpha \in V\),
            and \( (\alpha, \alpha) = 0 \) if and only if \( \alpha = 0 \);
        \item [Symmetry] \( (\alpha, \beta) = (\beta, \alpha), \quad \forall \alpha, \beta \in V\);
        \item [Linearity in the First Argument] 
            \( (k_1\alpha_1 + k_2\alpha_2, \beta) = k_1(\alpha_1, \beta) + k_2(\alpha_2, \beta), 
            \quad \forall k_1, k_2 \in \mathbb{R}, 
            \alpha_1, \alpha_2, \beta \in V. \)
    \end{description}
    The function \( (\cdot, \cdot) \) is called the (real) \textbf{inner product}\footnote{
        The inner product can be also defined as a positive-definite bilinear form.
    } on \( V \).

    Real inner product spaces with finite dimensions are called \textbf{Euclidean spaces}. 
\end{definition}


\begin{definition}{Normed Linear Space}
    A real \textbf{normed linear space} is a real linear space \( V \) 
    equipped with a function \( \| \cdot \|: V \to \mathbb{R} \) 
    satisfying the following properties\footnote{
        Similarly, the definition of norm can be given in complex linear spaces.
    }:
    \begin{description}
        \item [Positivity] \( \| \alpha \| \geq 0, \quad \forall \alpha \in V\),
            and \( \| \alpha \| = 0 \) if and only if \( \alpha = 0 \);
        \item [Homogeneity] \( \| k\alpha \| = |k| \|\alpha\|, 
            \quad \forall k \in \mathbb{R}, \alpha \in V; \)
        \item [Triangle Inequality] \( \| \alpha + \beta \| \leq \|\alpha\| + \|\beta\|, 
            \quad \forall \alpha, \beta \in V. \)
    \end{description}
    The function \( \| \cdot \| \) is called the (vector) \textbf{norm}\footnote{
        If replace positivity with semi-positivity in the above definition,
        i.e., \( \| \alpha \| \geq 0, \quad \forall \alpha \in V\),
        then we get the definition of \textbf{semi-norm}.
    } on \( V \).
\end{definition}

In Euclidean spaces, the norm can be induced by the inner product:
\[
\| \alpha \| = (\alpha, \alpha)^{\frac{1}{2}}, \quad \forall \alpha \in V,
\]
which is called the \textbf{Euclidean norm}.

\begin{remark}
    The definition of Euclidean space can be also derived from normed linear space
    or metric space.
\end{remark}
\vspace{0.7cm}
\begin{theorem}{Cauchy-Буняко́вский-Schwarz Inequality}
    Let \( V \) be an inner product space\footnote{
        It \emph{does not} require in real inner product spaces.
    }.
    For any vectors \( \alpha, \beta \in V \), the following inequality holds:
    \[
    |(\alpha, \beta)| \leq \|\alpha\| \|\beta\|,
    \]
    with equality if and only if \( \alpha \) and \( \beta \) are linearly dependent.
\end{theorem}

\begin{note}
    \begin{enumerate}
        \item In linear space \(\mathbb{R}^{n}\), for vectors \( \alpha, \beta \in \mathbb{R}^{n} \), 
            define the inner product as:
            \[
            (\alpha, \beta) = \alpha^{\mathrm{T}} \beta = \sum_{i=1}^{n} x_{i} y_{i},
            \]
            then \(\mathbb{R}^{n}\) forms a Euclidean space, called the \textbf{standard Euclidean space}.
            We still denote it as \(\mathbb{R}^{n}\) without confusion.

            In this case, Cauchy-Буняко́вский-Schwarz inequality becomes:
            \[
            \left| \sum_{i=1}^{n} x_{i} y_{i} \right| \leqslant
            \left( \sum_{i=1}^{n} x_{i}^{2} \right)^{\frac{1}{2}}
            \left( \sum_{i=1}^{n} y_{i}^{2} \right)^{\frac{1}{2}}.
            \]

        \item In \(C[a, b]\), which is the linear space of continuous real-valued functions on interval \([a, b]\),
            for functions \( f(x), g(x) \in C[a, b] \),
            define the inner product as:
            \[
            (f, g) = \int_{a}^{b} f(x) g(x) \, \mathrm{d}x,
            \]
            then \(C[a, b]\) forms a Euclidean space.

            In this case, Cauchy-Буняко́вский-Schwarz inequality becomes:
            \[
            \left| \int_{a}^{b} f(x) g(x) \, \mathrm{d}x \right| \leqslant
            \left( \int_{a}^{b} f^{2}(x) \, \mathrm{d}x \right)^{\frac{1}{2}}
            \left( \int_{a}^{b} g^{2}(x) \, \mathrm{d}x \right)^{\frac{1}{2}}.
            \]
    \end{enumerate}
\end{note}

\vspace{0.7cm}
Some concepts in real inner product spaces can be derived:
\begin{description}
    \item[Distance] The distance between two vectors \( \alpha, \beta \in V \) is defined as:
    \[
    d(\alpha, \beta) = \| \alpha - \beta \|
    \]
    \item[Angle] The angle \( \theta \) between two non-zero vectors \( \alpha, \beta \in V \) is defined using the inner product:
    \[
    \cos \theta = \frac{(\alpha, \beta)}{\|\alpha\| \|\beta\|}, 0 \leq \theta \leq \pi.
    \]
    \item[Orthogonality] Two vectors \( \alpha, \beta \in V \) are said to be orthogonal if:
    \[
    (\alpha, \beta) = 0,
    \]
    denoted as \( \alpha \perp \beta \).
\end{description}

\section{Metric Matrices and Orthonormal Bases} % 度量矩阵与标准正交基
\begin{leftbarTitle}{Metric Matrices}\end{leftbarTitle}
\begin{definition}{Gram Matrix}
    Let \( V \) be a \(n\)-dimensional Euclidean space, 
    and let \( \{\varepsilon_1, \varepsilon_2, \dots, \varepsilon_n\} \) be a basis of \( V \).
    The matrix
    \[
    G = ((\varepsilon_i, \varepsilon_j))_{n \times n} =
    \begin{pmatrix}
        (\varepsilon_1, \varepsilon_1) & (\varepsilon_1, \varepsilon_2) & \cdots & (\varepsilon_1, \varepsilon_n) \\
        (\varepsilon_2, \varepsilon_1) & (\varepsilon_2, \varepsilon_2) & \cdots & (\varepsilon_2, \varepsilon_n) \\
        \vdots & \vdots & \ddots & \vdots \\
        (\varepsilon_n, \varepsilon_1) & (\varepsilon_n, \varepsilon_2) & \cdots & (\varepsilon_n, \varepsilon_n)
    \end{pmatrix}
    \]
    is called the \textbf{Gram matrix} (or metric matrix) of the inner product on \( V \)
    under the basis \( \{\varepsilon_1, \varepsilon_2, \dots, \varepsilon_n\} \).
\end{definition}

For all vectors
\begin{gather*}
    \alpha = x_{1}\varepsilon_{1}+x_{2}\varepsilon_{2}+\cdots+x_{n}\varepsilon_{n},  \\
    \beta = y_{1}\varepsilon_{1}+y_{2}\varepsilon_{2}+\cdots+y_{n}\varepsilon_{n} \in V ,
\end{gather*}
since \(( \alpha, \beta)=\sum_{i=1}^{n} \sum_{j=1}^{n}  (\varepsilon_i, \varepsilon_j)x_{i}y_{j} \), 
it can be expressed in matrix form as:
\[
( \alpha, \beta) = X^{\mathrm{T}} G Y, \quad
X = \begin{pmatrix} x_1 \\ x_2 \\ \vdots \\ x_n \end{pmatrix}, \quad
Y = \begin{pmatrix} y_1 \\ y_2 \\ \vdots \\ y_n \end{pmatrix}.
\]

\begin{property}
    \begin{enumerate}
        \item Gram matrix \( G \) is symmetric; 
        \item \( G \) is positive semi-definite;
        \item In \(V\), Gram matrices under different bases are congruent;
        \item \(|G|\geqslant 0\), equality holds if and only if the basis is linearly dependent.
    \end{enumerate}
\end{property}

\begin{leftbarTitle}{Orthonormal Bases}\end{leftbarTitle}
\begin{definition}{Orthonormal Vector Set and Orthonormal Basis}
    Let \( V \) be a \(n\)-dimensional real inner product space.
    A set of non-zero vectors \( \{ \varepsilon_1, \varepsilon_2, \dots, \varepsilon_n \} \) in \( V \) 
    is called an \textbf{orthonormal vector set} if they are pairwise orthogonal.

    A orthonormal vector set with \( n \) vectors is called a \textbf{orthonormal basis} of \( V \).
    Orthonormal basis made up of unit vectors is called a \textbf{orthonormal basis} of \( V \).
\end{definition}

\begin{property}
    In \(n\)-dimensional Euclidean space, 
    \begin{enumerate}
        \item Orthonormal vector set is linearly independent;
        \item A set of basis is orthonormal basis if and only if its Gram matrix is the identity matrix;
        \item Standard orthonormal basis always exists.
        \item Any orthonormal vector set can be extended to a orthonormal basis.
    \end{enumerate}
\end{property}
In \(n\)-dimensional Euclidean space, let \( \{ \varepsilon_1, \varepsilon_2, \dots, \varepsilon_n \} \) 
be a orthonormal basis, then
\begin{itemize}
    \item for all \(\alpha \in V:\alpha = \sum_{i=1}^{n} (\alpha, \varepsilon_i) \varepsilon_i\), 
        which is called the \textbf{Fourier expansion} of vector \( \alpha \) under the basis;
    \item if the coordinates of \(\alpha, \beta\) under this basis are
        \[
        X = \begin{pmatrix} x_1 \\ x_2 \\ \vdots \\ x_n \end{pmatrix}, \quad
        Y = \begin{pmatrix} y_1 \\ y_2 \\ \vdots \\ y_n \end{pmatrix},
        \]
        then
        \[
        (\alpha, \beta) = X^{\mathrm{T}} Y = \sum_{i=1}^{n} x_i y_i.
        \]
\end{itemize}

\begin{definition}{Orthonormal Matrix}
    A square matrix \( T \in \mathbb{R}^{n \times n} \) is called an \textbf{orthonormal matrix} if:
    \[
    T^{\mathrm{T}} T = T T^{\mathrm{T}} = E,
    \]
    where \( E \) is the identity matrix of order \( n \).
\end{definition}
Let \( V \) be a \(n\)-dimensional Euclidean space,
\(\varepsilon_{1}, \varepsilon_{2},\cdots, \varepsilon_{n}\) and \(\eta_{1}, \eta_{2},\cdots, \eta_{n}\)
be orthonormal bases of \( V \), and let \( T \) be the transition matrix between them,
i.e., 
\[
(\eta_{1}, \eta_{2}, \cdots, \eta_{n}) = (\varepsilon_{1}, \varepsilon_{2}, \cdots, \varepsilon_{n}) T.
\]
Then 
\begin{enumerate}
    \item \(T\) is an orthonormal matrix; 
    \item \(T\) is upper triangular matrix;
\end{enumerate}

\begin{theorem}
    Let \(\varepsilon_{1}, \varepsilon_{2},\cdots, \varepsilon_{n}\) be a orthonormal basis of 
    \(n\)-dimensional Euclidean space \( V \), and
    \[
    (\eta_{1}, \eta_{2}, \cdots, \eta_{n}) = (\varepsilon_{1}, \varepsilon_{2}, \cdots, \varepsilon_{n}) T.
    \]
    Then \( \eta_{1}, \eta_{2},\cdots, \eta_{n} \) is a orthonormal basis of \( V \)
    if and only if \( T \) is an orthonormal matrix.
\end{theorem}


\begin{leftbarTitle}{Gram-Schmidt Process}\end{leftbarTitle}
\begin{theorem}
    For any basis \( \{ \varepsilon_1, \varepsilon_2, \dots, \varepsilon_n \} \) of \(n\)-dimensional Euclidean space \( V \),
    there exists a orthonormal basis \( \{ \eta_1, \eta_2, \dots, \eta_n \} \) such that:
    \[
    \langle \varepsilon_1, \varepsilon_2, \dots, \varepsilon_k \rangle
    = \langle \eta_1, \eta_2, \dots, \eta_k \rangle, \quad k = 1, 2, \dots, n.
    \]
\end{theorem}

\begin{proof}
    
\end{proof}

\vspace{0.7cm}
\begin{example}
    Let \(n\geqslant 2\), \(A\) is an \(n\)-order real symmetric matrix.
    \(\alpha=(a_{1},a_{2},\ldots,a_{n})^{\mathrm{T}}, \beta=(b_{1},b_{2},\ldots,b_{n})^{\mathrm{T}}\),
    which are two eigenvectors of \(A\) corresponding to different eigenvalues \(\lambda_{1}, \lambda_{2}\) respectively.
    Let 
    \[
    B = \begin{pmatrix} 
        a_{1}+b_{1} & a_{1}+b_{2} & \cdots & a_{1}+b_{n} \\
        a_{2}+b_{1} & a_{2}+b_{2} & \cdots & a_{2}+b_{n} \\
        \vdots & \vdots & \ddots & \vdots \\ 
        a_{n}+b_{1} & a_{n}+b_{2} & \cdots & a_{n}+b_{n} 
    \end{pmatrix},
    \]
    find all eigenvalues of matrix \(B\).
\end{example}

\begin{solution}
    Let \(\xi_{1}=\frac{\alpha}{\|\alpha\|}, \xi_{2}=\frac{\beta}{\|\beta\|}\),
    and \(S_{a}=\sum_{i=1}^{n}a_{i}, S_{b}=\sum_{i=1}^{n}b_{i}\).
    \newline It is easy to see that \(\xi_{1}, \xi_{2}\) are orthonormal unit vectors.
    \newline Extend \(\{\xi_{1}, \xi_{2}\}\) to an orthonormal basis \(\{\xi_{1}, \xi_{2}, \ldots, \xi_{n}\}\) of \(\mathbb{R}^{n}\).
    \newline Let \(P=(\xi_{1}, \xi_{2}, \ldots, \xi_{n})\), then \(P\) is an orthonormal matrix.
    \newline \(B\) can be expressed as: 
    \[
    B = \alpha e^{\mathrm{T}} + e \beta^{\mathrm{T}},
    \]
    where \(e=(1,1,\ldots,1)^{\mathrm{T}}\).
    \newline Therefore,
    \[
    P^{-1}BP = P^{\mathrm{T}}BP = \begin{pmatrix} 
        \xi_{1}^{\mathrm{T}} \\ \xi_{2}^{\mathrm{T}} \\ \vdots \\ \xi_{n}^{\mathrm{T}} 
    \end{pmatrix} 
    \begin{pmatrix} 
        B \xi_{1} & B \xi_{2} & \cdots & B \xi_{n}
    \end{pmatrix}.
    \]
    Since 
    \[
    B\xi_{1} = (\alpha e^{\mathrm{T}} + e \beta^{\mathrm{T}}) \frac{a_{1}}{\|\alpha\|} = 
    \frac{\alpha(e^{\mathrm{T}}\alpha)}{\|\alpha\|} + \frac{e(\beta^{\mathrm{T}}\alpha)}{\|\alpha\|} =
    \frac{S_{a}}{\|\alpha\|} \alpha + 0 = S_{a} \|\alpha\| \xi_{1},
    \]
    and similarly, \(B\xi_{2} = S_{b} \|\beta\| \xi_{2}\).
    For \(k\geqslant 3\), we have
    \[
    B\xi_{k} = (\alpha e^{\mathrm{T}} + e \beta^{\mathrm{T}}) \xi_{k} = 
    \alpha (e^{\mathrm{T}} \xi_{k}) + e (\beta^{\mathrm{T}} \xi_{k}) = 0 + 0 = 0.
    \]
    Then 
    \[
    P^{\mathrm{T}}BP = \begin{pmatrix} 
        S_{a}  & 0 & 0 & \cdots & 0 \\
        0 & S_{b} & 0 & \cdots & 0 \\
        0 & 0 & 0 & \cdots & 0 \\
        \vdots & \vdots & \vdots & \ddots & \vdots \\
        0 & 0 & 0 & \cdots & 0
    \end{pmatrix}.
    \]
    Therefore, the eigenvalues of matrix \(B\) are:
    \[
    \lambda_{1} = S_{a}, \quad \lambda_{2} = S_{b}, \quad \lambda_{3} = 0, \quad \ldots, \quad \lambda_{n} = 0.
    \]
\end{solution}

\section{Isomorphism of Real Inner Product Spaces}

\section{Orthogonal Completion and Orthogonal Projection}
\begin{leftbarTitle}{Orthogonal Completion}\end{leftbarTitle}
\begin{definition}{Orthogonal Complement}
    Let \( V \) be a real inner product space, and let \( W \) be a subspace of \( V \).
    The set
    \[
    W^{\perp} = \{ \alpha \in V \mid (\alpha, \beta) = 0, \quad \forall \beta \in W \}
    \]
    is called the \textbf{orthogonal complement}\footnote{
        Another equivalent definition is:
        \[
        W^{\perp} \perp W \text{ and } V = W + W^{\perp}.
        \]
    } of \( W \) in \( V \),
    and \( W^{\perp} \) is also a subspace of \( V \), called the \textbf{orthogonal subspace} of \( W \) in \( V \).
\end{definition}
\begin{property}
    \begin{enumerate}
        \item There exists a unique orthogonal complement \( W^{\perp} \) of \( W \) in Euclidean space \( V \);
        \item \(W \oplus W^{\perp} = V\);
        \item \(\left( W^{\perp} \right)^{\perp} = W\)
        \item \(\left( V_{1}+V_{2} \right)^{\perp} = V_{1}^{\perp} \cap V_{2}^{\perp}, \quad 
            \left( V_{1} \cap V_{2} \right)^{\perp} = V_{1}^{\perp} + V_{2}^{\perp} \),
            where \( V_{1}, V_{2} \) are subspaces of \( V \).
    \end{enumerate}
\end{property}


\begin{leftbarTitle}{Least Squares Method}\end{leftbarTitle}
\begin{definition}{Orthogonal Projection}
    Let \( V \) be a real inner product space, 
    and let \( W \) be a subspace of \( V \).
    For any vector \( \alpha \in V \),
    if there exists a vector \( \beta \in W \) such that:
    \[
    \alpha - \beta \in W^{\perp},
    \]
    then \( \beta \) is called the \textbf{orthogonal projection} of \( \alpha \) onto \( W \),
    denoted as \( \beta = \operatorname{proj}_{W} \alpha \).
\end{definition}

% 最佳逼近元
\begin{definition}{Best Approximation Element}
    Let \( V \) be a real inner product space, 
    and let \( W \) be a subspace of \( V \).
    For any vector \( \alpha \in V \),
    if there exists a vector \( \beta \in W \) such that:
    \[
    \| \alpha - \beta \| = \min_{\gamma \in W} \| \alpha - \gamma \|,
    \]
    then \( \beta \) is called the \textbf{best approximation element} of \( \alpha \) in \( W \).
\end{definition}

\begin{property}
    In Euclidean space \( V \), let \( W \) be an \(n\)-dimensional subspace of \( V \).
    Take an orthogonal basis \( \{ \varepsilon_1, \varepsilon_2, \dots, \varepsilon_n \} \) of \( W \).
    \begin{enumerate}
        \item For any vector \( \alpha \in V \),
            the orthogonal projection of \( \alpha \) onto \( W \) exists and is unique,
            and it is also the best approximation element of \( \alpha \) in \( W \);
        \item The orthogonal projection of \( \alpha \) onto \( W \) can be expressed as:
            \[
            \operatorname{proj}_{W} \alpha = 
            \sum_{i=1}^{n} \frac{(\alpha, \varepsilon_i)}{(\varepsilon_i, \varepsilon_i)} \varepsilon_i
            =: \sum_{i=1}^{n} c_{i} \varepsilon_i.
            \]
        \item The remainder of the orthogonal projection satisfies:
            \[
            \| \alpha - \operatorname{proj}_{W} \alpha \|^{2} = \| \alpha \|^{2} - \| \operatorname{proj}_{W} \alpha \|^{2}
            = \| \alpha \|^{2} - \sum_{i=1}^{n} c_{i}^{2} \|\varepsilon_i\|^{2} .
            \]
    \end{enumerate}
\end{property}

\section{Orthogonal Transformations and Symmetric Transformations}
\begin{leftbarTitle}{Orthogonal Transformations}\end{leftbarTitle}
\begin{leftbarTitle}{Symmetric Transformations}\end{leftbarTitle}
\begin{definition}{Symmetric Transformation}
    Let \( V \) be a real inner product space, 
    and let \( \mathcal{A}\in \operatorname{Hom}(V) \).
    If
    \[
    (\mathcal{A}\alpha, \beta) = (\alpha, \mathcal{A}\beta), \quad \forall \alpha, \beta \in V,
    \]
    then \( \mathcal{A} \) is called a \textbf{symmetric transformation} on \( V \).
\end{definition}
Obviously, symmetric transformations are linear transformations.

\begin{lemma}
    \begin{enumerate}
        \item If \( A \) is real symmetric matrix, then \( A \) has and only has \( n \) real eigenvalues (counting multiplicities). 
        \item Let \(V\) be a \(n\)-dimensional Euclidean space, 
            then any linear transformation \( \mathcal{A}: V \to V \) is symmetric if and only if
            there exists a orthonormal basis of \( V \) such that 
            the matrix representation of \( \mathcal{A} \) under this basis is a symmetric matrix.
        \item If \( \mathcal{A} \) is a symmetric transformation on a real inner product space \( V \),
            then if \(W\) is an invariant subspace of \( V \) under \( \mathcal{A} \),
            then its orthogonal complement \( W^{\perp} \) is also an invariant subspace of \( V \) under \( \mathcal{A} \).
        \item If \(A\) is a symmetric transformation on a Euclidean space \( V \),
            then eigenvectors corresponding to distinct eigenvalues are definitely orthogonal.
    \end{enumerate}
\end{lemma}

\begin{theorem}{Orthogonal Diagonalization of Symmetric Transformations}
    Let \( V \) be a \( n \)-dimensional Euclidean space, 
    and let \( \mathcal{A}: V \to V \) be a symmetric transformation on \( V \).
    Then there exists a orthonormal basis of \( V \) such that 
    the matrix representation of \( \mathcal{A} \) under this basis is a diagonal matrix.
    
    From the perspective of matrices,
    let \( A \in \mathbb{R}^{n \times n} \) be a real symmetric matrix.
    Then there exists an orthogonal matrix \( T \) such that:
    \[
    T^{-1} A T = T^{\mathrm{T}} A T = \Lambda = 
        \begin{pmatrix}
        \lambda_1 & 0 & \cdots & 0 \\
        0 & \lambda_2 & \cdots & 0 \\
        \vdots & \vdots & \ddots & \vdots \\
        0 & 0 & \cdots & \lambda_n
        \end{pmatrix},
    \]
    where \( \lambda_1, \lambda_2, \dots, \lambda_n \) are the eigenvalues of \( A \).
\end{theorem}

\begin{note}
    According to the theorem above,
    for all real quadratic forms \( f(x_1, x_2, \dots, x_n) = X^{\mathrm{T}} A X \),
    there exists an orthogonal transformation \( X = T Y \) such that:
    \[
    f(x_1, x_2, \dots, x_n) = Y^{\mathrm{T}} \Lambda Y = \sum_{i=1}^{n} \lambda_i y_i^2.
    \]
\end{note}

\section{Unitary Spaces and Normal Transformations}
\begin{leftbarTitle}{Unitary Spaces}\end{leftbarTitle}
\begin{definition}{Complex Inner Product Space}
    A \textbf{complex inner product space} is a complex linear space \( V \) 
    equipped with a function \( (\cdot, \cdot): V \times V \to \mathbb{C} \) 
    satisfying the following properties:
    \begin{description}
        \item [Positivity] \( (\alpha, \alpha) \geq 0, \quad \forall \alpha \in V\),
            and \( (\alpha, \alpha) = 0 \) if and only if \( \alpha = 0 \);
        \item [Conjugate Symmetry (Hermitian)] \( (\alpha, \beta) = \overline{(\beta, \alpha)}, \quad \forall \alpha, \beta \in V\);
        \item [Linearity in the First Argument] 
            \( (k_1\alpha_1 + k_2\alpha_2, \beta) = k_1(\alpha_1, \beta) + k_2(\alpha_2, \beta), 
            \quad \forall k_1, k_2 \in \mathbb{C}, 
            \alpha_1, \alpha_2, \beta \in V. \)
    \end{description}
    The function \( (\cdot, \cdot) \) is called the (complex) \textbf{inner product} on \( V \).

    Complex inner product spaces with finite dimensions are called \textbf{unitary spaces}.
\end{definition}
The norm in unitary spaces can be also induced by the inner product:
\[
\| \alpha \| = (\alpha, \alpha)^{\frac{1}{2}}, \quad \forall \alpha \in V.
\]




\begin{leftbarTitle}{Normal Transformation and Diagonalization}\end{leftbarTitle}
Define the conjugate transpose (Hermitian adjoint) of a complex matrix \( A \in \mathbb{C}^{m \times n} \) as:
\[
A^{\dagger} = \overline{A}^{\mathrm{T}}.
\]
Similarly, for a linear transformation \( \mathcal{A} \in \operatorname{Hom}(V) \) on a complex inner product space \( V \),
the \textbf{adjoint transformation} \( \mathcal{A}^{\dagger} \) of \( \mathcal{A} \) is defined as: % 伴随变换
\[
(\mathcal{A}\alpha, \beta) = (\alpha, \mathcal{A}^{\dagger}\beta), \quad \forall \alpha, \beta \in V.
\]

\begin{definition}{Normal Transformation and Normal Matrix}
    Let \( V \) be a complex inner product space, 
    and let \( \mathcal{A}\in \operatorname{Hom}(V) \).
    If
    \[
    \mathcal{A} \mathcal{A}^{\dagger} = \mathcal{A}^{\dagger} \mathcal{A},
    \]
    then \( \mathcal{A} \) is called a \textbf{normal transformation} on \( V \).

    A square matrix \( A \in \mathbb{C}^{n \times n} \) is called a \textbf{normal matrix} if:
    \[
    A A^{\dagger} = A^{\dagger} A.
    \]
\end{definition}


\begin{theorem}
    Let \(\mathcal{A}\) be a normal transformation on a unitary space \( V \).
    Then there exists an orthonormal basis of \( V \) such that
    the matrix representation of \( \mathcal{A} \) under this basis is a diagonal matrix.

    From the perspective of matrices,
    let \( A \in \mathbb{C}^{n \times n} \) be a normal matrix.
    Then there exists a unitary matrix \( U \) such that:
    \[
    U^{-1} A U = U^{\dagger} A U = \Lambda = 
        \begin{pmatrix}
        \lambda_1 & 0 & \cdots & 0 \\
        0 & \lambda_2 & \cdots & 0 \\
        \vdots & \vdots & \ddots & \vdots \\
        0 & 0 & \cdots & \lambda_n
        \end{pmatrix},
    \]
    where \( \lambda_1, \lambda_2, \dots, \lambda_n \) are the eigenvalues of \( A \).
\end{theorem}

\begin{leftbarTitle}{Special Normal Transformation}\end{leftbarTitle}
\begin{definition}{Unitary Transformation and Unitary Matrix}
    Let \( V \) be a complex inner product space, 
    and let \( \mathcal{A}\in \operatorname{Hom}(V) \).
    If
    \[
    \mathcal{A}^{\dagger} = \mathcal{A}^{-1}, \quad \text{or} \quad
    (\mathcal{A}\alpha, \mathcal{A}\beta) = (\alpha, \beta), \quad \forall \alpha, \beta \in V,
    \]
    then \( \mathcal{A} \) is called a \textbf{unitary transformation} on \( V \).

    A square matrix \( A \in \mathbb{C}^{n \times n} \) is called a \textbf{unitary matrix} if:
    \[
    A^{\dagger} = A^{-1},\quad \text{or} \quad A A^{\dagger} = A^{\dagger} A = E.
    \]
\end{definition}
Obviously, unitary transformations are generalizations of orthogonal transformations in real inner product spaces.


\begin{definition}{Hermitian Transformation and Hermitian Matrix}
    Let \( V \) be a complex inner product space, 
    and let \( \mathcal{A}\in \operatorname{Hom}(V) \).
    If
    \[
    (\mathcal{A}\alpha, \beta) = (\alpha, \mathcal{A}\beta), \quad \forall \alpha, \beta \in V,
    \]
    then \( \mathcal{A} \) is called a \textbf{Hermitian transformation} on \( V \).

    A square matrix \( A \in \mathbb{C}^{n \times n} \) is called a \textbf{Hermitian matrix} if:
    \[
    A^{\dagger} = A.
    \]
\end{definition}
Obviously, Hermitian transformations are generalizations of symmetric transformations in real inner product spaces.







\section{Symplectic Spaces} % 幂级数
\chapter{Pólya Counting} % Pólya 计数
 % Euclidean Spaces 上的极限与连续性
\chapter{Multi-variable Differential Calculus} % 多元微分学
\section{Directional Derivatives and Total Differential}
\begin{leftbarTitle}{Directional Derivative}\end{leftbarTitle}
\begin{definition}{Directional Derivative}
    Let \(U\subset \mathbb{R}^n\) be an open set, \(f: U\to \mathbb{R}^{1}\),
    \(\mathbf{e}\) is a unit vector in \(\mathbb{R}^{n}\), \(\mathbf{x}^{0}\in U\). Define
    \[
    u(t) = f(\mathbf{x}^{0} + t\mathbf{e}).
    \]
    If the derivative of \(u\) at \(t=0\) 
    \[ 
        u'(0) = \lim_{t \to 0} \frac{u(t) - u(0)}{t} = 
        \lim_{t \to 0} \frac{f(\mathbf{x}^{0} + t\mathbf{e}) - f(\mathbf{x}^{0})}{t} 
    \] 
    exists and is finite, 
    it is called the \textbf{directional derivative} of \(f\) at \(\mathbf{x}^{0}\) in the direction \(\mathbf{e}\), 
    denoted by \(\frac{\partial f}{\partial \mathbf{e}}(\mathbf{x}^{0})\). 
    It is the rate of change of \(f\) at \(\mathbf{x}^{0}\) in the direction \(\mathbf{e}\).
\end{definition}

Consider the following set of unit coordinate vectors: \(\mathbf{e}_{1},\mathbf{e}_{2},\cdots,\mathbf{e}_{n}\).
Let \(\mathbf{e}_{i}=\left( 0, 0, \cdots, 0, 1, 0, \cdots, 0 \right)  \) denote the standard orthonormal basis 
in \(\mathbb{R}^{n}\), where the 1 appears in the \(i\)-th position. That is,
\[
    \langle \mathbf{e}_{i}, \mathbf{e}_{j} \rangle = \delta_{i j} = \begin{cases}
    1, & i = j, \\
    0, & i \neq j.
    \end{cases}
\]
For a function \( f \), the directional derivative of \( f \) at the point \( \mathbf{x}_{0} \) 
in the direction of \( \mathbf{e}_{i} \) 
is called the \( i \)th first-order \textbf{partial derivative} of \( f \) at \(\mathbf{x}^{0}\), denoted by
\[
\frac{\partial f}{\partial x_i}(\mathbf{x}^{0}) \quad \text{or} 
\quad \mathrm{D}_i f(\mathbf{x}^{0})  \quad \text{or} 
\quad f_{x_i}(\mathbf{x}^{0}) \quad (i = 1, 2, \cdots, n).
\]
\( \mathrm{D}_i = \frac{\partial}{\partial x_i} \) is called the \( i \)th partial differential operator (\( i = 1, 2, \cdots, n \)).

Let \(\mathbf{e}_{i}=\sum_{i=0}^{n} \mathbf{e}_{i}\cos\alpha_{i}\) be a unit vector, 
where \(\sum_{i=0}^{n} \cos^2\alpha_{i} = 1\).
If \(\frac{\partial f}{\partial x_{i}}\) is continuous at \(\mathbf{x}^0\), 
then the directional derivative of \(f\) at \(\mathbf{x}^0\) along the direction \(\mathbf{e}\) is given by:
\[
\frac{\partial f}{\partial \mathbf{e}}(\mathbf{x}^0) = \sum_{i=1}^n \frac{\partial f}{\partial x_i}(\mathbf{x}^0) \cos \alpha_{i}.
\]

This is the formula for \textbf{expressing a directional derivative using partial derivatives}.


\begin{note}
    Let \(\mathbf{e}\) be a direction, then \(\|-\mathbf{e}\| = \|\mathbf{e}\| = 1\), 
    which implies that \(-\mathbf{e}\) is also a direction. At this point, we have:
    \[
    \frac{\partial f}{\partial (-\mathbf{e})}(\mathbf{x}^{0}) = -\frac{\partial f}{\partial \mathbf{e}}(\mathbf{x}^{0}).
    \]
\end{note}



\begin{definition}{Jacobian Matrix (Gradient)}
    Let
    \[
    Jf(\mathbf{x}) = (\mathrm{D}_1 f(\mathbf{x}), \mathrm{D}_2 f(\mathbf{x}), \dots, \mathrm{D}_n f(\mathbf{x})),
    \]
    which is called the \textbf{Jacobian matrix} of the function \( f \) at the point \( \mathbf{x} \), 
    (a \( 1 \times n \) matrix) whose counterpart is the first-order derivative of a single-variable function.

    Henceforth, we represent the point \(\mathbf{x}\) in \( \mathbb{R}^n \) 
    and its increments \(\mathbf{h}\) as column vectors.
    In this way, the differential of the function can be expressed using matrix multiplication as follows:
    \[
    \mathrm{d}f(\mathbf{x}^{0})(\mathbf{\Delta x}) = Jf(\mathbf{x}^{0}) \mathbf{\Delta x}.
    \]
    The Jacobian matrix of the function \( f \) is also frequently denoted as 
    \(\mathrm{grad}\,f\) (or \(\nabla f\)), that is,
    \[
    \nabla f(\mathbf{x}) =\mathrm{grad}\,f(\mathbf{x}) = Jf(\mathbf{x}),
    \]
    which is called the \textbf{gradient} of the scalar function \( f \).
\end{definition}



\begin{leftbarTitle}{Total Differential}\end{leftbarTitle}
\begin{definition}{Total Differential}
    Let \(U\subset \mathbb{R}^n\) be an open set, \(f: U\to \mathbb{R}^{1}\), \(\mathbf{x}^{0}\in U\),
    \(\mathbf{\Delta x}=\left( \Delta x_{1},\Delta x_{2},\cdots,\Delta x_{n} \right) \in \mathbb{R}^{n}\). If
    \[
    f(\mathbf{x}^{0} + \mathbf{\Delta x}) - f(\mathbf{x}^{0}) = 
    \sum_{i=1}^n A_{i} \Delta x_{i} + o(\|\mathbf{\Delta x}\|) \qquad (\|\mathbf{\Delta x}\| \to 0),
    \]
    where \(A_{1}, A_{2}, \dots, A_{n}\) are constants independent of \(\mathbf{\Delta x}\), 
    then the function \(f\) is said to be \textbf{differentiable} at the point \(\mathbf{x}^{0}\), 
    and the linear main part \(\sum_{i=1}^n A_{i} \Delta x_{i}\) is called the \textbf{total differential} 
    of \(f\) at \(\mathbf{x}^{0}\), 
    denoted as
    \[
    df(\mathbf{x}^{0})(\mathbf{\Delta x}) = \sum_{i=1}^n A_{i} \Delta x_{i}.
    \]
    If \(f\) is differentiable at every point in the open set \(U\), 
    then \(f\) is called a differentiable function on \(U\).    
\end{definition}

\begin{theorem}{Conditions of Differentiability}
    \begin{description}
        \item[Necessary Condition] If an \(n\)-variable function \(f\) is differentiable at the point \(\mathbf{x}_{0}\), 
        then \(f\) is continuous at \(\mathbf{x}^{0}\) and 
        possesses first-order partial derivatives \(\frac{\partial f}{\partial x_{i}}(\mathbf{x}^{0})\) 
        at \(\mathbf{x}^{0}\) for \(i = 1, 2, \dots, n\), and\footnote{
            It is referred to as the total differential formula, and the more common form is
            \[
                \mathrm{d}f(x_{0},y_{0})=
                \frac{\partial f}{\partial x}(x_{0},y_{0})\,\mathrm{d}x+\frac{\partial f}{\partial y}(x_{0},y_{0})\,\mathrm{d}y.
            \]
        }
        \[
        \mathbf{A} = \left( A_{1}, A_{2}, \dots, A_{n} \right)  = Jf(\mathbf{x}^{0}) = 
        \left(\mathrm{D}_{1}f(\mathbf{x}^{0}), \mathrm{D}_{2}f(\mathbf{x}^{0}), \dots, \mathrm{D}_{n}f(\mathbf{x}^{0}) \right).
        \]
        However, the converse is not true.
        %%%%%%%%%%%%%%%%%%%%%%%%%%%%%%%%%%%%%%%%%%%%%%%%%%%%%%%%%%%%%%%%%%%%%%%%%%%%%%%%%%%%%%%%%
        \item[Sufficient Condition] Let \(U \subset \mathbb{R}^n\) be an open set, 
        and let \(f: U \to \mathbb{R}^1\) be an \(n\)-variable function. 
        If \(Jf = \left( \mathrm{D}_{1}f, \mathrm{D}_{2}f, \dots, \mathrm{D}_{n}f \right)\) 
        is continuous at \(\mathbf{x}^{0}\) 
        (i.e., \(\frac{\partial f}{\partial x_{i}}\) is continuous at \(\mathbf{x}^{0}\) for \(i = 1, 2, \dots, n\)), 
        then \(f\) is differentiable at \(\mathbf{x}^{0}\)\footnote{
            In fact, this condition can be relaxed to require that one partial derivative exists at the point, 
            while the remaining \(n-1\) partial derivative functions are continuous at that point.
        }.

        However, the converse is not necessarily true.
    \end{description}    
\end{theorem}

\begin{note}
    The continuity of the derivative function at \(\mathbf{x}^{0}\) implies that 
    the original function \(f\) is differentiable in some neighborhood of \(\mathbf{x}^{0}\).
\end{note}
\begin{proof}
    Taking the function of three variables as an example.

    Assume the \(3\)-ary function \(f: \mathbb{R}^3 \to \mathbb{R}\) meets:
    \begin{enumerate}
        \item There exists \(f_{z}(x_{0},y_{0},z_{0})\). 
        \item The partial derivative functions \(f_{x}(x,y,z)\) and \(f_{y}(x,y,z)\) are continuous at \((x_{0},y_{0},z_{0})\),
            i.e. there are partial derivatives in some neighborhood of \((x_{0},y_{0},z_{0})\).
    \end{enumerate}
    Consider the total increment of \(f\) at the point \((x_{0},y_{0},z_{0})\):
    \begin{align*}
        \Delta f 
        &= \underbrace{\left[ f(x_0 + \Delta x, y_0 + \Delta y, z_0 + \Delta z) - f(x_0, y_0 + \Delta y, z_0 + \Delta z) \right]}_{I_1}\\
        &+ \underbrace{\left[ f(x_0, y_0 + \Delta y, z_0 + \Delta z) - f(x_0, y_0, z_0 + \Delta z) \right]}_{I_2}\\
        &+ \underbrace{\left[ f(x_0, y_0, z_0 + \Delta z) - f(x_0, y_0, z_0) \right]}_{I_3}.
    \end{align*}
    For \(I_{1}, I_{2}\), by the Lagrange's Mean Value Theorem of unary functions, 
    there exist \(\theta_{1}, \theta_{2} \in (0,1)\) such that
    \begin{gather*}
        I_{1}=f_{x}(x_{0}+\theta_{1}\Delta x,y_{0}+\Delta y,z_{0}+\Delta z)\Delta x,\\
        I_{2}=f_{y}(x_{0},y_{0}+\theta_{2}\Delta y,z_{0}+\Delta z)\Delta y.
    \end{gather*}
    Then by the continuity of the their partial derivatives at \((x_{0},y_{0},z_{0})\), we have
    \[
        \lim_{\Delta x, \Delta y, \Delta z \to 0} I_{1} = f_{x}(x_{0},y_{0},z_{0})\Delta x, \quad
        \lim_{\Delta x, \Delta y, \Delta z \to 0} I_{2} = f_{y}(x_{0},y_{0},z_{0})\Delta y.
    \]
    They can be expressed in terms of infinitesimals(\(\rho = \sqrt{\Delta x^2 + \Delta y^2 + \Delta z^2}\)):
    \begin{align*}
        I_{1}=f_{x}(x_{0},y_{0},z_{0})\Delta x + \alpha_{1}\Delta x, \quad \alpha_{1}\to 0(\rho\to 0),\\
        I_{2}=f_{y}(x_{0},y_{0},z_{0})\Delta y + \alpha_{2}\Delta y, \quad \alpha_{2}\to 0(\rho\to 0).
    \end{align*}
    For \(I_{3}\), by the definition of the partial derivative \(f_{z}(x,y,z)\) at \((x_{0},y_{0},z_{0})\), we have
    \[
        I_{3}=f_{z}(x_{0},y_{0},z_{0})\Delta z + \alpha_{3}\Delta z, \quad \alpha_{3}\to 0(\rho\to 0).
    \]
    Accordingly, 
    \begin{align*}
        \Delta f &= I_{1} + I_{2} + I_{3} \\
        &= \left[ f_{x}(x_{0},y_{0},z_{0})\Delta x + \alpha_{1}\Delta x \right] + \left[ f_{y}(x_{0},y_{0},z_{0})\Delta y + \alpha_{2}\Delta y \right] + \left[ f_{z}(x_{0},y_{0},z_{0})\Delta z + \alpha_{3}\Delta z \right] \\
        &= f_{x}(x_{0},y_{0},z_{0})\Delta x + f_{y}(x_{0},y_{0},z_{0})\Delta y + f_{z}(x_{0},y_{0},z_{0})\Delta z + \left[ \alpha_{1}\Delta x + \alpha_{2}\Delta y + \alpha_{3}\Delta z \right].
    \end{align*}
    Apparently, 
    \[
        \lim_{\rho \to 0} \frac{\alpha_{1}\Delta x + \alpha_{2}\Delta y + \alpha_{3}\Delta z}{\rho} = 0,
    \]
    i.e. \(\alpha_{1}\Delta x + \alpha_{2}\Delta y + \alpha_{3}\Delta z = o(\rho)\). 
    Therefore, \(f(x,y,z)\) is differentiable at \((x_{0},y_{0},z_{0})\), which completes the proof.
\end{proof}

\begin{note}{(At some point)}
    \begin{enumerate}
        % 偏导数存在, 未必连续
        \item The existence of partial derivatives at a point does not necessarily imply their continuity at that point.
            A classic counterexample is:
            \[
            f(x,y) = \begin{cases}
            \frac{xy}{x^2 + y^2}, & (x,y) \neq (0,0), \\
            0, & (x,y) = (0,0).
            \end{cases}
            \]
            Here, \(f_x(0,0) = 0\) and \(f_y(0,0) = 0\), but \(f_x(x,y)\) and \(f_y(x,y)\) are not continuous at \((0,0)\).
        % 邻域偏导数有界, 必定连续
        \item (\underline{partial derivatives bounded \(\Rightarrow\) continuous}) If the partial derivatives exist and are bounded in a neighborhood of a point, 
            then they are continuous at that point.
        % 一点连续且所有方向导数存在, 不一定可微
        \item Even if all directional derivatives exist at a point and the function is continuous at that point, 
            it does not necessarily imply that the function is differentiable at that point.
            A classic counterexample is:
            \[
            f(x,y) = \begin{cases}
            \frac{x^3}{x^2 + y^2}, & (x,y) \neq (0,0), \\
            0, & (x,y) = (0,0).
            \end{cases}
            \]
            Here, all directional derivatives of \(f\) exist at \((0,0)\), 
            and \(f\) is continuous at \((0,0)\), but \(f\) is not differentiable at \((0,0)\).
            Another counterexample is:
            \[
            f(x,y) = \sqrt{|xy|},
            \]
            which is continuous at \((0,0)\) and has all directional derivatives equal to \(0\) at \((0,0)\),
            but is not differentiable at \((0,0)\).
    \end{enumerate}
\end{note}

\begin{proof}
    Take the function of two variables as an example.
    Assume the bivariate function \(f: \mathbb{R}^2 \to \mathbb{R}\) meets:
    \(\frac{\partial f}{\partial x}\) and \(\frac{\partial f}{\partial y}\) exist and are bounded
    in some neighborhood of \((x_{0},y_{0})\).

    Consider the total increment of \(f\) at the point \((x_{0},y_{0})\):
    \begin{align*}
        \Delta f 
        &= \left[ f(x_0 + \Delta x, y_0 + \Delta y) - f(x_0, y_0 + \Delta y) \right]\\
        &+ \left[ f(x_0, y_0 + \Delta y) - f(x_0, y_0) \right].
    \end{align*}
    By the Lagrange's Mean Value Theorem of unary functions, there exist \(\theta_{1}, \theta_{2} \in (0,1)\) such that
    \begin{gather*}
        \Delta f = f_{x}(x_{0}+\theta_{1}\Delta x,y_{0}+\Delta y)\Delta x + f_{y}(x_{0},y_{0}+\theta_{2}\Delta y)\Delta y.
    \end{gather*}
    Since \(\frac{\partial f}{\partial x}\) and \(\frac{\partial f}{\partial y}\) are bounded 
    in some neighborhood of \((x_{0},y_{0})\),
    \[
    \lim_{(\Delta x, \Delta y) \to (0,0)} \Delta f = 0,
    \]
    i.e. \(f(x,y)\) is continuous at \((x_{0},y_{0})\), which completes the proof.
\end{proof}

\section{Higher-Order Partial Derivatives and Differentiability}
\begin{leftbarTitle}{Higher-Order Partial Derivatives}\end{leftbarTitle}
If the first-order partial derivative of \(f\), \(\frac{\partial f}{\partial x_i}\), 
itself possesses partial derivatives, then the second-order partial derivative of \(f\) is defined, 
and is denoted as follows(the first is also called the mixed partial derivative):
\[
f_{x_i x_j} = \frac{\partial^2 f}{\partial x_i \partial x_j} = \frac{\partial}{\partial x_j} \left( \frac{\partial f}{\partial x_i} \right), 
\quad f_{x_i x_i} = \frac{\partial^2 f}{\partial x_i^2} = \frac{\partial}{\partial x_i} \left( \frac{\partial f}{\partial x_i} \right), 
\quad i, j = 1, 2, \dots, n.
\]

Similarly, higher-order partial derivatives of order \(3,4,\cdots m,\cdots\) can be defined.

The following theorem provides the conditions under which mixed partial derivatives are equal.

\begin{theorem}{Conditions for Equality of Mixed Partial Derivatives}
    \begin{enumerate}\label{thm:condition_equality}
        \item Let $U \subset \mathbb{R}^2$ be an open set, and $f: U \to \mathbb{R}$ be a function of two variables. 
            If the partial derivatives $f_{x}, f_{y}$ and $f_{xy}$ exist in some neighborhood of $(x_0, y_0) \in U$, 
            and \(f_{xy}\) is continuous at $(x_0, y_0)$, then \(f_{yx}\) also exists at \((x_0, y_0)\), and
            \[
            f_{yx}(x_0, y_0) = f_{xy}(x_0, y_0).
            \]
        \item Let \(U \subset \mathbb{R}^n\) be an open set, and \(f: U \to \mathbb{R}\) be a function of \(n\) variables. 
            If the partial derivatives $f_{x_i}, f_{x_j}$ and \(f_{x_{i}x_{j}}\) exist in some neighborhood of 
            \(\mathbf{x}^{0} = (x_1^0, x_2^0, \ldots, x_n^0) \in U\), 
            and \(f_{x_{i}x_{j}}\) is continuous at \(\mathbf{x}^{0}\), then \(f_{x_j x_i}\) exist at \(\mathbf{x}^{0}\), and
            \[
            f_{x_j x_i}(\mathbf{x}^{0}) = f_{x_i x_j}(\mathbf{x}^{0}).
            \]
    \end{enumerate}
\end{theorem}


\begin{proof}

\end{proof}

\begin{leftbarTitle}{Higher-Order Differentiability}\end{leftbarTitle}
Suppose \(z=f(x,y)\) has continuous partial derivatives in the domain \(U\subset \mathbb{R}^2\). 
Then \(z\) is differentiable, and
\[
    \mathrm{d}z = \frac{\partial z}{\partial x} \mathrm{d}x + \frac{\partial z}{\partial y} \mathrm{d}y.
\]
If \(z\) also has continuous second-order partial derivatives, 
then \(\frac{\partial z}{\partial x}\) and \(\frac{\partial z}{\partial y}\) are also differentiable, 
and thus \(\mathrm{d}z\) is differentiable. 
We call the differential of \(\mathrm{d}z\) the second-order differential of \(z\), denoted as
\[
    \mathrm{d}^2z = \mathrm{d}(\mathrm{d}z).
\]
In general, based on the \(k\)-th order differential \(\mathrm{d}^kz\) of \(z\), 
its \((k+1)\)-th order differential (if it exists) is defined as
\[
    \mathrm{d}^{k+1}z = \mathrm{d}(\mathrm{d}^kz), \quad k = 1, 2, \cdots .
\]
Due to the fact that for the independent variables \( x \) and \( y \), we always have
\[
    \mathrm{d}^2 x = \mathrm{d}(\mathrm{d}x) = 0, \qquad \mathrm{d}^2 y = \mathrm{d}(\mathrm{d}y) = 0,
\]
the second-order differential of \( z = f(x, y) \) is given by
\begin{align*}
    \mathrm{d}^2 z &= \mathrm{d}(\mathrm{d}z) 
        = \mathrm{d}\left( \frac{\partial z}{\partial x} \mathrm{d}x + \frac{\partial z}{\partial y} \mathrm{d}y \right) \\
    &= \mathrm{d}\left( \frac{\partial z}{\partial x} \right) \mathrm{d}x + \frac{\partial z}{\partial x} \mathrm{d}^2 x 
        + \mathrm{d}\left( \frac{\partial z}{\partial y} \right) \mathrm{d}y + \frac{\partial z}{\partial y} \mathrm{d}^2 y\\
    &= \left( \frac{\partial^2 z}{\partial x^2} \mathrm{d}x + \frac{\partial^2 z}{\partial x \partial y} \mathrm{d}y \right) \mathrm{d}x
        + \left( \frac{\partial^2 z}{\partial y \partial x} \mathrm{d}x + \frac{\partial^2 z}{\partial y^2} \mathrm{d}y \right) \mathrm{d}y\\
    &= \frac{\partial^2 z}{\partial x^2} (\mathrm{d}x)^2 + 2 \frac{\partial^2 z}{\partial x \partial y} \mathrm{d}x \mathrm{d}y + \frac{\partial^2 z}{\partial y^2} (\mathrm{d}y)^2,
\end{align*}
where \( (\mathrm{d}x)^2 \) and \( (\mathrm{d}y)^2 \) denote \( \mathrm{d}^2 x \) and \( \mathrm{d}^2 y \) respectively.
If we treat \( \frac{\partial}{\partial x} \), \( \frac{\partial}{\partial y} \) as operators for partial differentiation 
and define
\[
    \left( \frac{\partial}{\partial x} \right)^2 = \frac{\partial^2}{\partial x^2}, \quad
    \left( \frac{\partial}{\partial y} \right)^2 = \frac{\partial^2}{\partial y^2}, \quad
    \left( \frac{\partial}{\partial x} \frac{\partial}{\partial y} \right) = \frac{\partial^2}{\partial x \partial y},
\]
then the formulas for the first and second differentials can be written as
\[
    \mathrm{d}z = \left( \mathrm{d}x \frac{\partial}{\partial x} + \mathrm{d}y \frac{\partial}{\partial y} \right) z,
\]
\[
    \mathrm{d}^2 z = \left( \mathrm{d}x \frac{\partial}{\partial x} + \mathrm{d}y \frac{\partial}{\partial y} \right)^2 z.
\]
Similarly, we define
\[
    \left( \frac{\partial}{\partial x} \right)^p
    \left( \frac{\partial}{\partial y} \right)^q
    = \frac{\partial^{p+q}}{\partial x^p \partial y^q}
    = \frac{\partial^q}{\partial y^q}
    \left( \frac{\partial}{\partial x} \right)^p,
    \quad (p, q = 1, 2, \dots)
\]
It is easy to use mathematical induction to prove the formula for higher-order differentials:
\[
    \mathrm{d}^k z = \left( \mathrm{d}x \frac{\partial}{\partial x} + \mathrm{d}y \frac{\partial}{\partial y} \right)^k z, 
    \quad k = 1, 2, \cdots.
\]
For an \( n \)-variable function \( u = f(x_1, x_2, \dots, x_n) \), higher-order differentials can be similarly defined, and the following holds:
\[
    \mathrm{d}^k u 
    = \left( \mathrm{d}x_1 \frac{\partial}{\partial x_1} + \mathrm{d}x_2 \frac{\partial}{\partial x_2} 
    + \cdots + \mathrm{d}x_n \frac{\partial}{\partial x_n} \right)^k u, \quad k = 1, 2, \dots.
\]
\section{Differential of Vector-Valued Functions}
Consider an $n$-dimensional vector-valued function defined on a domain $U \subset \mathbb{R}^n$:
\begin{gather*}
    f: U \to \mathbb{R}^m, \\ 
    \mathbf{x} \mapsto \mathbf{y} = f(\mathbf{x})
\end{gather*}
Expressed in coordinate vector form:
\[
    \mathbf{y} =
    \begin{pmatrix}
    y_1 \\ y_2 \\ \vdots \\ y_m
    \end{pmatrix}
    =
    \begin{pmatrix}
    f_1(x_1, x_2, \dots, x_n) \\
    f_2(x_1, x_2, \dots, x_n) \\
    \vdots \\
    f_m(x_1, x_2, \dots, x_n)
    \end{pmatrix},
    \qquad \mathbf{x} = \begin{pmatrix} 
    x_1 \\ x_2 \\ \vdots \\ x_n 
    \end{pmatrix}  \in U
\]

\begin{enumerate}
    \item  If each component function $f_i(x_1, x_2, \dots, x_n)$ ($i=1,2,\dots,m$) is partially differentiable at $\mathbf{x}^0$, 
        then the vector-valued function $\mathbf{f}$ is differentiable at $\mathbf{x}^0$, and we define the matrix
        \[
            \left( \frac{\partial f}{\partial x_j} (\mathbf{x}^0) \right)_{m \times n}
            =
            \begin{pmatrix}
            \frac{\partial f_1}{\partial x_1}(\mathbf{x}^0) & \frac{\partial f_1}{\partial x_2}(\mathbf{x}^0) & \cdots & \frac{\partial f_1}{\partial x_n}(\mathbf{x}^0) \\
            \frac{\partial f_2}{\partial x_1}(\mathbf{x}^0) & \frac{\partial f_2}{\partial x_2}(\mathbf{x}^0) & \cdots & \frac{\partial f_2}{\partial x_n}(\mathbf{x}^0) \\
            \vdots & \vdots & \ddots & \vdots \\
            \frac{\partial f_m}{\partial x_1}(\mathbf{x}^0) & \frac{\partial f_m}{\partial x_2}(\mathbf{x}^0) & \cdots & \frac{\partial f_m}{\partial x_n}(\mathbf{x}^0)
            \end{pmatrix}
        \]

        This matrix is called the Jacobian matrix of $\mathbf{f}$ at $\mathbf{x}^0$, 
        denoted by $f'(\mathbf{x}^0)$ (or $\mathrm{D}f(\mathbf{x}^0)$, $J_f(\mathbf{x}^0)$).

        For the special case $m=1$, i.e., $n$-variable scalar function $z=f(x_1,x_2,\dots,x_n)$, 
        the derivative at $\mathbf{x}^0$ is
        \[
            f'(\mathbf{x}^0) = 
            \left( 
                \frac{\partial f}{\partial x_1}(\mathbf{x}^0), \frac{\partial f}{\partial x_2}(\mathbf{x}^0), 
                \cdots, \frac{\partial f}{\partial x_n}(\mathbf{x}^0) 
            \right)
        \]
        If the vector-valued function $\mathbf{f}$ is differentiable at every point in $U$, 
        then $\mathbf{f}$ is said to be differentiable on $U$, and the corresponding relationship is
        \[
        \mathbf{x} \in U \mapsto f'(\mathbf{x}) = J_f(\mathbf{x})
        \]
        where $f'(\mathbf{x})$ (or $\mathrm{D}f(\mathbf{x})$, $J_f(\mathbf{x})$) 
        denotes the derivative of $\mathbf{f}$ at $\mathbf{x}$ in $U$.
    \item  If every component function $f_i(x_1, x_2, \dots, x_n)$ $(i=1,2,\dots,m)$ of $\mathbf{f}$ 
        has continuous partial derivatives at $\mathbf{x}^0$, 
        then every element of the Jacobian matrix of $\mathbf{f}$ is continuous at $\mathbf{x}^0$. 
        In this case, $\mathbf{f}$ is said to have a continuous derivative at $\mathbf{x}^0$ as a vector-valued function.
        
        If the derivative of a vector-valued function $\mathbf{f}$ is continuous at every point in $U$, 
        then $\mathbf{f}$ is said to have a continuous derivative on $U$.
    \item  If there exists an $m \times n$ matrix $A$ that depends only on $\mathbf{x}^0$ (and not on $\Delta \mathbf{x}$), 
        such that in the neighborhood of $\mathbf{x}^0$,
        \[
        \Delta \mathbf{y} = f(\mathbf{x}^0 + \Delta \mathbf{x}) - f(\mathbf{x}^0) = A \Delta \mathbf{x} + o(\|\Delta \mathbf{x}\|)
        \]
        (where $\Delta \mathbf{x} = (\Delta x_1, \Delta x_2, \dots, \Delta x_n)^T$ is a column vector and 
        $\|\Delta \mathbf{x}\|$ denotes its norm), 
        then $f$ is said to be differentiable at $\mathbf{x}^0$ as a vector-valued function, 
        and $A\Delta \mathbf{x}$ is called the differential of $f$ at $\mathbf{x}^0$, denoted as $\mathrm{d}\mathbf{y}$. 
        If we denote $\Delta \mathbf{x}$ by 
        $\mathrm{d}\mathbf{x}$ ($\mathrm{d}\mathbf{x} = (\mathrm{d}x_1, \mathrm{d}x_2, \dots, \mathrm{d}x_n)^T$), then
        \[
            \mathrm{d}\mathbf{y} = A\,\mathrm{d}\mathbf{x}.
        \]

        If the vector-valued function $\mathbf{f}$ is differentiable at every point in $U$, 
        then $\mathbf{f}$ is said to be differentiable on $U$.
\end{enumerate}

Combining the above three points, we obtain the following unified statement:

A vector-valued function \(\mathbf{f}\) is continuous, differentiable, 
and has derivatives if and only if each of its coordinate component functions 
\(f_i(x_1, x_2, \dots, x_n)\) (\(i = 1, 2, \dots, m\)) is continuous, differentiable, and has derivatives.


\section{Derivatives of Composite Mappings (Chain Rule)}
Let \( U \subset \mathbb{R}^l \) and \( V \subset \mathbb{R}^n \) be open sets, and let 
\[
\mathbf{g}: U \to V \quad \text{and} \quad \mathbf{f}: V \to \mathbb{R}^m
\]
be mappings. If \( \mathbf{g} \) is derivative at \( \mathbf{u}^{0} \in U \) 
and \( \mathbf{f} \) is differentiable at \( \mathbf{x}^{0} = \mathbf{g}(\mathbf{u}^{0}) \), 
then the composite mapping \( \mathbf{f} \circ \mathbf{g} \) is differentiable at \( \mathbf{u}^{0} \), and:
\[
J(\mathbf{f} \circ \mathbf{g})(\mathbf{u}^{0}) = 
J\mathbf{f}(\mathbf{x}^{0}) J\mathbf{g}(\mathbf{u}^{0}).
\]

\begin{note}
    \begin{enumerate}
        \item  outer differentiable + inner derivative = total derivative
        \item  outer differentiable + inner differentiable = total differentiable
    \end{enumerate}
\end{note}

Specially, define \( z = f(x, y), (x,y)\subset D_{f}\subset \mathbb{R}^{2} \), 
\(\mathbf{g}:D_{g}\to \mathbb{R}^{2}, (u,v)\mapsto (x(u,v), y(u,v))\), 
and \(g(D_{g})\subset D_{f}\), 
then we have composite function
\[
z = f \circ \mathbf{g} = f\left[x(u,v), y(u,v)\right],\quad (u,v)\in D_{g}.
\]
\[
\mathbb{R}^{2}\xrightarrow{\mathbf{g}:\text{derivative}}\mathbb{R}^{2}\xrightarrow{f:\text{differentiable}}\mathbb{R}
\]
If \(\mathbf{g}\) is derivative at \((u_{0}, v_{0})\in D_{g}\), 
and \(f\) is differentiable at \((x_{0}, y_{0}) = \mathbf{g}(u_{0}, v_{0})\), 
then \(z = f \circ \mathbf{g}\) is differentiable at \((u_{0}, v_{0})\), and at the point,
\[
    \begin{bmatrix} 
        \frac{\partial z}{\partial u}   & \frac{\partial z}{\partial v} 
    \end{bmatrix} 
    =
    \begin{bmatrix}
        \frac{\partial z}{\partial x} & \frac{\partial z}{\partial y}
    \end{bmatrix}
    \begin{bmatrix} 
        \frac{\partial x}{\partial u} & \frac{\partial x}{\partial v} \\
        \frac{\partial y}{\partial u} & \frac{\partial y}{\partial v}
    \end{bmatrix} 
\]

\begin{proof}
    
\end{proof}

\begin{leftbarTitle}{Applications}\end{leftbarTitle}
As an important application of the chain rule, we have the following theorem on the \underline{differentiation of} 
\underline{determinants}.
% 行列式求导
\begin{theorem}
    For 
    \[
    \Delta(t) =
    \begin{vmatrix} 
    a_{11}(t) & a_{12}(t) & \cdots & a_{1n}(t) \\
    a_{21}(t) & a_{22}(t) & \cdots & a_{2n}(t) \\
    \vdots & \vdots & \ddots & \vdots \\
    a_{n1}(t) & a_{n2}(t) & \cdots & a_{nn}(t)
    \end{vmatrix},
    \]
    where each element \(a_{ij}(t)\) is differentiable with respect to \(t\),
    then \(\Delta(t)\) is differentiable with respect to \(t\), and
    \[
    \frac{\mathrm{d}\Delta(t)}{\mathrm{d}t} =
    \sum_{j=1}^{n}
    \begin{vmatrix} 
    a_{11}(t) & a_{12}(t) & \cdots & a_{1n}(t) \\
    a_{21}(t) & a_{22}(t) & \cdots & a_{2n}(t) \\
    \vdots & \vdots & \ddots & \vdots \\
    \frac{\mathrm{d}}{\mathrm{d}t}a_{1j}(t) & \frac{\mathrm{d}}{\mathrm{d}t}a_{2j}(t) & \cdots & \frac{\mathrm{d}}{\mathrm{d}t}a_{nj}(t) \\
    \vdots & \vdots & \ddots & \vdots \\
    a_{n1}(t) & a_{n2}(t) & \cdots & a_{nn}(t)
    \end{vmatrix}
    \]
    where in each determinant on the right-hand side,
    the \(j\)-th column is replaced by the derivative of the \(j\)-th column of \(\Delta(t)\).
\end{theorem}

\vspace{0.7cm}
Another important application is \underline{homogeneous functions}.

\begin{proposition}
    The following statements can be generalized for \(n\) variables:
    \begin{enumerate}
        \item Let \(f(x, y)\in C^{1}\), then \(f\) is a homogeneous function of degree \(m\) if and only if
            \[
            x \frac{\partial f}{\partial x} + y \frac{\partial f}{\partial y} = m f(x, y).
            \]
        \item Let \(f(x, y)\in C^{2}\) be a homogeneous function of degree \(m\), then
            \[
            \left( x \frac{\partial}{\partial x} + y \frac{\partial}{\partial y} \right)^2 f(x, y) = m(m-1) f(x, y),
            \]
            where 
            \[
            \left( x \frac{\partial}{\partial x} + y \frac{\partial}{\partial y} \right) = 
            x^2 \frac{\partial^2}{\partial x^2} + 2xy \frac{\partial^2}{\partial x \partial y} + y^2 \frac{\partial^2}{\partial y^2},
            \]
            which is just a formal notation, not an operator multiplication.
        \item Let \(f(x, y)\in C^{2}\) be a homogeneous function of degree \(m\), then \(f_{x}(x, y), f_{y}(x, y)\) 
            are homogeneous functions of degree \(m-1\).
        \item Let \(f(x, y)\in C(\mathbb{R}^{2}\setminus \{(0,0)\})\) be a homogeneous function of degree \(m\), then 
            \[
            \left| f(x,y) \right| \leqslant C \rho^{m}, \quad \rho = \sqrt{x^2 + y^2},
            \]
            where \(C = \max_{\rho=1} |f(x,y)|\).
    \end{enumerate}
\end{proposition}

\begin{example}
    Let \(f(x, y)\) be a differential function on \(\mathbb{R}^2\),
    and satisfy the equation
    \[
    x \frac{\mathrm{d}f}{\mathrm{d}x} + y \frac{\mathrm{d}f}{\mathrm{d}y} = 0,
    \]
    prove that \(f(x, y)\) is always constant.
\end{example}

\section{Mean Value Theorem and Taylor's Formula}
\begin{leftbarTitle}{Mean Value Theorem}\end{leftbarTitle}
\begin{definition}{Convex Region}
    Let \(D \subseteq \mathbb{R}^n\) be a region. 
    If every line segment connecting any two points \(\mathbf{x}_0, \mathbf{x}_1 \in D\) 
    (denoted by \(\overline{\mathbf{x}_0 \mathbf{x}_1}\))
    is entirely contained in \(D\), i.e., for any \(\lambda \in [0, 1]\), we have  
    \[
    \mathbf{x}_0 + \lambda (\mathbf{x}_1 - \mathbf{x}_0) \in D,
    \]
    then \(D\) is called a convex region.
\end{definition}

\begin{theorem}{Lagrange's Mean Value Theorem}\label{thm:Multi_Lagrange}
    Let \(f\) be \underline{differentiable} on \underline{a convex region} \(D \subseteq \mathbb{R}^n\). 
    For any two points \(\mathbf{a}, \mathbf{b} \in D\), 
    there exists a point \(\mathbf{\xi}\in \overline{\mathbf{a} \mathbf{b}}\)
    such that:  
    \[
    f(\mathbf{b}) - f(\mathbf{a}) = Jf(\mathbf{\xi})(\mathbf{b} - \mathbf{a}).
    \]
\end{theorem}

\begin{proof}
    
\end{proof}

For mappings, Lagrange's mean value theorem can not be generalized directly, 
we need introduce inner product:
\begin{theorem}{Lagrange's Mean Value Theorem for Mappings}
    Let \(\mathbf{f}: D \to \mathbb{R}^m\) be \underline{differentiable} on \underline{an open set} \(D \subseteq \mathbb{R}^n\). 
    For any two points \(\mathbf{a}, \mathbf{b} \in D\), 
    there exists a point \(\mathbf{\xi}\in \overline{\mathbf{a} \mathbf{b}}\)
    such that:  
    \[
    \mathbf{a} \cdot [\mathbf{f}(\mathbf{b}) - \mathbf{f}(\mathbf{a})] = 
    \mathbf{a} \cdot [J\mathbf{f}(\mathbf{\xi})(\mathbf{b} - \mathbf{a})],
    \quad \forall \mathbf{a} \in \mathbb{R}^m.
    \]
\end{theorem}
\begin{note}
    If it does not contain the inner product, 
    then it is not necessarily true.
    For example, let
    \[
    \mathbf{f}(t) = (\cos t, \sin t), \quad t \in [0, 2\pi],
    \]
    then 
    \[
    J \mathbf{f}(t) = (-\sin t, \cos t),
    \]
    note that \(\mathbf{f}(2\pi) = \mathbf{f}(0)\),
    then there does not exist \(\theta \in (0, 1)\) such that
    \[
    \mathbf{f}(2\pi) - \mathbf{f}(0) = J\mathbf{f}(\theta \cdot 2\pi)(2\pi - 0).
    \]
    In fact, 
    \[
    J \mathbf{f}(t) \not\equiv 0, \quad \forall t \in [0, 2\pi].
    \]
\end{note}

And we have global estimation for the difference of mappings:
\begin{theorem}{Quasi-Differential Mean Value Theorem for Mappings}
    Let \(\mathbf{f}: D \to \mathbb{R}^m\) be \underline{differentiable} on \underline{a convex region} \(D \subseteq \mathbb{R}^n\). 
    For any two points \(\mathbf{a}, \mathbf{b} \in D\), 
    there exists a point \(\mathbf{\xi}\in \overline{\mathbf{a} \mathbf{b}}\)
    such that:  
    \[
    \|\mathbf{f}(\mathbf{b}) - \mathbf{f}(\mathbf{a})\| 
    \leq \|J\mathbf{f}(\mathbf{\xi})\| \cdot \|\mathbf{b} - \mathbf{a}\|.
    \]
\end{theorem}

\begin{corollary}
    Let \(D\) be a region in \(\mathbb{R}^n\). If for any \(\mathbf{x} \in D\), we have  
    \[
    J\mathbf{f}(\mathbf{x}) = 0,
    \]
    then \(\mathbf{f}\) is constant mapping on \(D\).
\end{corollary}

\begin{proof}
    
\end{proof}

\begin{leftbarTitle}{Taylor's Formula}\end{leftbarTitle}
\begin{theorem}{Taylor's Formula}
    \begin{description}
        \item[Lagrange's Remainder]  Let \(D \subseteq \mathbb{R}^n\) be a convex region, 
            and let \(f: D \to \mathbb{R}\) have \(m+1\) continuous partial derivatives. 
            For \(\mathbf{x}^0 = (x_1^0, x_2^0, \dots, x_n^0) \in D\) and \(\mathbf{x} = (x_1, x_2, \dots, x_n) \in D\), 
            there exists \(\mathbf{\xi} \in \overline{\mathbf{x}^0 \mathbf{x}}\) such that:  
            \[
            f(\mathbf{x}) = f(\mathbf{x}^0) 
            + \sum_{k=1}^m \frac{1}{k!} \left( \sum_{i=1}^n (x_i - x_i^0) \frac{\partial}{\partial x_i} \right)^k f(\mathbf{x}^0) 
            + \frac{1}{(m+1)!} \left( \sum_{i=1}^n (x_i - x_i^0) \frac{\partial}{\partial x_i} \right)^{m+1} f(\mathbf{\xi}).
            \]
        \item[Peano's Remainder] Let \(D \subseteq \mathbb{R}^n\) be a convex region, 
            and let \(f: D \to \mathbb{R}\) have \(m\) continuous partial derivatives. 
            Then:
            \[
            f(\mathbf{x}) = f(\mathbf{x}^0) 
            + \sum_{k=1}^m \frac{1}{k!} \sum_{i_1, i_2, \dots, i_k=1}^n 
            \frac{\partial^k f}{\partial x_{i_1} \partial x_{i_2} \dots \partial x_{i_k}}(\mathbf{x}^0) 
            \prod_{j=1}^{k} (x_{i_j} - x_{i_j}^0) 
            + R_m(\mathbf{x} - \mathbf{x}^0),
            \]
            where \(R_m(\mathbf{x} - \mathbf{x}^0) = O(\|\mathbf{x} - \mathbf{x}^0\|^{m+1})\) or \(o(\|\mathbf{x} - \mathbf{x}^0\|^{m})\), as \(\|\mathbf{x} - \mathbf{x}^0\| \to 0\).

    \end{description}
\end{theorem}

In applications, particularly important is the expression of the first three terms in Taylor's formula, which is given as
(let \(x_1 - x_1^0\) be denoted by \(\Delta x_1\), and similarly for other variables;
\(\Delta \mathbf{x} = (\Delta x_1, \Delta x_2, \dots, \Delta x_n)\)):
\[
    f(\mathbf{x}) = f(\mathbf{x}^0) + Jf(\mathbf{x}^0)(\Delta\mathbf{x})
    + \frac{1}{2!}(\Delta\mathbf{x})Hf(\mathbf{x}^0)(\Delta \mathbf{x})^{\mathrm{T}}+ \cdots,
\]
where the matrix
\[
Hf(\mathbf{x}^0) =
\begin{bmatrix}
\frac{\partial^2 f}{\partial x_1^2} & \frac{\partial^2 f}{\partial x_1 \partial x_2} 
    & \cdots & \frac{\partial^2 f}{\partial x_1 \partial x_n} \\
\frac{\partial^2 f}{\partial x_2 \partial x_1} & \frac{\partial^2 f}{\partial x_2^2} 
    & \cdots & \frac{\partial^2 f}{\partial x_2 \partial x_n} \\
\vdots & \vdots & \ddots & \vdots \\
\frac{\partial^2 f}{\partial x_n \partial x_1} & \frac{\partial^2 f}{\partial x_n \partial x_2} 
    & \cdots & \frac{\partial^2 f}{\partial x_n^2}
\end{bmatrix}_{\mathbf{x}^0}
\]
is called the \textbf{Hessian matrix} of the function \(f\).

\section{Implicit Function Theorem}
\begin{leftbarTitle}{Implicit Mapping}\end{leftbarTitle}
\begin{theorem}{Implicit Function Theorem}\label{thm:Implicit Function Theorem}
    Let \(U \subset \mathbb{R}^{n+1}\) be an open set, and \(F: U \to \mathbb{R}\) be an \(n+1\)-variable function. If:  
    \begin{enumerate}
        \item \(F \in C^k(U, \mathbb{R})\), where \(1 \leqslant k \leqslant +\infty\);
        \item \(F(\mathbf{x}^0, y^0) = 0\), 
            where \(\mathbf{x}^0 = (x_1^0, x_2^0, \dots, x_n^0) \in \mathbb{R}^n\), \(y^0 \in \mathbb{R}\), 
            and \((\mathbf{x}^0, y^0) \in U\) 
            (i.e., the equation \(F(\mathbf{x}, y) = 0\) has a solution \((\mathbf{x}^0, y^0)\));
        \item \(F'_y(\mathbf{x}^0, y^0) \neq 0\).
    \end{enumerate}

    Then there exists an open interval \(I \times J\) containing \((\mathbf{x}^0, y^0)\) 
    (\(I\) being an open interval in \(\mathbb{R}^n\) containing \(\mathbf{x}^0\), 
    and \(J\) being an open interval in \(\mathbb{R}\) containing \(y^0\)), 
    as shown in Fig.~\ref{fig:ImplicitFunction}, such that:  
    \begin{enumerate}
        \item \(\forall x \in I\), the equation \(F(\mathbf{x}, y) = 0\) has a unique solution \(y = f(\mathbf{x})\), 
            where \(f: I \to J\) is an \(n\)-variable function 
            (called the \textbf{implicit function} \(f\), hidden within the equation \(F(\mathbf{x}, f(\mathbf{x})) = 0\), 
            though not necessarily explicitly expressed);
        \item \(y^0 = f(\mathbf{x}^0)\);
        \item \(f \in C^k(I, \mathbb{R})\);
        \item When \(x \in I\), 
            \(\frac{\partial f}{\partial x_i} = \frac{\partial y}{\partial x_i} = -\frac{F_x(\mathbf{x}, y)}{F_y(\mathbf{x}, y)}\), 
            \(i = 1, 2, \dots, n\), where \(y = f(x)\).
    \end{enumerate}
\end{theorem}  
\begin{figure}[h]
    \centering
    \includegraphics[width=0.5\textwidth]{img/ImplicitFunction.png}
    \caption{Implicit Function}
    \label{fig:ImplicitFunction}
\end{figure}

\begin{proof}
    Only the single-variable implicit function theorem is proved; 
    the multi-variable case can be derived using mathematical induction.
    
    Without loss of generality, assume \(F_y(x^0, y^0) > 0\).

    First, prove the \underline{existence of the implicit function}.  
    From the continuity of \(F_y(x^{0}, y^{0}) > 0\) and \(F_y(x, y)\), 
    it is known that there exist closed rectangle:
    \[
    D^* = \{(x, y) \mid |x - x_0| \leqslant \alpha, |y - y_0| \leqslant \beta\} \subset U,
    \]
    where the following holds:
    \[
    F_y(x, y) > 0.
    \]
    Thus, for fixed \(x_0\), the function \(F(x^{0}, y)\) is strictly monotonically increasing 
    within \([y^{0} - \beta, y^{0} + \beta]\). Furthermore, since:
    \[
    F(x^{0}, y^{0}) = 0,
    \]
    it follows that:
    \[
    F(x^{0}, y^{0} - \beta) < 0, \quad F(x^{0}, y^{0} + \beta) > 0.
    \]
    Due to the continuity of \(F(x, y)\) within \(D^*\), there exists \(\rho > 0\) such that along the line segment:
    \[
    x = x^{0} + \rho, \, y = y^{0} + \beta,
    \]
    we have \(F(x, y) > 0\), and along the line segment:
    \[
    x = x^{0} + \rho, \, y = y^{0} - \beta,
    \]
    we have \(F(x, y) < 0\).
    Therefore, for any point \(\bar{x} \in (x^{0} - \rho, x^{0} + \rho)\), treat \(F(x, y)\) as a single-variable function of \(y\). 
    Within \([y^{0} - \beta, y^{0} + \beta]\), this function is continuous. From the previous discussion, we know:
    \[
    F(\bar{x}, y^{0} - \beta) < 0, \quad F(\bar{x}, y^{0} + \beta) > 0.
    \]
    According to the zero point existence theorem~\ref{thm:Zero Point Existence Theorem}, 
    there must exist a unique \(\bar{y} \in [y^{0} - \beta, y^{0} + \beta]\) 
    such that \(F(\bar{x}, \bar{y}) = 0\). 
    Furthermore, because \(F_y(x, y) > 0\) within \(D^*\), this \(\bar{y}\) is unique.
    Denote the corresponding relationship as \(\bar{y} = f(\bar{x})\), 
    then the function \(y = f(x)\) is defined within \((x^{0} - \rho, x^{0} + \rho)\), 
    satisfying \(F(x, f(x)) = 0\), and clearly:
    \[
    y^{0} = f(x^{0}).
    \]

    Further proving \underline{the continuity of the implicit function} \(y = f(x)\) on \((x^{0} - \rho, x^{0} + \rho)\):  
    Let \(\bar{x} \in (x^{0} - \rho, x^{0} + \rho)\) be any point. 
    For any given \(\varepsilon > 0\) (\(\varepsilon\) being sufficiently small), 
    since \(F(\bar{x}, \bar{y}) = 0\) (\(\bar{y} = f(\bar{x})\)), from the previous discussion we know:  
    \[
    F(\bar{x}, \bar{y} - \varepsilon) < 0, \quad F(\bar{x}, \bar{y} + \varepsilon) > 0.
    \]
    Furthermore, due to the continuity of \(F(x, y)\) on \(D^*\), there exists \(\delta > 0\) such that:  
    \[
    F(x, \bar{y} - \varepsilon) < 0, \quad F(x, \bar{y} + \varepsilon) > 0, \quad \text{when} \quad x \in O(x^{0}, \delta).
    \]
    By reasoning similar to the previous discussion, 
    it can be obtained that when \(x \in O(x^{0}, \delta)\), 
    the corresponding implicit function value must satisfy \(f(x) \in (\bar{y} - \varepsilon, \bar{y} + \varepsilon)\), i.e.,  
    \[
    \left|f(x) - f(x^{0})\right| < \varepsilon.
    \]
    This implies that \(y = f(x)\) is continuous on \((x^{0} - \rho, x^{0} + \rho)\).  

    Finally, prove the \underline{differentiability} of \(y = f(x)\) on \((x^{0} - \rho, x^{0} + \rho)\):  
    Let \(\bar{x} \in (x^{0} - \rho, x^{0} + \rho)\) be any point. 
    Take \(\Delta x\) sufficiently small such that \(\bar{x} = x + \Delta x \in (x^{0} - \rho, x^{0} + \rho)\). 
    Denote \(\bar{y} = f(\bar{x})\) and \(\bar{y} + \Delta y = f(\bar{x})\). Clearly,  
    \[
    F(\bar{x}, \bar{y}) = 0 \quad \text{and} \quad F(\bar{x}, \bar{y} + \Delta y) = 0.
    \]
    Using the multi-variable function's mean value theorem~\ref{thm:Multi_Lagrange}, we obtain:  
    \begin{align*}
        0   &= F(\bar{x}, \bar{y} + \Delta y) - F(\bar{x}, \bar{y}) \\
            &= F_x(\bar{x} + \theta \Delta x, \bar{y} + \theta \Delta y) \Delta x + F_y(\bar{x} + \theta \Delta x, \bar{y} + \theta \Delta y) \Delta y,
    \end{align*}
    where \(0 < \theta < 1\).  
    Note that \(F_y \neq 0\) on \(D^*\), hence:  
    \[
    \frac{\Delta y}{\Delta x} = 
        -\frac{F_x(\bar{x} + \theta \Delta x, \bar{y} + \theta \Delta y)}{F_y(\bar{x} + \theta \Delta x, \bar{y} + \theta \Delta y)}.
    \]
    Let \(\Delta x \to 0\). Considering the continuity of \(F_x\) and \(F_y\), we obtain:  
    \[
    \frac{dy}{dx} \Big|_{x = \bar{x}} = -\frac{F_x(\bar{x}, \bar{y})}{F_y(\bar{x}, \bar{y})}.
    \]
    Thus:  
    \[
    f'(\bar{x}) = -\frac{F_x(\bar{x}, \bar{y})}{F_y(\bar{x}, \bar{y})}.
    \]

    The proof is complete.
\end{proof}

\begin{note}
    From the proof process of the implicit function theorem,
    it can be observed that if only require the continuity of the implicit function \(y = f(x)\), 
    then the theorem can be restated as follows:
    \newline If 
    \begin{enumerate}
        \item \(F\in C(U, \mathbb{R})\); 
        \item \(F(\mathbf{x}^0, y^0) = 0\);
        \item For fixed \(\mathbf{x} = \mathbf{x}^0\), \(F(\mathbf{x}^0, y)\) is strictly monotonic with respect to \(y\).
    \end{enumerate}
    Then the function derived from the implicit function \(F(\mathbf{x}, y) = 0\),
    i.e., \(y = f(\mathbf{x})\), is continuous at \(I\).
\end{note}

\begin{theorem}{Implicit Mapping Theorem}\label{thm:Implicit Mapping Theorem}
    Let \( U \subset \mathbb{R}^{n+m} \) be an open set, and \( \mathbf{F}: U \to \mathbb{R}^m \) be a mapping. If:
    \begin{enumerate}
        \item \( \mathbf{F} \in C^k(U, \mathbb{R}^m) \), \( 1 \leqslant k \leqslant \infty \);
        \item \( \mathbf{F}(\mathbf{x}^0, \mathbf{y}^0) = 0 \), 
            where \( \mathbf{x}^0 = (x_1, x_2, \dots, x_n) \), \( \mathbf{y}^0 = (y_1, y_2, \dots, y_m) \), 
            \( (\mathbf{x}^0, \mathbf{y}^0) \in U \) 
            (implying \( \mathbf{F}(\mathbf{x}, \mathbf{y}) = \mathbf{0} \) 
            has a solution at \( (\mathbf{x}^0, \mathbf{y}^0) \));
        \item The determinant
        \[
        \det
        \begin{pmatrix}
            \frac{\partial F_1}{\partial y_1} & \cdots & \frac{\partial F_1}{\partial y_m} \\
            \vdots & \ddots & \vdots \\
            \frac{\partial F_m}{\partial y_1} & \cdots & \frac{\partial F_m}{\partial y_m}
        \end{pmatrix}_{(\mathbf{x}^0, \mathbf{y}^0)}
        = \det J_{\mathbf{y}} \mathbf{F}(\mathbf{x}^0, \mathbf{y}^0) \neq 0,
        \]
    \end{enumerate}
    then there exists an open neighborhood \( I \times J \subset U \subset \mathbb{R}^{n+m} \) 
    containing \( (\mathbf{x}^0, \mathbf{y}^0) \), such that:
    \begin{enumerate}
        \item For all \( \mathbf{x} \in I \), the system \( \mathbf{F}(\mathbf{x}, \mathbf{y}) = \mathbf{0} \) 
            has a unique solution \( \mathbf{y} = \mathbf{f}(\mathbf{x}) \), 
            where \( \mathbf{f}: I \to J \) is a mapping 
            (called \( \mathbf{f} \) the implicit function hidden in 
            \( \mathbf{F}(\mathbf{x}, \mathbf{f}(\mathbf{x})) = \mathbf{0} \));
        \item \( \mathbf{y}^0 = \mathbf{f}(\mathbf{x}^0) \);
        \item \( \mathbf{f} \in C^k(I, \mathbb{R}^m) \);
        \item For \( x \in I \),
        \[
        J\mathbf{f} = 
            - (J_{\mathbf{y}} \mathbf{F})^{-1} J_{\mathbf{x}} \mathbf{F}
        = -
        \begin{pmatrix}
            \frac{\partial F_1}{\partial y_1} & \cdots & \frac{\partial F_1}{\partial y_m} \\
            \vdots & \ddots & \vdots \\
            \frac{\partial F_m}{\partial y_1} & \cdots & \frac{\partial F_m}{\partial y_m}
        \end{pmatrix}^{-1}
        \begin{pmatrix}
            \frac{\partial F_1}{\partial x_1} & \cdots & \frac{\partial F_1}{\partial x_n} \\
            \vdots & \ddots & \vdots \\
            \frac{\partial F_m}{\partial x_1} & \cdots & \frac{\partial F_m}{\partial x_n}
        \end{pmatrix},
        \]
    \end{enumerate}
    where \( \mathbf{y} = \mathbf{f}(\mathbf{x}) \).
\end{theorem}

\begin{example}
    \[
    \begin{cases}
        x = x(z), \\
        y = y(z),
    \end{cases}
    \]
    is an mapping solved from the implicit function defined by the equations:
    \[
    \begin{cases}
        F(y-z, x+z) = 0, &  \\
        G(\frac{y}{z}, xz) = 0, &  \\
    \end{cases}
    \]
    where \(F, G \in C^1\).
    Find \(\frac{\mathrm{d}x}{\mathrm{d}z}\) and \(\frac{\mathrm{d}y}{\mathrm{d}z}\).
\end{example}
\begin{remark}
    % 用 F_{1} 表示 F 对第一个位置的变量求偏导, 它等价于F(u, v) 中的 F_{u}, 其他类似
    Here, we use \(F_{1}\) to represent the partial derivative of \(F\) with respect to its first variable, 
    which is equivalent to \(F_{u}\) in \(F(u, v)\).
    Other notations follow similarly.
\end{remark}

\begin{solution}

\noindent{\color{violet!80}\textbf{Method 1: Direct Derivative}}
Derivative both sides of the equations with respect to \(z\):
\begin{gather*}
    F_{1}(y' - 1) + F_{2}(x' + 1) = 0, \\
    G_{1}(\frac{y'z - y}{z^2}) + G_{2}(x'z + x) = 0.
\end{gather*}
Solve the above equations to get:
\begin{gather*}
    \frac{\mathrm{d}x}{\mathrm{d}z} = 
    \frac{zG_{1}(F_{1}-F_{2})-F_{1}(yG_{1}-xz^{2}G_{2})}{z(F_{2}G_{1}F_{1}G_{2}z^{2})}, \\
    \frac{\mathrm{d}y}{\mathrm{d}z} =  
    \frac{F_{2}(yG_{1}-xz^{2}G_{2})-G_{2}z^{3}(F_{1}-F_{2})}{z(F_{2}G_{1}-F_{1}G_{2}z^{2})}.
\end{gather*}

\noindent{\color{violet!80}\textbf{Method 2: Implicit Function Theorem}}
By the implicit function theorem, we have:
\begin{align*}
    \begin{pmatrix}
    \frac{\mathrm{d}x}{\mathrm{d}z} \\
    \frac{\mathrm{d}y}{\mathrm{d}z}
\end{pmatrix}
&= - 
\begin{pmatrix}
    \frac{\partial F}{\partial x} & \frac{\partial F}{\partial y} \\
    \frac{\partial G}{\partial x} & \frac{\partial G}{\partial y}
\end{pmatrix}^{-1}
\begin{pmatrix}
    \frac{\partial F}{\partial z} \\
    \frac{\partial G}{\partial z}
\end{pmatrix} \\
&= 
\begin{pmatrix} 
    \frac{zG_{1}(F_{1}-F_{2})-F_{1}(yG_{1}-xz^{2}G_{2})}{z(F_{2}G_{1}-F_{1}G_{2}z^{2})} \\
    \frac{F_{2}(yG_{1}-xz^{2}G_{2})-G_{2}z^{3}(F_{1}-F_{2})}{z(F_{2}G_{1}-F_{1}G_{2}z^{2})}
\end{pmatrix}.
\end{align*}
\end{solution}

\begin{example}
    Let \(u(x, y)\) is a function solved from the implicit function defined by the equation:
    \[
    \begin{cases} u=f(x, y, z, t), \\ g(y, z, t)=0,\\ h(z, t)=0 ,\end{cases}
    \]
    where \(f, g, h \in C^1\) and \(\frac{\partial(g, h)}{\partial(z, t)} \neq 0\).
    Find \(\frac{\partial u}{\partial y}\).
\end{example}
\begin{solution}
    
\noindent{\color{violet!80}\textbf{Method 1.}}{\color{violet!80}}
    Since \(\frac{\partial(g, h)}{\partial(z, t)} \neq 0\),
    and \(g, h \in C^1\), \(g(y, z, t)=0, h(z, t)=0\),
    by the implicit mapping theorem~\ref{thm:Implicit Mapping Theorem},
    we can express \(z\) and \(t\) as functions of \(y\):
    \[
    \begin{cases}
        z = z(y), \\
        t = t(y).
    \end{cases}
    \]
    Derivative both sides with respect to \(y\):
    \begin{gather*}
        g_{y} + g_{z} \frac{\mathrm{d}z}{\mathrm{d}y} + g_{t} \frac{\mathrm{d}t}{\mathrm{d}y} = 0, \\
        h_{z} \frac{\mathrm{d}z}{\mathrm{d}y} + h_{t} \frac{\mathrm{d}t}{\mathrm{d}y} = 0.
    \end{gather*}
    And \(u\) is a function of \(x\) and \(y\): \(u = u(x, y) = f(x, y, z(y), t(y))\).
    Thus:
    \[
    \frac{\partial u}{\partial y} = f_{2} + f_{3} \frac{\mathrm{d}z}{\mathrm{d}y} + f_{4} \frac{\mathrm{d}t}{\mathrm{d}y}.
    \]
    Solve the above equations to get:
    \[
    \frac{\partial u}{\partial y} = 
    f_{y} - g_{y}(f_{z} h_{t} - f_{t} h_{z}) \left( \frac{\partial(g, h)}{\partial(z, t)} \right)^{-1}.
    \]
    
\noindent{\color{violet!80}\textbf{Method 2.}}{\color{violet!80}}
Considering
\[
\begin{cases} F(x, y, z, t, u)=u-f(x,y,z,t)=0, \\ g(y,z,t)=0, \\h(z,t)=0.  \end{cases}
\]
Since \(\frac{\partial(F,g,h)}{\partial(u,z,t)}=\frac{\partial(g,h)}{\partial(z,t)}\neq0\),
by the implicit mapping theorem~\ref{thm:Implicit Mapping Theorem},
we have
\[
\begin{cases} u=u(x,y),  \\ z=z(x,y), \\ t=t(x,y).  \end{cases}
\]
Derivative both sides with respect to \(y\):
\begin{gather*}
    u_{y} -f_{y} - f_{z} z_{y} - f_{t} t_{y} = 0, \\
    g_{y} + g_{z} z_{y} + g_{t} t_{y} = 0, \\
    h_{z} z_{y} + h_{t} t_{y} = 0.
\end{gather*}
Solve the above equations to get the same result.
\end{solution}


\begin{leftbarTitle}{Inverse Mapping}\end{leftbarTitle} % 逆映射定理
\begin{theorem}{Local Inverse Mapping Theorem}\label{thm:Local Inverse Mapping Theorem} % 局部逆映射定理
    Let \( U \subset \mathbb{R}^n \) be an open set, and \( \mathbf{f}: U \to \mathbb{R}^n \) be a mapping. If:
    \begin{enumerate}
        \item \( \mathbf{f} \in C^k(U, \mathbb{R}^n) \), \( 1 \leqslant k \leqslant +\infty \);
        \item At point \( \mathbf{x}^0 \in U \), the Jacobian determinant
        \[
        \det J\mathbf{f}(\mathbf{x}^0) \neq 0.
        \]
    \end{enumerate}
    Then there exist open neighborhoods \( V \subset U \) of \( \mathbf{x}^0 \) 
    and \( W \subset \mathbb{R}^n \) of \( \mathbf{f}(\mathbf{x}^0 = \mathbf{y}^{0}) \), such that:
    \begin{enumerate}
        \item The restriction of \( \mathbf{f} \) to \( V \), denoted as \( \mathbf{f}|_V: V \to W \), is a bijection;
        \item The inverse mapping \( \mathbf{f}^{-1}: W \to V \) exists and belongs to \( C^k(W, \mathbb{R}^n) \);
        \item For any \( \mathbf{y} = \mathbf{f}(\mathbf{x}) \in W \),
        \[
        J\mathbf{f}^{-1}(\mathbf{y}) = [J\mathbf{f}(\mathbf{x})]^{-1},
        \]
        where \( \mathbf{x} = \mathbf{f}^{-1}(\mathbf{y}) \).
    \end{enumerate}
    % 此时, f 为 C^k 微分同胚 
    At this time, \( \mathbf{f} \) is called a \( C^k \) diffeomorphism.
\end{theorem}
% 如果加强条件, 则有全局逆映射定理
If the conditions are strengthened, 
then a global inverse mapping theorem can be established.
\begin{theorem}{Inverse Mapping Theorem}\label{thm:Inverse Mapping Theorem}
    Let \( U \subset \mathbb{R}^n \) be a convex region, and \( \mathbf{f}: U \to \mathbb{R}^n \) be a mapping. If:
    \begin{enumerate}
        \item \( \mathbf{f} \in C^k(U, \mathbb{R}^n) \), \( 1 \leqslant k \leqslant +\infty \);
        \item For any \( \mathbf{x} \in U \), the Jacobian determinant
        \[
        \det J\mathbf{f}(\mathbf{x}) \neq 0.
        \]
    \end{enumerate}
    Then \( \mathbf{f}: U \to \mathbf{f}(U) \) is a bijection, 
    and the inverse mapping \( \mathbf{f}^{-1}: \mathbf{f}(U) \to U \) exists and 
    belongs to \( C^k(\mathbf{f}(U), \mathbb{R}^n) \).
\end{theorem}

\begin{example}
    Here are substitutions:
    \[
    x=t, y=\frac{t}{1+tu}, z=\frac{t}{1+tv}.
    \]
    Transform the following equation to the form of dependent variables \(v\) and independent variables \(t, u\):
    \[
    x^{2} \frac{\partial z}{\partial x} + y^{2} \frac{\partial z}{\partial y} = z^{2}.
    \]
\end{example}


\section{Extremum of Multi-variable Functions}

\begin{leftbarTitle}{Unconditional Extremum}\end{leftbarTitle}



\begin{proposition}{Fermat's Three Villiges Problem} % 费马三村问题
    There are three villages located at points \(A\), \(B\), and \(C\) on a flat plane.
    A supply station needs to be established at point \(P\) on the plane,
    such that the total distance from \(P\) to the three villages \(A\), \(B\), and \(C\) is minimized.
    Such a point \(P\) is called the Fermat point of triangle \(ABC\),
    which can be determined as follows:
    \begin{enumerate}
        \item If any angle of triangle \(ABC\) is greater than or equal to \(120^{\circ}\),
            then the Fermat point is the vertex of that angle.
        \item If all angles of triangle \(ABC\) are less than \(120^{\circ}\),
            then the Fermat point \(P\) is located inside triangle \(ABC\),
            and the angles between the segments \(PA\), \(PB\), and \(PC\) are all equal to \(120^{\circ}\).
    \end{enumerate}
\end{proposition}

\begin{leftbarTitle}{Conditional Extremum}\end{leftbarTitle}
\begin{definition}{Conditional Extremum}
    Let \( f: D \to \mathbb{R} \) be a function with \(n+m\) variables defined on a open set \( D \subseteq \mathbb{R}^{n+m} \), 
    and let \(\mathbf{\Phi}: D \to \mathbb{R}^m\) be a mapping, \( M=\{ \mathbf{x} \in D \mid \mathbf{\Phi}(\mathbf{x}) = 0 \} \).
    If there exists \( \mathbf{x}^0 \in M \) satisfying the constraints such that:  
    \[
    f(\mathbf{x}^0) \leq f(\mathbf{x}) \quad (\text{or } f(\mathbf{x}^0) \geq f(\mathbf{x})),
    \]
    for all \( \mathbf{x} \in M \) that also satisfy the constraints, 
    then \( f \) is said to have a conditional minimum (or maximum) at point \( \mathbf{x}^0 \) under the given constraints.
\end{definition}

\begin{theorem}{Lagrange Multiplier Method}
    Let \( f: D \to \mathbb{R} \) be a function with \( n+m \) variables defined on an open set \( D \subseteq \mathbb{R}^{n+m} \), 
    and let \( \mathbf{\Phi}: D \to \mathbb{R}^m \) be a mapping, 
    \( M = \{ \mathbf{x} \in D \mid \mathbf{\Phi}(\mathbf{x}) = 0 \} \). 
    If:
    \begin{enumerate}
        \item \(f \in C^1(D,\mathbb{R}), \mathbf{\Phi} \in C^1(D,\mathbb{R}^m)\);
        \item \(\operatorname{rank}(J\mathbf{\Phi}(\mathbf{x}^0)) = m\);
        \item \(\mathbf{x}^{0}\) is a conditional extremum point of \(f\) on \(M\);
    \end{enumerate}
    then there exist \(\lambda_1, \lambda_2, \dots, \lambda_m \in \mathbb{R}\), such that:
    \[
    \nabla f(\mathbf{x}^0) + \sum_{i=1}^m \lambda_i \nabla \Phi_i(\mathbf{x}^0) = 0.
    \]
\end{theorem}
 % 多元微分学
\chapter{Multiple Integrals} % 多重积分
\section{Multiple Integrals on Bounded Closed Regions}
\underline{How to define a region with measurable area?} % 怎么定义可求面积的区域?
Generally speaking, there are two approaches to define regions with measurable area:
\begin{enumerate}
    \item Consider the integral over a closed rectangle, 
        and then extend it to a bounded closed region within the rectangle
        with the help of characteristic functions;
    \item Define that a bounded closed region \(D\) is measurable if 
        \(\forall \varepsilon > 0\), there exist two polygonal regions \(\Sigma_1\) and \(\Sigma_2\)
        consisting of finite rectangles, such that \(\Sigma_{1}\subset D \subset \Sigma_{2}\) 
        and the area of \(\Sigma_2 \setminus \Sigma_1\) is less than \(\varepsilon\).
\end{enumerate}

\begin{leftbarTitle}{Definition of Multiple Integral}\end{leftbarTitle}
Here, we introduce the definition of double integrals using the first approach.

Initially, we define the double integral on a closed interval (rectangle).
\begin{definition}{Double Integral on a Closed Interval}
    Let \( I = [a, b] \times [c, d] \) be a closed interval in \( \mathbb{R}^2 \), 
    (i.e., each boundary is parallel to the coordinate axes). Partition \( [a, b] \):
    \[
    T_x: a = x_0 < x_1 < \cdots < x_n = b.
    \]
    Partition \( [c, d] \):
    \[
    T_y: c = y_0 < y_1 < \cdots < y_m = d.
    \]
    Two sets of parallel lines \( x = x_i \, (i = 0, 1, \ldots, n) \) and \( y = y_j \, (j = 0, 1, \ldots, m) \) 
    divide \( I \) into \( n \times m \) subrectangles:
    \[
    [x_{i-1}, x_i] \times [y_{j-1}, y_j], \quad i = 1, \ldots, n, \, j = 1, \ldots, m.
    \]

    The union of these \( k \) subrectangles forms a partition \( T = T_x \times T_y = \{ I_1, I_2, \ldots, I_k \} \). 
    For each \( \xi^i \in I_i \, (i = 1, 2, \ldots, k) \), define the \textbf{Riemann sum} (also called a sum of integrals) as:
    \[
    \sum_{i=1}^k f(\boldsymbol{\xi}^i) v(I_i),
    \]
    where \( v(I_i) \) is the area of the rectangle \( I_i \), i.e., the product of its length and width. Denote:
    \[
    \lambda = \max(\text{diam}(I_1), \text{diam}(I_2), \ldots, \text{diam}(I_k)),
    \]
    where \( \text{diam}(I) \) is the diagonal length of the rectangle \( I \), 
    and \( \lambda \) is called the modulus or width of the partition \( T \). 
    The points 
    \( \boldsymbol{\xi} = (\boldsymbol{\xi}^1, \boldsymbol{\xi}^2, \ldots, \boldsymbol{\xi}^k) 
    \in I_1 \times I_2 \times \cdots \times I_k \) 
    are called sampling points for the Riemann sum.

    If there exists \( J \in \mathbb{R} \), such that \( \forall \varepsilon > 0 \), there exists \( \delta > 0 \), 
    such that when \( \lambda < \delta \), for all \( \boldsymbol{\xi} \in I_1 \times I_2 \times \cdots \times I_k \), we have:
    \[
    \left| \sum_{i=1}^k f(\boldsymbol{\xi}^i) v(I_i) - J \right| < \varepsilon,
    \]
    then \( f \) is said to be Riemann integrable on \( I \), and:
    \[
    J = \lim_{\lambda \to 0} \sum_{i=1}^k f(\boldsymbol{\xi}^i) v(I_i) =: 
    \iint_I f(x, y) \, \mathrm{d}x \mathrm{d}y \quad \text{or} \quad \int_I f \, \mathrm{d}v \quad \text{or} \quad \int_{I} f.
    \]

    The function \( f \) is said to have a double integral on \( I \), or simply \( f \) is integrable on \( I \). 
    Here \( f \) is called the integrand, \( I \) is called the integration region, 
    and \( \mathrm{d}v = \mathrm{d}x \mathrm{d}y \) is called the integration element.
\end{definition}

The defined double integral possesses properties similar to those of single-variable integrals.

On the basis of the above definition, we can extend it to the case of a bounded set.

\begin{definition}{Double Integral on a Bounded Set}
    Let \( \Omega \subset \mathbb{R}^2 \) be a bounded set, and \( f: \Omega \to \mathbb{R} \) a two-dimensional function. 
    Define:
    \[
    f_\Omega(\mathbf{x}) = f_\Omega(x, y) =
    \begin{cases} 
    f(x, y), & \text{if } \mathbf{x} = (x, y) \in \Omega, \\
    0, & \text{if } \mathbf{x} = (x, y) \not\in \Omega,
    \end{cases}
    \]
    and call this the \textbf{zero extension} (or \textbf{characteristic function}) of \( f \). 
    For any closed interval \( I \supset \Omega \), if \( f_\Omega \) is Riemann integrable on \( I \), 
    then \( f \) is said to be \textbf{Riemann integrable} on \( \Omega \) (abbreviated as integrable). 
    The integral of \( f \) on \( \Omega \), denoted as:
    \[
    \iint_\Omega f(x, y) \, \mathrm{d}x \mathrm{d}y = 
    \int_\Omega f \, \mathrm{dV} = \int_{\Omega} f = \int_{\Omega} f_\Omega = 
    \iint_I f_\Omega(x, y) \, \mathrm{d}x \mathrm{d}y,
    \]
    represents the Riemann integral of \( f \) on \( \Omega \).
\end{definition}

In above definition, the integral \( \int_\Omega f \) is independent of the choice of 
the closed interval \( I \) containing \( \Omega \) (this confirms the consistency of the definition).

It is worth noting that all the definitions and properties of double integrals 
can be \underline{extended} to triple integrals and higher-dimensional integrals without excessive inconvenience.

\begin{leftbarTitle}{About the Second Approach}\end{leftbarTitle}

\begin{definition}{Set with Zero Area and Set with Zero Measure (Null Set)}
    Let \( A \subset \mathbb{R}^2 \). If for any \( \varepsilon > 0 \), 
    there exist \underline{finitely many} closed intervals \( I_1, I_2, \dots, I_k \) such that:
    \[
    \bigcup_{i=1}^k I_i \supset A, \quad \text{and} \quad \sum_{i=1}^k v(I_i) < \varepsilon,
    \]
    then \( A \) is called a \textbf{set with zero area}.

    Let \( A \subset \mathbb{R}^2 \). If for any \( \varepsilon > 0 \), 
    there exist at most \underline{countably many} closed intervals \( I_1, I_2, \dots, I_k, \dots \) such that:
    \[
    \bigcup_{i=1}^\infty I_i \supset A, \quad \text{and} \quad \sum_{i=1}^\infty v(I_i) < \varepsilon,
    \]
    then \( A \) is called a \textbf{set with zero measure (null set)}.
\end{definition}

\begin{definition}{Set with Finite Area}
    Let \( \Omega \subset \mathbb{R}^2 \) be a bounded set. 
    If the constant function \( 1 \) is integrable on \( \Omega \), 
    then \( \Omega \) is called a \textbf{set with finite area}, and the area of \( \Omega \) is defined as:
    \[
    v(\Omega) = \int_\Omega 1 = \iint_\Omega \mathrm{d}x \mathrm{d}y = \int_I 1_\Omega.
    \]
\end{definition}

Obviously, \(\Omega\) is a set with zero area if and only if \(\Omega\) has finite area, 
and \(v(\Omega) = \int_\Omega 1 = 0\).

\begin{proposition}
    A bounded closed region \(\Omega \subset \mathbb{R}^2\) is measurable if and only if
    its boundary \(\partial \Omega\) is a set with zero area.
\end{proposition}
In the definition of multiple integrals derived from the second approach,
the key point is the division \(T\) of the bounded closed region \(\Omega\) 
into two polygonal regions \(\Sigma_1\) and \(\Sigma_2\).
With above statements, we can see that the division \(T\) is implemented by
infinitely many curves net with zero area.




\begin{leftbarTitle}{Necessary and Sufficient Conditions for Integrability}\end{leftbarTitle}

\begin{proposition}
    Let non-negative function \(f\in R(D)\),
    then \(\iint_{D}f(x,y)\mathrm{d}x\mathrm{d}y=0\) if and only if
    for any continuous points \((x,y)\in D\), \(f(x,y)=0\).
\end{proposition}

\section{Properties of Multiple Integrals}

\begin{leftbarTitle}{Reduction of Double Integral to Iterated Integral}\end{leftbarTitle}

\begin{theorem}{Reduction of Double Integral to Iterated Integral on a Closed Interval}\label{thm:Reduction of Double Integral to Iterated Integral on a Closed Interval}
    Let \( f \) be integrable on the closed interval \( I = [a, b] \times [c, d] \). 
    
    If \( \forall x \in [a, b] \), the integral \(\phi(x)=\int_c^d f(x,y)\,\mathrm{d}y\) exists,
    then \( \phi \) is integrable on \( [a, b] \), and:
    \[
    \iint_I f = \int_a^b \left( \int_c^d f(x, y) \, \mathrm{d}y \right) \mathrm{d}x
    =: \int_{a}^{b}\mathrm{d}x\int_{c}^{d}f(x,y)\,\mathrm{d}y.
    \]

    Similarly, if \( \forall y \in [c, d] \), the integral \(\psi(y)=\int_a^b f(x,y)\,\mathrm{d}x\) exists,
    then \( \psi \) is integrable on \( [c, d] \), and:
    \[
    \iint_I f = \int_c^d \left( \int_a^b f(x, y) \, \mathrm{d}x \right) \mathrm{d}y
    =: \int_{c}^{d}\mathrm{d}y\int_{a}^{b}f(x,y)\,\mathrm{d}x.
    \]
\end{theorem}
\begin{note}
    That is, if \(f\in C(I)\), then two iterated integrals above all exist, 
    and they are equal to the double integral of \(f\) on \(I\) (they can exchange the order of integration).
\end{note}


On the basis of the above theorem, we can extend it to the case of a bounded region.

\begin{theorem}{Reduction of Double Integral to Iterated Integral on a Bounded Set}
    Let \( \Omega \subset \mathbb{R}^2 \) be a set with infinite area, 
    and \( f: \Omega \to \mathbb{R} \) be bounded and continuous
    ~(\ref{fig:Double Integral on a Bounded Set}). 
    Denote the vertical projection of \( \Omega \) onto the \( x \)-axis as:
    \[
    I = \{ x \in \mathbb{R} \mid \exists y, \text{ s.t. } (x, y) \in \Omega \}.
    \]

    If \( \forall x \in I \), let \( \Omega_x = \{ y \in \mathbb{R} \mid (x, y) \in \Omega \} \) 
    be an interval (possibly reducing to a single point), then:
    \[
    \int_\Omega f = \int_I \mathrm{d}y \int_{\Omega_x} f(x, y) \, \mathrm{d}x.
    \]

    Similarly, denote the vertical projection of \( \Omega \) onto the \( y \)-axis as:
    \[
    J = \{ y \in \mathbb{R} \mid \exists x, \text{ s.t. } (x, y) \in \Omega \}.
    \]

    If \( \forall y \in J \), let \( \Omega_y = \{ x \in \mathbb{R} \mid (x, y) \in \Omega \} \) be an interval (possibly reducing to a single point), then:
    \[
    \int_\Omega f = \int_J \mathrm{d}y \int_{\Omega_y} f(x, y) \, \mathrm{d}x.
    \]
\end{theorem}
\begin{figure}[h]
    \centering
    \includegraphics[width=0.5\textwidth]{img/IntegralImg.png}
    \caption{Double Integral on a Bounded Set}
    \label{fig:Double Integral on a Bounded Set}
\end{figure}

Specially, Let:
\[
\Omega = \{ (x, y) \in \mathbb{R}^2 \mid y_1(x) \leqslant y \leqslant y_2(x), \, a \leqslant x \leqslant b \},
\]
where the functions \( y_1 \) and \( y_2 \) are continuous on \( [a, b] \)~(\ref{fig:Double Integral on a Bounded Set}) 
and the function \( f \) is integrable on \( \Omega \). If \( \forall x \in [a, b] \), the single-variable integral:
\[
\int_{y_1(x)}^{y_2(x)} f(x, y) \, \mathrm{d}y
\]
exists, then:
\[
\int_\Omega f = \int_a^b \mathrm{d}x \int_{y_1(x)}^{y_2(x)} f(x, y) \, \mathrm{d}y.
\]
This area called the \textbf{type X region}, similarly, we can define the \textbf{type Y region}.


According to~\ref{thm:Reduction of Double Integral to Iterated Integral on a Closed Interval},
we can derive the formula of multiplicative property for double integral.

\begin{theorem}{Formula of Multiplicative Property for Double Integral}
    Let \( f \in C([a, b]) \), \( g \in C([c, d]) \). 
    Then the function \( h(x, y) = f(x) g(y) \) is integrable on the closed interval \( I = [a, b] \times [c, d] \), 
    and:
    \[
    \iint_I h(x, y) \, \mathrm{d}x \mathrm{d}y = \left( \int_a^b f(x) \, \mathrm{d}x \right) 
    \left( \int_c^d g(y) \, \mathrm{d}y \right).
    \]
\end{theorem}

\begin{example}
    Let \(p(x)\in R[a,b],p(x)>0,x\in [a,b]\), the monotonicity of \(f(x), g(x)\) is same,
    prove that
    \[
    \int_{a}^{b}p(x)f(x)\mathrm{d}x \int_{a}^{b}p(x)g(x)\mathrm{d}x
    \leqslant  \int_{a}^{b}p(x)\mathrm{d}x \int_{a}^{b}p(x)f(x)g(x)\mathrm{d}x
    \]
\end{example}
\begin{proof}
    Let
    \[
    I = \int_{a}^{b}p(x)\mathrm{d}x \int_{a}^{b}p(x)f(x)g(x)\mathrm{d}x
        - \int_{a}^{b}p(x)f(x)\mathrm{d}x \int_{a}^{b}p(x)g(x)\mathrm{d}x,
    \]
    then
    \[
    I = \int_{a}^{b}\int_{a}^{b}p(x)p(y)g(y)(f(x)-f(y))\mathrm{d}x\mathrm{d}y,
    \]
    similarly,
    \[
    I = \int_{a}^{b}\int_{a}^{b}p(x)p(y)g(x)(f(x)-f(y))\mathrm{d}x\mathrm{d}y.
    \]
    Then 
    \[
    2I = \int_{a}^{b}\int_{a}^{b}p(x)p(y)(g(y)-g(x))(f(x)-f(y))\mathrm{d}x\mathrm{d}y \geqslant  0,
    \]
    which implies
    \[
    I \geqslant  0.
    \]
    The proof is complete.
\end{proof}

%%%%%%%%%%%%%%%%%%%%%%%%%%%%%%%%%%%%%%%%%%%%%%%%%%%%%%%%%%%%%%%%%%%%%%
\section{Calculation of Multiple Integrals} % 多重积分的计算
\begin{leftbarTitle}{Variable Substitution in Multiple Integrals}\end{leftbarTitle} % 多重积分中的变量替换
\begin{theorem}{Variable Substitution in Double Integral}
    Let \( \Omega \subset \mathbb{R}^2 \) be an open set, and let the mapping:
    \[
    \mathbf{F}: \Omega \to \mathbb{R}^2, \quad (u, v) \mapsto \mathbf{F}(u, v) = (x(u, v), y(u, v))
    \]
    satisfy the following conditions:
    \begin{enumerate}
        \item \( \mathbf{F} \in C^1(\Omega, \mathbb{R}^2) \);
        \item \( \frac{\partial (x, y)}{\partial (u, v)} 
            = \det J\mathbf{F}(u, v) = \det J\mathbf{F}(\mathbf{p}) \neq 0, \quad \mathbf{p} = (u, v) \in \Omega \);
        \item \( \mathbf{F} \) is injective.
    \end{enumerate}

    If the set \( \Delta \) is a set with finite area and 
    \( \overline{\Delta} \subset \Omega \), and \( f \) is continuous on \( \mathbf{F}(\Omega) \), 
    then \( \mathbf{F}(\Delta) \) is also a set with finite area, and:
    \[
    \iint_{\mathbf{F}(\Delta)} f = \iint_{\Delta} f \circ \mathbf{F} \left| \det J\mathbf{F} \right|,
    \]
    i.e.,
    \[
    \iint_{F(\Delta)} f(x, y) \, \mathrm{d}x \mathrm{d}y = 
    \iint_{\Delta} f(x(u, v), y(u, v)) \left| \frac{\partial (x, y)}{\partial (u, v)} \right| \, \mathrm{d}u \mathrm{d}v.
    \]
\end{theorem}

For triple and higher-dimensional integrals, the variable substitution theorem is similar to the above theorem.

\vspace{0.7cm}
Some common variable substitutions in multiple integrals are as follows:
\begin{description}
\item[Polar Coordinates]
\[
\begin{cases} 
    x = r \cos \theta, \\ 
    y = r \sin \theta,
\end{cases}
\qquad
\begin{cases} 
    r = \sqrt{x^2 + y^2},\quad r\geqslant 0 \\ 
    \theta = \arctan\left(\frac{y}{x}\right)\quad x\neq 0, \theta\in [0, 2\pi].
\end{cases}
\]
and
\[
\frac{\partial (x,y)}{\partial (r,\theta)} = r.
\]

\item[Cylindrical Coordinate System]
\[
\begin{cases} 
    x = r \cos \theta, \\ 
    y = r \sin \theta, \\
    z = z,
\end{cases}
\qquad
\begin{cases} 
    r = \sqrt{x^2 + y^2},\quad r\geqslant 0 \\ 
    \theta = \arctan\left(\frac{y}{x}\right)\quad x\neq 0, \theta\in [0, 2\pi], \\
    z = z.
\end{cases}
\]
and
\[
\frac{\partial (x,y,z)}{\partial (r,\theta,\varphi)} = r.
\]

\item[Spherical Coordinate System]
\[
\begin{cases} 
    x = r \sin \varphi \cos \theta, \\ 
    y = r \sin \varphi \sin \theta, \\
    z = r \cos \varphi,
\end{cases}
\qquad
\begin{cases} 
    r = \sqrt{x^2 + y^2 + z^2},\quad r\geqslant 0 \\ 
    \varphi = \arccos\left(\frac{z}{r}\right)\quad r\neq 0, \varphi\in [0, \pi], \\
    \theta = \arctan\left(\frac{y}{x}\right)\quad x\neq 0, \theta\in [0, 2\pi].
\end{cases}
\]
and
\[
\frac{\partial (x,y,z)}{\partial (r,\theta,\varphi)} = r^2 \sin \varphi.
\]

\end{description}
\begin{figure}[h]
    \centering
    \includegraphics[width=0.8\textwidth]{img/coordinate.png}
    \caption{Cylindrical and Spherical Coordinate Systems}
\end{figure}
\begin{leftbarTitle}{Calculation of Triple Integrals}\end{leftbarTitle} % 三重积分的计算
\begin{example}\label{eg:Triple Integral of Cone}
    Calculating \(I = \iiint_{ \Omega }z^{2}\mathrm{d}x\mathrm{d}y\mathrm{d}z\),
    where \(\Omega\) is the cone defined by \(z^{2} = \frac{h^{2}}{R^{2}}(x^{2}+y^{2})\) 
    and \(z = h\)~(\ref{fig:Cone}).
    \begin{figure}[h]
        \centering
        \includegraphics[width=0.4\textwidth]{img/cone.png}
        \caption{Cone Example.}
        \label{fig:Cone}
    \end{figure}
\end{example}

\begin{example}\label{eg:Project Method Example}
    Calculating \(I = \iiint_{ \Omega }xy\mathrm{d}x\mathrm{d}y\mathrm{d}z\),
    where \(\Omega\) is the region defined 
    by \(0 \leqslant z \leqslant xy, 0\leqslant y \leqslant 1-x, 0\leqslant x \leqslant 1\)
    ~(\ref{fig:Project Method Example}).
    \begin{figure}[h]
        \centering
        \includegraphics[width=0.4\textwidth]{img/project_method_example.png}
        \caption{Project Method Example.}
        \label{fig:Project Method Example}
    \end{figure}
\end{example}

\vspace{1cm}
With the help of examples above, we can derive \textbf{two methods for calculating triple integrals}.

\begin{description}
\item[First 2 then 1 (Section Method)] 
Fix one variable (e.g., \(z\)), first perform a double integral over the other two variables (e.g., \(x,y\)) 
on the "section region" corresponding to the fixed variable, 
and then perform a definite integral over the fixed variable (\(z\)) within its range of values.

This method is convenient when the area of the section region is easy to calculate, 
or when the integrand is only related to the "later-integrated variable" (e.g., only related to \(z\)).

In the example~\ref{eg:Triple Integral of Cone}, the following steps are taken:
\begin{enumerate}
    \item Determine the range of z: \(z \in [0, h]\). 
    \item Determine the section region \(D_z\): 
        For a fixed \(z\), \(D_z\) is the region on the \(xy\)-plane satisfying \(\frac{h^2}{R^2}(x^2 + y^2) \leqslant z^2\), 
        which is a circle with radius \(\frac{R}{h}z\).
    \item Split the integral: 
        \[
        I = \int_{0}^{h} \left( \iint_{D_z} z^2 \,\mathrm{d}x\mathrm{d}y \right) \,\mathrm{d}z.
        \] 
        Since \(z^2\) is independent of \(x\) and \(y\), it can be factored out:
        \(I = \int_{0}^{h} z^2 \left( \iint_{D_z} \,\mathrm{d}x\mathrm{d}y \right) dz\).
    \item Calculate the double integral (area of the section): 
        \[
        \iint_{D_z} \,\mathrm{d}x\mathrm{d}y = \pi \left( \frac{R}{h}z \right)^2 = \pi \frac{R^2}{h^2} z^{2}.
        \]
    \item Calculate the definite integral: 
        \[
        I = \int_{0}^{h} z^2 \cdot \pi \frac{R^2}{h^2} z^2 \,\mathrm{d}z = \frac{\pi R^2 h^3}{5}.\
        \]
\end{enumerate}

\item[First 1 then 2 (Project Method)]
Fix two variables (e.g., \(x,y\)), first perform a definite integral over the third variable (e.g., \(z\)) 
on the "vertical line segment" corresponding to the fixed variables, 
and then perform a double integral over the fixed two variables ( \(x,y\)) on their "projection region. 

This method is convenient when the projection region of the integral region on 
a certain coordinate plane (e.g., \(xy\)-plane) is easy to determine, 
and the upper and lower limits of a single variable (e.g., \(z\)) can 
be easily expressed by the other two variables.

In the example~\ref{eg:Project Method Example}, the following steps are taken:
\begin{enumerate}
    \item Determine the projection region \(D_{xy}\):\(D_{xy}\) is the region on the \(xy\)-plane 
        bounded by \(x + y \leqslant 1\), \(x \geq 0\), and \(y \geq 0\), 
        which can be expressed as \(0 \leqslant x \leqslant 1\) and \(0 \leqslant y \leqslant 1 - x\).
    \item Determine the range of \(z\): \(z \in [0, xy]\) (since \(z\) is bounded below by \(z = 0\) and above by \(z = xy\)).
    \item Split the integral: 
        \[
        I = \iint_{D_{xy}}\, \left( \int_{0}^{xy} xy \mathrm{d}z \right)  \mathrm{d}x \mathrm{d}y,
        \]
        split the double integral on \(D_{xy}\) as:
        \(I = \int_{0}^{1} \, \mathrm{d}x \int_{0}^{1 - x} \, \mathrm{d}y  \int_{0}^{xy} xy \, \mathrm{d}z\).
        (Since \(xy\) is independent of \(z\), it can be factored out without affecting the integral:
        \(I = \int_{0}^{1} \, \mathrm{d}x \int_{0}^{1 - x} xy \, \mathrm{d}y  \int_{0}^{xy} \, \mathrm{d}z\).)
    \item Calculate the inner integral (with respect to \(z\)):
        \(\int_{0}^{xy} xy \, dz = xy \cdot \int_{0}^{xy} dz 
        = xy \cdot \left. z \right|_{0}^{xy} = xy \cdot xy = x^2 y^2\).
    \item Calculate the middle integral (with respect to \(y\)):
        Substitute the result of the inner integral,
        \[
        \int_{0}^{1 - x} x^2 y^2 \, dy 
        = x^2 \cdot \left. \frac{y^3}{3} \right|_{0}^{1 - x} = \frac{x^2 (1 - x)^3}{3}.
        \]
    \item Calculate the outer integral (with respect to \(x\)):Substitute the result of the middle integral:
        \begin{align*}
            \int_{0}^{1} \frac{x^2 (1 - x)^3}{3} \, dx &= \frac{1}{3} \int_{0}^{1} (x^2 - 3x^3 + 3x^4 - x^5) \, dx \\
            &= \frac{1}{3} \left( \left. \frac{x^3}{3} - \frac{3x^4}{4} + \frac{3x^5}{5} - \frac{x^6}{6} \right|_{0}^{1} \right) \\
            &= \frac{1}{3} \left( \frac{1}{3} - \frac{3}{4} + \frac{3}{5} - \frac{1}{6} \right) \\
            &= \frac{1}{180}.
        \end{align*}
\end{enumerate}
\end{description}
% 选择哪种方法的技巧
Some tips for choosing between the two methods (take the above two examples as reference):

\begin{tabular}{p{0.45\textwidth} p{0.45\textwidth}}
    \textbf{First 2 then 1 (Section Method)} & \textbf{First 1 then 2 (Project Method)} \\
    \toprule
    Section area \(D_{z}\) is easy to calculate & Projection region \(D_{xy}\) is easy to determine \\
    \hline
    Integrand is only related to \(z\) & Upper and lower limits \(z\) can be easily expressed by the other two variables \(x,y\) \\
    \bottomrule
\end{tabular}

\vspace{0.7cm}
\begin{example}
    Find the volume of region bounded by the half \textbf{Viviani's curve}:
    sphere \(x^{2}+y^{2}+z^{2}\leqslant a^{2}\) and cylinder \(x^{2}+y^{2}\leqslant ax\) (\(a>0\)).
\end{example}
\begin{figure}[h]
    \centering
    \includegraphics[width=0.4\textwidth]{img/viviani.png}
\end{figure}




\section{Improper Multiple Integrals}
Improper multiple integrals can be also classified into two types, infinite integrals and defective integrals.

\begin{definition}{Infinite Multiple Integral}
    Let \( D \subset \mathbb{R}^2 \) be an unbounded region,
    whose boundary consists of finite or countably many smooth curves, 
    and \( f: D \to \mathbb{R} \) be a function,
    which is integrable on any measurable bounded closed set \( D' \subset D \). 
    If there exists an increasing sequence of bounded closed regions \( \{ D_k \} \) such that:
    \[
    D_1 \subset D_2 \subset \cdots \subset D_k \subset \cdots, 
    \quad \bigcup_{k=1}^{\infty} D_k = D,
    \]
    which is called an \textbf{exhaustion} of \( D \),
    and for each \( k \), the integral \( I(D_k) = \iint_{D_k} f \) exists, 
    and the limit:
    \[
    I = \lim_{k\to\infty} I(D_k)
    \]
    exists, then \( I \) is called the \textbf{improper multiple integral} of \( f \) on \( D \), denoted as:
    \[
    I = \iint_{D} f = \lim_{k\to\infty} \iint_{D_k} f.
    \]
\end{definition}
\begin{remark}
    There are also other ways to define improper multiple integrals,
    such as using limit definitions based on distance to infinity.
    They are equivalent to the above definition.
\end{remark}

\begin{theorem}
    Improper multiple integral is integrable if and only if it is absolutely integrable.
\end{theorem}

\begin{example}
    Calculate
    \[
    \iint_{\mathbb{R}^2} e^{-(x^2+y^2)} \, \mathrm{d}x \mathrm{d}y,
    \]
    and find the value of Poisson integral
    \[
    \int_{-\infty}^{+\infty} e^{-x^2} \, \mathrm{d}x.
    \]
\end{example}

 % 多元积分学
\chapter{Introduction to Surface Theory} % 曲面论导论
\section{Parameterization of Surface} % 参数化曲面

\begin{definition}{Parameterization of Surface}\label{def:Parameterization of Surface}
    Let \( \Delta \) be an open subset in \( \mathbb{R}^s \), 
    and \( \mathbf{r}: \Delta \to \mathbb{R}^n \) be a mapping, 
    where \( \mathbf{u} = (u_1, u_2, \dots, u_s) \to \mathbf{x}(\mathbf{u}) = 
    (x_1(u_1, u_2, \dots, u_s), x_2(u_1, u_2, \dots, u_s), \dots, x_n(u_1, u_2, \dots, u_s)) \). 
    Then \( M = \mathbf{r}(\Delta) = \{ \mathbf{r}(\mathbf{u}) \mid \mathbf{u} \in \Delta \} \) 
    is called an \( s \)-dimensional \textbf{surface (patch)}, and \( \mathbf{r}(\mathbf{u}) \) 
    is referred to as the parameterization of \( M \). 
    
    When \( \mathbf{r}(\mathbf{u}) \in C^k \) (\( k \geq 0 \)), 
    \( \mathbf{r} \) or \( M \) is called an \( s \)-dimensional \( C^k \) surface. 
    
    If \( \mathbf{r} \in C^k \) (\( k \geq 1 \)), \( \mathbf{r} \) or \( M \) 
    is called an \textbf{\( s \)-dimensional \( C^k \) smooth surface}. 
    
    When
    \[
    \operatorname{rank}(r'_1(\mathbf{u}^0), r'_2(\mathbf{u}^0), \dots, r'_s(\mathbf{u}^0)) = 
    \operatorname{rank}
    \begin{pmatrix}
    \frac{\partial r_1}{\partial u_1} & \cdots & \frac{\partial r_1}{\partial u_s} \\
    \vdots & \ddots & \vdots \\
    \frac{\partial r_n}{\partial u_1} & \cdots & \frac{\partial r_n}{\partial u_s}
    \end{pmatrix}_{\mathbf{u}^0}
    = s,
    \]
    we call \( \mathbf{u}^0 \) or \( \mathbf{r}(\mathbf{u}^0) \) a \textbf{regular point} of the surface \( M \). 
    Otherwise, it is called a singular point. 
    
    Every point that is a regular point of the surface is referred to as an \textbf{\( s \)-dimensional \( C^k \) regular surface}. 
    
    At regular points, \( \{r'_1, \dots, r'_s\} \) are linearly independent.
\end{definition}

When \( s = 1 \), \( t \) represents the parameter, a one-dimensional surface is commonly referred to as a curve. 
Considering a \( C^k \) (\( k \geq 1 \)) curve \( \mathbf{r}(t) \), we have:
\[
\mathbf{r}'(t) = \left( r_{1}'(t), r_{2}'(t), \cdots, r_{n}'(t) \right) .
\]
If \( t \) is a regular point, then 
\( \operatorname{rank}(\mathbf{r}'(t)) = \text{rank}(r'_1(t), r'_2(t), \dots, r'_n(t)) = 1 \); 
this is equivalent to \( \mathbf{r}'(t) \neq 0 \), which means \( r'_1(t), r'_2(t), \dots, r'_n(t) \) are not all zero.

We refer to \( \mathbf{r}'(t) \) as the tangent vector of the curve \( \mathbf{r}(t) \) at point \( t \). 
When \( t \) varies, a tangent vector field along the curve \( \mathbf{r}(t) \) is obtained. 
If \( \mathbf{r}(t) \) is a regular curve, 
\( \frac{\mathbf{r}'(t)}{\|\mathbf{r}'(t)\|} \) is the unit tangent vector field along the curve \( \mathbf{r}(t) \). 
It should be emphasized that \( \mathbf{r}'(t) \) or \( \frac{\mathbf{r}'(t)}{\|\mathbf{r}'(t)\|} \) 
always points outward from point \( t \).


\section{Tangent Space and Normal Space} % 切空间与法空间
% 切空间与法空间
\begin{definition}{Tangent Space and Normal Space}
    \(M\) is an \(s\)-dimensional smooth surface in \(\mathbb{R}^n\) defined above,
    and \(\mathbf{u}^{0}\) is a regular point of \(M\).
    The \textbf{tangent space} of \(M\) at point \(\mathbf{r}(\mathbf{u}^{0})\) is the linear space spanned 
    by \(s\) tangent vectors:
    \[
    T_{\mathbf{u}^{0}}M = \operatorname{span}\{ r'_{1}(\mathbf{u}^{0}), r'_{2}(\mathbf{u}^{0}), \dots, r'_{s}(\mathbf{u}^{0}) \}.
    \]
    Accordingly, the \textbf{normal space} of \(M\) at point \(\mathbf{r}(\mathbf{u}^{0})\) is 
    the orthogonal complement of the tangent space:
    \[
    N_{\mathbf{u}^{0}}M = (T_{\mathbf{u}^{0}}M)^{\perp}.
    \]
\end{definition}
% 给出特殊情况的几个切空间法空间表达式
Some special cases of tangent space and normal space expressions are given below:
\begin{leftbarTitle}{Curve}\end{leftbarTitle}
When \( n = 3, s = 1 \), \( M \) is a curve in three-dimensional space. 
% 三维空间中的曲线: 切线与法平面
\begin{enumerate}
    \item If the curve is parameterized as 
        \[
        \mathbf{r}(t) = (x(t), y(t), z(t)), \quad t \in I \subseteq \mathbb{R}.
        \]
        At the regular point \( \mathbf{r}(t^{0})= (x(t^{0}), y(t^{0}), z(t^{0})) \), the tangent line and normal plane are:
        \[
        T_{t^{0}}M = \operatorname{span}\{ \mathbf{r}'(t^{0}) \}: \frac{x-x(t^{0})}{x'(t^{0})} 
            = \frac{y-y(t^{0})}{y'(t^{0})} = \frac{z-z(t^{0})}{z'(t^{0})},
        \]
        \begin{align*}
            N_{t^{0}}M:\quad &x'(t^{0})(x-x(t^{0})) + y'(t^{0})(y-y(t^{0})) + z'(t^{0})(z-z(t^{0})) = 0 \\
            \Leftrightarrow& \mathbf{r}'(t^{0}) \cdot (\mathbf{r} - \mathbf{r}(t^{0})) = 0.
        \end{align*}
    \item If the curve is described by:
        \[
        \begin{cases}
            F(x, y, z) = 0, \\
            G(x, y, z) = 0,
        \end{cases}
        \]
        and the regular point is \( \mathbf{x}^{0} = (x^{0}, y^{0}, z^{0}) \).
        \newline For the Jacobian matrix:
        \[
            J = 
            \begin{pmatrix}
                F_x(\mathbf{x}^{0}) & F_y(\mathbf{x}^{0}) & F_z(\mathbf{x}^{0}) \\
                G_x(\mathbf{x}^{0}) & G_y(\mathbf{x}^{0}) & G_z(\mathbf{x}^{0})
            \end{pmatrix},
        \]
        since \(\operatorname{rank}J = 2\), without loss of generality, assume:
        \[
        \frac{\partial (F, G)}{\partial (y, z)} =
        \begin{vmatrix}
            F_y(\mathbf{x}^{0}) & F_z(\mathbf{x}^{0}) \\
            G_y(\mathbf{x}^{0}) & G_z(\mathbf{x}^{0})
        \end{vmatrix} \neq 0.
        \]
        By the implicit mapping theorem (\ref{thm:Implicit Mapping Theorem}), we can express:
        \[
        y = f(x), \quad z = g(x), \quad x \in U(x^{0}) \subseteq \mathbb{R}.
        \]
        Then 
        \[
        f'(x^{0}) = \frac{\frac{\partial (F, G)}{\partial (z, x)}(\mathbf{x}^{0})}
        {\frac{\partial (F, G)}{\partial (y, z)}(\mathbf{x}^{0})}, \quad
        g'(x^{0}) = \frac{\frac{\partial (F, G)}{\partial (x, y)}(\mathbf{x}^{0})}
        {\frac{\partial (F, G)}{\partial (y, z)}(\mathbf{x}^{0})}.
        \]
        Therefore, the tangent line and normal plane at point \(\mathbf{x}^{0}\) are:
        \begin{gather*}
            T_{x^{0}}M: \quad
            \frac{x - x^{0}}{1} = \frac{y - y^{0}}{f'(x^{0})} = \frac{z - z^{0}}{g'(x^{0})} 
            \Leftrightarrow \frac{x-x^{0}}{\frac{\partial (F, G)}{\partial (y, z)}(\mathbf{x}^{0})} 
            = \frac{y - y^{0}}{\frac{\partial (F, G)}{\partial (z, x)}(\mathbf{x}^{0})}
            = \frac{z - z^{0}}{\frac{\partial (F, G)}{\partial (x, y)}(\mathbf{x}^{0})}, \\
            N_{x^{0}}M: \quad
            \frac{\partial (F, G)}{\partial (y, z)}(\mathbf{x}^{0})(x - x^{0}) +
            \frac{\partial (F, G)}{\partial (z, x)}(\mathbf{x}^{0})(y - y^{0}) +
            \frac{\partial (F, G)}{\partial (x, y)}(\mathbf{x}^{0})(z - z^{0}) = 0.
        \end{gather*}
\end{enumerate}

\begin{leftbarTitle}{Surface}\end{leftbarTitle}
% 三维空间中的曲面: 切平面与法线
When \( n = 3, s = 2 \), \( M \) is a surface in three-dimensional space. 
\begin{enumerate}
    \item If the surface can be described explicitly as:
        \[
        z = f(x, y), \quad (x, y) \in D \subseteq \mathbb{R}^2,
        \]
        at the regular point \( \overline{\mathbf{x}}^{0} = (x^{0}, y^{0}, z^{0}) \) (\(\mathbf{x}^{0}=(x^{0}, y^{0})\)), 
        the tangent plane and normal line are:
        \begin{gather*}
            T_{\mathbf{x}^{0}}M: \quad z - z^{0} = f_x(\mathbf{x}^{0})(x - x^{0}) + f_y(\mathbf{x}^{0})(y - y^{0}), \\
            N_{\mathbf{x}^{0}}M: \quad \frac{x - x^{0}}{f_x(\mathbf{x}^{0})} 
            = \frac{y - y^{0}}{f_y(\mathbf{x}^{0})} = \frac{z - z^{0}}{-1},
        \end{gather*}
        where the expression of \(T_{\mathbf{x}^{0}}M\) is derived from 
        the total differential of \(z = f(x, y)\) at point \(\mathbf{x}^{0}\):
        \[
        \mathrm{d}z = f_x(\mathbf{x}^{0}) \mathrm{d}x + f_y(\mathbf{x}^{0}) \mathrm{d}y.
        \]

    \item If the surface is parameterized as 
        \[
        \mathbf{r}(u, v) = (x(u, v), y(u, v), z(u, v)), \quad (u, v) \in D \subseteq \mathbb{R}^2,
        \]
        at the regular point \( \mathbf{x}^{0} = (x^{0}, y^{0}, z^{0}) \) . 
        \newline For the Jacobian matrix:
        \[
        J = 
        \begin{pmatrix}
            x_{u}(\mathbf{x}^{0}) & x_{v}(\mathbf{x}^{0})  \\
            y_{u}(\mathbf{x}^{0}) & y_{v}(\mathbf{x}^{0})  \\
            z_{u}(\mathbf{x}^{0}) & z_{v}(\mathbf{x}^{0})
        \end{pmatrix},
        \]
        since \(\operatorname{rank}J = 2\), without loss of generality, assume:
        \[
        \frac{\partial (x, y)}{\partial (u, v)}(\mathbf{x}^{0}) = 
        \begin{vmatrix}
            x_{u}(\mathbf{x}^{0}) & x_{v}(\mathbf{x}^{0}) \\
            y_{u}(\mathbf{x}^{0}) & y_{v}(\mathbf{x}^{0})
        \end{vmatrix} \neq 0.
        \]
        By the inverse mapping theorem (\ref{thm:Inverse Mapping Theorem}), we can determine
        the inverse mapping of
        \[
        \begin{cases} x=x(u,v), &  \\ y=y(u,v), &  \end{cases}
        \]
        in a neighborhood of point \(\mathbf{x}^{0}\):
        \[
        \begin{cases} u = u(x, y), &  \\ v = v(x, y), &  \end{cases}
        \]
        where \(u^{0} = u(x^{0}, y^{0})\), \(v^{0} = v(x^{0}, y^{0})\).
        Then we obtain the explicit representation of the surface:
        \[
        z = z(u(x, y), v(x, y)) = f(x, y), \quad (x, y) \in U(x^{0}, y^{0}) \subseteq \mathbb{R}^2.
        \]
        Therefore, the tangent plane and normal line at point \(\mathbf{x}^{0}\) are:
        \begin{gather*}
            T_{\mathbf{x}^{0}}M: \quad 
            \left. \frac{\partial (y,z)}{\partial (u,v)} \right|_{\left(u^{0}, v^{0}\right)}\left( x- x^{0} \right)
            + \left. \frac{\partial (z,x)}{\partial (u,v)} \right|_{\left(u^{0}, v^{0}\right)}\left( y- y^{0} \right)
            + \left. \frac{\partial (x,y)}{\partial (u,v)} \right|_{\left(u^{0}, v^{0}\right)}\left( z- z^{0} \right) = 0, \\
            N_{\mathbf{x}^{0}}M: \quad \frac{x - x^{0}}{\left. \frac{\partial (y,z)}{\partial (u,v)} \right|_{\left(u^{0}, v^{0}\right)}}
            = \frac{y - y^{0}}{\left. \frac{\partial (z,x)}{\partial (u,v)} \right|_{\left(u^{0}, v^{0}\right)}}
            = \frac{z - z^{0}}{\left. \frac{\partial (x,y)}{\partial (u,v)} \right|_{\left(u^{0}, v^{0}\right)}}.
        \end{gather*}
\end{enumerate}




\section{Intrinsic Geometry} % 内在几何
This two sections will introduce the first and second fundamental forms of surfaces,
which can be all generalized to higher-dimensional manifolds;
here, we only discuss the case of two-dimensional surfaces in three-dimensional space.

Let \(\Delta \in \mathbb{R}^{2}\) be an open set,
and \(\mathbf{r}:\Delta \to \mathbb{R}^{3}\) be a \(C^{k}\) (\(k\geqslant 2\)) smooth regular surface parameterization,
\(M = \mathbf{r}(\Delta)\),
where \(\mathbf{u} = (u, v) \to \mathbf{r}(u, v) = (x(u, v), y(u, v), z(u, v))\).
We can obtain that:
\begin{enumerate}
    \item \(\mathbf{r}\in C^{k}(\Delta, \mathbb{R}^{3})\);
    \item For any \(p=(u, v) \in \Delta\),
        \(\operatorname{rank}(\mathbf{r}'_{u}(u, v), \mathbf{r}'_{v}(u, v)) = 2\),
        that is, \(\mathbf{r}'_{u}(u, v)\) and \(\mathbf{r}'_{v}(u, v)\) are linearly independent,
        where
        \[
        \mathbf{r}'_{u}(u, v) = \left( \frac{\partial x}{\partial u}, 
        \frac{\partial y}{\partial u}, \frac{\partial z}{\partial u} \right), \quad
        \mathbf{r}'_{v}(u, v) = \left( \frac{\partial x}{\partial v}, 
        \frac{\partial y}{\partial v}, \frac{\partial z}{\partial v} \right).
        \]
\end{enumerate}
At this time, the tangent space \(T_{p}M = \operatorname{span}(\mathbf{r}'_{u}(u, v), \mathbf{r}'_{v}(u, v))\),
which is a subspace of \(\mathbb{R}^{3}\).
Hence, it inherits the inner product from \(\mathbb{R}^{3}\).

The first fundamental form is the metric that a surface inherits from its ambient Euclidean space \(\mathbb{R}^{3}\). 
It is essentially a symmetric positive-definite bilinear form defined on the tangent space, 
which allows us to \underline{measure lengths, angles, and areas on the surface}.

\begin{definition}{The First Fundamental Form}
    In the above conditions,
    for any point \(p = (u, v) \in \Delta\),
    the \textbf{first fundamental form} of the surface \(M\) at point \(p\) is defined as:
    for any tangent vector \(\mathbf{w}_{1}, \mathbf{w}_{2} \in T_{p}M\),
    \[
    \mathrm{I}_{p}(\mathbf{w}_{1}, \mathbf{w}_{2}) := \mathbf{w}_{1} \cdot \mathbf{w}_{2},
    \]
    which is a symmetric positive-definite bilinear form on the tangent space \(T_{p}M\).
    This form is also called the \textbf{Riemann metric} of \textbf{metric tensor}, denoted as \(\mathrm{I}_{p}\) or \(g_{p}\).
\end{definition}
For convenience, we express \(\mathrm{I}_{p}\) in the basis \(\{\mathbf{r}'_{u}, \mathbf{r}'_{v}\}\) 
of the tangent space \(T_{p}M\).
Define:
\begin{align*}
    &E(u, v) := \mathrm{I}_{p}(\mathbf{r}_{u}, \mathbf{r}_{u}) = \mathbf{r}_{u} \cdot \mathbf{r}_{u} = \| \mathbf{r}_{u} \|^2, \\
    &F(u, v) := \mathrm{I}_{p}(\mathbf{r}_{u}, \mathbf{r}_{v}) = \mathbf{r}_{u} \cdot \mathbf{r}_{v}, \\
    &G(u, v) := \mathrm{I}_{p}(\mathbf{r}_{v}, \mathbf{r}_{v}) = \mathbf{r}_{v} \cdot \mathbf{r}_{v} = \| \mathbf{r}_{v} \|^2,
\end{align*}
which are called the \textbf{Gauß coefficients}.
\newline Then the matrix representation of the first fundamental form \(\mathrm{I}_{p}\) under the basis \(\{\mathbf{r}'_{u}, \mathbf{r}'_{v}\}\) is:
\[
\mathrm{I}_{p} =
\begin{pmatrix}
    E & F \\
    F & G
\end{pmatrix},
\]
which is symmetric and positive-definite.

The quadratic form corresponding to this bilinear form is also commonly called the first fundamental form, 
denoted as  \(\mathrm{d}s^{2}\). 
For a tangent vector \(\mathbf{w}\in T_{p}S\), it represents the square of the length of that vector:
\[
\mathrm{d}s^{2} := \mathrm{I}_{p}(\mathbf{w}, \mathbf{w}) = \| \mathbf{w} \|^{2}.
\]
If \(\mathbf{w}\) is the tangent vector to the curve \(\gamma(t)=\mathbf{r}(u(t), v(t))\), 
given by \(\gamma'(t)=\mathbf{r}_{u}u'(t)+\mathbf{r}_{v}v'(t)\), 
then \(\mathrm{d}s^{2}\) is conventionally written as: 
\[
\mathrm{d}s^{2} = E \, \mathrm{d}u^{2} + 2F \, \mathrm{d}u \, \mathrm{d}v + G \, \mathrm{d}v^{2}.
\]
Here,\(\mathrm{d}u\) and \(\mathrm{d}v\) are the coordinates under the basis \(\{ \mathrm{d}u, \mathrm{d}v \}\) , 
representing the components of the tangent vector \((u', v')\). 
This is a long-standing notation, and strictly speaking, 
it represents the value of the quadratic form on the vector \((u', v')\).




\begin{leftbarTitle}{Arc Length}\end{leftbarTitle} % 弧长
\begin{definition}{Arc Length}
    Let \(C = \overset{\frown}{AB}\) be a curve on the \(\mathbb{R}^{2}\) plane\footnote{
        Or in \(\mathbb{R}^{3}\) space, even in a higher-dimensional Euclidean space.
    },
    take any partition \( A = P_{0}, P_{1}, \ldots, P_{n} = B \),
    which divides the curve \(C\) into \(n\) segments, denoted as \(T\).
    Then connect every two adjacent points \(P_{i-1}\) and \(P_{i}\) with a straight line segment,
    obtaining \(n\) chords \(\overline{P_{i-1}P_{i}}\)(\(i=1, 2, \ldots, n\)),
    which in turn form an inscribed polygonal line \(C\).
    Let 
    \[
    \|T\| = \max_{1 \leqslant i \leqslant n} \|P_{i-1}P_{i}\|, \quad s_{T}= \sum_{i=1}^{n} \|P_{i-1}P_{i}\|.
    \]
    If the limit
    \[
    \lim_{\|T\| \to 0} s_{T} = s,
    \] 
    namely, 
    \[
    \forall \varepsilon > 0, \exists \delta > 0, \text{s.t.} \forall T(\|T\| < \delta): |s_{T} - s| < \varepsilon,
    \]
    and the limit is independent of the choice of partition \(T\),
    then \(C\) is said to be rectifiable, and the limit \(s\) is called the arc length of the curve \(C\).
\end{definition}

\begin{theorem}{Sufficient Condition for Rectifiability of Curves}
    Let the curve \(C\) in \(\mathbb{R}^{2}\) be given by the parametric equations 
    \[
    (x, y) = (x(t), y(t)), \quad t \in [\alpha, \beta],
    \] 
    and let it be a \(C^{1}\) smooth regular curve\footnote{
        I.e., \(x(t)\) and \(y(t)\) are continuously differentiable, and \(x'^{2}(t) + y'^{2}(t) \neq 0\); 
        a curve \(C\) satisfying this condition is called a regular point.
        Also see Definition~\ref{def:Parameterization of Surface}
    }
    Then \(C\) is rectifiable, 
    and its arc length is 
    \[
    s = \int_{\alpha}^{\beta} \sqrt{x'^{2}(t) + y'^{2}(t)} \, \mathrm{d}t.
    \]
\end{theorem}

\begin{leftbarTitle}{Area}\end{leftbarTitle} % 面积
For convenience, we study the area of a surface patch \(M\) parameterized by \(\mathbf{r}(u, v): \Delta \to \mathbb{R}^3\)
over the domain \(\Delta \subseteq \mathbb{R}^{2}\).

% 尝试用内接多边形曲面定义面积, 但是 Schwarz 的反例说明该定义不成立
Similar to the definition of arc length,
we try to define the area of surface patch \(M\) by approximating it with inscribed polygonal surfaces.
However, this definition does not hold, as demonstrated by Schwarz's counterexample
that is called \textbf{Schwartz's lantern} vividly.
% 在这个反例中, 我们能得到圆周率等于 4 的错误结论

In this counterexample, we can obtain the incorrect conclusion that \(\pi = 4\).
Here is a brief description of the construction of Schwartz's lantern:

Consider a cylinder with height \(1\) and base radius \(\frac{1}{2}\) (the left in Fig.~\ref{fig:Schwartz's lantern}).
Its lateral surface area is \(2\pi rh = 2\pi \times \frac{1}{2} \times 1 = \pi\).

Divide the cylinder into four equal cylinders, and place seven equally spaced red dots on each circle 
(the middle in Fig.~\ref{fig:Schwartz's lantern}).
Then connect these red dots to form a polygonal surface (the right in Fig.~\ref{fig:Schwartz's lantern}).
\begin{figure}[h]
    \centering
    \includegraphics[width=0.6\textwidth]{img/Schwartz1.png}
    \caption{Schwartz's lantern construction on a cylinder.}
    \label{fig:Schwartz's lantern}
\end{figure}
This particular lantern has \(4\) horizontal triangular bands, 
and on each level there are \(7\) equally spaced red dots, 
which can be expressed as \(b = 4, p = 7\).

To obtain increasingly precise approximations of the cylindrical lanterns, 
simply increase the number of bands and points
(Fig.~\ref{fig:More precise approximations of Schwartz's lantern}).
\begin{figure}[h]
    \centering
    \includegraphics[width=0.8\textwidth]{img/Schwartz2.png}
    \caption{More precise approximations of Schwartz's lantern.}
    \label{fig:More precise approximations of Schwartz's lantern}
\end{figure}

In fact, we can assign particular values to \(b\) and \(p\)
to make the area of the polygonal surface approach any value greater than \(\pi\),
\[
A(b, p) = \frac{bp}{2} \sin\frac{\pi}{p} \sqrt{\left( \frac{2}{b} \right)^2 + \left( \sin\frac{\pi}{p} \right)^2}.
\]

\vspace{0.7cm}
???????????????????????????????????????????????????????????????????

We derive the area of \(M\) using the first fundamental form.
Consider a small rectangle \(\Delta u \times \Delta v\) in the parameter domain \(\Delta\),
which is mapped to a small parallelogram on the surface \(M\) by the parameterization \(\mathbf{r}(u, v)\).
The two adjacent sides of this parallelogram can be approximated by the tangent vectors:
\[\mathbf{r}_{u} \Delta u, \quad \mathbf{r}_{v} \Delta v.\]
The area of this parallelogram is given by the magnitude of the cross product of these two vectors:
\[\|\mathbf{r}_{u} \times \mathbf{r}_{v}\| \Delta u \Delta v.\]
Using the properties of the dot product and the first fundamental form, we have:
\[\|\mathbf{r}_{u} \times \mathbf{r}_{v}\| = \sqrt{EG - F^{2}}.\]
Therefore, the area element \(\mathrm{d}A\) on the surface \(M\) is:
\[\mathrm{d}A = \sqrt{EG - F^{2}} \,\mathrm{d}u \, \mathrm{d}v.\]
Integrating over the entire parameter domain \(\Delta\), we obtain the total area of the surface patch \(M\):
\[\text{Area}(M) = \iint_{\Delta} \sqrt{EG - F^{2}} \,\mathrm{d}u \, \mathrm{d}v.\]
???????????????????????????????????????????????????????????????????

\section{Extrinsic Geometry} % 外在几何
The second fundamental form is a symmetric bilinear form defined on the tangent space 
that measures the change in the normal vector of a surface, 
thereby describing the \underline{extrinsic curvature} of the surface relative to its ambient space \(\mathbb{R}^3\).

On the regular surface patch \(M\) defined in the beginning of last section, 
we can define a continuous unit normal vector field \(\mathbf{n}: M \to \mathbb{S}^2\), 
where \(\mathbb{S}^2\) is the unit sphere in \(\mathbb{R}^3\):
\[
\mathbf{n}(p) = \frac{\mathbf{r}_u \times \mathbf{r}_v}{\|\mathbf{r}_u \times \mathbf{r}_v\|}(p).
\]
This mapping \(\mathbf{n}\) from the surface to the unit sphere is called the \textbf{Gauß map}. 
The second fundamental form is defined by studying the differential of the Gauß map.

\begin{definition}{The Second Fundamental Form}
    Under the above conditions,
    for any point \(p = (u, v) \in \Delta\),
    the \textbf{second fundamental form} of the surface \(M\) at point \(p\) is 
    a symmetric bilinear form on the tangent space \(T_{p}M\),
    which is defined as:
    for any tangent vector \(\mathbf{w}_{1}, \mathbf{w}_{2} \in T_{p}M\),
    \[
    \mathrm{II}_{p}(\mathbf{w}_{1}, \mathbf{w}_{2}) := -\mathrm{d}_{p}\mathbf{n}(\mathbf{w}_{1}) \cdot \mathbf{w}_{2},
    \]\footnote{ 
        About the formula,
        \begin{itemize}
            \item since \(\mathbf{n}(p)\) is a unit vector, \(T_{\mathbf{n}(p)}\mathbb{S}^{2}\) is 
                the plane orthogonal to \(\mathbf{n}(p)\),
                and \(T_{p}M\) itself is also orthogonal to  \(\mathbf{n}(p)\), 
                it follows that  \(\mathrm{d}_{p}\mathbf{n}(\mathbf{w}_1)\)  and  \(\mathbf{w}_2\)  lie in the same plane, 
                and their dot product is well-defined.
            \item the negative sign in this definition is a convention, 
                which makes the principal curvatures of a convex surface (like a sphere) positive.
        \end{itemize}
    }
    where \(\mathrm{d}_{p}\mathbf{n}: T_{p}M \to T_{\mathbf{n}(p)}\mathbb{S}^{2}\) is the differential (or Jacobian)
    of the Gauß map at point \(p\).

    The linear operator associated with \(\mathrm{d}_{p}\mathbf{n}\), defined as  
    \(W_{p}(\mathbf{w}) = -\mathrm{d}_{p}\mathbf{n}(\mathbf{w})\), 
    is called the Weingarten map or shape operator, 
    and it is a linear operator from  \(T_{p}M\)  to itself. Therefore, the second fundamental form can also be written as:
    \[
    \mathrm{II}_{p}(\mathbf{w}_{1}, \mathbf{w}_{2}) = W_{p}(\mathbf{w}_{1}) \cdot \mathbf{w}_{2}.
    \]
\end{definition}
For convenience, we express \(\mathrm{II}_{p}\) in the basis \(\{\mathbf{r}'_{u}, \mathbf{r}'_{v}\}\) of 
the tangent space \(T_{p}M\).
Define:
\begin{align*}
    &L(u, v):=\mathrm{II}_{p}(\mathbf{r}_{u}, \mathbf{r}_{u}) = 
    W_{p}(\mathbf{r}_{u}) \cdot \mathbf{r}_{u} = \mathbf{r}_{uu} \cdot \mathbf{n}; \\
    &M(u, v):=\mathrm{II}_{p}(\mathbf{r}_{u}, \mathbf{r}_{v}) = 
    W_{p}(\mathbf{r}_{u}) \cdot \mathbf{r}_{v} = \mathbf{r}_{uv} \cdot \mathbf{n}; \\
    &N(u, v):=\mathrm{II}_{p}(\mathbf{r}_{v}, \mathbf{r}_{v}) = 
    W_{p}(\mathbf{r}_{v}) \cdot \mathbf{r}_{v} = \mathbf{r}_{vv} \cdot \mathbf{n}.
\end{align*}
Then the matrix representation of the second fundamental form \(\mathrm{II}_{p}\) 
under the basis \(\{\mathbf{r}'_{u}, \mathbf{r}'_{v}\}\) is:
\[
\mathrm{II}_{p} =
\begin{pmatrix}
    L & M \\
    M & N
\end{pmatrix},
\]
which is symmetric, but not necessarily positive-definite.
And its sign reflects the way the surface is curved. 

The associated second fundamental form, also denoted by \(\mathrm{II}\), is an expression for the normal curvature:
\[
\mathrm{II} = L \, \mathrm{d}u^2 + 2M \, \mathrm{d}u \, \mathrm{d}v + N \, \mathrm{d}v^2.
\]
For a unit tangent vector  \(\mathbf{w} \in T_{p}M\), 
the value of  \(\mathrm{II}_{p}(\mathbf{w}, \mathbf{w})\)  
is the normal curvature of the surface in the direction of  \(\mathbf{w}\), denoted \(k_n(\mathbf{w})\).

\begin{leftbarTitle}{Curvature}\end{leftbarTitle}
Curvature is a mathematical quantity describing the "bending" degree of a geometric object, 
such as a curve or a surface. 

The meaning of curvature varies for geometric objects of different dimensions: 
\begin{itemize}
    \item Curvature on a curve: Describes the degree to which the curve deviates from a straight line.
    \item Description of curvature by a surface: Is more complex, involving directionality—the bending of 
        a surface can be completely different in different directions. 
\end{itemize}
The curvature of a surface is usually classified into the following typical types: 
normal curvature, principal curvatures, mean curvature, Gaussian curvature, etc.

\begin{definition}{Curvature of Curve}
    Let \(C\) be a \(C^{2}\) smooth regular curve in \(\mathbb{R}^{3}\),
    parameterized by arc length \(t\):
    \[
    \mathbf{r}(t) = (x(t), y(t), z(t)), \quad t \in [a, b].
    \]
    The unit tangent vector of the curve at point \(t\) is:
    \[
    \mathbf{T}(t) = \mathbf{r}'(t) = (x'(t), y'(t), z'(t)).
    \]
    The \textbf{curvature} of the curve at point \(t\) is defined as the magnitude of the derivative of the unit tangent vector with respect to arc length:
    \[
    \kappa(t) = \left\| \frac{\mathrm{d}\mathbf{T}(t)}{\mathrm{d}t} \right\| = 
    \frac{\|\mathbf{r}'(t) \times \mathbf{r}''(t)\|}{\|\mathbf{r}'(t)\|^3}.
    \]
    Geometrically, curvature measures how quickly the curve changes direction at point \(t\).

    If the best-fit circle is found based on the tangent and normal at a certain point, 
    the radius of this circle is called the radius of curvature \(R\), 
    and the curvature is its reciprocal: 
    \[
    \kappa = \frac{1}{R}.
    \] 
    This fitted circle is called the \textbf{osculating circle} of the curve at that point. % 渐屈圆
\end{definition}
Some special cases of curvature are given below:
\begin{enumerate}
    \item For a plane curve given by \(y = f(x)\), the curvature at point \(x\) is:
        \[
        \kappa(x) = \frac{|f''(x)|}{(1 + (f'(x))^2)^{3/2}}.
        \]
    \item For a circle with radius \(R\), the curvature is constant:
        \[
        \kappa = \frac{1}{R}.
        \]
\end{enumerate}


\section{Oriented Surface} % 定向曲面

\section{Bounded Variation Functions} % 有界变差函数
\begin{definition}{Bounded Variation}
    Let \(f: [a, b] \to \mathbb{R}\) be a real-valued function defined on the closed interval \([a, b]\).
    For any partition \(P = \{ x_0, x_1, \ldots, x_n \}\) of \([a, b]\) with \(a = x_0 < x_1 < \cdots < x_n = b\),
    define the variation of \(f\) on the partition \(P\) as:
    \[
    V(f, P) = \sum_{i=1}^{n} |f(x_i) - f(x_{i-1})|.
    \]
    The total variation of \(f\) on \([a, b]\) is defined as:
    \[
    V_a^b(f) = \sup_{P} V(f, P),
    \]
    where the supremum is taken over all possible partitions \(P\) of \([a, b]\).
    If \(V_a^b(f) < \infty\), then \(f\) is said to be of \textbf{bounded variation} on \([a, b]\),
    denoted as \(f \in BV[a, b]\).
\end{definition}
\begin{property}
    \begin{enumerate}
        \item \(BV[a, b]\subset B[a, b]\)
        \item For any \(f, g\in BV[a, b]\), 
            and any scalars \(\alpha, \beta \in \mathbb{R}\), 
            the linear combination \(\alpha f + \beta g \in BV[a, b]\),
            and 
            \[
            V_a^b(\alpha f + \beta g) \leqslant |\alpha| V_a^b(f) + |\beta| V_a^b(g).
            \]
            Specially, if \(|g(x)|\geqslant \sigma > 0\), then \(\frac{f(x)}{g(x)} \in BV[a, b]\).
        \item 
    \end{enumerate}
\end{property}

Some common bounded variation functions include:
\begin{description}
    \item[Monotonic Functions] Any monotonic function on a closed interval is of bounded variation,
        and \(V_a^b(f) = |f(b) - f(a)|\).
    \item[Piecewise Monotonic Functions] Functions that are monotonic on each subinterval of a finite partition of \([a, b]\)
        are also of bounded variation.
    \item[Lipschitz Continuous Functions] Any Lipschitz continuous function on \([a, b]\) is of bounded variation.
    \item[Functions with Finite Discontinuities] Functions that have only a finite number of jump discontinuities on \([a, b]\)
        are of bounded variation.
    \item[Absolutely Continuous Functions] Any absolutely continuous function on \([a, b]\) is of bounded variation.
\end{description}


\begin{theorem}{Jordan Decomposition Theorem} % Jordan 分解定理
    \(f\in BV[a, b]\) if and only if there exist two monotonic increasing functions
    \(g(x),  h(x) : [a, b] \to \mathbb{R}\), such that:
    \[
    f(x) = g(x) - h(x).
    \]
\end{theorem}

\vspace{0.7cm}
Bounded variation functions have important applications,
for example, in harmonic analysis, the Fourier series of bounded variation functions 
converge pointwise.
Other typical applications are as follows:
\begin{leftbarTitle}{Rectifiable Curves}\end{leftbarTitle} % 可求长曲线
\begin{theorem}{Jordan's Theorem on Rectifiable Curves}
    A curve \(C\) in \(\mathbb{R}^{2}\) defined by the parametric equations
    \[
    (x, y) = (x(t), y(t)), \quad t \in [a, b],
    \]
    is rectifiable if and only if both \(x(t)\) and \(y(t)\) are of bounded variation on \([a, b]\).
\end{theorem}

\begin{leftbarTitle}{Stieltjes Integral}\end{leftbarTitle} % Stieltjes 积分


 % 曲面论导论
\chapter{Line Integrals and Surface Integrals} % 曲线积分与曲面积分
\section{Line Integrals and Surface Integrals of scalar fields}
\begin{leftbarTitle}{Line Integral of Scalar Field}\end{leftbarTitle}
\begin{definition}{Line Integral of Scalar Field}
    Let \(L\) is a rectifiable continuous curve in \(\mathbb{R}^3\), whose endpoints are \(A\) and \(B\),
    and \(f(x, y, z)\) is bounded on \(L\).
    Partition \(L\) into \(n\) segments by points \(A = P_0, P_1, \ldots, P_n = B\),
    and select a point \(\boldsymbol{\xi}_{i}\) on each segment \(P_{i-1}P_i\) (\(i = 1, 2, \ldots, n\)).
    Remark that the length of segment \(P_{i-1}P_i\) is \(\Delta s_i\) (\(i=1,2,\cdots n\)),
    and make the sum:
    \[
    \sum_{i=1}^{n} f(\boldsymbol{\xi}_i) \Delta s_i.
    \]
    If when \( \lambda \) (the length of the longest segment) tends to \(0\),
    the above sum tends to a limit \(I\) independent of the partition and the choice of points \(\boldsymbol{\xi}_i\),
    then \(I\) is called the \textbf{line integral of the scalar field \(f\) along the curve \(L\)},
    denoted as:
    \[
    \int_{L} f \, \mathrm{d}s.
    \]
    That is,
    \[
    I = \int_{L} f(\boldsymbol{\xi}) \, \mathrm{d}s =
    \lim_{\lambda \to 0} \sum_{i=1}^{n} f(\boldsymbol{\xi}_i) \Delta s_i.
    \]
\end{definition}

\begin{theorem}
    Let \(L\) be a \(C^{1}\) smooth regular curve parameterized by \(\mathbf{x}(t) = (x(t), y(t), z(t)), t \in [\alpha, \beta]\),
    and \(f\) be continuous on \(L\).
    Then:
    \[
    \int_{L} f \, \mathrm{d}s = \int_{\alpha}^{\beta} f(\mathbf{x}(t)) \|\mathbf{x}'(t)\| \, \mathrm{d}t.
    = \int_{\alpha}^{\beta} f(x(t), y(t), z(t)) \sqrt{(x'(t))^2 + (y'(t))^2 + (z'(t))^2} \, \mathrm{d}t.
    \]
\end{theorem}
Specially, if the plane curve \(L\) is given by \(y = y(x), x \in [a, b]\),
then:
\[
\int_{L} f \, \mathrm{d}s = \int_{a}^{b} f(x, y(x)) \sqrt{1 + (y'(x))^2} \, \mathrm{d}x.
\]


\begin{leftbarTitle}{Surface Integrals of Scalar Fields}\end{leftbarTitle}
\begin{definition}{Surface Integral of Scalar Field}
    Let \(\Sigma\) be a piecewise smooth surface in \(\mathbb{R}^3\),
    and \(f(x, y, z)\) be bounded on \(\Sigma\).
    Partition \(\Sigma\) into \(n\) small pieces \(\Delta\Sigma_1, \Delta\Sigma_2, \ldots, \Delta\Sigma_n\)
    with smooth curve webs,
    and select a point \(\boldsymbol{\xi}_i\) on each piece \(\Delta\Sigma_i\) (\(i = 1, 2, \ldots, n\)).
    Remark that the area of piece \(\Delta\Sigma_i\) is \(\Delta S_i\) (\(i=1,2,\cdots n\)),
    and make the sum:
    \[
    \sum_{i=1}^{n} f(\boldsymbol{\xi}_i) \Delta S_i.
    \]
    If when \( \lambda \) (the area of the largest piece) tends to \(0\),
    the above sum tends to a limit \(I\) independent of the partition and the choice of points \(\boldsymbol{\xi}_i\),
    then \(I\) is called the \textbf{surface integral of the scalar field \(f\) over the surface \(\Sigma\)},
    denoted as:
    \[
    \iint_{\Sigma} f \, \mathrm{d}S.
    \]
    That is,
    \[
    I = \iint_{\Sigma} f(\boldsymbol{\xi}) \, \mathrm{d}S =
    \lim_{\lambda \to 0} \sum_{i=1}^{n} f(\boldsymbol{\xi}_i) \Delta S_i.
    \]
\end{definition}

\begin{theorem}
    Let \(\Sigma\) be a piecewise smooth closed surface parameterized by 
    \(\mathbf{r}(u, v) = (x(u, v), y(u, v), z(u, v)), (u, v) \in D\),
    and \(f\) be continuous on \(\Sigma\).
    \(x, y, z\) have continuous first-order partial derivatives with respect to \(u\) and \(v\) on \(D\),
    and according Jacobian matrix 
    \[
    J = \begin{pmatrix}
        \frac{\partial x}{\partial u} & \frac{\partial x}{\partial v} \\
        \frac{\partial y}{\partial u} & \frac{\partial y}{\partial v} \\
        \frac{\partial z}{\partial u} & \frac{\partial z}{\partial v}
    \end{pmatrix}
    \]
    is of full rank.
    Then:
    \[
    \iint_{\Sigma} f \, \mathrm{d}S = \iint_{D} f(\mathbf{r}(u, v)) 
    \left\| \frac{\partial \mathbf{r}}{\partial u} \times \frac{\partial \mathbf{r}}{\partial v} \right\| \, \mathrm{d}u \mathrm{d}v
    = \iint_{D} f(x(u, v), y(u, v), z(u, v)) 
    \sqrt{EG-F^{2}} \, \mathrm{d}u \mathrm{d}v,
    \]
    where \(E, G, F\) are the Gauß coefficients of the surface \(\Sigma\).
\end{theorem}
Specially, if the surface \(\Sigma\) is given by \(z = z(x, y), (x, y) \in D\),
then:
\[
\iint_{\Sigma} f \, \mathrm{d}S = 
\iint_{D} f(x, y, z(x, y)) 
\sqrt{1 + \left(\frac{\partial z}{\partial x}\right)^2 + \left(\frac{\partial z}{\partial y}\right)^2} \, \mathrm{d}x \mathrm{d}y.
\]

\section{Differential Form and Exterior Differentiation}
Let \(\mathrm{d}x_{i}, \mathrm{d}x_{j}\) be differentials of independent variables \(x_{i}, x_{j}\).

In \(\mathbb{R}^{1}\):
\begin{align*}
    &\text{0-form: } f(x), \\
    &\text{1-form: } \omega = f(x)\mathrm{d}x, \\
    &\text{k-form (\(k\geqslant 2\)): } \omega = \sum_{1 \leqslant i_{1} < i_{2} < \cdots < i_{k} \leqslant n}
        f_{i_{1} i_{2} \cdots i_{k}}(x_{1}, x_{2}, \cdots, x_{n})
        \mathrm{d}x_{i_{1}} \wedge \mathrm{d}x_{i_{2}} \wedge \cdots \wedge \mathrm{d}x_{i_{k}} = 0.
\end{align*}

In \(\mathbb{R}^{2}\):
\begin{align*}
    &\text{0-form: } f(x, y), \\
    &\text{1-form: } \omega = P(x, y)\mathrm{d}x + Q(x, y)\mathrm{d}y, \\
    &\text{2-form: } \omega = f(x, y)\mathrm{d}x \wedge \mathrm{d}y, \\
    &\text{k-form (\(k\geqslant 3\)): } \omega = \sum_{1 \leqslant i_{1} < i_{2} < \cdots < i_{k} \leqslant n}
        f_{i_{1} i_{2} \cdots i_{k}}(x_{1}, x_{2}, \cdots, x_{n})
        \mathrm{d}x_{i_{1}} \wedge \mathrm{d}x_{i_{2}} \wedge \cdots \wedge \mathrm{d}x_{i_{k}} = 0.
\end{align*}

In \(\mathbb{R}^{3}\):
\begin{align*}
    &\text{0-form: } f(x, y, z), \\
    &\text{1-form: } \omega = P(x, y, z)\mathrm{d}x + Q(x, y, z)\mathrm{d}y + R(x, y, z)\mathrm{d}z, \\
    &\text{2-form: } \omega = P(x, y, z)\mathrm{d}y \wedge \mathrm{d}z + Q(x, y, z)\mathrm{d}z \wedge \mathrm{d}x + R(x, y, z)\mathrm{d}x \wedge \mathrm{d}y, \\
    &\text{3-form: } \omega = f(x, y, z)\mathrm{d}x \wedge \mathrm{d}y \wedge \mathrm{d}z, \\
    &\text{k-form (\(k\geqslant 4\)): } \omega = \sum_{1 \leqslant i_{1} < i_{2} < \cdots < i_{k} \leqslant n}
        f_{i_{1} i_{2} \cdots i_{k}}(x_{1}, x_{2}, \cdots, x_{n})
        \mathrm{d}x_{i_{1}} \wedge \mathrm{d}x_{i_{2}} \wedge \cdots \wedge \mathrm{d}x_{i_{k}} = 0.
\end{align*}

Here, \(\wedge\) is called the \textbf{wedge product}, which satisfies:
\begin{enumerate}
    \item Skew symmetric: \(\mathrm{d}x_{i} \wedge \mathrm{d}x_{j} = -\mathrm{d}x_{j} \wedge \mathrm{d}x_{i}\), 
    \item Associative: \((\mathrm{d}x_{i} \wedge \mathrm{d}x_{j}) \wedge \mathrm{d}x_{k} = \mathrm{d}x_{i} \wedge (\mathrm{d}x_{j} \wedge \mathrm{d}x_{k})\),
    \item In a fixed dimension, the wedge product of k-forms becomes zero (as higher forms are not defined), 
        for example, in \(3\)-dimensional space, a \(4\)-form is equal to \(0\).
\end{enumerate}
\textbf{Differential form} is a skew symmetric tensor on vector field.


\begin{definition}{Exterior Differentiation}
    Let \(\omega\) be a \(k\)-form on \(\mathbb{R}^{n}\),
    \[
    \omega = \sum_{1 \leqslant i_{1} < i_{2} < \cdots < i_{k} \leqslant n}
        f_{i_{1} i_{2} \cdots i_{k}}(x_{1}, x_{2}, \cdots, x_{n})
        \mathrm{d}x_{i_{1}} \wedge \mathrm{d}x_{i_{2}} \wedge \cdots \wedge \mathrm{d}x_{i_{k}},
    \]
    where \(f_{i_{1} i_{2} \cdots i_{k}}\) are functions with continuous first-order partial derivatives.
    The exterior differentiation of \(\omega\) is defined as:
    \[
    \mathrm{d}\omega = \sum_{1 \leqslant i_{1} < i_{2} < \cdots < i_{k} \leqslant n}
        \mathrm{d}f_{i_{1} i_{2} \cdots i_{k}}(x_{1}, x_{2}, \cdots, x_{n})
        \wedge \mathrm{d}x_{i_{1}} \wedge \mathrm{d}x_{i_{2}} \wedge \cdots \wedge \mathrm{d}x_{i_{k}},
    \]
    where
    \[
    \mathrm{d}f = \frac{\partial f}{\partial x_1}\mathrm{d}x_1 + 
    \frac{\partial f}{\partial x_2}\mathrm{d}x_2 + 
    \cdots + 
    \frac{\partial f}{\partial x_n}\mathrm{d}x_n.
    \]
    Note that the exterior differentiation of a \(k\)-form is a \(k+1\)-form.
\end{definition}
\begin{property}
    \begin{description}
        \item[Linearity] \(\mathrm{d}(\alpha \omega+ \beta \eta) = \alpha \mathrm{d}\omega + \beta \mathrm{d}\eta\),
        where \(\alpha, \beta\) are constants.
        \item[Leibniz Rule] \(\mathrm{d}(\omega \wedge \eta) = \mathrm{d}\omega \wedge \eta + (-1)^{k} \omega \wedge \mathrm{d}\eta\),
        where \(\omega\) is a \(k\)-form.
        \item[Nilpotency] \(\mathrm{d}(\mathrm{d}\omega) = 0\).
    \end{description}
\end{property}




\section{Line Integrals and Surface Integrals of Vector Fields}
\begin{leftbarTitle}{Line Integral of Vector Field}\end{leftbarTitle}
\begin{definition}{Line Integral of Vector Field}
    Let \(\overset{\rightharpoonup}{L}\) be a orientated smooth curve in \(\mathbb{R}^3\),
    whose endpoints are \(A\) and \(B\).
    Take unit tangent vector \(\boldsymbol{\tau}=(\cos\alpha, \cos\beta, \cos\gamma)\) 
    at each point of \(\overset{\rightharpoonup}{L}\), making it consistent with the orientation of \(\overset{\rightharpoonup}{L}\).
    Let \(\mathbf{f}(x, y, z) = P(x, y, z)\mathbf{i} + Q(x, y, z)\mathbf{j} + R(x, y, z)\mathbf{k}\) 
    be a vector-valued function on \(\overset{\rightharpoonup}{L}\), then 
    \[
    \int_{\overset{\rightharpoonup}{L}} \mathbf{f} \cdot \boldsymbol{\tau} \mathrm{d}\mathbf{s} =
    \int_{\overset{\rightharpoonup}{L}} \left[P \cos\alpha + Q \cos\beta + R \cos\gamma\right] \, \mathrm{d}s
    \]
    is called the \textbf{line integral of the vector field \(\mathbf{f}\) along the oriented curve \(\overset{\rightharpoonup}{L}\)}
    (if the right-hand side exists).
\end{definition}
Consider a differential arc length element \( \mathrm{d}s \) at a point \((x, y, z)\) on the curve \( L \). 
We form the vector \( \mathrm{d}\mathbf{s} = \boldsymbol{\tau} \mathrm{d}s \), 
where \( \boldsymbol{\tau} = (\cos\alpha, \cos\beta, \cos\gamma) \) represents 
the unit tangent vector of curve \( L \) at \((x, y, z)\), pointing along the direction of \( L \). 
The projection of \( \mathrm{d}s \) onto the \( x \)-axis is given by \(\cos\alpha \, \mathrm{d}s\). 
Therefore, we denote:
\[
\mathrm{d}x = \cos\alpha \, \mathrm{d}s, \quad \mathrm{d}y = \cos\beta \, \mathrm{d}s, \quad 
\mathrm{d}z = \cos\gamma \, \mathrm{d}s.
\]
Thus, the second type of line integral can be expressed as:
\[
\int_{\overset{\rightharpoonup}{L}} \mathbf{f} \cdot \boldsymbol{\tau} \mathrm{d}s = \int_{\overset{\rightharpoonup}{L}} \mathbf{f} \, \mathrm{d} \mathbf{s} 
= \int_{\overset{\rightharpoonup}{L}} P(x, y, z) \mathrm{d}x + Q(x, y, z) \mathrm{d}y + R(x, y, z) \mathrm{d}z.
\]
This line integral is also referred to as the integral of the \(1\)-form:
\[
\omega = P(x, y, z) \mathrm{d}x + Q(x, y, z) \mathrm{d}y + R(x, y, z) \mathrm{d}z.
\]
The second type of line integral of \( \omega \) along the curve \( L \) is denoted as:
\[
\int_{\overset{\rightharpoonup}{L}} \omega.
\]

\begin{theorem}
    Let \(\overset{\rightharpoonup}{L}\) be a \(C^{1}\) smooth regular oriented curve parameterized by 
    \(\mathbf{x}(t) = (x(t), y(t), z(t)), t \in [\alpha, \beta]\),
    and \(\mathbf{f} = P\mathbf{i} + Q\mathbf{j} + R\mathbf{k}\) be continuous on \(\overset{\rightharpoonup}{L}\).
    Then:
    \begin{gather*}
    \int_{\overset{\rightharpoonup}{L}} \mathbf{f} \cdot \boldsymbol{\tau} \mathrm{d}s =
    \int_{\alpha}^{\beta} \mathbf{f}(\mathbf{x}(t)) \cdot \mathbf{x}'(t) \, \mathrm{d}t \\
    = \int_{\alpha}^{\beta} [P(x(t), y(t), z(t)) x'(t) + Q(x(t), y(t), z(t)) y'(t) + R(x(t), y(t), z(t)) z'(t)] \, \mathrm{d}t.
    \end{gather*}
\end{theorem}
Specially, if the plane curve \(\overset{\rightharpoonup}{L}\) is given by \(y = y(x), x: a \to b\),
then:
\[
\int_{\overset{\rightharpoonup}{L}} \mathbf{f} \cdot \boldsymbol{\tau} \mathrm{d}s = 
\int_{a}^{b} \mathbf{f}(x, y(x)) \cdot (1, y'(x)) \sqrt{1 + (y'(x))^2} \, \mathrm{d}x.
\]

\begin{leftbarTitle}{Surface Integral of Vector Field}\end{leftbarTitle}
\begin{definition}{Surface Integral of Vector Field}
    Let \(\overset{\rightharpoonup}{\Sigma}\) be an orientated smooth surface in \(\mathbb{R}^3\),
    and \(\mathbf{f}(x, y, z) = P(x, y, z)\mathbf{i} + Q(x, y, z)\mathbf{j} + R(x, y, z)\mathbf{k}\) 
    be a vector-valued function on \(\overset{\rightharpoonup}{\Sigma}\).
    Each point appoints a unit normal vector \(\mathbf{n}=(\cos\alpha, \cos\beta, \cos\gamma)\).
    Then 
    \[
    \iint_{\overset{\rightharpoonup}{\Sigma}} \mathbf{f} \cdot \mathbf{n} \mathrm{d}S =
    \iint_{\overset{\rightharpoonup}{\Sigma}} \left[P \cos\alpha + Q \cos\beta + R \cos\gamma\right] \, \mathrm{d}S
    \]
    is called the \textbf{surface integral of the vector field \(\mathbf{f}\) over the oriented surface \(\overset{\rightharpoonup}{\Sigma}\)}
    (if the right-hand side exists).
\end{definition}
Consider a differential area element \( \mathrm{d}S \) at a point \((x, y, z)\) on the surface \( \Sigma \). 
We form the vector \( \mathrm{d}\mathbf{S} = \mathbf{n} \mathrm{d}S \), 
where \( \mathbf{n} = (\cos\alpha, \cos\beta, \cos\gamma) \) represents
the unit normal vector of surface \( \Sigma \) at \((x, y, z)\),
pointing along the orientation of \( \Sigma \).
The projection of \( \mathrm{d}S \) onto the \( x \)-axis is given by \(\cos\alpha \, \mathbf{\mathrm{d}}S\). 
Therefore, we denote:
\[
\mathrm{d}y \wedge \mathrm{d}z = \cos\alpha \, \mathrm{d}S, \quad \mathrm{d}z \wedge \mathrm{d}x = \cos\beta \, \mathrm{d}S, \quad 
\mathrm{d}x \wedge \mathrm{d}y = \cos\gamma \, \mathrm{d}S.
\]
Thus, the surface integral can be expressed as:
\[
\iint_{\overset{\rightharpoonup}{\Sigma}} \mathbf{f} \cdot \mathbf{n} \mathrm{d}S = 
\iint_{\overset{\rightharpoonup}{\Sigma}} P \mathrm{d}y \wedge \mathrm{d}z + Q \mathrm{d}z \wedge \mathrm{d}x + R \mathrm{d}x \wedge \mathrm{d}y
= \iint_{\Sigma} P \mathrm{d}y\mathrm{d}z + Q \mathrm{d}z\mathrm{d}x + R \mathrm{d}x\mathrm{d}y,
\]
where \(\mathrm{d}y\mathrm{d}z\) is the simplified notation for \(\mathrm{d}y \wedge \mathrm{d}z\), etc.
This surface integral is also referred to as the integral of the \(2\)-form:
\[
\omega = P(x, y, z) \mathrm{d}y \wedge \mathrm{d}z + Q(x, y, z) \mathrm{d}z \wedge \mathrm{d}x + R(x, y, z) \mathrm{d}x \wedge \mathrm{d}y.
\]
The second type of surface integral of \( \omega \) over the surface \( \Sigma \) is denoted as:
\[
\iint_{\overset{\rightharpoonup}{\Sigma}} \omega.
\]

\begin{theorem}
    Let \(\overset{\rightharpoonup}{\Sigma}\) be a smooth oriented surface parameterized by 
    \(\mathbf{r}(u, v) = (x(u, v), y(u, v), z(u, v)), (u, v) \in D\),
    where \(D\) is a closed region with piecewise smooth boundary in \(uv\)-plane,
    and \(\mathbf{f} = P\mathbf{i} + Q\mathbf{j} + R\mathbf{k}\) be continuous on \(\overset{\rightharpoonup}{\Sigma}\).
    \(x, y, z\) have continuous first-order partial derivatives with respect to \(u\) and \(v\) on \(D\),
    and according Jacobian matrix is of full rank.
    Then:
    \begin{align*}
        &\iint_{\overset{\rightharpoonup}{\Sigma}} \mathbf{f} \cdot \mathbf{n} \mathrm{d}S \\
        =&\iint_{\overset{\rightharpoonup}{\Sigma}} [P \cos\alpha + Q \cos\beta + R \cos\gamma] \, \mathrm{d}S \\
        =& \iint_{D} \mathbf{f}(\mathbf{r}(u, v)) \cdot \left( \frac{\partial \mathbf{r}}{\partial u} \times \frac{\partial \mathbf{r}}{\partial v} \right) \, \mathrm{d}u \mathrm{d}v \\
        =& \pm \iint_{D} \bigg[P(x(u, v), y(u, v), z(u, v)) \cdot \frac{\partial (y, z)}{\partial (u, v)} 
        + Q(x(u, v), y(u, v), z(u, v)) \cdot \frac{\partial (z, x)}{\partial (u, v)} \\
        &+ R(x(u, v), y(u, v), z(u, v)) \cdot \frac{\partial (x, y)}{\partial (u, v)}\bigg] \, \mathrm{d}u \mathrm{d}v,
    \end{align*}
    where the sign \(\pm\) depends on whether the orientation of \(\overset{\rightharpoonup}{\Sigma}\) is consistent with
    the direction of \(\frac{\partial \mathbf{r}}{\partial u} \times \frac{\partial \mathbf{r}}{\partial v}\).
\end{theorem}
Specially, if the surface \(\overset{\rightharpoonup}{\Sigma}\) is given by \(z = z(x, y), (x, y) \in D_{xy}\),
where \(D_{xy}\) is a closed region with piecewise smooth boundary in \(xy\)-plane,
and \(R(x,y,z)\) is continuous on \(D_{xy}\),
then:
\[
\iint_{\overset{\rightharpoonup}{\Sigma}} R(x, y, z) \mathrm{d}x \mathrm{d}y = 
\pm \iint_{D_{xy}} R(x, y, z(x, y)) \, \mathrm{d}x \mathrm{d}y,
\]
where the sign \(\pm\) depends on whether the orientation of \(\overset{\rightharpoonup}{\Sigma}\) is upward or downward.



\section{Stokes' Formula}
\begin{leftbarTitle}{Newton-Leibniz Formula}\end{leftbarTitle}

\begin{leftbarTitle}{Green's Formula}\end{leftbarTitle}
Consider two kinds of special orientated closed regions in \(xy\)-plane as shown in Figure \ref{fig:SpecialRegion1}.
As for the first region \(\overset{\rightharpoonup}{M}\), it consists of four orientated curves:
\begin{description}
    \item[\(\overset{\rightharpoonup}{C_1}\)] \(y = \varphi_{1}(x), x \in [a, b]\),
    \item[\(\overset{\rightharpoonup}{C_2}\)] \(x = b, y \in [\varphi_{1}(b), \varphi_{2}(b)]\), can be reduced to a point,
    \item[\(\overset{\rightharpoonup}{C_3}\)] \(y = \varphi_{2}(x), x \in [a, b]\),
    \item[\(\overset{\rightharpoonup}{C_4}\)] \(x = a, y \in [\varphi_{1}(a), \varphi_{2}(a)]\), can be reduced to a point.
\end{description}
The second region is similar.
\begin{figure}[h]
    \centering
    \includegraphics[width=0.8\textwidth]{img/SpecialRegion1.png}
    \caption{Two special orientated closed regions.}
    \label{fig:SpecialRegion1}
\end{figure}

Denote \(\oint_{\overset{\rightharpoonup}{\partial M}}\) as the line integral 
along the boundary of region \(\overset{\rightharpoonup}{M}\), then we have the following lemma.
\begin{lemma}
    \begin{enumerate}
        \item Let \(\overset{\rightharpoonup}{\partial M}\) be the first region in Fig~\ref{fig:SpecialRegion1},
            \(P(x, y) \in C^{1}(M)\), then:
            \[
            \oint_{\overset{\rightharpoonup}{\partial M}} P \, \mathrm{d}x = -\iint_{\overset{\rightharpoonup}{M}} \frac{\partial P}{\partial y} \, \mathrm{d}x \wedge \mathrm{d}y,
            \] 
        \item Let \(\overset{\rightharpoonup}{\partial M}\) be the second region in Fig~\ref{fig:SpecialRegion1},
            \(Q(x, y) \in C^{1}(M)\), then:
            \[
            \oint_{\overset{\rightharpoonup}{\partial M}} Q \, \mathrm{d}y = \iint_{\overset{\rightharpoonup}{M}} \frac{\partial Q}{\partial x} \, \mathrm{d}x \wedge \mathrm{d}y.
            \]
    \end{enumerate}
\end{lemma}

\begin{theorem}{Green's Theorem}
    Let \(\overset{\rightharpoonup}{M}\) be an orientated closed region in \(\mathbb{R}^{2}\),
    and \(\omega = P\mathrm{d}x + Q \mathrm{d}y \in C^{1}(M)\).
    If \(\overset{\rightharpoonup}{\partial M}\) can be split into finitely many first and second regions 
    in Fig~\ref{fig:SpecialRegion1} simultaneously (non-overlapping, no shared interior points),
    then:
    \[
    \oint_{\overset{\rightharpoonup}{\partial M}} P \, \mathrm{d}x + Q \, \mathrm{d}y = 
    \iint_{\overset{\rightharpoonup}{M}} \left( \frac{\partial Q}{\partial x} - 
    \frac{\partial P}{\partial y} \right) \, \mathrm{d}x \wedge \mathrm{d}y =
        \iint_{M} \left( \frac{\partial Q}{\partial x} - 
    \frac{\partial P}{\partial y} \right) \, \mathrm{d}x \mathrm{d}y,
    \]\footnote{
        Note that \(\mathrm{d}x \wedge \mathrm{d}y \) is directed area element, 
        while \(\mathrm{d}x \mathrm{d}y\) is unsigned area element.
    }
    or equivalently,
    \[
    \oint_{\overset{\rightharpoonup}{\partial M}} \omega = \iint_{\overset{\rightharpoonup}{M}} \mathrm{d}\omega,
    \]
    where \(\overset{\rightharpoonup}{\partial M}\) is the induced orientation of \(\overset{\rightharpoonup}{M}\).
\end{theorem}


\begin{leftbarTitle}{Gauß's Formula}\end{leftbarTitle}
Consider three kinds of special orientated closed surfaces in \(\mathbb{R}^{3}\) as shown in Figure \ref{fig:SpecialRegion2}.
As for the first surface \(\overset{\rightharpoonup}{M}\)
($\overset{\rightharpoonup}{M}$ adopts a positive orientation (right-hand system), 
and $\overset{\rightharpoonup}{\partial M}$ adopts the outward normal orientation), 
it consists of three orientated surfaces:
\begin{description}
    \item[\(\overset{\rightharpoonup}{\Sigma_1}\)] \(z = \varphi_{1}(x, y), (x, y) \in \Delta_{1}\),
    \item[\(\overset{\rightharpoonup}{\Sigma_2}\)] \(z = \varphi_{2}(x, y), (x, y) \in \Delta_{1}\),
    \item[\(\overset{\rightharpoonup}{\Sigma_3}\)] A cylindrical taking \(\partial \Delta_1\) as the directrix,
        with the generatrix paralleling to the \(Oz\)-axis. 
        Of course, it can also be reduced as a closed curve.
\end{description}
The second and third surfaces are similar.
\begin{figure}[h]
    \centering
    \includegraphics[width=0.8\textwidth]{img/SpecialRegion2.png}
    \caption{Three special orientated closed surfaces (only the first two are shown).}
    \label{fig:SpecialRegion2}
\end{figure}

Denote \(\oiint_{\overset{\rightharpoonup}{\partial M}}\) as the surface integral
over the boundary of region \(\overset{\rightharpoonup}{M}\), then we have the following lemma.
\begin{lemma}
    \begin{enumerate}
        \item Let \(\overset{\rightharpoonup}{\partial M}\) be the first surface in Fig~\ref{fig:SpecialRegion2},
            \(R(x, y, z) \in C^{1}(M)\), then:
            \[
            \oiint_{\overset{\rightharpoonup}{\partial M}} R \, \mathrm{d}x \wedge \mathrm{d}y = 
            \iiint_{\overset{\rightharpoonup}{M}} \frac{\partial R}{\partial z} \, \mathrm{d}x \wedge \mathrm{d}y \wedge \mathrm{d}z,
            \] 
        \item Let \(\overset{\rightharpoonup}{\partial M}\) be the second surface in Fig~\ref{fig:SpecialRegion2},
            \(P(x, y, z) \in C^{1}(M)\), then:
            \[
            \oiint_{\overset{\rightharpoonup}{\partial M}} P \, \mathrm{d}y \wedge \mathrm{d}z = 
            \iiint_{\overset{\rightharpoonup}{M}} \frac{\partial P}{\partial x} \, \mathrm{d}x \wedge \mathrm{d}y \wedge \mathrm{d}z,
            \]
        \item Let \(\overset{\rightharpoonup}{\partial M}\) be the third surface in Fig~\ref{fig:SpecialRegion2},
            \(Q(x, y, z) \in C^{1}(M)\), then:
            \[
            \oiint_{\overset{\rightharpoonup}{\partial M}} Q \, \mathrm{d}z \wedge \mathrm{d}x = 
            \iiint_{\overset{\rightharpoonup}{M}} \frac{\partial Q}{\partial y} \, \mathrm{d}x \wedge \mathrm{d}y \wedge \mathrm{d}z.
            \]
    \end{enumerate}
\end{lemma}

\begin{theorem}{Gauß's Theorem}
    Let \(\overset{\rightharpoonup}{M}\) be an orientated closed region in \(\mathbb{R}^{3}\),
    and \(\omega = P\mathrm{d}y \wedge \mathrm{d}z + Q \mathrm{d}z \wedge \mathrm{d}x 
    + R \mathrm{d}x \wedge \mathrm{d}y \in C^{1}(M)\).
    If \(\overset{\rightharpoonup}{\partial M}\) can be split into finitely many first, second and third regions
    in Fig~\ref{fig:SpecialRegion1} simultaneously (non-overlapping, no shared interior points),
    then:
    then:
    \[
    \oiint_{\overset{\rightharpoonup}{\partial M}} P \, \mathrm{d}y \wedge \mathrm{d}z + Q \, \mathrm{d}z \wedge \mathrm{d}x + R \, \mathrm{d}x \wedge \mathrm{d}y = 
    \iiint_{\overset{\rightharpoonup}{M}} \left( 
        \frac{\partial P}{\partial x} + 
        \frac{\partial Q}{\partial y} + 
        \frac{\partial R}{\partial z} 
    \right) \, \mathrm{d}x \wedge \mathrm{d}y \wedge \mathrm{d}z,
    \]
    or equivalently,
    \[
    \oiint_{\overset{\rightharpoonup}{\partial M}} \omega = 
    \iiint_{\overset{\rightharpoonup}{M}} \mathrm{d}\omega,
    \]
    where \(\overset{\rightharpoonup}{\partial M}\) is the induced orientation of \(\overset{\rightharpoonup}{M}\).
\end{theorem}


\begin{leftbarTitle}{Stokes' Formula}\end{leftbarTitle}
\begin{theorem}{Stokes' Theorem}
    Let \(\overset{\rightharpoonup}{M}\) be an orientated smooth surface in \(\mathbb{R}^{3}\)
    with boundary \(\overset{\rightharpoonup}{\partial M}\),
    and \(\omega = P\mathrm{d}x + Q \mathrm{d}y + R \mathrm{d}z \in C^{1}(\Sigma)\).
    Then:
    \begin{align*}
        &\oint_{\overset{\rightharpoonup}{\partial M}} P \, \mathrm{d}x + Q \, \mathrm{d}y + R \, \mathrm{d}z\\
        =& \iint_{\overset{\rightharpoonup}{M}} \left( 
            \frac{\partial R}{\partial y} - \frac{\partial Q}{\partial z}
        \right) \, \mathrm{d}y \wedge \mathrm{d}z +
        \left( 
            \frac{\partial P}{\partial z} - \frac{\partial R}{\partial x}
        \right) \, \mathrm{d}z \wedge \mathrm{d}x +
        \left(
            \frac{\partial Q}{\partial x} - \frac{\partial P}{\partial y}
        \right) \, \mathrm{d}x \wedge \mathrm{d}y \\
        =& \iint_{\overset{\rightharpoonup}{M}} 
            \begin{vmatrix}\mathrm{d}y \wedge \mathrm{d}z&\mathrm{d}z \wedge \mathrm{d}x& \mathrm{d}x \wedge \mathrm{d}y\\
            \frac{\partial }{\partial x}&\frac{\partial }{\partial y}&\frac{\partial }{\partial z}\\
            P&Q&R            
            \end{vmatrix}\\
        =&\iint_{\overset{\rightharpoonup}{M}} 
        \begin{vmatrix}\cos\alpha&\cos\beta&\cos\gamma \\
        \frac{\partial }{\partial x}&\frac{\partial }{\partial y}&\frac{\partial }{\partial z}\\
        P&Q&R
        \end{vmatrix} \mathrm{d}S,
    \end{align*}
    or equivalently,
    \[
    \oint_{\overset{\rightharpoonup}{\partial M}} \omega = \iint_{\overset{\rightharpoonup}{M}} \mathrm{d}\omega,
    \]
    where \(\overset{\rightharpoonup}{\partial M}\) is the induced orientation of \(\overset{\rightharpoonup}{M}\).
\end{theorem}

\section{Closed and Exact Differential Forms}
\begin{definition}{Closed and Exact Differential Forms}
    Let \(U \subset \mathbb{R}^{n}\) be an open set and \(\omega\) be a \(C^{r}(r\geqslant 1)\) \(k\)-form on \(U\).
    \begin{enumerate}
        \item If \(\mathrm{d}\omega = 0\), then \(\omega\) is called a \textbf{closed form}.
        \item If there exists a \(C^{r+1}\) \((k-1)\)-form \(\eta\) such that \(\omega = \mathrm{d}\eta\),
            then \(\omega\) is called an \textbf{exact differential form}.
    \end{enumerate}
\end{definition}

\begin{theorem}{Necessary Condition for Exactness}
    Let \(U \subset \mathbb{R}^{n}\) be an open set and \(\omega\) be a \(C^{1}\) \(k\)-form on \(U\).
    If \(\omega\) is exact, then \(\omega\) is closed.
    The converse is not necessarily true.
\end{theorem}

\vspace{0.7cm}
We only discuss the case of \(1\)-forms in \(\mathbb{R}^{2}\) below.

Let \(\omega = P(x, y) \mathrm{d}x + Q(x, y) \mathrm{d}y\) be a \(C^{1}\) \(1\)-form on an open set \(U \subset \mathbb{R}^{2}\).
For any points \(A, B \in U\), a piecewise smooth simple closed curve on \(U\)
is called a \textbf{path} from \(A\) to \(B\) if it starts at \(A\) and ends at \(B\). 

For any path \(\overset{\rightharpoonup}{L}\) from \(A\) to \(B\),
if 
\[
\int_{\overset{\rightharpoonup}{L}} \omega = \int_{A}^{B} \omega,
\]
where the right-hand side is independent of the choice of path \(\overset{\rightharpoonup}{L}\),
then the line integral of \(\omega\) is said to be \textbf{path-independent} on \(U\).

\begin{theorem}
    Let \(U\in \mathbb{R}^{2}\) is a simply connected open region,
    and \(\omega = P(x, y) \mathrm{d}x + Q(x, y) \mathrm{d}y\) be a \(C^{1}\) \(1\)-form on \(U\).
    Then the following statements are equivalent:
    \begin{enumerate}[label=(\roman*)]
        \item \(\omega\) is exact on \(U\), i.e., there exists a \(C^{2}\) function \(F(x, y)\) on \(U\),
            such that
            \[
            \mathrm{d}F = \omega = P \, \mathrm{d}x + Q \, \mathrm{d}y.
            \]
            At this time, \(F(x, y)\) is called a \textbf{potential function} of \(\omega\) on \(U\)
            and 
            \[
            F(x, y) = \int_{(x_{0}, y_{0})}^{(x, y)} \omega + C
            = \int_{x_{0}}^{x} P(t, y_{0}) \, \mathrm{d}t + \int_{y_{0}}^{y} Q(x, s) \, \mathrm{d}s + C,
            \]
            where \((x_{0}, y_{0})\) is a fixed point in \(U\) and \(C\) is an arbitrary constant.
        \item \(\omega\) is closed on \(U\), i.e.,
            \[
                \frac{\partial P}{\partial y} = \frac{\partial Q}{\partial x}.
            \] 
        \item The line integral of \(\omega\) is path-independent on \(U\).
        \item For any piecewise smooth simple closed curve \(\overset{\rightharpoonup}{L}\) on \(U\),
            \[
            \oint_{\overset{\rightharpoonup}{L}} \omega = 0.
            \]
    \end{enumerate}
\end{theorem}


\begin{example}
    Calculate
    \[
    I = \oint_{\overset{\rightharpoonup}{C}} \frac{\cos(\mathbf{r},\mathbf{n})}{r} \, \mathrm{d}s,
    \]
    where \(\overset{\rightharpoonup}{C}\) is piecewise smooth simple closed curve,
    \(\mathbf{r}=(x,y)\), \(r=\|\mathbf{r}\|=\sqrt{x^2 + y^2}\),
    and \(\mathbf{n}\) is the unit outward normal vector of \(\overset{\rightharpoonup}{C}\).
\end{example} % 曲线积分与曲面积分
\chapter{Integrals with Variable Parameters} % 变参积分
\section{Definite Integrals with Variable Parameters}
\begin{definition}{Definite Integral with Variable Parameters}
    Let \(f(x, y)\) be defined on \([a, b] \times [c, d]\).
    For each fixed \(y \in [c, d]\), if the definite integral
    \[
    I(y) = \int_{a}^{b} f(x, y) \, \mathrm{d}x
    \]
    exists, then \(I(y)\) is called a \textbf{definite integral with variable parameter \(y\)}.
\end{definition}

\section{Elliptic Integrals} % 椭圆积分

\section{Improper Integrals with Variable Parameters}
There are two types of improper integrals with variable parameters: 
on infinite interval and with unbounded integrand.
Here we only give the definition of improper integrals on infinite interval with variable parameters.

\begin{definition}{Improper Integral with Variable Parameters}
    Let \(f(x, y)\) be defined on \([a, +\infty) \times [c, d]\).
    For some fixed \(y_{0} \in [c, d]\), if the improper integral
    \(
    I(y_{0}) = \int_{a}^{+\infty} f(x, y_{0}) \, \mathrm{d}x
    \)
    converges, then \(\int_{a}^{+\infty} f(x, y) \, \mathrm{d}x\) is called convergent at \(y_{0}\),
    and \(y_{0}\) is called its convergence point.
    
    Let the set of all convergence points be \(E\),
    then \(E\) is the domain of definition of the improper integral with variable parameters
    \[
    I(y) = \int_{a}^{+\infty} f(x, y) \, \mathrm{d}x,
    \]
    also called the convergence domain of the improper integral \(\int_{a}^{+\infty} f(x, y) \, \mathrm{d}x\).
\end{definition}

\begin{leftbarTitle}{Uniform Convergence and Its Tests}\end{leftbarTitle}
\begin{definition}{Uniform Convergence of Improper Integrals with Variable Parameters}
    Let \(f(x, y)\) be defined on \([a, +\infty) \times [c, d]\),
    where \([c, d]\) is the convergence domain of the improper integral
    \(\int_{a}^{+\infty} f(x, y) \, \mathrm{d}x\).
    If for every \(\varepsilon > 0\), there exists a number \(A_{0} > a\) independent of \(y\),
    such that for all \(A > A_{0}\) and for all \(y \in [c, d]\),
    \[
    \left| \int_{a}^{A} f(x, y) \, \mathrm{d}x - I(y) \right| = \left| \int_{A}^{+\infty} f(x, y) \, \mathrm{d}x \right| < \varepsilon,
    \]
    then the improper integral \(\int_{a}^{+\infty} f(x, y) \, \mathrm{d}x\)
    is said to be \textbf{uniformly convergent} on \([c, d]\).
\end{definition}

\begin{theorem}{Cauchy Criterion for Uniform Convergence of Improper Integrals with Variable Parameters}
    Let \(f(x, y)\) be defined on \([a, +\infty) \times [c, d]\),
    where \([c, d]\) is the convergence domain of the improper integral
    \(\int_{a}^{+\infty} f(x, y) \, \mathrm{d}x\).
    The improper integral \(\int_{a}^{+\infty} f(x, y) \, \mathrm{d}x\)
    is uniformly convergent on \([c, d]\) if and only if
    for every \(\varepsilon > 0\), there exists a number \(A_{0} > a\) independent of \(y\),
    such that for all \(A_{1}, A_{2} > A_{0}\) and for all \(y \in [c, d]\),
    \[
    \left| \int_{A_{1}}^{A_{2}} f(x, y) \, \mathrm{d}x \right| < \varepsilon.
    \]
\end{theorem}

\section{Analysis Properties of Uniform Convergence} % 一致收敛的分析性质 
\begin{lemma}
    
\end{lemma}

\begin{theorem}{Uniform Convergence and Continuity}
    Let \(f(x, y)\) be continuous on \([a, +\infty) \times [c, d]\),
    and \(\int_{a}^{+\infty} f(x, y) \, \mathrm{d}x\)
    is uniformly convergent on \([c, d]\) with respect to \(y\), then:
    \begin{enumerate}[label=(\roman*)]
        \item 
        \[
        I(y) = \int_{a}^{+\infty} f(x, y) \, \mathrm{d}x
        \] 
        is continuous on \([c, d]\), i.e.,
        \[
        \lim_{y\to y_0} \int_{a}^{+\infty} f(x, y) \, \mathrm{d}x
        = \int_{a}^{+\infty} \lim_{y\to y_0} f(x, y) \, \mathrm{d}x,
        \quad y_0 \in [c, d],
        \]
        that is, the limit and the integral can be interchanged.
        \item 
        \[
        \int_{c}^{d} \mathrm{d}y \int_{a}^{+\infty} f(x, y) \, \mathrm{d}x
        = \int_{a}^{+\infty} \mathrm{d}x \int_{c}^{d} f(x, y) \, \mathrm{d}y,
        \]
        that is, the order of integration can be interchanged.
    \end{enumerate}
\end{theorem}

When \([c, d]\) is replaced by \([c, +\infty)\), the above theorem fails,
but we have the following theorem.
\begin{theorem}
    On the region \(D = [a, +\infty) \times [c, +\infty)\),
    \begin{enumerate}
        % 无穷积分与无穷积分可交换
        \item if \(f(x, y)\) satisfies: 
        \begin{enumerate}
            \item \(f(x, y)\in C(D)\);
            \item \(\int_{a}^{+\infty} f(x, y) \, \mathrm{d}x\) internally closed uniformly converges with respect to \(y\);
                \(\int_{c}^{+\infty} f(x, y) \, \mathrm{d}y\) internally closed uniformly converges with respect to \(x\);
            \item One of the two integrals \(\int_{a}^{+\infty} \mathrm{d}x \int_{c}^{+\infty} |f(x, y)| \, \mathrm{d}y\)
                or \(\int_{c}^{+\infty} \mathrm{d}y \int_{a}^{+\infty} |f(x, y)| \, \mathrm{d}x\) converges;
        \end{enumerate}
        then
        \[
        \int_{c}^{+\infty} \, \mathrm{d}y\int_{a}^{+\infty} f(x, y) \, \mathrm{d}x = 
        \int_{a}^{+\infty} \, \mathrm{d}x\int_{c}^{+\infty} f(x, y) \, \mathrm{d}y
        \]
        % 非负函数的无穷积分与无穷积分可交换
        \item if \(f(x, y)\) satisfies:
        \begin{enumerate}
            \item \(f(x, y)\in C(D)\) and \(f(x)\geqslant 0\) on \(D\);
            \item \(\int_{a}^{+\infty} f(x, y) \, \mathrm{d}x \in C[c, +\infty)\);
                \(\int_{c}^{+\infty} f(x, y) \, \mathrm{d}y \in C[a,+\infty)\);
            \item One of the two integrals \(\int_{a}^{+\infty} \mathrm{d}x \int_{c}^{+\infty} f(x, y) \, \mathrm{d}y\)
                or \(\int_{c}^{+\infty} \mathrm{d}y \int_{a}^{+\infty} f(x, y) \, \mathrm{d}x\) converges;
        \end{enumerate}
        then
        \[
        \int_{c}^{+\infty} \, \mathrm{d}y\int_{a}^{+\infty} f(x, y) \, \mathrm{d}x = 
        \int_{a}^{+\infty} \, \mathrm{d}x\int_{c}^{+\infty} f(x, y) \, \mathrm{d}y
        \]
    \end{enumerate}
\end{theorem}

\begin{remark}
    One of the two integrals exists implies the other exists as well as the equality holds.
\end{remark}


\vspace{0.7cm}
\begin{theorem}{Uniform Convergence and Differentiation}
    On the region \(D = [a, +\infty] \times [c, d]\), if the following conditions are satisfied:
    \begin{enumerate}[label=(\roman*)]
        \item \(\frac{\partial }{\partial y}f(x, y) \in C(D)\);
        \item \(\int_{a}^{+\infty} \frac{\partial }{\partial y}f(x, y) \, \mathrm{d}x\) converges uniformly 
            with respect to \(y\) on \([c, d]\);
        \item There exists a point \(y_{0} \in [c, d]\), such that \(\int_{a}^{+\infty} f(x, y_{0}) \, \mathrm{d}x\) converges;
        \item For any \([\alpha, \beta]\subset [a, +\infty)\), \(\int_{\alpha}^{\beta} f(x, y) \, \mathrm{d}x\) exists.
    \end{enumerate}
    Then \(I(y) = \int_{a}^{+\infty} f(x, y) \, \mathrm{d}x\) is differentiable on \([c, d]\), and
    \[
    \frac{\mathrm{d}}{\mathrm{d}y} \int_{a}^{+\infty} f(x, y) \, \mathrm{d}x =
    \int_{a}^{+\infty} \frac{\partial }{\partial y} f(x, y) \, \mathrm{d}x.
    \]
\end{theorem}

\begin{example}
    Let 
    \[
    F(a) = \int_{0}^{+\infty} \frac{1}{t}(1-e^{-at})\cos bt \, \mathrm{d}t, \quad b\neq 0.
    \]
    \begin{enumerate}
        \item Prove that \(F(a)\in C[0,+\infty)\cap D(0,+\infty)\) 
        \item Find the expression of \(F(a)\).
    \end{enumerate}
\end{example}

\begin{leftbarTitle}{Imbedding Method}\end{leftbarTitle} % 嵌入法
If \(I=\int_{a}^{b}f(x)\mathrm{d}x\) is difficult to calculate directly,
we can introduce a parameter \(y\) and consider the integral
\[
I(y)=\int_{a}^{b}f(x,y)\mathrm{d}x,
\]
and let \(I=I(y_0)\) for some specific \(y_0\).
If we can calculate \(I(y)\) and then take \(y=y_0\),
then we can get the value of \(I\).
This method is called \textbf{imbedding method}.

\begin{example}
    Compute the integral:
    \[
    I = \int_{0}^{1} \frac{\ln(1+x)}{1+x^{2}} \, \mathrm{d}x.
    \]
\end{example}

\begin{example}
    Compute Dirichlet's integral:
    \[
    I = \int_{0}^{+\infty} \frac{\sin x}{x} \, \mathrm{d}x.
    \]
\end{example}
\begin{solution}

\end{solution}


\section{Euler Integrals}
\begin{leftbarTitle}{Beta Function}\end{leftbarTitle}
% 两种形式的 Beta 函数
Beta function can be defined in the following equivalent forms:
\begin{enumerate}
    \item For \(p>0, q>0\):
    \[
    B(p, q) = \int_{0}^{1} t^{p-1} (1-t)^{q-1} \, \mathrm{d}t.
    \]
    \item Via substitution \(t = \frac{u}{1+u}\):
    \[
    B(p, q) = \int_{0}^{+\infty} \frac{u^{p-1}}{(1+u)^{p+q}} \, \mathrm{d}u = 
    \int_{0}^{+\infty} \frac{u^{q-1}}{(1+u)^{p+q}} \, \mathrm{d}u.
    \]
    \item Via substitution \(t = \sin^{2}\theta\):
    \[
    B(p, q) = 2\int_{0}^{\frac{\pi}{2}} \sin^{2p-1}\theta \cos^{2q-1}\theta \, \mathrm{d}\theta.
    \]
    Then we have:
    \[
    B(\frac{1}{2}, \frac{1}{2}) = \pi, \quad B(\frac{3}{2}, \frac{1}{2}) = \frac{\pi}{2}, \quad B(1, 1) = 1.
    \]
\end{enumerate}

\begin{property}
    \begin{description}
        \item[Continuity] \(B(p, q)\in C(U)\), where \(U = \{(p, q) | p>0, q>0\}\).
        \item[Symmetry] \(B(p, q) = B(q, p)\).
        \item[Recurrence Relation] \(B(p, q) = \frac{q-1}{p+q-1} B(p, q-1)\) for \(p>0,q>1\).
    \end{description}
\end{property}


\begin{leftbarTitle}{Gamma Function}\end{leftbarTitle}
Gamma function can be defined in the following equivalent forms:
\begin{enumerate}
    \item For \(s>0\):
    \[
    \Gamma(s) = \int_{0}^{+\infty} x^{s-1} e^{-x} \, \mathrm{d}x.
    \]
    \item Via the limit:
    \[
    \Gamma(s) = \lim_{n\to\infty} \frac{n!}{s(s+1)(s+2)\cdots(s+n)}.
    \]
    \item Via substitution \(x = t^{2}\):
    \[
    \Gamma(s) = \frac{1}{2} \int_{0}^{+\infty} t^{2s-1} e^{-t^{2}} \, \mathrm{d}t.
    \]
    Then we have:
    \[
    \Gamma(\frac{1}{2}) = \sqrt{\pi}, \quad \Gamma(\frac{3}{2}) = \frac{\sqrt{\pi}}{2}, \quad \Gamma(1) = 1.
    \]
\end{enumerate}


\begin{property}
    \begin{description}
        \item[Continuity] \(\Gamma(s)\in C(0, +\infty)\).
        \item[Recurrence Relation] \(\Gamma(s+1) = s \Gamma(s)\) for \(s>0\).
    \end{description}
\end{property}

Gamma function can be \underline{extended} to the whole complex plane except for non-positive integers,
where it has simple poles.



\begin{leftbarTitle}{Relation between Beta and Gamma Functions}\end{leftbarTitle}
\begin{theorem}
    There holds the following relation between Beta and Gamma functions:
    \[
    B(p, q) = \frac{\Gamma(p) \Gamma(q)}{\Gamma(p+q)}, \quad p>0, q>0.
    \]
\end{theorem}

Next, we give three important formulas about Gamma function,
which can be extended to the complex domain as well.
\begin{theorem}{Bohr-Mollerup Theorem}
    The Gamma function is the unique function defined on \((0, +\infty)\)
    satisfying the following three conditions:
    \begin{enumerate}[label=(\roman*)]
        \item \(f(x)>0\) and \(f(1) = 1\);
        \item \(f(x+1) = x f(x)\) for all \(x > 0\);
        \item \(\ln f(x)\) is convex on \((0, +\infty)\).
    \end{enumerate}
    
\end{theorem}

\begin{theorem}{Legendre's Duplication Formula}
    For \(s > 0\), there holds:
    \[
    \Gamma(s) \Gamma(s + \frac{1}{2}) = \frac{\sqrt{\pi}}{2^{2s-1}} \Gamma(2s).
    \]    
\end{theorem}


\begin{theorem}{Reflection Formula} % 余元公式
    For \(0 < s < 1\), there holds:
    \[
    \Gamma(s) \Gamma(1 - s) = \frac{\pi}{\sin \pi s}.
    \]
\end{theorem}

\begin{theorem}{Stirling's Formula}
    \[
    \Gamma(s+1) = \sqrt{2 \pi s} \left( \frac{s}{e} \right)^{s} \exp\left( \frac{\theta}{12s} \right),
    \]
    where \(0 < \theta < 1\).

    Specially, when \(s = n \in \mathbb{N}\),
    \[
    \Gamma(n+1) = n! = \sqrt{2 \pi n} \left( \frac{n}{e} \right)^{n} \exp\left( \frac{\theta}{12n} \right),
    \]
    where \(0 < \theta < 1\).
\end{theorem}

\begin{example}
    Prove the integral form of Riemann \(\zeta\) function:
    \[
    \zeta(s) = \sum_{n=1}^{\infty} \frac{1}{n^s} 
    = \frac{1}{\Gamma(s)} \int_{0}^{+\infty} \frac{x^{s-1}}{e^{x}-1} \, \mathrm{d}x, \quad s > 1.
    \]
\end{example} % 变参积分





\begin{thebibliography}{99} 
\bibitem{1} 徐森林, 薛春华. \emph{数学分析 (1st edition) }. 清华大学出版社, 2005.
\bibitem{2} 陈纪修, 於崇华. \emph{数学分析 (3rd edition) }. 高等教育出版社, 2019.
\bibitem{3} 常庚哲, 史济怀. \emph{数学分析教程 (3rd edition) }. 中国科学技术大学出版社, 2012.
\bibitem{4} 裴礼文. \emph{数学分析中的典型问题与方法 (3rd edition) }. 高等教育出版社, 2021.
\bibitem{5} 汪林. \emph{数学分析中的问题与反例 (1st edition) }. 高等教育出版社, 2015.
\bibitem{6} 谢惠民, 恽自求, 易法槐, 钱定边. \emph{数学分析习题课讲义 (2nd edition) }. 高等教育出版社, 2019.
\bibitem{7} Walter Rudin. \emph{Principles of Mathematical Analysis (3rd edition) }. McGraw-Hill, 1976.
\bibitem{8} 菲赫金哥尔茨. \emph{微积分学教程 (8th edition) }. 高等教育出版社, 2006.
\bibitem{9} Wikipedia. \url{https://en.wikipedia.org/wiki/}.
\end{thebibliography}

\end{document}