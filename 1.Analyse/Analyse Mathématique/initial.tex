\documentclass[11pt]{elegantbook}

\title{Analyse Mathématique} % 这里放置书名
% \subtitle{Subtitle} % 这里放置副标题

\author{CatMono} % 这里放置作者名
\date{July, 2025} % 这里放置日期
\version{0.1} % 这里放置版本号
% \institute{Elegant\LaTeX{} Program} % 这里放置机构名
% \bioinfo{Custom Key}{Custom Value} % 这里放置自定义信息

% \extrainfo{extra information} % 这里放置额外信息,将显示在最下方中央

\setcounter{tocdepth}{2} % 设置目录深度
\setcounter{secnumdepth}{2} % 设置章节编号深度


% \logo{logo-blue.png} % 这里放置封面logo,默认从figure目录下寻找
% \cover{LogiqueMathematique.png} % 这里放置封面图片,默认从figure目录下寻找

% modify the color in the middle of titlepage
\definecolor{customcolor}{RGB}{32,178,170} % 自定义颜色
\colorlet{coverlinecolor}{customcolor}
\usepackage{cprotect} % 保护命令参数不被 LaTeX 解析器过早处理,允许在某些特殊环境中使用脆弱命令(fragile commands)。
\usepackage{xeCJK} % 使用 xeCJK 包支持中文


% ===== 开始文档 =====
\begin{document}

\maketitle %生成文档的标题页,根据之前定义的标题信息(如标题、作者、日期等)自动创建一个格式化的标题页

% === 前言部分 ===
\frontmatter        % 开始前言,页码为 i, ii, iii...
\tableofcontents    % 目录 (页码: i, ii)
% \listoffigures      % 图表目录 (页码: iii)
% \listoftables       % 表格目录 (页码: iv)

\chapter{Preface}   % 前言章节(无编号,页码: v, vi...)
This is the preface of the book...

% \chapter{Acknowledgments}  % 致谢(无编号)
% I would like to thank...
% === 正文部分 ===
\mainmatter         % 开始正文,页码从 1 重新开始

\chapter{Preliminaries} % 这里放置章节标题
\section{Section Title} % 这里放置小节标题
\subsection{Subsection Title} % 这里放置子小节标题

\chapter{Limits of Sequences and Continuity of Real Number System} 
\section{Limits of Sequences}

\section{Criteria for Convergence}

\section{Substitution}

\section{Continuity of Real Number System}

\chapter{Limits and Continuity of Functions}
\section{Limits of Functions}

\section{Continuous Functions}

\section{Infinitesimal and Infinite Quantities}

\section{Continuous Functions on Closed Intervals}

\section{Period Three Implies Chaos}

\section{Functional Equations}

\chapter{Series of Numbers}

\chapter{Series of Functions}

\chapter{Power Series}

\chapter{Limits and Continuity in Euclidean Spaces}

\chapter{Multivariable Differential Calculus}
\section{Directional Derivatives and Total Differential}
\begin{definition}{Directional Derivative}
    Let \(U\subset \mathbb{R}^n\) be an open set, \(f: U\to \mathbb{R}^{1}\),
    \(\boldsymbol{e}\) is a unit vector in \(\mathbb{R}^{n}\), \(\boldsymbol{x}^{0}\in U\). Define
    \[
    u(t) = f(\boldsymbol{x}^{0} + t\boldsymbol{e}).
    \]
    If the derivative of \(u\) at \(t=0\) 
    \[ 
        u'(0) = \lim_{t \to 0} \frac{u(t) - u(0)}{t} = 
        \lim_{t \to 0} \frac{f(\boldsymbol{x}^{0} + t\boldsymbol{e}) - f(\boldsymbol{x}^{0})}{t} 
    \] 
    exists and is finite, 
    it is called the \textbf{directional derivative} of \(f\) at \(\boldsymbol{x}_{0}\) in the direction \(\boldsymbol{e}\), 
    denoted by \(\frac{\partial f}{\partial \boldsymbol{e}}(\boldsymbol{x}_{0})\). 
    It is the rate of change of \(f\) at \(\boldsymbol{x}_{0}\) in the direction \(\boldsymbol{e}\).
\end{definition}

Consider the following set of unit coordinate vectors: \(\boldsymbol{e}_{1},\boldsymbol{e}_{2},\cdots,\boldsymbol{e}_{n}\).
For a function \( f \), the directional derivative of \( f \) at the point \( \boldsymbol{x}_{0} \) in the direction of \( \boldsymbol{e}_{i} \) 
is called the \( i \)th first-order \textbf{partial derivative} of \( f \) at \(\boldsymbol{x}^{0}\), denoted by
\[
\frac{\partial f}{\partial x_i}(\boldsymbol{x}^{0}) \quad \text{or} \quad \mathrm{D}_i f(\boldsymbol{x}^{0}).
\]
\( \mathrm{D}_i = \frac{\partial}{\partial x_i} \) is called the \( i \)th partial differential operator (\( i = 1, 2, \cdots, n \)).

\begin{note}
    Let \(\boldsymbol{e}\) be a direction, then \(\|-\boldsymbol{e}\| = \|\boldsymbol{e}\| = 1\), 
    which implies that \(-\boldsymbol{e}\) is also a direction. At this point, we have:
    \[
    \frac{\partial f}{\partial (-\boldsymbol{e})}(\boldsymbol{x}^{0}) = -\frac{\partial f}{\partial \boldsymbol{e}}(\boldsymbol{x}^{0}).
    \]
\end{note}

If the first-order partial derivative of \(f\), \(\frac{\partial f}{\partial x_i}\), 
itself possesses partial derivatives, then the second-order partial derivative of \(f\) is defined, 
and is denoted as follows:
\[
f_{x_i x_j} = \frac{\partial^2 f}{\partial x_i \partial x_j} = \frac{\partial}{\partial x_j} \left( \frac{\partial f}{\partial x_i} \right), 
\quad f_{x_i x_i} = \frac{\partial^2 f}{\partial x_i^2} = \frac{\partial}{\partial x_i} \left( \frac{\partial f}{\partial x_i} \right), 
\quad i, j = 1, 2, \dots, n.
\]

Similarly, higher-order partial derivatives of order \(3,4,\cdots m,\cdots\) can be defined.

\begin{definition}{Jacobian Matrix (Gradient)}
    Let
    \[
    \boldsymbol{J}f(\boldsymbol{x}) = (\mathrm{D}_1 f(\boldsymbol{x}), \mathrm{D}_2 f(\boldsymbol{x}), \dots, \mathrm{D}_n f(\boldsymbol{x})),
    \]
    which is called the \textbf{Jacobian matrix} of the function \( f \) at the point \( \boldsymbol{x} \), 
    (a \( 1 \times n \) matrix) whose counterpart is the first-order derivative of a single-variable function.

    Henceforth, we represent the point \(\boldsymbol{x}\) in \( \mathbb{R}^n \) 
    and its increments \(\boldsymbol{h}\) as column vectors.
    In this way, the differential of the function can be expressed using matrix multiplication as follows:
    \[
    \mathrm{d}f(\boldsymbol{x}^{0})(\boldsymbol{\Delta x}) = \boldsymbol{J}f(\boldsymbol{x}^{0}) \boldsymbol{\Delta x}.
    \]
    The Jacobian matrix of the function \( f \) is also frequently denoted as 
    \(\boldsymbol{\mathrm{grad}}\,f\) (or \(\nabla f\)), that is,
    \[
    \boldsymbol{\mathrm{grad}}\,f(\boldsymbol{x}) = \boldsymbol{J}f(\boldsymbol{x}),
    \]
    which is called the \textbf{gradient} of the scalar function \( f \).
\end{definition}


\begin{definition}{Total Differential}
    Let \(U\subset \mathbb{R}^n\) be an open set, \(f: U\to \mathbb{R}^{1}\), \(\boldsymbol{x}^{0}\in U\),
    \(\boldsymbol{\Delta x}=\left( \Delta x_{1},\Delta x_{2},\cdots,\Delta x_{n} \right) \in \mathbb{R}^{n}\). If
    \[
    f(\boldsymbol{x}^{0} + \boldsymbol{\Delta x}) - f(\boldsymbol{x}^{0}) = 
    \sum_{i=1}^n A_{i} \Delta x_{i} + o(\|\boldsymbol{\Delta x}\|) \qquad (\|\boldsymbol{\Delta x}\| \to 0),
    \]
    where \(A_{1}, A_{2}, \dots, A_{n}\) are constants independent of \(\boldsymbol{\Delta x}\), 
    then the function \(f\) is said to be \textbf{differentiable} at the point \(\boldsymbol{x}^{0}\), 
    and the linear main part \(\sum_{i=1}^n A_{i} \Delta x_{i}\) is called the \textbf{total differential} 
    of \(f\) at \(\boldsymbol{x}^{0}\), 
    denoted as
    \[
    df(\boldsymbol{x}^{0})(\boldsymbol{\Delta x}) = \sum_{i=1}^n A_{i} \Delta x_{i}.
    \]
    If \(f\) is differentiable at every point in the open set \(U\), 
    then \(f\) is called a differentiable function on \(U\).    
\end{definition}

\begin{theorem}{Conditions of Differentiability}
    \begin{description}
        \item[Necessary Condition] If an \(n\)-variable function \(f\) is differentiable at the point \(\boldsymbol{x}_{0}\), 
        then \(f\) is continuous at \(\boldsymbol{x}^{0}\) and 
        possesses first-order partial derivatives \(\frac{\partial f}{\partial x_{i}}(\boldsymbol{x}^{0})\) 
        at \(\boldsymbol{x}^{0}\) for \(i = 1, 2, \dots, n\), and
        \[
        \boldsymbol{A} = \left( A_{1}, A_{2}, \dots, A_{n} \right)  = \boldsymbol{J}f(\boldsymbol{x}^{0}) = 
        \left(\mathrm{D}_{1}f(\boldsymbol{x}^{0}), \mathrm{D}_{2}f(\boldsymbol{x}^{0}), \dots, \mathrm{D}_{n}f(\boldsymbol{x}^{0}) \right).
        \]\footnote{
            It is referred to as the total differential formula, and the more common form is
            \[
                \mathrm{d}f(x_{0},y_{0})=
                \frac{\partial f}{\partial x}(x_{0},y_{0})\,\mathrm{d}x+\frac{\partial f}{\partial y}(x_{0},y_{0})\,\mathrm{d}y.
            \]
        }
        However, the converse is not true.
        \item[Sufficient Condition] Let \(U \subset \mathbb{R}^n\) be an open set, 
        and let \(f: U \to \mathbb{R}^1\) be an \(n\)-variable function. If \(\boldsymbol{J}f = \left( \mathrm{D}_{1}f, \mathrm{D}_{2}f, \dots, \mathrm{D}_{n}f \right)\) 
        is continuous at \(\boldsymbol{x}^{0}\) 
        (i.e., \(\frac{\partial f}{\partial x_{i}}\) is continuous at \(\boldsymbol{x}^{0}\) for \(i = 1, 2, \dots, n\)), 
        then \(f\) is differentiable at \(\boldsymbol{x}^{0}\). 
        However, the converse is not necessarily true.
    \end{description}    
\end{theorem}

\begin{note}{(At some point)}
    \begin{enumerate}
        \item  Differentiable
            \begin{itemize}
                \item \(\implies\) Continuous
                \item \(\implies\) Partial derivatives exist: \(\mathrm{D}_{\vec{u}}=\nabla f\cdot\vec{u}\)
            \end{itemize}
        \item  Directional Derivative
            \begin{itemize}
                \item All directional derivatives exist \(\not\implies\) differentiable or continuous.
                \item All directional derivatives exist and are equal \(\not\implies\) differentiable.
            \end{itemize}
        \item  Partial Derivative
        \begin{itemize}
            \item  The continuity and existence of directional/partial derivatives are mutually exclusive.
        \end{itemize}
    \end{enumerate}
\end{note}




\chapter{Multiple Integrals}




\begin{thebibliography}{99} 
\bibitem{1} 徐森林, 薛春华. \emph{数学分析}. 第一版. 清华大学出版社, 2005.
\bibitem{2} 陈纪修, 於崇华. \emph{数学分析}. 第三版. 高等教育出版社, 2019.
\bibitem{3} 常庚哲, 史济怀. \emph{数学分析教程}. 第三版. 中国科学技术大学出版社, 2012.
\bibitem{4} 裴礼文. \emph{数学分析中的典型问题与方法}. 第三版. 高等教育出版社, 2021.
\bibitem{5} 汪林. \emph{数学分析中的问题与反例}. 第一版. 高等教育出版社, 2015.
\bibitem{6} 谢惠民, 恽自求, 易法槐, 钱定边. \emph{数学分析习题课讲义}. 第二版. 高等教育出版社, 2019.
\bibitem{7} Walter Rudin. \emph{Principles of Mathematical Analysis}. Third Edition. McGraw-Hill, 1976.
\bibitem{8} 菲赫金哥尔茨. \emph{微积分学教程}. 第八版. 高等教育出版社, 2006.
\end{thebibliography}

\end{document}