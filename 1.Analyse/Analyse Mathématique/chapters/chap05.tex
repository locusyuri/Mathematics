\chapter{Indefinite Integral} % 不定积分

\section{Two Common Integration Methods}
\begin{leftbarTitle}{Integration Methods}\end{leftbarTitle}
\begin{definition}{Integration by Parts}
    Let \( u(x) \) and \( v(x) \) be two differentiable functions,
    and at least one of them has an antiderivative.
    Then the integration by parts formula states that:
    \[
        \int u \, \mathrm{d}v = uv - \int v \, \mathrm{d}u.
    \]
\end{definition}

\begin{definition}{Substitution Method}
\end{definition}

Some common substitutions are as follows:
\begin{description}
    \item[Trigonometric Substitution] When restoring variables, auxiliary right triangles is often utilized.
        \begin{description}
            \item[Sine] \( \sqrt{a^2 - x^2} \): \( x = a \sin t \) or \( x = a \cos t \)
            \item[Tangent] \( \sqrt{a^2 + x^2} \): \( x = a \tan t \) or \( x = a \sinh t \)
            \item[Secant] \( \sqrt{x^2 - a^2} \): \( x = a \sec t \) or \( x = a \cosh t \)
        \end{description}
    \item[Irreational Substitution] 
        \begin{itemize}
            \item If the integrand contains \( \sqrt[n]{x} \),
                one can use the substitution \( t = \sqrt[n]{x} \) to simplify the expression.
            \item If the integrand contains \( \sqrt[n]{\frac{\alpha x + \beta}{\gamma x + \delta}} \),
                one can use the substitution \( t = \sqrt[n]{\frac{\alpha x + \beta}{\gamma x + \delta}} \) to simplify the expression.
        \end{itemize}
    \item[Reciprocal Substitution] If the degree of the numerator is lower than that of the denominator according to \(x\) 
        one can use the substitution \( x = \frac{1}{t} \) to reduce the degree.
\end{description}

\begin{leftbarTitle}{Basic Integration Formulas}\end{leftbarTitle}
\[
\begin{array}{|c|c|}
\hline
\textbf{Integral} & \textbf{Result} \\
\hline
\int a \, \mathrm{d}x & ax + C \quad (a \text{ is constant}) \\
\hline
\int x^n \, \mathrm{d}x & \frac{x^{n+1}}{n+1} + C \quad (n \neq -1) \\
\hline
\int \frac{1}{x} \, \mathrm{d}x & \ln|x| + C \\
\hline
\int e^x \, \mathrm{d}x & e^x + C \\
\hline
\int a^x \, \mathrm{d}x & \frac{a^x}{\ln a} + C \quad (a > 0, a \neq 1) \\ \hline
\rowcolor{gray!30} & \\ \hline
\int \sin x \, \mathrm{d}x & -\cos x + C \\ \hline
\int \cos x \, \mathrm{d}x & \sin x + C \\ \hline
\int \tan x \, \mathrm{d}x & -\ln|\cos x| + C \\ \hline
\int \cot x \, \mathrm{d}x & \ln|\sin x| + C \\ \hline
\int \sec x \, \mathrm{d}x & \ln|\sec x + \tan x| + C \\
\hline
\int \csc x \, \mathrm{d}x & \ln|\csc x - \cot x| + C \\
\hline
\int \sec x \tan x \, \mathrm{d}x & \sec x + C \\
\hline
\int \csc x \cot x \, \mathrm{d}x & -\csc x + C \\
\hline
\int \sec^2 x \, \mathrm{d}x & \tan x + C \\
\hline
\int \csc^2 x \, \mathrm{d}x & -\cot x + C \\ \hline
\rowcolor{gray!30} & \\ \hline
\int \frac{1}{\sqrt{a^{2}-x^2}} \, \mathrm{d}x & \arcsin \left( \frac{x}{a} \right) + C \\
\hline
\int \frac{-1}{\sqrt{a^{2}-x^2}} \, \mathrm{d}x & \arccos \left( \frac{x}{a} \right) + C \\
\hline
\int \frac{1}{a^{2}+x^2} \, \mathrm{d}x & \frac{1}{a}\arctan \left( \frac{x}{a} \right) + C \\
\hline
\int \frac{-1}{a^{2}+x^2} \, \mathrm{d}x &\frac{1}{a}\mathrm{arccot } \left( \frac{x}{a} \right) + C \\
\hline
\int \frac{1}{\sqrt{x^2+a^{2}}} \, \mathrm{d}x & \ln|x + \sqrt{x^2+a^{2}}| + C \\
\hline
\int \frac{1}{\sqrt{x^2-a^{2}}} \, \mathrm{d}x & \ln|x + \sqrt{x^2-a^{2}}| + C \quad (x > a \text{ or } x < -a) \\ \hline
\rowcolor{gray!30} & \\ \hline
\int \sinh x \, \mathrm{d}x & \cosh x + C \\
\hline
\int \cosh x \, \mathrm{d}x & \sinh x + C \\
\hline
\end{array}
\]
