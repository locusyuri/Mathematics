\chapter{Multiple Integrals} % 多重积分
\section{Multiple Integrals on Bounded Closed Regions}
\underline{How to define a region with measurable area?} % 怎么定义可求面积的区域?
Generally speaking, there are two approaches to define regions with measurable area:
\begin{enumerate}
    \item Consider the integral over a closed rectangle, 
        and then extend it to a bounded closed region within the rectangle
        with the help of characteristic functions;
    \item Define that a bounded closed region \(D\) is measurable if 
        \(\forall \varepsilon > 0\), there exist two polygonal regions \(\Sigma_1\) and \(\Sigma_2\)
        consisting of finite rectangles, such that \(\Sigma_{1}\subset D \subset \Sigma_{2}\) 
        and the area of \(\Sigma_2 \setminus \Sigma_1\) is less than \(\varepsilon\).
\end{enumerate}

\begin{leftbarTitle}{Definition of Multiple Integral}\end{leftbarTitle}
Here, we introduce the definition of double integrals using the first approach.

Initially, we define the double integral on a closed interval (rectangle).
\begin{definition}{Double Integral on a Closed Interval}
    Let \( I = [a, b] \times [c, d] \) be a closed interval in \( \mathbb{R}^2 \), 
    (i.e., each boundary is parallel to the coordinate axes). Partition \( [a, b] \):
    \[
    T_x: a = x_0 < x_1 < \cdots < x_n = b.
    \]
    Partition \( [c, d] \):
    \[
    T_y: c = y_0 < y_1 < \cdots < y_m = d.
    \]
    Two sets of parallel lines \( x = x_i \, (i = 0, 1, \ldots, n) \) and \( y = y_j \, (j = 0, 1, \ldots, m) \) 
    divide \( I \) into \( n \times m \) subrectangles:
    \[
    [x_{i-1}, x_i] \times [y_{j-1}, y_j], \quad i = 1, \ldots, n, \, j = 1, \ldots, m.
    \]

    The union of these \( k \) subrectangles forms a partition \( T = T_x \times T_y = \{ I_1, I_2, \ldots, I_k \} \). 
    For each \( \xi^i \in I_i \, (i = 1, 2, \ldots, k) \), define the \textbf{Riemann sum} (also called a sum of integrals) as:
    \[
    \sum_{i=1}^k f(\boldsymbol{\xi}^i) v(I_i),
    \]
    where \( v(I_i) \) is the area of the rectangle \( I_i \), i.e., the product of its length and width. Denote:
    \[
    \lambda = \max(\text{diam}(I_1), \text{diam}(I_2), \ldots, \text{diam}(I_k)),
    \]
    where \( \text{diam}(I) \) is the diagonal length of the rectangle \( I \), 
    and \( \lambda \) is called the modulus or width of the partition \( T \). 
    The points 
    \( \boldsymbol{\xi} = (\boldsymbol{\xi}^1, \boldsymbol{\xi}^2, \ldots, \boldsymbol{\xi}^k) 
    \in I_1 \times I_2 \times \cdots \times I_k \) 
    are called sampling points for the Riemann sum.

    If there exists \( J \in \mathbb{R} \), such that \( \forall \varepsilon > 0 \), there exists \( \delta > 0 \), 
    such that when \( \lambda < \delta \), for all \( \boldsymbol{\xi} \in I_1 \times I_2 \times \cdots \times I_k \), we have:
    \[
    \left| \sum_{i=1}^k f(\boldsymbol{\xi}^i) v(I_i) - J \right| < \varepsilon,
    \]
    then \( f \) is said to be Riemann integrable on \( I \), and:
    \[
    J = \lim_{\lambda \to 0} \sum_{i=1}^k f(\boldsymbol{\xi}^i) v(I_i) =: 
    \iint_I f(x, y) \, \mathrm{d}x \mathrm{d}y \quad \text{or} \quad \int_I f \, \mathrm{d}v \quad \text{or} \quad \int_{I} f.
    \]

    The function \( f \) is said to have a double integral on \( I \), or simply \( f \) is integrable on \( I \). 
    Here \( f \) is called the integrand, \( I \) is called the integration region, 
    and \( \mathrm{d}v = \mathrm{d}x \mathrm{d}y \) is called the integration element.
\end{definition}

The defined double integral possesses properties similar to those of single-variable integrals.

On the basis of the above definition, we can extend it to the case of a bounded set.

\begin{definition}{Double Integral on a Bounded Set}
    Let \( \Omega \subset \mathbb{R}^2 \) be a bounded set, and \( f: \Omega \to \mathbb{R} \) a two-dimensional function. 
    Define:
    \[
    f_\Omega(\mathbf{x}) = f_\Omega(x, y) =
    \begin{cases} 
    f(x, y), & \text{if } \mathbf{x} = (x, y) \in \Omega, \\
    0, & \text{if } \mathbf{x} = (x, y) \not\in \Omega,
    \end{cases}
    \]
    and call this the \textbf{zero extension} (or \textbf{characteristic function}) of \( f \). 
    For any closed interval \( I \supset \Omega \), if \( f_\Omega \) is Riemann integrable on \( I \), 
    then \( f \) is said to be \textbf{Riemann integrable} on \( \Omega \) (abbreviated as integrable). 
    The integral of \( f \) on \( \Omega \), denoted as:
    \[
    \iint_\Omega f(x, y) \, \mathrm{d}x \mathrm{d}y = 
    \int_\Omega f \, \mathrm{dV} = \int_{\Omega} f = \int_{\Omega} f_\Omega = 
    \iint_I f_\Omega(x, y) \, \mathrm{d}x \mathrm{d}y,
    \]
    represents the Riemann integral of \( f \) on \( \Omega \).
\end{definition}

In above definition, the integral \( \int_\Omega f \) is independent of the choice of 
the closed interval \( I \) containing \( \Omega \) (this confirms the consistency of the definition).

It is worth noting that all the definitions and properties of double integrals 
can be \underline{extended} to triple integrals and higher-dimensional integrals without excessive inconvenience.

\begin{leftbarTitle}{About the Second Approach}\end{leftbarTitle}

\begin{definition}{Set with Zero Area and Set with Zero Measure (Null Set)}
    Let \( A \subset \mathbb{R}^2 \). If for any \( \varepsilon > 0 \), 
    there exist \underline{finitely many} closed intervals \( I_1, I_2, \dots, I_k \) such that:
    \[
    \bigcup_{i=1}^k I_i \supset A, \quad \text{and} \quad \sum_{i=1}^k v(I_i) < \varepsilon,
    \]
    then \( A \) is called a \textbf{set with zero area}.

    Let \( A \subset \mathbb{R}^2 \). If for any \( \varepsilon > 0 \), 
    there exist at most \underline{countably many} closed intervals \( I_1, I_2, \dots, I_k, \dots \) such that:
    \[
    \bigcup_{i=1}^\infty I_i \supset A, \quad \text{and} \quad \sum_{i=1}^\infty v(I_i) < \varepsilon,
    \]
    then \( A \) is called a \textbf{set with zero measure (null set)}.
\end{definition}

\begin{definition}{Set with Finite Area}
    Let \( \Omega \subset \mathbb{R}^2 \) be a bounded set. 
    If the constant function \( 1 \) is integrable on \( \Omega \), 
    then \( \Omega \) is called a \textbf{set with finite area}, and the area of \( \Omega \) is defined as:
    \[
    v(\Omega) = \int_\Omega 1 = \iint_\Omega \mathrm{d}x \mathrm{d}y = \int_I 1_\Omega.
    \]
\end{definition}

Obviously, \(\Omega\) is a set with zero area if and only if \(\Omega\) has finite area, 
and \(v(\Omega) = \int_\Omega 1 = 0\).

\begin{proposition}
    A bounded closed region \(\Omega \subset \mathbb{R}^2\) is measurable if and only if
    its boundary \(\partial \Omega\) is a set with zero area.
\end{proposition}
In the definition of multiple integrals derived from the second approach,
the key point is the division \(T\) of the bounded closed region \(\Omega\) 
into two polygonal regions \(\Sigma_1\) and \(\Sigma_2\).
With above statements, we can see that the division \(T\) is implemented by
infinitely many curves net with zero area.




\begin{leftbarTitle}{Necessary and Sufficient Conditions for Integrability}\end{leftbarTitle}

\begin{proposition}
    Let non-negative function \(f\in R(D)\),
    then \(\iint_{D}f(x,y)\mathrm{d}x\mathrm{d}y=0\) if and only if
    for any continuous points \((x,y)\in D\), \(f(x,y)=0\).
\end{proposition}

\section{Properties of Multiple Integrals}

\begin{leftbarTitle}{Reduction of Double Integral to Iterated Integral}\end{leftbarTitle}

\begin{theorem}{Reduction of Double Integral to Iterated Integral on a Closed Interval}\label{thm:Reduction of Double Integral to Iterated Integral on a Closed Interval}
    Let \( f \) be integrable on the closed interval \( I = [a, b] \times [c, d] \). 
    
    If \( \forall x \in [a, b] \), the integral \(\phi(x)=\int_c^d f(x,y)\,\mathrm{d}y\) exists,
    then \( \phi \) is integrable on \( [a, b] \), and:
    \[
    \iint_I f = \int_a^b \left( \int_c^d f(x, y) \, \mathrm{d}y \right) \mathrm{d}x
    =: \int_{a}^{b}\mathrm{d}x\int_{c}^{d}f(x,y)\,\mathrm{d}y.
    \]

    Similarly, if \( \forall y \in [c, d] \), the integral \(\psi(y)=\int_a^b f(x,y)\,\mathrm{d}x\) exists,
    then \( \psi \) is integrable on \( [c, d] \), and:
    \[
    \iint_I f = \int_c^d \left( \int_a^b f(x, y) \, \mathrm{d}x \right) \mathrm{d}y
    =: \int_{c}^{d}\mathrm{d}y\int_{a}^{b}f(x,y)\,\mathrm{d}x.
    \]
\end{theorem}
\begin{note}
    That is, if \(f\in C(I)\), then two iterated integrals above all exist, 
    and they are equal to the double integral of \(f\) on \(I\) (they can exchange the order of integration).
\end{note}


On the basis of the above theorem, we can extend it to the case of a bounded region.

\begin{theorem}{Reduction of Double Integral to Iterated Integral on a Bounded Set}
    Let \( \Omega \subset \mathbb{R}^2 \) be a set with infinite area, 
    and \( f: \Omega \to \mathbb{R} \) be bounded and continuous
    ~(\ref{fig:Double Integral on a Bounded Set}). 
    Denote the vertical projection of \( \Omega \) onto the \( x \)-axis as:
    \[
    I = \{ x \in \mathbb{R} \mid \exists y, \text{ s.t. } (x, y) \in \Omega \}.
    \]

    If \( \forall x \in I \), let \( \Omega_x = \{ y \in \mathbb{R} \mid (x, y) \in \Omega \} \) 
    be an interval (possibly reducing to a single point), then:
    \[
    \int_\Omega f = \int_I \mathrm{d}y \int_{\Omega_x} f(x, y) \, \mathrm{d}x.
    \]

    Similarly, denote the vertical projection of \( \Omega \) onto the \( y \)-axis as:
    \[
    J = \{ y \in \mathbb{R} \mid \exists x, \text{ s.t. } (x, y) \in \Omega \}.
    \]

    If \( \forall y \in J \), let \( \Omega_y = \{ x \in \mathbb{R} \mid (x, y) \in \Omega \} \) be an interval (possibly reducing to a single point), then:
    \[
    \int_\Omega f = \int_J \mathrm{d}y \int_{\Omega_y} f(x, y) \, \mathrm{d}x.
    \]
\end{theorem}
\begin{figure}[h]
    \centering
    \includegraphics[width=0.5\textwidth]{img/IntegralImg.png}
    \caption{Double Integral on a Bounded Set}
    \label{fig:Double Integral on a Bounded Set}
\end{figure}

Specially, Let:
\[
\Omega = \{ (x, y) \in \mathbb{R}^2 \mid y_1(x) \leqslant y \leqslant y_2(x), \, a \leqslant x \leqslant b \},
\]
where the functions \( y_1 \) and \( y_2 \) are continuous on \( [a, b] \)~(\ref{fig:Double Integral on a Bounded Set}) 
and the function \( f \) is integrable on \( \Omega \). If \( \forall x \in [a, b] \), the single-variable integral:
\[
\int_{y_1(x)}^{y_2(x)} f(x, y) \, \mathrm{d}y
\]
exists, then:
\[
\int_\Omega f = \int_a^b \mathrm{d}x \int_{y_1(x)}^{y_2(x)} f(x, y) \, \mathrm{d}y.
\]
This area called the \textbf{type X region}, similarly, we can define the \textbf{type Y region}.


According to~\ref{thm:Reduction of Double Integral to Iterated Integral on a Closed Interval},
we can derive the formula of multiplicative property for double integral.

\begin{theorem}{Formula of Multiplicative Property for Double Integral}
    Let \( f \in C([a, b]) \), \( g \in C([c, d]) \). 
    Then the function \( h(x, y) = f(x) g(y) \) is integrable on the closed interval \( I = [a, b] \times [c, d] \), 
    and:
    \[
    \iint_I h(x, y) \, \mathrm{d}x \mathrm{d}y = \left( \int_a^b f(x) \, \mathrm{d}x \right) 
    \left( \int_c^d g(y) \, \mathrm{d}y \right).
    \]
\end{theorem}

\begin{example}
    Let \(p(x)\in R[a,b],p(x)>0,x\in [a,b]\), the monotonicity of \(f(x), g(x)\) is same,
    prove that
    \[
    \int_{a}^{b}p(x)f(x)\mathrm{d}x \int_{a}^{b}p(x)g(x)\mathrm{d}x
    \leqslant  \int_{a}^{b}p(x)\mathrm{d}x \int_{a}^{b}p(x)f(x)g(x)\mathrm{d}x
    \]
\end{example}
\begin{proof}
    Let
    \[
    I = \int_{a}^{b}p(x)\mathrm{d}x \int_{a}^{b}p(x)f(x)g(x)\mathrm{d}x
        - \int_{a}^{b}p(x)f(x)\mathrm{d}x \int_{a}^{b}p(x)g(x)\mathrm{d}x,
    \]
    then
    \[
    I = \int_{a}^{b}\int_{a}^{b}p(x)p(y)g(y)(f(x)-f(y))\mathrm{d}x\mathrm{d}y,
    \]
    similarly,
    \[
    I = \int_{a}^{b}\int_{a}^{b}p(x)p(y)g(x)(f(x)-f(y))\mathrm{d}x\mathrm{d}y.
    \]
    Then 
    \[
    2I = \int_{a}^{b}\int_{a}^{b}p(x)p(y)(g(y)-g(x))(f(x)-f(y))\mathrm{d}x\mathrm{d}y \geqslant  0,
    \]
    which implies
    \[
    I \geqslant  0.
    \]
    The proof is complete.
\end{proof}

%%%%%%%%%%%%%%%%%%%%%%%%%%%%%%%%%%%%%%%%%%%%%%%%%%%%%%%%%%%%%%%%%%%%%%
\section{Calculation of Multiple Integrals} % 多重积分的计算
\begin{leftbarTitle}{Variable Substitution in Multiple Integrals}\end{leftbarTitle} % 多重积分中的变量替换
\begin{theorem}{Variable Substitution in Double Integral}
    Let \( \Omega \subset \mathbb{R}^2 \) be an open set, and let the mapping:
    \[
    \mathbf{F}: \Omega \to \mathbb{R}^2, \quad (u, v) \mapsto \mathbf{F}(u, v) = (x(u, v), y(u, v))
    \]
    satisfy the following conditions:
    \begin{enumerate}
        \item \( \mathbf{F} \in C^1(\Omega, \mathbb{R}^2) \);
        \item \( \frac{\partial (x, y)}{\partial (u, v)} 
            = \det J\mathbf{F}(u, v) = \det J\mathbf{F}(\mathbf{p}) \neq 0, \quad \mathbf{p} = (u, v) \in \Omega \);
        \item \( \mathbf{F} \) is injective.
    \end{enumerate}

    If the set \( \Delta \) is a set with finite area and 
    \( \overline{\Delta} \subset \Omega \), and \( f \) is continuous on \( \mathbf{F}(\Omega) \), 
    then \( \mathbf{F}(\Delta) \) is also a set with finite area, and:
    \[
    \iint_{\mathbf{F}(\Delta)} f = \iint_{\Delta} f \circ \mathbf{F} \left| \det J\mathbf{F} \right|,
    \]
    i.e.,
    \[
    \iint_{F(\Delta)} f(x, y) \, \mathrm{d}x \mathrm{d}y = 
    \iint_{\Delta} f(x(u, v), y(u, v)) \left| \frac{\partial (x, y)}{\partial (u, v)} \right| \, \mathrm{d}u \mathrm{d}v.
    \]
\end{theorem}

For triple and higher-dimensional integrals, the variable substitution theorem is similar to the above theorem.

\vspace{0.7cm}
Some common variable substitutions in multiple integrals are as follows:
\begin{description}
\item[Polar Coordinates]
\[
\begin{cases} 
    x = r \cos \theta, \\ 
    y = r \sin \theta,
\end{cases}
\qquad
\begin{cases} 
    r = \sqrt{x^2 + y^2},\quad r\geqslant 0 \\ 
    \theta = \arctan\left(\frac{y}{x}\right)\quad x\neq 0, \theta\in [0, 2\pi].
\end{cases}
\]
and
\[
\frac{\partial (x,y)}{\partial (r,\theta)} = r.
\]

\item[Cylindrical Coordinate System]
\[
\begin{cases} 
    x = r \cos \theta, \\ 
    y = r \sin \theta, \\
    z = z,
\end{cases}
\qquad
\begin{cases} 
    r = \sqrt{x^2 + y^2},\quad r\geqslant 0 \\ 
    \theta = \arctan\left(\frac{y}{x}\right)\quad x\neq 0, \theta\in [0, 2\pi], \\
    z = z.
\end{cases}
\]
and
\[
\frac{\partial (x,y,z)}{\partial (r,\theta,\varphi)} = r.
\]

\item[Spherical Coordinate System]
\[
\begin{cases} 
    x = r \sin \varphi \cos \theta, \\ 
    y = r \sin \varphi \sin \theta, \\
    z = r \cos \varphi,
\end{cases}
\qquad
\begin{cases} 
    r = \sqrt{x^2 + y^2 + z^2},\quad r\geqslant 0 \\ 
    \varphi = \arccos\left(\frac{z}{r}\right)\quad r\neq 0, \varphi\in [0, \pi], \\
    \theta = \arctan\left(\frac{y}{x}\right)\quad x\neq 0, \theta\in [0, 2\pi].
\end{cases}
\]
and
\[
\frac{\partial (x,y,z)}{\partial (r,\theta,\varphi)} = r^2 \sin \varphi.
\]

\end{description}
\begin{figure}[h]
    \centering
    \includegraphics[width=0.8\textwidth]{img/coordinate.png}
    \caption{Cylindrical and Spherical Coordinate Systems}
\end{figure}
\begin{leftbarTitle}{Calculation of Triple Integrals}\end{leftbarTitle} % 三重积分的计算
\begin{example}\label{eg:Triple Integral of Cone}
    Calculating \(I = \iiint_{ \Omega }z^{2}\mathrm{d}x\mathrm{d}y\mathrm{d}z\),
    where \(\Omega\) is the cone defined by \(z^{2} = \frac{h^{2}}{R^{2}}(x^{2}+y^{2})\) 
    and \(z = h\)~(\ref{fig:Cone}).
    \begin{figure}[h]
        \centering
        \includegraphics[width=0.4\textwidth]{img/cone.png}
        \caption{Cone Example.}
        \label{fig:Cone}
    \end{figure}
\end{example}

\begin{example}\label{eg:Project Method Example}
    Calculating \(I = \iiint_{ \Omega }xy\mathrm{d}x\mathrm{d}y\mathrm{d}z\),
    where \(\Omega\) is the region defined 
    by \(0 \leqslant z \leqslant xy, 0\leqslant y \leqslant 1-x, 0\leqslant x \leqslant 1\)
    ~(\ref{fig:Project Method Example}).
    \begin{figure}[h]
        \centering
        \includegraphics[width=0.4\textwidth]{img/project_method_example.png}
        \caption{Project Method Example.}
        \label{fig:Project Method Example}
    \end{figure}
\end{example}

\vspace{1cm}
With the help of examples above, we can derive \textbf{two methods for calculating triple integrals}.

\begin{description}
\item[First 2 then 1 (Section Method)] 
Fix one variable (e.g., \(z\)), first perform a double integral over the other two variables (e.g., \(x,y\)) 
on the "section region" corresponding to the fixed variable, 
and then perform a definite integral over the fixed variable (\(z\)) within its range of values.

This method is convenient when the area of the section region is easy to calculate, 
or when the integrand is only related to the "later-integrated variable" (e.g., only related to \(z\)).

In the example~\ref{eg:Triple Integral of Cone}, the following steps are taken:
\begin{enumerate}
    \item Determine the range of z: \(z \in [0, h]\). 
    \item Determine the section region \(D_z\): 
        For a fixed \(z\), \(D_z\) is the region on the \(xy\)-plane satisfying \(\frac{h^2}{R^2}(x^2 + y^2) \leqslant z^2\), 
        which is a circle with radius \(\frac{R}{h}z\).
    \item Split the integral: 
        \[
        I = \int_{0}^{h} \left( \iint_{D_z} z^2 \,\mathrm{d}x\mathrm{d}y \right) \,\mathrm{d}z.
        \] 
        Since \(z^2\) is independent of \(x\) and \(y\), it can be factored out:
        \(I = \int_{0}^{h} z^2 \left( \iint_{D_z} \,\mathrm{d}x\mathrm{d}y \right) dz\).
    \item Calculate the double integral (area of the section): 
        \[
        \iint_{D_z} \,\mathrm{d}x\mathrm{d}y = \pi \left( \frac{R}{h}z \right)^2 = \pi \frac{R^2}{h^2} z^{2}.
        \]
    \item Calculate the definite integral: 
        \[
        I = \int_{0}^{h} z^2 \cdot \pi \frac{R^2}{h^2} z^2 \,\mathrm{d}z = \frac{\pi R^2 h^3}{5}.\
        \]
\end{enumerate}

\item[First 1 then 2 (Project Method)]
Fix two variables (e.g., \(x,y\)), first perform a definite integral over the third variable (e.g., \(z\)) 
on the "vertical line segment" corresponding to the fixed variables, 
and then perform a double integral over the fixed two variables ( \(x,y\)) on their "projection region. 

This method is convenient when the projection region of the integral region on 
a certain coordinate plane (e.g., \(xy\)-plane) is easy to determine, 
and the upper and lower limits of a single variable (e.g., \(z\)) can 
be easily expressed by the other two variables.

In the example~\ref{eg:Project Method Example}, the following steps are taken:
\begin{enumerate}
    \item Determine the projection region \(D_{xy}\):\(D_{xy}\) is the region on the \(xy\)-plane 
        bounded by \(x + y \leqslant 1\), \(x \geq 0\), and \(y \geq 0\), 
        which can be expressed as \(0 \leqslant x \leqslant 1\) and \(0 \leqslant y \leqslant 1 - x\).
    \item Determine the range of \(z\): \(z \in [0, xy]\) (since \(z\) is bounded below by \(z = 0\) and above by \(z = xy\)).
    \item Split the integral: 
        \[
        I = \iint_{D_{xy}}\, \left( \int_{0}^{xy} xy \mathrm{d}z \right)  \mathrm{d}x \mathrm{d}y,
        \]
        split the double integral on \(D_{xy}\) as:
        \(I = \int_{0}^{1} \, \mathrm{d}x \int_{0}^{1 - x} \, \mathrm{d}y  \int_{0}^{xy} xy \, \mathrm{d}z\).
        (Since \(xy\) is independent of \(z\), it can be factored out without affecting the integral:
        \(I = \int_{0}^{1} \, \mathrm{d}x \int_{0}^{1 - x} xy \, \mathrm{d}y  \int_{0}^{xy} \, \mathrm{d}z\).)
    \item Calculate the inner integral (with respect to \(z\)):
        \(\int_{0}^{xy} xy \, dz = xy \cdot \int_{0}^{xy} dz 
        = xy \cdot \left. z \right|_{0}^{xy} = xy \cdot xy = x^2 y^2\).
    \item Calculate the middle integral (with respect to \(y\)):
        Substitute the result of the inner integral,
        \[
        \int_{0}^{1 - x} x^2 y^2 \, dy 
        = x^2 \cdot \left. \frac{y^3}{3} \right|_{0}^{1 - x} = \frac{x^2 (1 - x)^3}{3}.
        \]
    \item Calculate the outer integral (with respect to \(x\)):Substitute the result of the middle integral:
        \begin{align*}
            \int_{0}^{1} \frac{x^2 (1 - x)^3}{3} \, dx &= \frac{1}{3} \int_{0}^{1} (x^2 - 3x^3 + 3x^4 - x^5) \, dx \\
            &= \frac{1}{3} \left( \left. \frac{x^3}{3} - \frac{3x^4}{4} + \frac{3x^5}{5} - \frac{x^6}{6} \right|_{0}^{1} \right) \\
            &= \frac{1}{3} \left( \frac{1}{3} - \frac{3}{4} + \frac{3}{5} - \frac{1}{6} \right) \\
            &= \frac{1}{180}.
        \end{align*}
\end{enumerate}
\end{description}
% 选择哪种方法的技巧
Some tips for choosing between the two methods (take the above two examples as reference):

\begin{tabular}{p{0.45\textwidth} p{0.45\textwidth}}
    \textbf{First 2 then 1 (Section Method)} & \textbf{First 1 then 2 (Project Method)} \\
    \toprule
    Section area \(D_{z}\) is easy to calculate & Projection region \(D_{xy}\) is easy to determine \\
    \hline
    Integrand is only related to \(z\) & Upper and lower limits \(z\) can be easily expressed by the other two variables \(x,y\) \\
    \bottomrule
\end{tabular}

\vspace{0.7cm}
\begin{example}
    Find the volume of region bounded by the half \textbf{Viviani's curve}:
    sphere \(x^{2}+y^{2}+z^{2}\leqslant a^{2}\) and cylinder \(x^{2}+y^{2}\leqslant ax\) (\(a>0\)).
\end{example}
\begin{figure}[h]
    \centering
    \includegraphics[width=0.4\textwidth]{img/viviani.png}
\end{figure}




\section{Improper Multiple Integrals}
Improper multiple integrals can be also classified into two types, infinite integrals and defective integrals.

\begin{definition}{Infinite Multiple Integral}
    Let \( D \subset \mathbb{R}^2 \) be an unbounded region,
    whose boundary consists of finite or countably many smooth curves, 
    and \( f: D \to \mathbb{R} \) be a function,
    which is integrable on any measurable bounded closed set \( D' \subset D \). 
    If there exists an increasing sequence of bounded closed regions \( \{ D_k \} \) such that:
    \[
    D_1 \subset D_2 \subset \cdots \subset D_k \subset \cdots, 
    \quad \bigcup_{k=1}^{\infty} D_k = D,
    \]
    which is called an \textbf{exhaustion} of \( D \),
    and for each \( k \), the integral \( I(D_k) = \iint_{D_k} f \) exists, 
    and the limit:
    \[
    I = \lim_{k\to\infty} I(D_k)
    \]
    exists, then \( I \) is called the \textbf{improper multiple integral} of \( f \) on \( D \), denoted as:
    \[
    I = \iint_{D} f = \lim_{k\to\infty} \iint_{D_k} f.
    \]
\end{definition}
\begin{remark}
    There are also other ways to define improper multiple integrals,
    such as using limit definitions based on distance to infinity.
    They are equivalent to the above definition.
\end{remark}

\begin{theorem}
    Improper multiple integral is integrable if and only if it is absolutely integrable.
\end{theorem}

\begin{example}
    Calculate
    \[
    \iint_{\mathbb{R}^2} e^{-(x^2+y^2)} \, \mathrm{d}x \mathrm{d}y,
    \]
    and find the value of Poisson integral
    \[
    \int_{-\infty}^{+\infty} e^{-x^2} \, \mathrm{d}x.
    \]
\end{example}

