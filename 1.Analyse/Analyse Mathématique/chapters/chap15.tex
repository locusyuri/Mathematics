\chapter{Line Integrals and Surface Integrals} % 曲线积分与曲面积分
\section{Line Integrals and Surface Integrals of scalar fields}
\begin{leftbarTitle}{Line Integral of Scalar Field}\end{leftbarTitle}
\begin{definition}{Line Integral of Scalar Field}
    Let \(L\) is a rectifiable continuous curve in \(\mathbb{R}^3\), whose endpoints are \(A\) and \(B\),
    and \(f(x, y, z)\) is bounded on \(L\).
    Partition \(L\) into \(n\) segments by points \(A = P_0, P_1, \ldots, P_n = B\),
    and select a point \(\boldsymbol{\xi}_{i}\) on each segment \(P_{i-1}P_i\) (\(i = 1, 2, \ldots, n\)).
    Remark that the length of segment \(P_{i-1}P_i\) is \(\Delta s_i\) (\(i=1,2,\cdots n\)),
    and make the sum:
    \[
    \sum_{i=1}^{n} f(\boldsymbol{\xi}_i) \Delta s_i.
    \]
    If when \( \lambda \) (the length of the longest segment) tends to \(0\),
    the above sum tends to a limit \(I\) independent of the partition and the choice of points \(\boldsymbol{\xi}_i\),
    then \(I\) is called the \textbf{line integral of the scalar field \(f\) along the curve \(L\)},
    denoted as:
    \[
    \int_{L} f \, \mathrm{d}s.
    \]
    That is,
    \[
    I = \int_{L} f(\boldsymbol{\xi}) \, \mathrm{d}s =
    \lim_{\lambda \to 0} \sum_{i=1}^{n} f(\boldsymbol{\xi}_i) \Delta s_i.
    \]
\end{definition}

\begin{theorem}
    Let \(L\) be a \(C^{1}\) smooth regular curve parameterized by \(\mathbf{x}(t) = (x(t), y(t), z(t)), t \in [\alpha, \beta]\),
    and \(f\) be continuous on \(L\).
    Then:
    \[
    \int_{L} f \, \mathrm{d}s = \int_{\alpha}^{\beta} f(\mathbf{x}(t)) \|\mathbf{x}'(t)\| \, \mathrm{d}t.
    = \int_{\alpha}^{\beta} f(x(t), y(t), z(t)) \sqrt{(x'(t))^2 + (y'(t))^2 + (z'(t))^2} \, \mathrm{d}t.
    \]
\end{theorem}
Specially, if the plane curve \(L\) is given by \(y = y(x), x \in [a, b]\),
then:
\[
\int_{L} f \, \mathrm{d}s = \int_{a}^{b} f(x, y(x)) \sqrt{1 + (y'(x))^2} \, \mathrm{d}x.
\]


\begin{leftbarTitle}{Surface Integrals of Scalar Fields}\end{leftbarTitle}
\begin{definition}{Surface Integral of Scalar Field}
    Let \(\Sigma\) be a piecewise smooth surface in \(\mathbb{R}^3\),
    and \(f(x, y, z)\) be bounded on \(\Sigma\).
    Partition \(\Sigma\) into \(n\) small pieces \(\Delta\Sigma_1, \Delta\Sigma_2, \ldots, \Delta\Sigma_n\)
    with smooth curve webs,
    and select a point \(\boldsymbol{\xi}_i\) on each piece \(\Delta\Sigma_i\) (\(i = 1, 2, \ldots, n\)).
    Remark that the area of piece \(\Delta\Sigma_i\) is \(\Delta S_i\) (\(i=1,2,\cdots n\)),
    and make the sum:
    \[
    \sum_{i=1}^{n} f(\boldsymbol{\xi}_i) \Delta S_i.
    \]
    If when \( \lambda \) (the area of the largest piece) tends to \(0\),
    the above sum tends to a limit \(I\) independent of the partition and the choice of points \(\boldsymbol{\xi}_i\),
    then \(I\) is called the \textbf{surface integral of the scalar field \(f\) over the surface \(\Sigma\)},
    denoted as:
    \[
    \iint_{\Sigma} f \, \mathrm{d}S.
    \]
    That is,
    \[
    I = \iint_{\Sigma} f(\boldsymbol{\xi}) \, \mathrm{d}S =
    \lim_{\lambda \to 0} \sum_{i=1}^{n} f(\boldsymbol{\xi}_i) \Delta S_i.
    \]
\end{definition}

\begin{theorem}
    Let \(\Sigma\) be a piecewise smooth closed surface parameterized by 
    \(\mathbf{r}(u, v) = (x(u, v), y(u, v), z(u, v)), (u, v) \in D\),
    and \(f\) be continuous on \(\Sigma\).
    \(x, y, z\) have continuous first-order partial derivatives with respect to \(u\) and \(v\) on \(D\),
    and according Jacobian matrix 
    \[
    J = \begin{pmatrix}
        \frac{\partial x}{\partial u} & \frac{\partial x}{\partial v} \\
        \frac{\partial y}{\partial u} & \frac{\partial y}{\partial v} \\
        \frac{\partial z}{\partial u} & \frac{\partial z}{\partial v}
    \end{pmatrix}
    \]
    is of full rank.
    Then:
    \[
    \iint_{\Sigma} f \, \mathrm{d}S = \iint_{D} f(\mathbf{r}(u, v)) 
    \left\| \frac{\partial \mathbf{r}}{\partial u} \times \frac{\partial \mathbf{r}}{\partial v} \right\| \, \mathrm{d}u \mathrm{d}v
    = \iint_{D} f(x(u, v), y(u, v), z(u, v)) 
    \sqrt{EG-F^{2}} \, \mathrm{d}u \mathrm{d}v,
    \]
    where \(E, G, F\) are the Gauß coefficients of the surface \(\Sigma\).
\end{theorem}
Specially, if the surface \(\Sigma\) is given by \(z = z(x, y), (x, y) \in D\),
then:
\[
\iint_{\Sigma} f \, \mathrm{d}S = 
\iint_{D} f(x, y, z(x, y)) 
\sqrt{1 + \left(\frac{\partial z}{\partial x}\right)^2 + \left(\frac{\partial z}{\partial y}\right)^2} \, \mathrm{d}x \mathrm{d}y.
\]

\section{Differential Form and Exterior Differentiation}
Let \(\mathrm{d}x_{i}, \mathrm{d}x_{j}\) be differentials of independent variables \(x_{i}, x_{j}\).

In \(\mathbb{R}^{1}\):
\begin{align*}
    &\text{0-form: } f(x), \\
    &\text{1-form: } \omega = f(x)\mathrm{d}x, \\
    &\text{k-form (\(k\geqslant 2\)): } \omega = \sum_{1 \leqslant i_{1} < i_{2} < \cdots < i_{k} \leqslant n}
        f_{i_{1} i_{2} \cdots i_{k}}(x_{1}, x_{2}, \cdots, x_{n})
        \mathrm{d}x_{i_{1}} \wedge \mathrm{d}x_{i_{2}} \wedge \cdots \wedge \mathrm{d}x_{i_{k}} = 0.
\end{align*}

In \(\mathbb{R}^{2}\):
\begin{align*}
    &\text{0-form: } f(x, y), \\
    &\text{1-form: } \omega = P(x, y)\mathrm{d}x + Q(x, y)\mathrm{d}y, \\
    &\text{2-form: } \omega = f(x, y)\mathrm{d}x \wedge \mathrm{d}y, \\
    &\text{k-form (\(k\geqslant 3\)): } \omega = \sum_{1 \leqslant i_{1} < i_{2} < \cdots < i_{k} \leqslant n}
        f_{i_{1} i_{2} \cdots i_{k}}(x_{1}, x_{2}, \cdots, x_{n})
        \mathrm{d}x_{i_{1}} \wedge \mathrm{d}x_{i_{2}} \wedge \cdots \wedge \mathrm{d}x_{i_{k}} = 0.
\end{align*}

In \(\mathbb{R}^{3}\):
\begin{align*}
    &\text{0-form: } f(x, y, z), \\
    &\text{1-form: } \omega = P(x, y, z)\mathrm{d}x + Q(x, y, z)\mathrm{d}y + R(x, y, z)\mathrm{d}z, \\
    &\text{2-form: } \omega = P(x, y, z)\mathrm{d}y \wedge \mathrm{d}z + Q(x, y, z)\mathrm{d}z \wedge \mathrm{d}x + R(x, y, z)\mathrm{d}x \wedge \mathrm{d}y, \\
    &\text{3-form: } \omega = f(x, y, z)\mathrm{d}x \wedge \mathrm{d}y \wedge \mathrm{d}z, \\
    &\text{k-form (\(k\geqslant 4\)): } \omega = \sum_{1 \leqslant i_{1} < i_{2} < \cdots < i_{k} \leqslant n}
        f_{i_{1} i_{2} \cdots i_{k}}(x_{1}, x_{2}, \cdots, x_{n})
        \mathrm{d}x_{i_{1}} \wedge \mathrm{d}x_{i_{2}} \wedge \cdots \wedge \mathrm{d}x_{i_{k}} = 0.
\end{align*}

Here, \(\wedge\) is called the \textbf{wedge product}, which satisfies:
\begin{enumerate}
    \item Skew symmetric: \(\mathrm{d}x_{i} \wedge \mathrm{d}x_{j} = -\mathrm{d}x_{j} \wedge \mathrm{d}x_{i}\), 
    \item Associative: \((\mathrm{d}x_{i} \wedge \mathrm{d}x_{j}) \wedge \mathrm{d}x_{k} = \mathrm{d}x_{i} \wedge (\mathrm{d}x_{j} \wedge \mathrm{d}x_{k})\),
    \item In a fixed dimension, the wedge product of k-forms becomes zero (as higher forms are not defined), 
        for example, in \(3\)-dimensional space, a \(4\)-form is equal to \(0\).
\end{enumerate}
\textbf{Differential form} is a skew symmetric tensor on vector field.


\begin{definition}{Exterior Differentiation}
    Let \(\omega\) be a \(k\)-form on \(\mathbb{R}^{n}\),
    \[
    \omega = \sum_{1 \leqslant i_{1} < i_{2} < \cdots < i_{k} \leqslant n}
        f_{i_{1} i_{2} \cdots i_{k}}(x_{1}, x_{2}, \cdots, x_{n})
        \mathrm{d}x_{i_{1}} \wedge \mathrm{d}x_{i_{2}} \wedge \cdots \wedge \mathrm{d}x_{i_{k}},
    \]
    where \(f_{i_{1} i_{2} \cdots i_{k}}\) are functions with continuous first-order partial derivatives.
    The exterior differentiation of \(\omega\) is defined as:
    \[
    \mathrm{d}\omega = \sum_{1 \leqslant i_{1} < i_{2} < \cdots < i_{k} \leqslant n}
        \mathrm{d}f_{i_{1} i_{2} \cdots i_{k}}(x_{1}, x_{2}, \cdots, x_{n})
        \wedge \mathrm{d}x_{i_{1}} \wedge \mathrm{d}x_{i_{2}} \wedge \cdots \wedge \mathrm{d}x_{i_{k}},
    \]
    where
    \[
    \mathrm{d}f = \frac{\partial f}{\partial x_1}\mathrm{d}x_1 + 
    \frac{\partial f}{\partial x_2}\mathrm{d}x_2 + 
    \cdots + 
    \frac{\partial f}{\partial x_n}\mathrm{d}x_n.
    \]
    Note that the exterior differentiation of a \(k\)-form is a \(k+1\)-form.
\end{definition}
\begin{property}
    \begin{description}
        \item[Linearity] \(\mathrm{d}(\alpha \omega+ \beta \eta) = \alpha \mathrm{d}\omega + \beta \mathrm{d}\eta\),
        where \(\alpha, \beta\) are constants.
        \item[Leibniz Rule] \(\mathrm{d}(\omega \wedge \eta) = \mathrm{d}\omega \wedge \eta + (-1)^{k} \omega \wedge \mathrm{d}\eta\),
        where \(\omega\) is a \(k\)-form.
        \item[Nilpotency] \(\mathrm{d}(\mathrm{d}\omega) = 0\).
    \end{description}
\end{property}




\section{Line Integrals and Surface Integrals of Vector Fields}
\begin{leftbarTitle}{Line Integral of Vector Field}\end{leftbarTitle}
\begin{definition}{Line Integral of Vector Field}
    Let \(\overset{\rightharpoonup}{L}\) be a orientated smooth curve in \(\mathbb{R}^3\),
    whose endpoints are \(A\) and \(B\).
    Take unit tangent vector \(\boldsymbol{\tau}=(\cos\alpha, \cos\beta, \cos\gamma)\) 
    at each point of \(\overset{\rightharpoonup}{L}\), making it consistent with the orientation of \(\overset{\rightharpoonup}{L}\).
    Let \(\mathbf{f}(x, y, z) = P(x, y, z)\mathbf{i} + Q(x, y, z)\mathbf{j} + R(x, y, z)\mathbf{k}\) 
    be a vector-valued function on \(\overset{\rightharpoonup}{L}\), then 
    \[
    \int_{\overset{\rightharpoonup}{L}} \mathbf{f} \cdot \boldsymbol{\tau} \mathrm{d}\mathbf{s} =
    \int_{\overset{\rightharpoonup}{L}} \left[P \cos\alpha + Q \cos\beta + R \cos\gamma\right] \, \mathrm{d}s
    \]
    is called the \textbf{line integral of the vector field \(\mathbf{f}\) along the oriented curve \(\overset{\rightharpoonup}{L}\)}
    (if the right-hand side exists).
\end{definition}
Consider a differential arc length element \( \mathrm{d}s \) at a point \((x, y, z)\) on the curve \( L \). 
We form the vector \( \mathrm{d}\mathbf{s} = \boldsymbol{\tau} \mathrm{d}s \), 
where \( \boldsymbol{\tau} = (\cos\alpha, \cos\beta, \cos\gamma) \) represents 
the unit tangent vector of curve \( L \) at \((x, y, z)\), pointing along the direction of \( L \). 
The projection of \( \mathrm{d}s \) onto the \( x \)-axis is given by \(\cos\alpha \, \mathrm{d}s\). 
Therefore, we denote:
\[
\mathrm{d}x = \cos\alpha \, \mathrm{d}s, \quad \mathrm{d}y = \cos\beta \, \mathrm{d}s, \quad 
\mathrm{d}z = \cos\gamma \, \mathrm{d}s.
\]
Thus, the second type of line integral can be expressed as:
\[
\int_{\overset{\rightharpoonup}{L}} \mathbf{f} \cdot \boldsymbol{\tau} \mathrm{d}s = \int_{\overset{\rightharpoonup}{L}} \mathbf{f} \, \mathrm{d} \mathbf{s} 
= \int_{\overset{\rightharpoonup}{L}} P(x, y, z) \mathrm{d}x + Q(x, y, z) \mathrm{d}y + R(x, y, z) \mathrm{d}z.
\]
This line integral is also referred to as the integral of the \(1\)-form:
\[
\omega = P(x, y, z) \mathrm{d}x + Q(x, y, z) \mathrm{d}y + R(x, y, z) \mathrm{d}z.
\]
The second type of line integral of \( \omega \) along the curve \( L \) is denoted as:
\[
\int_{\overset{\rightharpoonup}{L}} \omega.
\]

\begin{theorem}
    Let \(\overset{\rightharpoonup}{L}\) be a \(C^{1}\) smooth regular oriented curve parameterized by 
    \(\mathbf{x}(t) = (x(t), y(t), z(t)), t \in [\alpha, \beta]\),
    and \(\mathbf{f} = P\mathbf{i} + Q\mathbf{j} + R\mathbf{k}\) be continuous on \(\overset{\rightharpoonup}{L}\).
    Then:
    \begin{gather*}
    \int_{\overset{\rightharpoonup}{L}} \mathbf{f} \cdot \boldsymbol{\tau} \mathrm{d}s =
    \int_{\alpha}^{\beta} \mathbf{f}(\mathbf{x}(t)) \cdot \mathbf{x}'(t) \, \mathrm{d}t \\
    = \int_{\alpha}^{\beta} [P(x(t), y(t), z(t)) x'(t) + Q(x(t), y(t), z(t)) y'(t) + R(x(t), y(t), z(t)) z'(t)] \, \mathrm{d}t.
    \end{gather*}
\end{theorem}
Specially, if the plane curve \(\overset{\rightharpoonup}{L}\) is given by \(y = y(x), x: a \to b\),
then:
\[
\int_{\overset{\rightharpoonup}{L}} \mathbf{f} \cdot \boldsymbol{\tau} \mathrm{d}s = 
\int_{a}^{b} \mathbf{f}(x, y(x)) \cdot (1, y'(x)) \sqrt{1 + (y'(x))^2} \, \mathrm{d}x.
\]

\begin{leftbarTitle}{Surface Integral of Vector Field}\end{leftbarTitle}
\begin{definition}{Surface Integral of Vector Field}
    Let \(\overset{\rightharpoonup}{\Sigma}\) be an orientated smooth surface in \(\mathbb{R}^3\),
    and \(\mathbf{f}(x, y, z) = P(x, y, z)\mathbf{i} + Q(x, y, z)\mathbf{j} + R(x, y, z)\mathbf{k}\) 
    be a vector-valued function on \(\overset{\rightharpoonup}{\Sigma}\).
    Each point appoints a unit normal vector \(\mathbf{n}=(\cos\alpha, \cos\beta, \cos\gamma)\).
    Then 
    \[
    \iint_{\overset{\rightharpoonup}{\Sigma}} \mathbf{f} \cdot \mathbf{n} \mathrm{d}S =
    \iint_{\overset{\rightharpoonup}{\Sigma}} \left[P \cos\alpha + Q \cos\beta + R \cos\gamma\right] \, \mathrm{d}S
    \]
    is called the \textbf{surface integral of the vector field \(\mathbf{f}\) over the oriented surface \(\overset{\rightharpoonup}{\Sigma}\)}
    (if the right-hand side exists).
\end{definition}
Consider a differential area element \( \mathrm{d}S \) at a point \((x, y, z)\) on the surface \( \Sigma \). 
We form the vector \( \mathrm{d}\mathbf{S} = \mathbf{n} \mathrm{d}S \), 
where \( \mathbf{n} = (\cos\alpha, \cos\beta, \cos\gamma) \) represents
the unit normal vector of surface \( \Sigma \) at \((x, y, z)\),
pointing along the orientation of \( \Sigma \).
The projection of \( \mathrm{d}S \) onto the \( x \)-axis is given by \(\cos\alpha \, \mathbf{\mathrm{d}}S\). 
Therefore, we denote:
\[
\mathrm{d}y \wedge \mathrm{d}z = \cos\alpha \, \mathrm{d}S, \quad \mathrm{d}z \wedge \mathrm{d}x = \cos\beta \, \mathrm{d}S, \quad 
\mathrm{d}x \wedge \mathrm{d}y = \cos\gamma \, \mathrm{d}S.
\]
Thus, the surface integral can be expressed as:
\[
\iint_{\overset{\rightharpoonup}{\Sigma}} \mathbf{f} \cdot \mathbf{n} \mathrm{d}S = 
\iint_{\overset{\rightharpoonup}{\Sigma}} P \mathrm{d}y \wedge \mathrm{d}z + Q \mathrm{d}z \wedge \mathrm{d}x + R \mathrm{d}x \wedge \mathrm{d}y
= \iint_{\Sigma} P \mathrm{d}y\mathrm{d}z + Q \mathrm{d}z\mathrm{d}x + R \mathrm{d}x\mathrm{d}y,
\]
where \(\mathrm{d}y\mathrm{d}z\) is the simplified notation for \(\mathrm{d}y \wedge \mathrm{d}z\), etc.
This surface integral is also referred to as the integral of the \(2\)-form:
\[
\omega = P(x, y, z) \mathrm{d}y \wedge \mathrm{d}z + Q(x, y, z) \mathrm{d}z \wedge \mathrm{d}x + R(x, y, z) \mathrm{d}x \wedge \mathrm{d}y.
\]
The second type of surface integral of \( \omega \) over the surface \( \Sigma \) is denoted as:
\[
\iint_{\overset{\rightharpoonup}{\Sigma}} \omega.
\]

\begin{theorem}
    Let \(\overset{\rightharpoonup}{\Sigma}\) be a smooth oriented surface parameterized by 
    \(\mathbf{r}(u, v) = (x(u, v), y(u, v), z(u, v)), (u, v) \in D\),
    where \(D\) is a closed region with piecewise smooth boundary in \(uv\)-plane,
    and \(\mathbf{f} = P\mathbf{i} + Q\mathbf{j} + R\mathbf{k}\) be continuous on \(\overset{\rightharpoonup}{\Sigma}\).
    \(x, y, z\) have continuous first-order partial derivatives with respect to \(u\) and \(v\) on \(D\),
    and according Jacobian matrix is of full rank.
    Then:
    \begin{align*}
        &\iint_{\overset{\rightharpoonup}{\Sigma}} \mathbf{f} \cdot \mathbf{n} \mathrm{d}S \\
        =&\iint_{\overset{\rightharpoonup}{\Sigma}} [P \cos\alpha + Q \cos\beta + R \cos\gamma] \, \mathrm{d}S \\
        =& \iint_{D} \mathbf{f}(\mathbf{r}(u, v)) \cdot \left( \frac{\partial \mathbf{r}}{\partial u} \times \frac{\partial \mathbf{r}}{\partial v} \right) \, \mathrm{d}u \mathrm{d}v \\
        =& \pm \iint_{D} \bigg[P(x(u, v), y(u, v), z(u, v)) \cdot \frac{\partial (y, z)}{\partial (u, v)} 
        + Q(x(u, v), y(u, v), z(u, v)) \cdot \frac{\partial (z, x)}{\partial (u, v)} \\
        &+ R(x(u, v), y(u, v), z(u, v)) \cdot \frac{\partial (x, y)}{\partial (u, v)}\bigg] \, \mathrm{d}u \mathrm{d}v,
    \end{align*}
    where the sign \(\pm\) depends on whether the orientation of \(\overset{\rightharpoonup}{\Sigma}\) is consistent with
    the direction of \(\frac{\partial \mathbf{r}}{\partial u} \times \frac{\partial \mathbf{r}}{\partial v}\).
\end{theorem}
Specially, if the surface \(\overset{\rightharpoonup}{\Sigma}\) is given by \(z = z(x, y), (x, y) \in D_{xy}\),
where \(D_{xy}\) is a closed region with piecewise smooth boundary in \(xy\)-plane,
and \(R(x,y,z)\) is continuous on \(D_{xy}\),
then:
\[
\iint_{\overset{\rightharpoonup}{\Sigma}} R(x, y, z) \mathrm{d}x \mathrm{d}y = 
\pm \iint_{D_{xy}} R(x, y, z(x, y)) \, \mathrm{d}x \mathrm{d}y,
\]
where the sign \(\pm\) depends on whether the orientation of \(\overset{\rightharpoonup}{\Sigma}\) is upward or downward.



\section{Stokes' Formula}
\begin{leftbarTitle}{Newton-Leibniz Formula}\end{leftbarTitle}

\begin{leftbarTitle}{Green's Formula}\end{leftbarTitle}
Consider two kinds of special orientated closed regions in \(xy\)-plane as shown in Figure \ref{fig:SpecialRegion1}.
As for the first region \(\overset{\rightharpoonup}{M}\), it consists of four orientated curves:
\begin{description}
    \item[\(\overset{\rightharpoonup}{C_1}\)] \(y = \varphi_{1}(x), x \in [a, b]\),
    \item[\(\overset{\rightharpoonup}{C_2}\)] \(x = b, y \in [\varphi_{1}(b), \varphi_{2}(b)]\), can be reduced to a point,
    \item[\(\overset{\rightharpoonup}{C_3}\)] \(y = \varphi_{2}(x), x \in [a, b]\),
    \item[\(\overset{\rightharpoonup}{C_4}\)] \(x = a, y \in [\varphi_{1}(a), \varphi_{2}(a)]\), can be reduced to a point.
\end{description}
The second region is similar.
\begin{figure}[h]
    \centering
    \includegraphics[width=0.8\textwidth]{img/SpecialRegion1.png}
    \caption{Two special orientated closed regions.}
    \label{fig:SpecialRegion1}
\end{figure}

Denote \(\oint_{\overset{\rightharpoonup}{\partial M}}\) as the line integral 
along the boundary of region \(\overset{\rightharpoonup}{M}\), then we have the following lemma.
\begin{lemma}
    \begin{enumerate}
        \item Let \(\overset{\rightharpoonup}{\partial M}\) be the first region in Fig~\ref{fig:SpecialRegion1},
            \(P(x, y) \in C^{1}(M)\), then:
            \[
            \oint_{\overset{\rightharpoonup}{\partial M}} P \, \mathrm{d}x = -\iint_{\overset{\rightharpoonup}{M}} \frac{\partial P}{\partial y} \, \mathrm{d}x \wedge \mathrm{d}y,
            \] 
        \item Let \(\overset{\rightharpoonup}{\partial M}\) be the second region in Fig~\ref{fig:SpecialRegion1},
            \(Q(x, y) \in C^{1}(M)\), then:
            \[
            \oint_{\overset{\rightharpoonup}{\partial M}} Q \, \mathrm{d}y = \iint_{\overset{\rightharpoonup}{M}} \frac{\partial Q}{\partial x} \, \mathrm{d}x \wedge \mathrm{d}y.
            \]
    \end{enumerate}
\end{lemma}

\begin{theorem}{Green's Theorem}
    Let \(\overset{\rightharpoonup}{M}\) be an orientated closed region in \(\mathbb{R}^{2}\),
    and \(\omega = P\mathrm{d}x + Q \mathrm{d}y \in C^{1}(M)\).
    If \(\overset{\rightharpoonup}{\partial M}\) can be split into finitely many first and second regions 
    in Fig~\ref{fig:SpecialRegion1} simultaneously (non-overlapping, no shared interior points),
    then:
    \[
    \oint_{\overset{\rightharpoonup}{\partial M}} P \, \mathrm{d}x + Q \, \mathrm{d}y = 
    \iint_{\overset{\rightharpoonup}{M}} \left( \frac{\partial Q}{\partial x} - 
    \frac{\partial P}{\partial y} \right) \, \mathrm{d}x \wedge \mathrm{d}y =
        \iint_{M} \left( \frac{\partial Q}{\partial x} - 
    \frac{\partial P}{\partial y} \right) \, \mathrm{d}x \mathrm{d}y,
    \]\footnote{
        Note that \(\mathrm{d}x \wedge \mathrm{d}y \) is directed area element, 
        while \(\mathrm{d}x \mathrm{d}y\) is unsigned area element.
    }
    or equivalently,
    \[
    \oint_{\overset{\rightharpoonup}{\partial M}} \omega = \iint_{\overset{\rightharpoonup}{M}} \mathrm{d}\omega,
    \]
    where \(\overset{\rightharpoonup}{\partial M}\) is the induced orientation of \(\overset{\rightharpoonup}{M}\).
\end{theorem}


\begin{leftbarTitle}{Gauß's Formula}\end{leftbarTitle}
Consider three kinds of special orientated closed surfaces in \(\mathbb{R}^{3}\) as shown in Figure \ref{fig:SpecialRegion2}.
As for the first surface \(\overset{\rightharpoonup}{M}\)
($\overset{\rightharpoonup}{M}$ adopts a positive orientation (right-hand system), 
and $\overset{\rightharpoonup}{\partial M}$ adopts the outward normal orientation), 
it consists of three orientated surfaces:
\begin{description}
    \item[\(\overset{\rightharpoonup}{\Sigma_1}\)] \(z = \varphi_{1}(x, y), (x, y) \in \Delta_{1}\),
    \item[\(\overset{\rightharpoonup}{\Sigma_2}\)] \(z = \varphi_{2}(x, y), (x, y) \in \Delta_{1}\),
    \item[\(\overset{\rightharpoonup}{\Sigma_3}\)] A cylindrical taking \(\partial \Delta_1\) as the directrix,
        with the generatrix paralleling to the \(Oz\)-axis. 
        Of course, it can also be reduced as a closed curve.
\end{description}
The second and third surfaces are similar.
\begin{figure}[h]
    \centering
    \includegraphics[width=0.8\textwidth]{img/SpecialRegion2.png}
    \caption{Three special orientated closed surfaces (only the first two are shown).}
    \label{fig:SpecialRegion2}
\end{figure}

Denote \(\oiint_{\overset{\rightharpoonup}{\partial M}}\) as the surface integral
over the boundary of region \(\overset{\rightharpoonup}{M}\), then we have the following lemma.
\begin{lemma}
    \begin{enumerate}
        \item Let \(\overset{\rightharpoonup}{\partial M}\) be the first surface in Fig~\ref{fig:SpecialRegion2},
            \(R(x, y, z) \in C^{1}(M)\), then:
            \[
            \oiint_{\overset{\rightharpoonup}{\partial M}} R \, \mathrm{d}x \wedge \mathrm{d}y = 
            \iiint_{\overset{\rightharpoonup}{M}} \frac{\partial R}{\partial z} \, \mathrm{d}x \wedge \mathrm{d}y \wedge \mathrm{d}z,
            \] 
        \item Let \(\overset{\rightharpoonup}{\partial M}\) be the second surface in Fig~\ref{fig:SpecialRegion2},
            \(P(x, y, z) \in C^{1}(M)\), then:
            \[
            \oiint_{\overset{\rightharpoonup}{\partial M}} P \, \mathrm{d}y \wedge \mathrm{d}z = 
            \iiint_{\overset{\rightharpoonup}{M}} \frac{\partial P}{\partial x} \, \mathrm{d}x \wedge \mathrm{d}y \wedge \mathrm{d}z,
            \]
        \item Let \(\overset{\rightharpoonup}{\partial M}\) be the third surface in Fig~\ref{fig:SpecialRegion2},
            \(Q(x, y, z) \in C^{1}(M)\), then:
            \[
            \oiint_{\overset{\rightharpoonup}{\partial M}} Q \, \mathrm{d}z \wedge \mathrm{d}x = 
            \iiint_{\overset{\rightharpoonup}{M}} \frac{\partial Q}{\partial y} \, \mathrm{d}x \wedge \mathrm{d}y \wedge \mathrm{d}z.
            \]
    \end{enumerate}
\end{lemma}

\begin{theorem}{Gauß's Theorem}
    Let \(\overset{\rightharpoonup}{M}\) be an orientated closed region in \(\mathbb{R}^{3}\),
    and \(\omega = P\mathrm{d}y \wedge \mathrm{d}z + Q \mathrm{d}z \wedge \mathrm{d}x 
    + R \mathrm{d}x \wedge \mathrm{d}y \in C^{1}(M)\).
    If \(\overset{\rightharpoonup}{\partial M}\) can be split into finitely many first, second and third regions
    in Fig~\ref{fig:SpecialRegion1} simultaneously (non-overlapping, no shared interior points),
    then:
    then:
    \[
    \oiint_{\overset{\rightharpoonup}{\partial M}} P \, \mathrm{d}y \wedge \mathrm{d}z + Q \, \mathrm{d}z \wedge \mathrm{d}x + R \, \mathrm{d}x \wedge \mathrm{d}y = 
    \iiint_{\overset{\rightharpoonup}{M}} \left( 
        \frac{\partial P}{\partial x} + 
        \frac{\partial Q}{\partial y} + 
        \frac{\partial R}{\partial z} 
    \right) \, \mathrm{d}x \wedge \mathrm{d}y \wedge \mathrm{d}z,
    \]
    or equivalently,
    \[
    \oiint_{\overset{\rightharpoonup}{\partial M}} \omega = 
    \iiint_{\overset{\rightharpoonup}{M}} \mathrm{d}\omega,
    \]
    where \(\overset{\rightharpoonup}{\partial M}\) is the induced orientation of \(\overset{\rightharpoonup}{M}\).
\end{theorem}


\begin{leftbarTitle}{Stokes' Formula}\end{leftbarTitle}
\begin{theorem}{Stokes' Theorem}
    Let \(\overset{\rightharpoonup}{M}\) be an orientated smooth surface in \(\mathbb{R}^{3}\)
    with boundary \(\overset{\rightharpoonup}{\partial M}\),
    and \(\omega = P\mathrm{d}x + Q \mathrm{d}y + R \mathrm{d}z \in C^{1}(\Sigma)\).
    Then:
    \begin{align*}
        &\oint_{\overset{\rightharpoonup}{\partial M}} P \, \mathrm{d}x + Q \, \mathrm{d}y + R \, \mathrm{d}z\\
        =& \iint_{\overset{\rightharpoonup}{M}} \left( 
            \frac{\partial R}{\partial y} - \frac{\partial Q}{\partial z}
        \right) \, \mathrm{d}y \wedge \mathrm{d}z +
        \left( 
            \frac{\partial P}{\partial z} - \frac{\partial R}{\partial x}
        \right) \, \mathrm{d}z \wedge \mathrm{d}x +
        \left(
            \frac{\partial Q}{\partial x} - \frac{\partial P}{\partial y}
        \right) \, \mathrm{d}x \wedge \mathrm{d}y \\
        =& \iint_{\overset{\rightharpoonup}{M}} 
            \begin{vmatrix}\mathrm{d}y \wedge \mathrm{d}z&\mathrm{d}z \wedge \mathrm{d}x& \mathrm{d}x \wedge \mathrm{d}y\\
            \frac{\partial }{\partial x}&\frac{\partial }{\partial y}&\frac{\partial }{\partial z}\\
            P&Q&R            
            \end{vmatrix}\\
        =&\iint_{\overset{\rightharpoonup}{M}} 
        \begin{vmatrix}\cos\alpha&\cos\beta&\cos\gamma \\
        \frac{\partial }{\partial x}&\frac{\partial }{\partial y}&\frac{\partial }{\partial z}\\
        P&Q&R
        \end{vmatrix} \mathrm{d}S,
    \end{align*}
    or equivalently,
    \[
    \oint_{\overset{\rightharpoonup}{\partial M}} \omega = \iint_{\overset{\rightharpoonup}{M}} \mathrm{d}\omega,
    \]
    where \(\overset{\rightharpoonup}{\partial M}\) is the induced orientation of \(\overset{\rightharpoonup}{M}\).
\end{theorem}

\section{Closed and Exact Differential Forms}
\begin{definition}{Closed and Exact Differential Forms}
    Let \(U \subset \mathbb{R}^{n}\) be an open set and \(\omega\) be a \(C^{r}(r\geqslant 1)\) \(k\)-form on \(U\).
    \begin{enumerate}
        \item If \(\mathrm{d}\omega = 0\), then \(\omega\) is called a \textbf{closed form}.
        \item If there exists a \(C^{r+1}\) \((k-1)\)-form \(\eta\) such that \(\omega = \mathrm{d}\eta\),
            then \(\omega\) is called an \textbf{exact differential form}.
    \end{enumerate}
\end{definition}

\begin{theorem}{Necessary Condition for Exactness}
    Let \(U \subset \mathbb{R}^{n}\) be an open set and \(\omega\) be a \(C^{1}\) \(k\)-form on \(U\).
    If \(\omega\) is exact, then \(\omega\) is closed.
    The converse is not necessarily true.
\end{theorem}

\vspace{0.7cm}
We only discuss the case of \(1\)-forms in \(\mathbb{R}^{2}\) below.

Let \(\omega = P(x, y) \mathrm{d}x + Q(x, y) \mathrm{d}y\) be a \(C^{1}\) \(1\)-form on an open set \(U \subset \mathbb{R}^{2}\).
For any points \(A, B \in U\), a piecewise smooth simple closed curve on \(U\)
is called a \textbf{path} from \(A\) to \(B\) if it starts at \(A\) and ends at \(B\). 

For any path \(\overset{\rightharpoonup}{L}\) from \(A\) to \(B\),
if 
\[
\int_{\overset{\rightharpoonup}{L}} \omega = \int_{A}^{B} \omega,
\]
where the right-hand side is independent of the choice of path \(\overset{\rightharpoonup}{L}\),
then the line integral of \(\omega\) is said to be \textbf{path-independent} on \(U\).

\begin{theorem}
    Let \(U\in \mathbb{R}^{2}\) is a simply connected open region,
    and \(\omega = P(x, y) \mathrm{d}x + Q(x, y) \mathrm{d}y\) be a \(C^{1}\) \(1\)-form on \(U\).
    Then the following statements are equivalent:
    \begin{enumerate}[label=(\roman*)]
        \item \(\omega\) is exact on \(U\), i.e., there exists a \(C^{2}\) function \(F(x, y)\) on \(U\),
            such that
            \[
            \mathrm{d}F = \omega = P \, \mathrm{d}x + Q \, \mathrm{d}y.
            \]
            At this time, \(F(x, y)\) is called a \textbf{potential function} of \(\omega\) on \(U\)
            and 
            \[
            F(x, y) = \int_{(x_{0}, y_{0})}^{(x, y)} \omega + C
            = \int_{x_{0}}^{x} P(t, y_{0}) \, \mathrm{d}t + \int_{y_{0}}^{y} Q(x, s) \, \mathrm{d}s + C,
            \]
            where \((x_{0}, y_{0})\) is a fixed point in \(U\) and \(C\) is an arbitrary constant.
        \item \(\omega\) is closed on \(U\), i.e.,
            \[
                \frac{\partial P}{\partial y} = \frac{\partial Q}{\partial x}.
            \] 
        \item The line integral of \(\omega\) is path-independent on \(U\).
        \item For any piecewise smooth simple closed curve \(\overset{\rightharpoonup}{L}\) on \(U\),
            \[
            \oint_{\overset{\rightharpoonup}{L}} \omega = 0.
            \]
    \end{enumerate}
\end{theorem}


\begin{example}
    Calculate
    \[
    I = \oint_{\overset{\rightharpoonup}{C}} \frac{\cos(\mathbf{r},\mathbf{n})}{r} \, \mathrm{d}s,
    \]
    where \(\overset{\rightharpoonup}{C}\) is piecewise smooth simple closed curve,
    \(\mathbf{r}=(x,y)\), \(r=\|\mathbf{r}\|=\sqrt{x^2 + y^2}\),
    and \(\mathbf{n}\) is the unit outward normal vector of \(\overset{\rightharpoonup}{C}\).
\end{example}