\chapter{Differential} % 微分学
\section{Differential and Derivative}

\begin{leftbarTitle}{Basic Differential Rules and Formulas}\end{leftbarTitle}
\[
\begin{array}{|c|c|c|}
\hline
 & \textbf{Derivative Rules} & \textbf{Differential Rules} \\
\hline
\text{Linear Combination} & (c_{1}f+c_{2}g)' = c_{1}f' + c_{2}g' & 
    \mathrm{d}(c_{1}f+c_{2}g) = c_{1}\mathrm{d}f + c_{2}\mathrm{d}g \\
\hline
\text{Product Rule} & (fg)' = f'g + fg' & \mathrm{d}(fg) = g\mathrm{d}f + f\mathrm{d}g \\
\hline
\text{Quotient Rule} & \left( \frac{f}{g} \right)' = \frac{f'g - fg'}{g^2} & 
    \mathrm{d}\left( \frac{f}{g} \right) = \frac{g\mathrm{d}f - f\mathrm{d}g}{g^2} \\
\hline
\text{Inverse Function} & [f^{-1}(y)]' = \frac{1}{f'(x)} & \mathrm{d}x = \frac{\mathrm{d}y}{f'(x)} = [f^{-1}(y)]'\mathrm{d}y \\
\hline
\text{Chain Rule} & [f(g(x))]' = f'(u)g'(x) & \mathrm{d}[f(g(x))] = f'(u)g'(x)\mathrm{d}x \\
\hline
\end{array}
\]



\[
\begin{array}{|c|c|}
\hline
\textbf{Derivative} & \textbf{Differential} \\
\hline
(C)'=0 & \mathrm{d}(C)=0\cdot \mathrm{d}x=0 \\
\hline
(x^\alpha)'=\alpha x^{\alpha-1} & \mathrm{d}(x^\alpha)=\alpha x^{\alpha-1}\mathrm{d}x \\ \hline
\rowcolor{gray!30} & \\ \hline
(\sin x)'=\cos x & \mathrm{d}(\sin x)=\cos x\mathrm{d}x \\
\hline
(\cos x)'=-\sin x & \mathrm{d}(\cos x)=-\sin x\mathrm{d}x \\
\hline
(\tan x)'=\sec^2x & \mathrm{d}(\tan x)=\sec^2x\mathrm{d}x \\
\hline
(\cot x)'=-\csc^2x & \mathrm{d}(\cot x)=-\csc^2x\mathrm{d}x \\
\hline
(\sec x)'=\tan x\sec x & \mathrm{d}(\sec x)=\tan x\sec x\mathrm{d}x \\
\hline
(\csc x)'=-\cot x\csc x & \mathrm{d}(\csc x)=-\cot x\csc x\mathrm{d}x \\
\hline
(\arcsin x)'=\frac{1}{\sqrt{1-x^2}} & \mathrm{d}(\arcsin x)=\frac{1}{\sqrt{1-x^2}}\mathrm{d}x \\
\hline
(\arccos x)'=-\frac{1}{\sqrt{1-x^2}} & \mathrm{d}(\arccos x)=-\frac{1}{\sqrt{1-x^2}}\mathrm{d}x \\
\hline
(\arctan x)'=\frac{1}{1+x^2} & \mathrm{d}(\arctan x)=\frac{1}{1+x^2}\mathrm{d}x \\
\hline
(\mathrm{arccot} x)'=-\frac{1}{1+x^2} & \mathrm{d}(\mathrm{arccot} x)=-\frac{1}{1+x^2}\mathrm{d}x \\
\hline
\hline
(a^x)'=\ln a\cdot a^x, \, (e^x)'=e^x & \mathrm{d}(a^x)=\ln a\cdot a^x\mathrm{d}x, \, \mathrm{d}(e^x)=e^x\mathrm{d}x \\
\hline
(\log_{a}x)'=\frac{1}{x\ln a}, \, (\ln x)'=\frac{1}{x} & \mathrm{d}(\log_{a}x)=\frac{1}{x\ln a}\mathrm{d}x, \, \mathrm{d}(\ln x)=\frac{1}{x}\mathrm{d}x \\ \hline
\rowcolor{gray!30} & \\ \hline
(\mathrm{sh}\,x)'=\mathrm{ch}\,x & \mathrm{d}(\mathrm{sh}\,x)=\mathrm{ch}\,x\mathrm{d}x \\
\hline
(\mathrm{ch}\,x)'=\mathrm{sh}\,x & \mathrm{d}(\mathrm{ch}\,x)=\mathrm{sh}\,x\mathrm{d}x \\
\hline
(\mathrm{th}\,x)'=\mathrm{sech}^2\,x & \mathrm{d}(\mathrm{th}\,x)=\mathrm{sech}^2\,x\mathrm{d}x \\
\hline
(\mathrm{cth}\,x)'=-\mathrm{csch}^2\,x & \mathrm{d}(\mathrm{cth}\,x)=-\mathrm{csch}^2\,x\mathrm{d}x \\
\hline
(\mathrm{arcsh}\,x)'=\frac{1}{\sqrt{1+x^2}} & \mathrm{d}(\mathrm{arcsh}\,x)=\frac{1}{\sqrt{1+x^2}}\mathrm{d}x \\
\hline
(\mathrm{arcch}\,x)'=\frac{1}{\sqrt{x^2-1}} & \mathrm{d}(\mathrm{arcch}\,x)=\frac{1}{\sqrt{x^2-1}}\mathrm{d}x \\
\hline
(\mathrm{arcth}\,x)'=(\mathrm{arccth}\,x)'=\frac{1}{1-x^2} & 
    \mathrm{d}(\mathrm{arcth}\,x)=\mathrm{d}(\mathrm{arccth}\,x)=\frac{1}{1-x^2}\mathrm{d}x \\ \hline
\rowcolor{gray!30} & \\ \hline
\ln(x+\sqrt{x^2+a^2})'=\frac{1}{\sqrt{x^2+a^2}} & \mathrm{d}[\ln(x+\sqrt{x^2+a^2})]=\frac{\mathrm{d}x}{\sqrt{x^2+a^2}} \\
\hline
\end{array}
\]

\section{Higher-Order Derivatives}

Some useful formulas of higher-order derivatives:
\begin{gather*}
    (a^{x})^{(n)} = (\ln a)^n a^x,\\
    (\sin \alpha x)^{(n)} = \alpha^n \sin\left(\alpha x + \frac{n\pi}{2}\right), \\
    (\cos \alpha x)^{(n)} = \alpha^n \cos\left(\alpha x + \frac{n\pi}{2}\right),\\
    (\ln x)^{(n)} = \frac{(-1)^{n-1}(n-1)!}{x^n}, \\ 
    (x^\alpha)^{(n)} = \alpha(\alpha-1)\cdots(\alpha-n+1)x^{\alpha-n}.
\end{gather*}

In order to obtain the higher-order derivative of two or more functions' linear combination and product,
we need to use the following theorems.

\begin{theorem}{Linear Operation of Higher-Order Derivatives}
    If \(f,g\in D^{(n)}(I)\), then for any constants \(c_{1}, c_{2}\in \mathbb{R}\),
    \[
        (c_{1}f + c_{2}g)^{(n)} = c_{1}f^{(n)} + c_{2}g^{(n)}.
    \]
\end{theorem}

\begin{theorem}{Leibniz's Formula}
    If \(f,g\in D^{(n)}(I)\), then
    \[
        (fg)^{(n)} = \sum_{k=0}^{n} \binom{n}{k} f^{(k)} g^{(n-k)}.
    \]
\end{theorem}

\begin{caution}
    Note the distinction: 
    \begin{itemize}
        \item \(\mathrm{d}x^2\) represents the square of the differential of the independent variable, i.e., \((\mathrm{d}x)^2\);
        \item \(\mathrm{d}^2x\) represents the second differential of the independent variable, \(\mathrm{d}(\mathrm{d}x)\);
        \item \(\mathrm{d}(x^2)\) represents the differential of \(x^2\), which is \(2x\mathrm{d}x\).
    \end{itemize}
\end{caution}

\section{Differential Mean Value Theorems}
\begin{definition}{Argmax and Argmin}
    Let \(f(x)\) is defined on \((a,b)\), \(x_{0}\in (a,b)\).
    If there exists \(U(x_{0}, \delta)\subset (a,b)\) such that \(f(x)\leqslant f(x_{0})\) on it,
    then \(x_{0}\) is called a arguments of the maxima point of \(f\),
    and \(f(x_{0})\) is referred to as the corresponding arguments of the maxima (abbreviated arg max or argmax).

    The definition of the argmin is analogous.
\end{definition}


\begin{lemma}{Fermat's Lemma}
    If \(f\) is differentiable at \(x_{0}\) which is a local extremum, then \(f'(x_{0}) = 0\).
\end{lemma}

\begin{theorem}{Rolle's Theorem}
    If \(f\in C[a,b], f\in D(a,b)\) and \(f(a) = f(b)\), then there exists \(\xi\in (a,b)\) such that \(f'(\xi) = 0\).

    \underline{\textbf{Enhanced Version:}}If \(f\in D(a,b)\)(finite or infinite interval), 
    and \(\lim_{x \to a^{+}} f(x) = \lim_{x \to b^{-}} f(x) \) , 
    then there exists \(\xi\in (a,b)\) such that \(f'(\xi) = 0\).
\end{theorem}

\begin{theorem}{Lagrange's Mean Value Theorem}
    If \(f\in C[a,b], f\in D(a,b)\), then there exists \(\xi\in (a,b)\) such that
    \[
        f'(\xi) = \frac{f(b) - f(a)}{b - a}.
    \]
\end{theorem}

\begin{theorem}{Cauchy's Mean Value Theorem}
    If \(f,g\in C[a,b], f,g\in D(a,b)\) and \(g'(x) \neq 0\) for all \(x\in (a,b)\), 
    then there exists \(\xi\in (a,b)\) such that
    \[
        \frac{f'(\xi)}{g'(\xi)} = \frac{f(b) - f(a)}{g(b) - g(a)}.
    \]
\end{theorem}


\vspace{0.7cm}
\begin{note}
The following types of problems commonly appear in proofs related to intermediate values in differential calculus:
\begin{enumerate}
    \item Prove the existence of a point \(\xi\) such that \(F(\xi, f(\xi), f'(\xi)) = 0\).
        Problems of this type generally involve constructing auxiliary functions and applying Rolle's theorem. 
        The commonly used auxiliary functions include:
        \[
        \begin{aligned}
        \xi f'(\xi) + f(\xi) &= 0, \quad x f(x), \\
        \xi f'(\xi) + nf(\xi) &= 0, \quad x^n f(x), \\
        \xi f'(\xi) - f(\xi) &= 0, \quad e^x f(x), \\
        f'(\xi) + \lambda f(\xi) &= 0, \quad e^{-x} f(x), \\
        f'(\xi) + f(\xi) &= 0, \quad x^n f(x), \\
        f'(\xi) - f(\xi) &= 0, \quad x f(x). \\
        \end{aligned}
        \]
    
    \item Prove the existence of two points \(\xi, \eta\) (i.e., two intermediate values) 
        such that \(F(\xi, f(\xi), f'(\xi), \eta, f(\eta), f'(\eta)) = 0\).
        These problems can be divided into the following categories:
        \begin{description}
            \item [\(\xi \neq \eta\)] Problems of this type usually occur in the same interval \([a, b]\) and 
                employ theorems of \underline{double} differentiation intermediate values 
                such as the Lagrange mean value theorem or Cauchy's mean value theorem. 
                The specific choice of auxiliary functions often includes terms like \(\xi\) and 
                other variables determined after \underline{decomposition}.
            \item [\(\xi = \eta\)] Such problems cannot occur within the same interval \([a, b]\). 
                They use double differentiation mean value theorems by \underline{splitting} \([a, b]\) into 
                two intervals \([a,c]\) and \([c,b]\), 
                applying the Lagrange mean value theorem separately to each interval. 
                Here, the \underline{selection} of \(\xi\) and \(\eta\) is key.
        \end{description}

    \item As a rule, when conditions in a theorem involve additional constraints about \underline{higher-order} derivatives, 
        it is necessary to use Taylor's intermediate value theorem.
\end{enumerate}
\end{note}



\section{Theorems about Derivatives}

\begin{theorem}{Darboux's Intermediate Value Theorem for Derivatives}
    If \(f(x)\in D[a,b]\), and \(f'_{+}(a)\cdot f'_{-}(b)<0\),
    then there at least exists \(\xi\in (a,b)\) such that \(f'(\xi) = 0\).
\end{theorem}

\begin{theorem}{Theorem on the Limit of Derivatives}
    If \(f(x)\in C(U(x_{0})),D(\mathring{U}(x_{0}))\), and \(\lim_{x \to x_{0}} f'(x) = A\),
    then \(f\) is differentiable at \(x_{0}\) and \(f'(x_{0}) = A\).
\end{theorem}
\begin{remark}
    In fact, \(\lim_{x \to x_{0}} f'(x) = A\) has already been shown to imply that \(f\in D(\mathring{U}(x_{0}))\).

    The mnemonic for this theorem is:
    Continuous function + limit of derivative \(\Rightarrow\) derivative at the point.
\end{remark}

\section{Taylor Theorem}
\begin{leftbarTitle}{L'Hôpital's Rule}\end{leftbarTitle}

\begin{leftbarTitle}{Taylor Formula}\end{leftbarTitle}


\begin{leftbarTitle}{Maclaurin Formula}\end{leftbarTitle}
\begin{lemma}
    If \(f(x)\) has \(n+2\) derivatives in some neighborhood of \(x_{0}\), 
    then the derivative of its \(n+1\)th degree Taylor polynomial 
    is exactly the \(n\)th degree Taylor polynomial of \(f'(x)\).
\end{lemma}

Taylor formula at \(x_{0} = 0\) is called the \textbf{Maclaurin formula}.
Some common Maclaurin formulas are as follows:
\begin{gather*}
    e^{x} = 1 + \frac{x}{1!} + \frac{x^2}{2!} + \frac{x^3}{3!} + \cdots + \frac{x^n}{n!} + o(x^n), \\
    \ln(1+x) = x - \frac{x^2}{2} + \frac{x^3}{3} - \cdots + (-1)^{n-1}\frac{x^n}{n} + o(x^n), \\
\end{gather*}
\begin{gather*}
    \sin x = x - \frac{x^3}{3!} + \frac{x^5}{5!} - \cdots + (-1)^{n-1}\frac{x^{2n-1}}{(2n-1)!} + o(x^{2n}), \\
    \cos x = 1 - \frac{x^2}{2!} + \frac{x^4}{4!} - \cdots + (-1)^{n}\frac{x^{2n}}{(2n)!} + o(x^{2n+1}), \\
\end{gather*}
\begin{gather*}
    \arctan x = x - \frac{x^3}{3} + \frac{x^5}{5} - \cdots + (-1)^{n-1}\frac{x^{2n-1}}{2n-1} + o(x^{2n}), \\
    \arcsin x = x + \frac{1}{2}\frac{x^3}{3} + \frac{1\cdot 3}{2\cdot 4}\frac{x^5}{5} + \cdots + 
        \frac{(2n-1)!!}{(2n)!!}\frac{x^{2n+1}}{2n+1} + o(x^{2n+2}). \\
\end{gather*}
Specially, 
\[
(1+x)^{\alpha} = \sum_{k=0}^{\alpha} \binom{\alpha}{k} x^k + o(x^{n}),
\]
\begin{itemize}
    \item if \(\alpha = n\in \mathbb{N}^{+}\), that is Newton's binomial formula 
        \((1+x)^n = 1 + \binom{n}{1}x + \binom{n}{2}x^2 + \cdots + \binom{n}{n}x^n\); 
    \item if \(\alpha = \frac{1}{2}\), then
        \((1+x)^{\frac{1}{2}} = 1 + \frac{1}{2}x - \frac{1}{8}x^2 + \cdots\);
    \item if \(\alpha = -1\), then
        \((1+x)^{-1} = 1 - x + x^2 - x^3 + \cdots\);
    \item if \(\alpha = -\frac{1}{2}\), then
        \((1+x)^{-\frac{1}{2}} = 1 - \frac{1}{2}x + \frac{3}{8}x^2 - \cdots\).
\end{itemize}

\begin{leftbarTitle}{Euler and Bernoulli Numbers}\end{leftbarTitle}
\begin{definition}{Euler Numbers}
    The Euler numbers \(E_n\) are defined by the Taylor series expansion of the secant function:
    \[
        \operatorname{sech} x = \frac{2}{e^x + e^{-x}} = \sum_{n=0}^{\infty} E_n \frac{x^n}{n!}.
    \]
    The odd-indexed Euler numbers are all zero. 
    The even-indexed ones have alternating signs. Some values are:
    \[
        E_0 = 1, \quad E_2 = -1, \quad E_4 = 5, \quad E_6 = -61, \quad E_8 = 1385.
    \]
\end{definition}

\begin{definition}{Bernoulli Numbers}
    The Bernoulli numbers \(B_n\) are defined by the Taylor series expansion of the function \(\frac{x}{e^x - 1}\):
    \[
        \frac{x}{e^x - 1} = \sum_{n=0}^{\infty} B_n \frac{x^n}{n!}.
    \]
    Some values are:
    \[
        B_0 = 1, \quad B_2 = \frac{1}{6}, \quad B_4 = -\frac{1}{30}, \quad B_6 = \frac{1}{42}, \quad B_8 = -\frac{1}{30}.
    \]
    Notably, all odd-indexed Bernoulli numbers (except \(B_1 = -\frac{1}{2}\)) are zero.
\end{definition}

\begin{remark}
    Euler and Bernoulli numbers are widely used in number theory, combinatorics, and numerical analysis.
    For example, in the infinite series:
    \[
    \sum_{n=1}^{\infty}  \frac{1}{n^{2k}} = (-1)^{k-1} \frac{(2\pi)^{2k}}{2(2k)!} B_{2k},\quad k \in \mathbb{N}^{+},
    \]
    when \(k=1\), it gives the famous Basel problem result:
    \[
    \sum_{n=1}^{\infty}  \frac{1}{n^{2}} = \frac{\pi^2}{6}.
    \]
\end{remark}

With the help of Bernoulli numbers, we have
\[
\tan x = \sum_{n=0}^{\infty} \frac{B_{2n}}{2n} \frac{x^{2n}}{(2n)!} = 
x + \frac{x^3}{3} + \frac{2}{15}x^5 + \cdots.
\]


\section{Properties of Functions}
\begin{leftbarTitle}{Monotonicity and Convexity}\end{leftbarTitle}
\begin{definition}{Convex Function}
    A function \(f\) is called \textbf{convex} on an interval \(I\) if for any \(x_1, x_2 \in I\) and 
    \(t \in [0,1]\), the following inequality holds:
    \[
        f(t x_1 + (1-t)x_2) \leqslant t f(x_1) + (1-t)f(x_2).
    \]
    If the inequality is strict for \(x_1 \neq x_2\) and \(t \in (0,1)\), 
    then \(f\) is called \textbf{strictly convex} on \(I\).

    Conversely, if the inequality is reversed, then \(f\) is called \textbf{concave} or \textbf{concave down} on \(I\).
\end{definition}
A related concept is that of \textbf{inflection points}:
a point on the graph of a function at which the concavity changes.

\begin{figure}[h]
    \centering
    \includegraphics[width=0.8\textwidth]{img/ConvexFunction.png}
\end{figure}

\begin{theorem}
    Mark above definition as Definition 1, give the following statements:
    \begin{enumerate}
        \setcounter{enumi}{1} % 设置计数器,从 2 开始
        \item (\textbf{Jensen Definition}) A function \(f\) is called convex on an interval \(I\) 
            if for any \(x_1, x_2 \in I\):
            \[
                f\left( \frac{x_1 + x_2}{2} \right) \leqslant \frac{f(x_1) + f(x_2)}{2}.
            \]
        \item A function \(f\) is called convex on an interval \(I\) 
            if for any \(x_1, x_2, \cdots, x_n \in I\):
            \[
                f\left( \frac{x_1 + x_2 + \cdots + x_n}{n} \right) \leqslant \frac{f(x_1) + f(x_2) + \cdots + f(x_n)}{n}.
            \]
        \item A function \(f\) is called convex on an interval \(I\) 
            if the tangent line at any point lies below the graph of the function.
    \end{enumerate}
    Then,
    \begin{itemize}
        \item Definitions 2 and 3 are equivalent. 
        \item When \(f\) is continuous, Definition 1, 2, 3 is equivalent.
        \item When \(f\) is differentiable, all four definitions are equivalent.
    \end{itemize}
\end{theorem}


\begin{theorem}{Jensen Inequality}
    If \(f\) is convex on an interval \(I\), 
    then for any \(x_1, x_2, \cdots, x_n \in I\) and any \(t_1, t_2, \cdots, t_n > 0\) 
    such that \(t_1 + t_2 + \cdots + t_n = 1\),
    the following inequality holds:
    \[
        f(t_1 x_1 + t_2 x_2 + \cdots + t_n x_n) \leqslant t_1 f(x_1) + t_2 f(x_2) + \cdots + t_n f(x_n).
    \]

    Specially, when \(t_1 = t_2 = \cdots = t_n = \frac{1}{n}\), it reduces to Definition 3.
\end{theorem}

\vspace{0.7cm}
Next, we present derivative-based criteria for monotonicity and convexity:
\begin{theorem}
    \begin{enumerate}
        \item If \(f \in D(I)\), then \(f\) is increasing (decreasing) on \(I\) 
            if and only if \(f'(x) \geq 0\) (\(f'(x) \leqslant 0\)) for all \(x \in I\). 
        \item If \(f \in D^{(2)}(I)\), then \(f\) is convex (concave) on \(I\) 
            if and only if \(f''(x) \geq 0\) (\(f''(x) \leqslant 0\)) for all \(x \in I\).
    \end{enumerate}
\end{theorem}

\begin{note}
    If \(f'(x) > 0\) (\(f''(x) > 0\)) for all \(x \in I\), then \(f\) is strictly increasing (convex) on \(I\).
    Even though the condition weakens to holding except at finitely many points,
    the conclusion of strict monotonicity (convexity) still holds.
    For example, \(f(x) = x^3\) is strictly increasing on \(\mathbb{R}\)
    despite \(f'(0) = 0\).
\end{note}



\begin{leftbarTitle}{Argmax and Argmin}\end{leftbarTitle}

\begin{definition}{Stationary Point}
    Stationary points are points where the first derivative of a function is \underline{zero or non-existent}.
\end{definition}
Stationary points can be classified into three types:
\begin{description}
    \item[Argmax and argmin points] Points where the function attains its maximum or minimum values.
    \item[Inflection points] Points where the function changes concavity.
    \item[Trivial points] Points that are neither local maxima nor local minima.
\end{description}


\begin{leftbarTitle}{Asymptote}\end{leftbarTitle}







\section{Applications}


