\chapter{Introduction to Surface Theory} % 曲面论导论
\section{Parameterization of Surface} % 参数化曲面

\begin{definition}{Parameterization of Surface}\label{def:Parameterization of Surface}
    Let \( \Delta \) be an open subset in \( \mathbb{R}^s \), 
    and \( \mathbf{r}: \Delta \to \mathbb{R}^n \) be a mapping, 
    where \( \mathbf{u} = (u_1, u_2, \dots, u_s) \to \mathbf{x}(\mathbf{u}) = 
    (x_1(u_1, u_2, \dots, u_s), x_2(u_1, u_2, \dots, u_s), \dots, x_n(u_1, u_2, \dots, u_s)) \). 
    Then \( M = \mathbf{r}(\Delta) = \{ \mathbf{r}(\mathbf{u}) \mid \mathbf{u} \in \Delta \} \) 
    is called an \( s \)-dimensional \textbf{surface (patch)}, and \( \mathbf{r}(\mathbf{u}) \) 
    is referred to as the parameterization of \( M \). 
    
    When \( \mathbf{r}(\mathbf{u}) \in C^k \) (\( k \geq 0 \)), 
    \( \mathbf{r} \) or \( M \) is called an \( s \)-dimensional \( C^k \) surface. 
    
    If \( \mathbf{r} \in C^k \) (\( k \geq 1 \)), \( \mathbf{r} \) or \( M \) 
    is called an \textbf{\( s \)-dimensional \( C^k \) smooth surface}. 
    
    When
    \[
    \operatorname{rank}(r'_1(\mathbf{u}^0), r'_2(\mathbf{u}^0), \dots, r'_s(\mathbf{u}^0)) = 
    \operatorname{rank}
    \begin{pmatrix}
    \frac{\partial r_1}{\partial u_1} & \cdots & \frac{\partial r_1}{\partial u_s} \\
    \vdots & \ddots & \vdots \\
    \frac{\partial r_n}{\partial u_1} & \cdots & \frac{\partial r_n}{\partial u_s}
    \end{pmatrix}_{\mathbf{u}^0}
    = s,
    \]
    we call \( \mathbf{u}^0 \) or \( \mathbf{r}(\mathbf{u}^0) \) a \textbf{regular point} of the surface \( M \). 
    Otherwise, it is called a singular point. 
    
    Every point that is a regular point of the surface is referred to as an \textbf{\( s \)-dimensional \( C^k \) regular surface}. 
    
    At regular points, \( \{r'_1, \dots, r'_s\} \) are linearly independent.
\end{definition}

When \( s = 1 \), \( t \) represents the parameter, a one-dimensional surface is commonly referred to as a curve. 
Considering a \( C^k \) (\( k \geq 1 \)) curve \( \mathbf{r}(t) \), we have:
\[
\mathbf{r}'(t) = \left( r_{1}'(t), r_{2}'(t), \cdots, r_{n}'(t) \right) .
\]
If \( t \) is a regular point, then 
\( \operatorname{rank}(\mathbf{r}'(t)) = \text{rank}(r'_1(t), r'_2(t), \dots, r'_n(t)) = 1 \); 
this is equivalent to \( \mathbf{r}'(t) \neq 0 \), which means \( r'_1(t), r'_2(t), \dots, r'_n(t) \) are not all zero.

We refer to \( \mathbf{r}'(t) \) as the tangent vector of the curve \( \mathbf{r}(t) \) at point \( t \). 
When \( t \) varies, a tangent vector field along the curve \( \mathbf{r}(t) \) is obtained. 
If \( \mathbf{r}(t) \) is a regular curve, 
\( \frac{\mathbf{r}'(t)}{\|\mathbf{r}'(t)\|} \) is the unit tangent vector field along the curve \( \mathbf{r}(t) \). 
It should be emphasized that \( \mathbf{r}'(t) \) or \( \frac{\mathbf{r}'(t)}{\|\mathbf{r}'(t)\|} \) 
always points outward from point \( t \).


\section{Tangent Space and Normal Space} % 切空间与法空间
% 切空间与法空间
\begin{definition}{Tangent Space and Normal Space}
    \(M\) is an \(s\)-dimensional smooth surface in \(\mathbb{R}^n\) defined above,
    and \(\mathbf{u}^{0}\) is a regular point of \(M\).
    The \textbf{tangent space} of \(M\) at point \(\mathbf{r}(\mathbf{u}^{0})\) is the linear space spanned 
    by \(s\) tangent vectors:
    \[
    T_{\mathbf{u}^{0}}M = \operatorname{span}\{ r'_{1}(\mathbf{u}^{0}), r'_{2}(\mathbf{u}^{0}), \dots, r'_{s}(\mathbf{u}^{0}) \}.
    \]
    Accordingly, the \textbf{normal space} of \(M\) at point \(\mathbf{r}(\mathbf{u}^{0})\) is 
    the orthogonal complement of the tangent space:
    \[
    N_{\mathbf{u}^{0}}M = (T_{\mathbf{u}^{0}}M)^{\perp}.
    \]
\end{definition}
% 给出特殊情况的几个切空间法空间表达式
Some special cases of tangent space and normal space expressions are given below:
\begin{leftbarTitle}{Curve}\end{leftbarTitle}
When \( n = 3, s = 1 \), \( M \) is a curve in three-dimensional space. 
% 三维空间中的曲线: 切线与法平面
\begin{enumerate}
    \item If the curve is parameterized as 
        \[
        \mathbf{r}(t) = (x(t), y(t), z(t)), \quad t \in I \subseteq \mathbb{R}.
        \]
        At the regular point \( \mathbf{r}(t^{0})= (x(t^{0}), y(t^{0}), z(t^{0})) \), the tangent line and normal plane are:
        \[
        T_{t^{0}}M = \operatorname{span}\{ \mathbf{r}'(t^{0}) \}: \frac{x-x(t^{0})}{x'(t^{0})} 
            = \frac{y-y(t^{0})}{y'(t^{0})} = \frac{z-z(t^{0})}{z'(t^{0})},
        \]
        \begin{align*}
            N_{t^{0}}M:\quad &x'(t^{0})(x-x(t^{0})) + y'(t^{0})(y-y(t^{0})) + z'(t^{0})(z-z(t^{0})) = 0 \\
            \Leftrightarrow& \mathbf{r}'(t^{0}) \cdot (\mathbf{r} - \mathbf{r}(t^{0})) = 0.
        \end{align*}
    \item If the curve is described by:
        \[
        \begin{cases}
            F(x, y, z) = 0, \\
            G(x, y, z) = 0,
        \end{cases}
        \]
        and the regular point is \( \mathbf{x}^{0} = (x^{0}, y^{0}, z^{0}) \).
        \newline For the Jacobian matrix:
        \[
            J = 
            \begin{pmatrix}
                F_x(\mathbf{x}^{0}) & F_y(\mathbf{x}^{0}) & F_z(\mathbf{x}^{0}) \\
                G_x(\mathbf{x}^{0}) & G_y(\mathbf{x}^{0}) & G_z(\mathbf{x}^{0})
            \end{pmatrix},
        \]
        since \(\operatorname{rank}J = 2\), without loss of generality, assume:
        \[
        \frac{\partial (F, G)}{\partial (y, z)} =
        \begin{vmatrix}
            F_y(\mathbf{x}^{0}) & F_z(\mathbf{x}^{0}) \\
            G_y(\mathbf{x}^{0}) & G_z(\mathbf{x}^{0})
        \end{vmatrix} \neq 0.
        \]
        By the implicit mapping theorem (\ref{thm:Implicit Mapping Theorem}), we can express:
        \[
        y = f(x), \quad z = g(x), \quad x \in U(x^{0}) \subseteq \mathbb{R}.
        \]
        Then 
        \[
        f'(x^{0}) = \frac{\frac{\partial (F, G)}{\partial (z, x)}(\mathbf{x}^{0})}
        {\frac{\partial (F, G)}{\partial (y, z)}(\mathbf{x}^{0})}, \quad
        g'(x^{0}) = \frac{\frac{\partial (F, G)}{\partial (x, y)}(\mathbf{x}^{0})}
        {\frac{\partial (F, G)}{\partial (y, z)}(\mathbf{x}^{0})}.
        \]
        Therefore, the tangent line and normal plane at point \(\mathbf{x}^{0}\) are:
        \begin{gather*}
            T_{x^{0}}M: \quad
            \frac{x - x^{0}}{1} = \frac{y - y^{0}}{f'(x^{0})} = \frac{z - z^{0}}{g'(x^{0})} 
            \Leftrightarrow \frac{x-x^{0}}{\frac{\partial (F, G)}{\partial (y, z)}(\mathbf{x}^{0})} 
            = \frac{y - y^{0}}{\frac{\partial (F, G)}{\partial (z, x)}(\mathbf{x}^{0})}
            = \frac{z - z^{0}}{\frac{\partial (F, G)}{\partial (x, y)}(\mathbf{x}^{0})}, \\
            N_{x^{0}}M: \quad
            \frac{\partial (F, G)}{\partial (y, z)}(\mathbf{x}^{0})(x - x^{0}) +
            \frac{\partial (F, G)}{\partial (z, x)}(\mathbf{x}^{0})(y - y^{0}) +
            \frac{\partial (F, G)}{\partial (x, y)}(\mathbf{x}^{0})(z - z^{0}) = 0.
        \end{gather*}
\end{enumerate}

\begin{leftbarTitle}{Surface}\end{leftbarTitle}
% 三维空间中的曲面: 切平面与法线
When \( n = 3, s = 2 \), \( M \) is a surface in three-dimensional space. 
\begin{enumerate}
    \item If the surface can be described explicitly as:
        \[
        z = f(x, y), \quad (x, y) \in D \subseteq \mathbb{R}^2,
        \]
        at the regular point \( \overline{\mathbf{x}}^{0} = (x^{0}, y^{0}, z^{0}) \) (\(\mathbf{x}^{0}=(x^{0}, y^{0})\)), 
        the tangent plane and normal line are:
        \begin{gather*}
            T_{\mathbf{x}^{0}}M: \quad z - z^{0} = f_x(\mathbf{x}^{0})(x - x^{0}) + f_y(\mathbf{x}^{0})(y - y^{0}), \\
            N_{\mathbf{x}^{0}}M: \quad \frac{x - x^{0}}{f_x(\mathbf{x}^{0})} 
            = \frac{y - y^{0}}{f_y(\mathbf{x}^{0})} = \frac{z - z^{0}}{-1},
        \end{gather*}
        where the expression of \(T_{\mathbf{x}^{0}}M\) is derived from 
        the total differential of \(z = f(x, y)\) at point \(\mathbf{x}^{0}\):
        \[
        \mathrm{d}z = f_x(\mathbf{x}^{0}) \mathrm{d}x + f_y(\mathbf{x}^{0}) \mathrm{d}y.
        \]

    \item If the surface is parameterized as 
        \[
        \mathbf{r}(u, v) = (x(u, v), y(u, v), z(u, v)), \quad (u, v) \in D \subseteq \mathbb{R}^2,
        \]
        at the regular point \( \mathbf{x}^{0} = (x^{0}, y^{0}, z^{0}) \) . 
        \newline For the Jacobian matrix:
        \[
        J = 
        \begin{pmatrix}
            x_{u}(\mathbf{x}^{0}) & x_{v}(\mathbf{x}^{0})  \\
            y_{u}(\mathbf{x}^{0}) & y_{v}(\mathbf{x}^{0})  \\
            z_{u}(\mathbf{x}^{0}) & z_{v}(\mathbf{x}^{0})
        \end{pmatrix},
        \]
        since \(\operatorname{rank}J = 2\), without loss of generality, assume:
        \[
        \frac{\partial (x, y)}{\partial (u, v)}(\mathbf{x}^{0}) = 
        \begin{vmatrix}
            x_{u}(\mathbf{x}^{0}) & x_{v}(\mathbf{x}^{0}) \\
            y_{u}(\mathbf{x}^{0}) & y_{v}(\mathbf{x}^{0})
        \end{vmatrix} \neq 0.
        \]
        By the inverse mapping theorem (\ref{thm:Inverse Mapping Theorem}), we can determine
        the inverse mapping of
        \[
        \begin{cases} x=x(u,v), &  \\ y=y(u,v), &  \end{cases}
        \]
        in a neighborhood of point \(\mathbf{x}^{0}\):
        \[
        \begin{cases} u = u(x, y), &  \\ v = v(x, y), &  \end{cases}
        \]
        where \(u^{0} = u(x^{0}, y^{0})\), \(v^{0} = v(x^{0}, y^{0})\).
        Then we obtain the explicit representation of the surface:
        \[
        z = z(u(x, y), v(x, y)) = f(x, y), \quad (x, y) \in U(x^{0}, y^{0}) \subseteq \mathbb{R}^2.
        \]
        Therefore, the tangent plane and normal line at point \(\mathbf{x}^{0}\) are:
        \begin{gather*}
            T_{\mathbf{x}^{0}}M: \quad 
            \left. \frac{\partial (y,z)}{\partial (u,v)} \right|_{\left(u^{0}, v^{0}\right)}\left( x- x^{0} \right)
            + \left. \frac{\partial (z,x)}{\partial (u,v)} \right|_{\left(u^{0}, v^{0}\right)}\left( y- y^{0} \right)
            + \left. \frac{\partial (x,y)}{\partial (u,v)} \right|_{\left(u^{0}, v^{0}\right)}\left( z- z^{0} \right) = 0, \\
            N_{\mathbf{x}^{0}}M: \quad \frac{x - x^{0}}{\left. \frac{\partial (y,z)}{\partial (u,v)} \right|_{\left(u^{0}, v^{0}\right)}}
            = \frac{y - y^{0}}{\left. \frac{\partial (z,x)}{\partial (u,v)} \right|_{\left(u^{0}, v^{0}\right)}}
            = \frac{z - z^{0}}{\left. \frac{\partial (x,y)}{\partial (u,v)} \right|_{\left(u^{0}, v^{0}\right)}}.
        \end{gather*}
\end{enumerate}




\section{Intrinsic Geometry} % 内在几何
This two sections will introduce the first and second fundamental forms of surfaces,
which can be all generalized to higher-dimensional manifolds;
here, we only discuss the case of two-dimensional surfaces in three-dimensional space.

Let \(\Delta \in \mathbb{R}^{2}\) be an open set,
and \(\mathbf{r}:\Delta \to \mathbb{R}^{3}\) be a \(C^{k}\) (\(k\geqslant 2\)) smooth regular surface parameterization,
\(M = \mathbf{r}(\Delta)\),
where \(\mathbf{u} = (u, v) \to \mathbf{r}(u, v) = (x(u, v), y(u, v), z(u, v))\).
We can obtain that:
\begin{enumerate}
    \item \(\mathbf{r}\in C^{k}(\Delta, \mathbb{R}^{3})\);
    \item For any \(p=(u, v) \in \Delta\),
        \(\operatorname{rank}(\mathbf{r}'_{u}(u, v), \mathbf{r}'_{v}(u, v)) = 2\),
        that is, \(\mathbf{r}'_{u}(u, v)\) and \(\mathbf{r}'_{v}(u, v)\) are linearly independent,
        where
        \[
        \mathbf{r}'_{u}(u, v) = \left( \frac{\partial x}{\partial u}, 
        \frac{\partial y}{\partial u}, \frac{\partial z}{\partial u} \right), \quad
        \mathbf{r}'_{v}(u, v) = \left( \frac{\partial x}{\partial v}, 
        \frac{\partial y}{\partial v}, \frac{\partial z}{\partial v} \right).
        \]
\end{enumerate}
At this time, the tangent space \(T_{p}M = \operatorname{span}(\mathbf{r}'_{u}(u, v), \mathbf{r}'_{v}(u, v))\),
which is a subspace of \(\mathbb{R}^{3}\).
Hence, it inherits the inner product from \(\mathbb{R}^{3}\).

The first fundamental form is the metric that a surface inherits from its ambient Euclidean space \(\mathbb{R}^{3}\). 
It is essentially a symmetric positive-definite bilinear form defined on the tangent space, 
which allows us to \underline{measure lengths, angles, and areas on the surface}.

\begin{definition}{The First Fundamental Form}
    In the above conditions,
    for any point \(p = (u, v) \in \Delta\),
    the \textbf{first fundamental form} of the surface \(M\) at point \(p\) is defined as:
    for any tangent vector \(\mathbf{w}_{1}, \mathbf{w}_{2} \in T_{p}M\),
    \[
    \mathrm{I}_{p}(\mathbf{w}_{1}, \mathbf{w}_{2}) := \mathbf{w}_{1} \cdot \mathbf{w}_{2},
    \]
    which is a symmetric positive-definite bilinear form on the tangent space \(T_{p}M\).
    This form is also called the \textbf{Riemann metric} of \textbf{metric tensor}, denoted as \(\mathrm{I}_{p}\) or \(g_{p}\).
\end{definition}
For convenience, we express \(\mathrm{I}_{p}\) in the basis \(\{\mathbf{r}'_{u}, \mathbf{r}'_{v}\}\) 
of the tangent space \(T_{p}M\).
Define:
\begin{align*}
    &E(u, v) := \mathrm{I}_{p}(\mathbf{r}_{u}, \mathbf{r}_{u}) = \mathbf{r}_{u} \cdot \mathbf{r}_{u} = \| \mathbf{r}_{u} \|^2, \\
    &F(u, v) := \mathrm{I}_{p}(\mathbf{r}_{u}, \mathbf{r}_{v}) = \mathbf{r}_{u} \cdot \mathbf{r}_{v}, \\
    &G(u, v) := \mathrm{I}_{p}(\mathbf{r}_{v}, \mathbf{r}_{v}) = \mathbf{r}_{v} \cdot \mathbf{r}_{v} = \| \mathbf{r}_{v} \|^2,
\end{align*}
which are called the \textbf{Gauß coefficients}.
\newline Then the matrix representation of the first fundamental form \(\mathrm{I}_{p}\) under the basis \(\{\mathbf{r}'_{u}, \mathbf{r}'_{v}\}\) is:
\[
\mathrm{I}_{p} =
\begin{pmatrix}
    E & F \\
    F & G
\end{pmatrix},
\]
which is symmetric and positive-definite.

The quadratic form corresponding to this bilinear form is also commonly called the first fundamental form, 
denoted as  \(\mathrm{d}s^{2}\). 
For a tangent vector \(\mathbf{w}\in T_{p}S\), it represents the square of the length of that vector:
\[
\mathrm{d}s^{2} := \mathrm{I}_{p}(\mathbf{w}, \mathbf{w}) = \| \mathbf{w} \|^{2}.
\]
If \(\mathbf{w}\) is the tangent vector to the curve \(\gamma(t)=\mathbf{r}(u(t), v(t))\), 
given by \(\gamma'(t)=\mathbf{r}_{u}u'(t)+\mathbf{r}_{v}v'(t)\), 
then \(\mathrm{d}s^{2}\) is conventionally written as: 
\[
\mathrm{d}s^{2} = E \, \mathrm{d}u^{2} + 2F \, \mathrm{d}u \, \mathrm{d}v + G \, \mathrm{d}v^{2}.
\]
Here,\(\mathrm{d}u\) and \(\mathrm{d}v\) are the coordinates under the basis \(\{ \mathrm{d}u, \mathrm{d}v \}\) , 
representing the components of the tangent vector \((u', v')\). 
This is a long-standing notation, and strictly speaking, 
it represents the value of the quadratic form on the vector \((u', v')\).




\begin{leftbarTitle}{Arc Length}\end{leftbarTitle} % 弧长
\begin{definition}{Arc Length}
    Let \(C = \overset{\frown}{AB}\) be a curve on the \(\mathbb{R}^{2}\) plane\footnote{
        Or in \(\mathbb{R}^{3}\) space, even in a higher-dimensional Euclidean space.
    },
    take any partition \( A = P_{0}, P_{1}, \ldots, P_{n} = B \),
    which divides the curve \(C\) into \(n\) segments, denoted as \(T\).
    Then connect every two adjacent points \(P_{i-1}\) and \(P_{i}\) with a straight line segment,
    obtaining \(n\) chords \(\overline{P_{i-1}P_{i}}\)(\(i=1, 2, \ldots, n\)),
    which in turn form an inscribed polygonal line \(C\).
    Let 
    \[
    \|T\| = \max_{1 \leqslant i \leqslant n} \|P_{i-1}P_{i}\|, \quad s_{T}= \sum_{i=1}^{n} \|P_{i-1}P_{i}\|.
    \]
    If the limit
    \[
    \lim_{\|T\| \to 0} s_{T} = s,
    \] 
    namely, 
    \[
    \forall \varepsilon > 0, \exists \delta > 0, \text{s.t.} \forall T(\|T\| < \delta): |s_{T} - s| < \varepsilon,
    \]
    and the limit is independent of the choice of partition \(T\),
    then \(C\) is said to be rectifiable, and the limit \(s\) is called the arc length of the curve \(C\).
\end{definition}

\begin{theorem}{Sufficient Condition for Rectifiability of Curves}
    Let the curve \(C\) in \(\mathbb{R}^{2}\) be given by the parametric equations 
    \[
    (x, y) = (x(t), y(t)), \quad t \in [\alpha, \beta],
    \] 
    and let it be a \(C^{1}\) smooth regular curve\footnote{
        I.e., \(x(t)\) and \(y(t)\) are continuously differentiable, and \(x'^{2}(t) + y'^{2}(t) \neq 0\); 
        a curve \(C\) satisfying this condition is called a regular point.
        Also see Definition~\ref{def:Parameterization of Surface}
    }
    Then \(C\) is rectifiable, 
    and its arc length is 
    \[
    s = \int_{\alpha}^{\beta} \sqrt{x'^{2}(t) + y'^{2}(t)} \, \mathrm{d}t.
    \]
\end{theorem}

\begin{leftbarTitle}{Area}\end{leftbarTitle} % 面积
For convenience, we study the area of a surface patch \(M\) parameterized by \(\mathbf{r}(u, v): \Delta \to \mathbb{R}^3\)
over the domain \(\Delta \subseteq \mathbb{R}^{2}\).

% 尝试用内接多边形曲面定义面积, 但是 Schwarz 的反例说明该定义不成立
Similar to the definition of arc length,
we try to define the area of surface patch \(M\) by approximating it with inscribed polygonal surfaces.
However, this definition does not hold, as demonstrated by Schwarz's counterexample
that is called \textbf{Schwartz's lantern} vividly.
% 在这个反例中, 我们能得到圆周率等于 4 的错误结论

In this counterexample, we can obtain the incorrect conclusion that \(\pi = 4\).
Here is a brief description of the construction of Schwartz's lantern:

Consider a cylinder with height \(1\) and base radius \(\frac{1}{2}\) (the left in Fig.~\ref{fig:Schwartz's lantern}).
Its lateral surface area is \(2\pi rh = 2\pi \times \frac{1}{2} \times 1 = \pi\).

Divide the cylinder into four equal cylinders, and place seven equally spaced red dots on each circle 
(the middle in Fig.~\ref{fig:Schwartz's lantern}).
Then connect these red dots to form a polygonal surface (the right in Fig.~\ref{fig:Schwartz's lantern}).
\begin{figure}[h]
    \centering
    \includegraphics[width=0.6\textwidth]{img/Schwartz1.png}
    \caption{Schwartz's lantern construction on a cylinder.}
    \label{fig:Schwartz's lantern}
\end{figure}
This particular lantern has \(4\) horizontal triangular bands, 
and on each level there are \(7\) equally spaced red dots, 
which can be expressed as \(b = 4, p = 7\).

To obtain increasingly precise approximations of the cylindrical lanterns, 
simply increase the number of bands and points
(Fig.~\ref{fig:More precise approximations of Schwartz's lantern}).
\begin{figure}[h]
    \centering
    \includegraphics[width=0.8\textwidth]{img/Schwartz2.png}
    \caption{More precise approximations of Schwartz's lantern.}
    \label{fig:More precise approximations of Schwartz's lantern}
\end{figure}

In fact, we can assign particular values to \(b\) and \(p\)
to make the area of the polygonal surface approach any value greater than \(\pi\),
\[
A(b, p) = \frac{bp}{2} \sin\frac{\pi}{p} \sqrt{\left( \frac{2}{b} \right)^2 + \left( \sin\frac{\pi}{p} \right)^2}.
\]

\vspace{0.7cm}
???????????????????????????????????????????????????????????????????

We derive the area of \(M\) using the first fundamental form.
Consider a small rectangle \(\Delta u \times \Delta v\) in the parameter domain \(\Delta\),
which is mapped to a small parallelogram on the surface \(M\) by the parameterization \(\mathbf{r}(u, v)\).
The two adjacent sides of this parallelogram can be approximated by the tangent vectors:
\[\mathbf{r}_{u} \Delta u, \quad \mathbf{r}_{v} \Delta v.\]
The area of this parallelogram is given by the magnitude of the cross product of these two vectors:
\[\|\mathbf{r}_{u} \times \mathbf{r}_{v}\| \Delta u \Delta v.\]
Using the properties of the dot product and the first fundamental form, we have:
\[\|\mathbf{r}_{u} \times \mathbf{r}_{v}\| = \sqrt{EG - F^{2}}.\]
Therefore, the area element \(\mathrm{d}A\) on the surface \(M\) is:
\[\mathrm{d}A = \sqrt{EG - F^{2}} \,\mathrm{d}u \, \mathrm{d}v.\]
Integrating over the entire parameter domain \(\Delta\), we obtain the total area of the surface patch \(M\):
\[\text{Area}(M) = \iint_{\Delta} \sqrt{EG - F^{2}} \,\mathrm{d}u \, \mathrm{d}v.\]
???????????????????????????????????????????????????????????????????

\section{Extrinsic Geometry} % 外在几何
The second fundamental form is a symmetric bilinear form defined on the tangent space 
that measures the change in the normal vector of a surface, 
thereby describing the \underline{extrinsic curvature} of the surface relative to its ambient space \(\mathbb{R}^3\).

On the regular surface patch \(M\) defined in the beginning of last section, 
we can define a continuous unit normal vector field \(\mathbf{n}: M \to \mathbb{S}^2\), 
where \(\mathbb{S}^2\) is the unit sphere in \(\mathbb{R}^3\):
\[
\mathbf{n}(p) = \frac{\mathbf{r}_u \times \mathbf{r}_v}{\|\mathbf{r}_u \times \mathbf{r}_v\|}(p).
\]
This mapping \(\mathbf{n}\) from the surface to the unit sphere is called the \textbf{Gauß map}. 
The second fundamental form is defined by studying the differential of the Gauß map.

\begin{definition}{The Second Fundamental Form}
    Under the above conditions,
    for any point \(p = (u, v) \in \Delta\),
    the \textbf{second fundamental form} of the surface \(M\) at point \(p\) is 
    a symmetric bilinear form on the tangent space \(T_{p}M\),
    which is defined as:
    for any tangent vector \(\mathbf{w}_{1}, \mathbf{w}_{2} \in T_{p}M\),
    \[
    \mathrm{II}_{p}(\mathbf{w}_{1}, \mathbf{w}_{2}) := -\mathrm{d}_{p}\mathbf{n}(\mathbf{w}_{1}) \cdot \mathbf{w}_{2},
    \]\footnote{ 
        About the formula,
        \begin{itemize}
            \item since \(\mathbf{n}(p)\) is a unit vector, \(T_{\mathbf{n}(p)}\mathbb{S}^{2}\) is 
                the plane orthogonal to \(\mathbf{n}(p)\),
                and \(T_{p}M\) itself is also orthogonal to  \(\mathbf{n}(p)\), 
                it follows that  \(\mathrm{d}_{p}\mathbf{n}(\mathbf{w}_1)\)  and  \(\mathbf{w}_2\)  lie in the same plane, 
                and their dot product is well-defined.
            \item the negative sign in this definition is a convention, 
                which makes the principal curvatures of a convex surface (like a sphere) positive.
        \end{itemize}
    }
    where \(\mathrm{d}_{p}\mathbf{n}: T_{p}M \to T_{\mathbf{n}(p)}\mathbb{S}^{2}\) is the differential (or Jacobian)
    of the Gauß map at point \(p\).

    The linear operator associated with \(\mathrm{d}_{p}\mathbf{n}\), defined as  
    \(W_{p}(\mathbf{w}) = -\mathrm{d}_{p}\mathbf{n}(\mathbf{w})\), 
    is called the Weingarten map or shape operator, 
    and it is a linear operator from  \(T_{p}M\)  to itself. Therefore, the second fundamental form can also be written as:
    \[
    \mathrm{II}_{p}(\mathbf{w}_{1}, \mathbf{w}_{2}) = W_{p}(\mathbf{w}_{1}) \cdot \mathbf{w}_{2}.
    \]
\end{definition}
For convenience, we express \(\mathrm{II}_{p}\) in the basis \(\{\mathbf{r}'_{u}, \mathbf{r}'_{v}\}\) of 
the tangent space \(T_{p}M\).
Define:
\begin{align*}
    &L(u, v):=\mathrm{II}_{p}(\mathbf{r}_{u}, \mathbf{r}_{u}) = 
    W_{p}(\mathbf{r}_{u}) \cdot \mathbf{r}_{u} = \mathbf{r}_{uu} \cdot \mathbf{n}; \\
    &M(u, v):=\mathrm{II}_{p}(\mathbf{r}_{u}, \mathbf{r}_{v}) = 
    W_{p}(\mathbf{r}_{u}) \cdot \mathbf{r}_{v} = \mathbf{r}_{uv} \cdot \mathbf{n}; \\
    &N(u, v):=\mathrm{II}_{p}(\mathbf{r}_{v}, \mathbf{r}_{v}) = 
    W_{p}(\mathbf{r}_{v}) \cdot \mathbf{r}_{v} = \mathbf{r}_{vv} \cdot \mathbf{n}.
\end{align*}
Then the matrix representation of the second fundamental form \(\mathrm{II}_{p}\) 
under the basis \(\{\mathbf{r}'_{u}, \mathbf{r}'_{v}\}\) is:
\[
\mathrm{II}_{p} =
\begin{pmatrix}
    L & M \\
    M & N
\end{pmatrix},
\]
which is symmetric, but not necessarily positive-definite.
And its sign reflects the way the surface is curved. 

The associated second fundamental form, also denoted by \(\mathrm{II}\), is an expression for the normal curvature:
\[
\mathrm{II} = L \, \mathrm{d}u^2 + 2M \, \mathrm{d}u \, \mathrm{d}v + N \, \mathrm{d}v^2.
\]
For a unit tangent vector  \(\mathbf{w} \in T_{p}M\), 
the value of  \(\mathrm{II}_{p}(\mathbf{w}, \mathbf{w})\)  
is the normal curvature of the surface in the direction of  \(\mathbf{w}\), denoted \(k_n(\mathbf{w})\).

\begin{leftbarTitle}{Curvature}\end{leftbarTitle}
Curvature is a mathematical quantity describing the "bending" degree of a geometric object, 
such as a curve or a surface. 

The meaning of curvature varies for geometric objects of different dimensions: 
\begin{itemize}
    \item Curvature on a curve: Describes the degree to which the curve deviates from a straight line.
    \item Description of curvature by a surface: Is more complex, involving directionality—the bending of 
        a surface can be completely different in different directions. 
\end{itemize}
The curvature of a surface is usually classified into the following typical types: 
normal curvature, principal curvatures, mean curvature, Gaussian curvature, etc.

\begin{definition}{Curvature of Curve}
    Let \(C\) be a \(C^{2}\) smooth regular curve in \(\mathbb{R}^{3}\),
    parameterized by arc length \(t\):
    \[
    \mathbf{r}(t) = (x(t), y(t), z(t)), \quad t \in [a, b].
    \]
    The unit tangent vector of the curve at point \(t\) is:
    \[
    \mathbf{T}(t) = \mathbf{r}'(t) = (x'(t), y'(t), z'(t)).
    \]
    The \textbf{curvature} of the curve at point \(t\) is defined as the magnitude of the derivative of the unit tangent vector with respect to arc length:
    \[
    \kappa(t) = \left\| \frac{\mathrm{d}\mathbf{T}(t)}{\mathrm{d}t} \right\| = 
    \frac{\|\mathbf{r}'(t) \times \mathbf{r}''(t)\|}{\|\mathbf{r}'(t)\|^3}.
    \]
    Geometrically, curvature measures how quickly the curve changes direction at point \(t\).

    If the best-fit circle is found based on the tangent and normal at a certain point, 
    the radius of this circle is called the radius of curvature \(R\), 
    and the curvature is its reciprocal: 
    \[
    \kappa = \frac{1}{R}.
    \] 
    This fitted circle is called the \textbf{osculating circle} of the curve at that point. % 渐屈圆
\end{definition}
Some special cases of curvature are given below:
\begin{enumerate}
    \item For a plane curve given by \(y = f(x)\), the curvature at point \(x\) is:
        \[
        \kappa(x) = \frac{|f''(x)|}{(1 + (f'(x))^2)^{3/2}}.
        \]
    \item For a circle with radius \(R\), the curvature is constant:
        \[
        \kappa = \frac{1}{R}.
        \]
\end{enumerate}


\section{Oriented Surface} % 定向曲面

\section{Bounded Variation Functions} % 有界变差函数
\begin{definition}{Bounded Variation}
    Let \(f: [a, b] \to \mathbb{R}\) be a real-valued function defined on the closed interval \([a, b]\).
    For any partition \(P = \{ x_0, x_1, \ldots, x_n \}\) of \([a, b]\) with \(a = x_0 < x_1 < \cdots < x_n = b\),
    define the variation of \(f\) on the partition \(P\) as:
    \[
    V(f, P) = \sum_{i=1}^{n} |f(x_i) - f(x_{i-1})|.
    \]
    The total variation of \(f\) on \([a, b]\) is defined as:
    \[
    V_a^b(f) = \sup_{P} V(f, P),
    \]
    where the supremum is taken over all possible partitions \(P\) of \([a, b]\).
    If \(V_a^b(f) < \infty\), then \(f\) is said to be of \textbf{bounded variation} on \([a, b]\),
    denoted as \(f \in BV[a, b]\).
\end{definition}
\begin{property}
    \begin{enumerate}
        \item \(BV[a, b]\subset B[a, b]\)
        \item For any \(f, g\in BV[a, b]\), 
            and any scalars \(\alpha, \beta \in \mathbb{R}\), 
            the linear combination \(\alpha f + \beta g \in BV[a, b]\),
            and 
            \[
            V_a^b(\alpha f + \beta g) \leqslant |\alpha| V_a^b(f) + |\beta| V_a^b(g).
            \]
            Specially, if \(|g(x)|\geqslant \sigma > 0\), then \(\frac{f(x)}{g(x)} \in BV[a, b]\).
        \item 
    \end{enumerate}
\end{property}

Some common bounded variation functions include:
\begin{description}
    \item[Monotonic Functions] Any monotonic function on a closed interval is of bounded variation,
        and \(V_a^b(f) = |f(b) - f(a)|\).
    \item[Piecewise Monotonic Functions] Functions that are monotonic on each subinterval of a finite partition of \([a, b]\)
        are also of bounded variation.
    \item[Lipschitz Continuous Functions] Any Lipschitz continuous function on \([a, b]\) is of bounded variation.
    \item[Functions with Finite Discontinuities] Functions that have only a finite number of jump discontinuities on \([a, b]\)
        are of bounded variation.
    \item[Absolutely Continuous Functions] Any absolutely continuous function on \([a, b]\) is of bounded variation.
\end{description}


\begin{theorem}{Jordan Decomposition Theorem} % Jordan 分解定理
    \(f\in BV[a, b]\) if and only if there exist two monotonic increasing functions
    \(g(x),  h(x) : [a, b] \to \mathbb{R}\), such that:
    \[
    f(x) = g(x) - h(x).
    \]
\end{theorem}

\vspace{0.7cm}
Bounded variation functions have important applications,
for example, in harmonic analysis, the Fourier series of bounded variation functions 
converge pointwise.
Other typical applications are as follows:
\begin{leftbarTitle}{Rectifiable Curves}\end{leftbarTitle} % 可求长曲线
\begin{theorem}{Jordan's Theorem on Rectifiable Curves}
    A curve \(C\) in \(\mathbb{R}^{2}\) defined by the parametric equations
    \[
    (x, y) = (x(t), y(t)), \quad t \in [a, b],
    \]
    is rectifiable if and only if both \(x(t)\) and \(y(t)\) are of bounded variation on \([a, b]\).
\end{theorem}

\begin{leftbarTitle}{Stieltjes Integral}\end{leftbarTitle} % Stieltjes 积分


