\chapter{Preliminaries} % 预备知识
\section{Trigonometric Formulas} % 这里放置小节标题
% Trigonometric identities and formulas (Academic English LaTeX)

% Product-to-Sum (Product to Sum and Difference)
\textbf{Product-to-Sum Formulas:}
\begin{align*}
\sin\alpha \cos\beta &= \frac{1}{2} \left[ \sin(\alpha + \beta) + \sin(\alpha - \beta) \right] \\
\cos\alpha \sin\beta &= \frac{1}{2} \left[ \sin(\alpha + \beta) - \sin(\alpha - \beta) \right] \\
\cos\alpha \cos\beta &= \frac{1}{2} \left[ \cos(\alpha + \beta) + \cos(\alpha - \beta) \right] \\
\sin\alpha \sin\beta &= -\frac{1}{2} \left[ \cos(\alpha + \beta) - \cos(\alpha - \beta) \right]
\end{align*}

% Sum and Difference Formulas
\textbf{Sum and Difference Formulas:}
\begin{align*}
\sin(\alpha + \beta) &= \sin\alpha \cos\beta + \cos\alpha \sin\beta \\
\sin(\alpha - \beta) &= \sin\alpha \cos\beta - \cos\alpha \sin\beta \\
\cos(\alpha + \beta) &= \cos\alpha \cos\beta - \sin\alpha \sin\beta \\
\cos(\alpha - \beta) &= \cos\alpha \cos\beta + \sin\alpha \sin\beta
\end{align*}

% Sum-to-Product (Sum and Difference to Product)
\textbf{Sum-to-Product Formulas:}
\begin{align*}
\sin\alpha + \sin\beta &= 2 \sin\left( \frac{\alpha + \beta}{2} \right) \cos\left( \frac{\alpha - \beta}{2} \right) \\
\sin\alpha - \sin\beta &= 2 \sin\left( \frac{\alpha - \beta}{2} \right) \cos\left( \frac{\alpha + \beta}{2} \right) \\
\cos\alpha + \cos\beta &= 2 \cos\left( \frac{\alpha + \beta}{2} \right) \cos\left( \frac{\alpha - \beta}{2} \right) \\
\cos\alpha - \cos\beta &= -2 \sin\left( \frac{\alpha + \beta}{2} \right) \sin\left( \frac{\alpha - \beta}{2} \right)
\end{align*}

% Double Angle Formulas
\textbf{Double Angle Formulas:}
\begin{align*}
\sin 2\alpha &= 2\sin\alpha \cos\alpha \\
\cos 2\alpha &= \cos^2\alpha - \sin^2\alpha = 2\cos^2\alpha - 1 = 1 - 2\sin^2\alpha \\
\tan 2\alpha &= \frac{2\tan\alpha}{1 - \tan^2\alpha}
\end{align*}

% Half Angle Formulas
\textbf{Half Angle Formulas:}
\begin{align*}
\sin \frac{\alpha}{2} &= \pm \sqrt{ \frac{1 - \cos\alpha}{2} } \\
\cos \frac{\alpha}{2} &= \pm \sqrt{ \frac{1 + \cos\alpha}{2} } \\
\tan \frac{\alpha}{2} &= \frac{1 - \cos\alpha}{\sin\alpha} = \frac{\sin\alpha}{1 + \cos\alpha}
\end{align*}

% Power-Reducing Formulas
\textbf{Power-Reducing Formulas:}
\begin{align*}
\sin^2\alpha &= \frac{1 - \cos 2\alpha}{2} \\
\cos^2\alpha &= \frac{1 + \cos 2\alpha}{2}
\end{align*}

% Angle Decomposition Formulas
\textbf{Angle Decomposition Formulas:}
\begin{align*}
\sin^2\alpha - \sin^2\beta &= \sin(\alpha + \beta) \sin(\alpha - \beta) \\
\cos^2\alpha - \sin^2\beta &= \cos(\alpha + \beta) \cos(\alpha - \beta)
\end{align*}

\begin{figure}[h]
    \centering
    \includegraphics[width=0.4\textwidth]{img/triangle.png}
\end{figure}

% Geometric remarks
\begin{property}
    \begin{itemize}
        \item On the gray triangle, the sum of the squares of the two numbers above is equal to the square of the number below,
            for instance, \(\tan^{2} x + 1 = \sec^{2}x\)
        \item The three trigonometric functions in the clockwise direction have the following properties: 
            $\tan x = \frac{\sin x}{\cos x}$, etc.
    \end{itemize}
\end{property}

\begin{theorem}{Weierstrass Substitution (All-Powerful Formula)}
    Let \(t = \tan \frac{x}{2}\), then:
    \begin{align*}
        &\sin x = \frac{2t}{1+t^2}, \\
        &\cos x = \frac{1-t^2}{1+t^2}, \\
        &\mathrm{d}x = \frac{2}{1+t^2} \mathrm{d}t.
    \end{align*}
\end{theorem}


\section{Common Inequalities}

Some common inequalities:
\begin{gather*}
    \frac{x}{1+x} < \ln(1+x) < x, \quad x > 0; \\
\end{gather*}


\section{Factorial Power}
\begin{definition}
    Rising factorials and falling factorials can be expressed in multiple notations.

    The Pochhammer symbol, introduced by Leo August Pochhammer, is one of the commonly used notations, 
    represented as \( x^{(n)} \) or \( (x)_n \).

    Ronald Graham, Donald Ervin Knuth, and Oren Patashnik introduced the symbols 
    \( x^{\bar{n}} \) and \( x^{\underline{n}} \) in their book \textit{Concrete Mathematics}.

    \paragraph{Definitions:}
    \begin{itemize}
        \item \textbf{Rising factorial:}
        \[
        x^{\bar{n}} = x(x+1)(x+2)\dots(x+n-1) = \frac{(x+n-1)!}{(x-1)!}.
        \]
        \item \textbf{Falling factorial:}
        \[
        x^{\underline{n}} = x(x-1)(x-2)\dots(x-n+1) = \frac{x!}{(x-n)!}.
        \]
    \end{itemize}

    \paragraph{Relationships:}
    \begin{itemize}
        \item Relationship between rising and falling factorials:
        \[
        x^{\bar{n}} = (x+n-1)^{\underline{n}}.
        \]
        \item Relationship with factorial:
        \[
        1^{\bar{n}} = n^{\underline{n}} = n!.
        \]
    \end{itemize}
\end{definition}
