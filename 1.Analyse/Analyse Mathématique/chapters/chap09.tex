\chapter{Series of Functions} % 函数项级数
\section{Pointwise and Uniform Convergence}
\begin{leftbarTitle}{Pointwise Convergence}\end{leftbarTitle}
\begin{definition}{Function Term Series}
    Let \( u_n(x) \) (\( n = 1, 2, 3, \dots \)) be a sequence of functions with a common domain \( E \). 
    The sum of these infinitely many functions \( u_1(x) + u_2(x) + \dots + u_n(x) + \dots \) 
    is called a \textbf{function term series}, denoted as:
    \[
    \sum_{n=1}^{\infty} u_n(x).
    \]

    For any fixed point \( x_0 \in E \), if the numerical series \( \sum_{n=1}^{\infty} u_n(x_0) \) converges, 
    then the function term series \( \sum_{n=1}^{\infty} u_n(x) \) is said to 
    converge at \( x_0 \), or equivalently, \( x_0 \) is called 
    a \textbf{convergence point} of \( \sum_{n=1}^{\infty} u_n(x) \).

    The set of all convergence points is called the \textbf{domain of convergence} of \( \sum_{n=1}^{\infty} u_n(x) \).
\end{definition}

\begin{definition}{Pointwise Convergence}
    Let the domain of convergence of the function term series \( \sum_{n=1}^\infty u_n(x) \) be \( D \subset E \). 
    Then \( \sum_{n=1}^\infty u_n(x) \) defines a function \( S(x) \) on the set \( D \), where:
    \[
    S(x) = \sum_{n=1}^\infty u_n(x), \quad x \in D.
    \]
    The function \( S(x) \) is called the \textbf{sum function} of the series, 
    and \( \sum_{n=1}^\infty u_n(x) \) is said to \textbf{converge pointwise} to \( S(x) \) on \( D \).
\end{definition}

Define the \textbf{partial sum function} of the series as:
\[
S_n(x) = \sum_{k=1}^n u_k(x).
\]
It is evident that the set of all \( x \) for which \( \{ S_n(x) \} \) converges is precisely \( D \). 
Therefore, on \( D \), we have:
\[
S(x) = \lim_{n \to \infty} S_n(x) = \lim_{n \to \infty} \sum_{k=1}^n u_k(x).
\]
Conversely, if a sequence of functions \( \{ S_n(x) \} \) (\( x \in E \)) is given, we can define:
\[
\begin{cases}
u_1(x) = S_1(x), \\
u_{n+1}(x) = S_{n+1}(x) - S_n(x), \quad n = 1, 2, \dots
\end{cases}
\]
to obtain the corresponding function term series.

Thus, the convergence behavior of a function term series and the corresponding sequence 
of partial sum functions is essentially the same.

However, it is important to note that the pointwise convergence has certain limitations.

\begin{description}
    \item[Continuity]
    The sum of finitely many continuous functions satisfies additive continuity:
    \[
    \lim_{x \to x_0} [u_1(x) + u_2(x) + \dots + u_n(x)] 
    = \lim_{x \to x_0} u_1(x) + \lim_{x \to x_0} u_2(x) + \dots + \lim_{x \to x_0} u_n(x).
    \]

    If this property can be extended to infinitely many functions, that is:
    If \( u_n(x) \) is continuous on \( D \), the sum function \( S(x) = \sum_{n=1}^\infty u_n(x) \) 
    is also continuous on \( D \). Moreover:
    \[
    \lim_{x \to x_0} \sum_{n=1}^\infty u_n(x) = \sum_{n=1}^\infty \lim_{x \to x_0} u_n(x),
    \]
    meaning that \underline{the limit operation and infinite summation can be interchanged}
    (also known as the fact that function term series can be evaluated termwise).

    For the sequence of partial sums \( \{ S_n(x) \} \), 
    the corresponding conclusion is that the limit function 
    \( S(x) = \lim_{n \to \infty} S_n(x) \) is also continuous on \( D \), and:
    \[
    \lim_{x \to x_0} \lim_{n \to \infty} S_n(x) = \lim_{n \to \infty} \lim_{x \to x_0} S_n(x),
    \]
    meaning that the two limit operations can be interchanged.

    Unfortunately, in the case of pointwise convergence, this property \underline{does not hold}.

    \item[Derivability]
    The sum of finitely many differentiable functions satisfies additive differentiability:
    \[
    \frac{\mathrm{d}}{\mathrm{d}x} [u_1(x) + u_2(x) + \dots + u_n(x)] 
    = \frac{\mathrm{d}}{\mathrm{d}x} u_1(x) + \frac{\mathrm{d}}{\mathrm{d}x} u_2(x) + \dots + \frac{\mathrm{d}}{\mathrm{d}x} u_n(x).
    \]

    If this property can be extended to infinitely many functions, that is:
    If \( u_n(x) \) is differentiable on \( D \), 
    the sum function \( S(x) = \sum_{n=1}^\infty u_n(x) \) is also differentiable on \( D \). Moreover:
    \[
    \frac{\mathrm{d}}{\mathrm{d}x} \sum_{n=1}^\infty u_n(x) = \sum_{n=1}^\infty \frac{\mathrm{d}}{\mathrm{d}x} u_n(x),
    \]
    meaning that \underline{the differentiation operation and infinite summation can be interchanged }
    (also known as the fact that function term series can be differentiated termwise).

    For the sequence of partial sums \( \{ S_n(x) \} \), 
    the corresponding conclusion is that the limit function 
    \( S(x) = \lim_{n \to \infty} S_n(x) \) is also differentiable on \( D \), and:
    \[
    \frac{\mathrm{d}}{\mathrm{d}x} \lim_{n \to \infty} S_n(x) = \lim_{n \to \infty} \frac{\mathrm{d}}{\mathrm{d}x} S_n(x),
    \]
    meaning that the two operations can be interchanged.

    Unfortunately, in the case of pointwise convergence, this property \underline{does not hold}.


    \item[Integrability]
    The sum of finitely many integrable functions satisfies additive integrability:
    \[
    \int_a^b [u_1(x) + u_2(x) + \dots + u_n(x)] \, \mathrm{d}x 
    = \int_a^b u_1(x) \, \mathrm{d}x + \int_a^b u_2(x) \, \mathrm{d}x + \dots + \int_a^b u_n(x) \, \mathrm{d}x.
    \]

    If this property can be extended to infinitely many functions, that is:
    If \( u_n(x) \) is integrable on \( [a, b] \subset D \), 
    the sum function \( S(x) = \sum_{n=1}^\infty u_n(x) \) is also integrable on \( [a, b] \subset D \). Moreover:
    \[
    \int_a^b \sum_{n=1}^\infty u_n(x) \, \mathrm{d}x = \sum_{n=1}^\infty \int_a^b u_n(x) \, \mathrm{d}x,
    \]
    meaning that \underline{the integration operation and infinite summation can be interchanged }
    (also known as the fact that function term series can be integrated termwise).

    For the sequence of partial sums \( \{ S_n(x) \} \), 
    the corresponding conclusion is that the limit function \( S(x) = \lim_{n \to \infty} S_n(x) \) is also integrable on \( [a, b] \subset D \), and:
    \[
    \int_a^b \lim_{n \to \infty} S_n(x) \, \mathrm{d}x = \lim_{n \to \infty} \int_a^b S_n(x) \, \mathrm{d}x,
    \]
    meaning that the two operations can be interchanged.

    Unfortunately, in the case of pointwise convergence, this property \underline{does not hold}.
\end{description}


\begin{leftbarTitle}{Uniform Convergence}\end{leftbarTitle}
\begin{definition}{Uniform Convergence}
    Let \( \{ S_n(x) \} (x \in D) \) be a sequence of functions. If:
    \[
    \forall \varepsilon > 0, \exists N(\varepsilon) \in \mathbb{N}^+, \forall n > N(\varepsilon): 
    |S_n(x) - S(x)| < \varepsilon \quad (\forall x \in D),
    \]
    then \( \{ S_n \} \) is said to \textbf{converge uniformly} to \( S(x) \) on \( D \), denoted as:
    \[
    S_n(x) \mathop{\rightrightarrows}\limits^{D} S(x).
    \]

    If the partial sum sequence \( \{ S_n(x) \} \) of the function term series 
    \( \sum_{n=1}^\infty u_n(x) (x \in D) \) converges uniformly to \( S(x) \) on \( D \), 
    then \( \sum_{n=1}^\infty u_n(x) \) is said to converge uniformly to \( S(x) \) on \( D \).
\end{definition}
Obviously, if the partial sum sequence \( \{ S_n(x) \} \) of \( \sum_{n=1}^\infty u_n(x) \) satisfies:
\[
S_n(x) \mathop{\rightrightarrows}\limits^{D} S(x),
\]
then:
\[
u_n(x) \mathop{\rightrightarrows}\limits^{D} 0.
\]

\begin{theorem}{Cauchy Criterion for Uniform Convergence}
    The necessary and sufficient condition for the sequence of functions \( \{ S_n(x) \} \) to converge uniformly on \( D \) is:
    \[
    \forall \varepsilon > 0, \exists N \in \mathbb{N}^*, \forall m > n > N: 
    |S_m(x) - S_n(x)| < \varepsilon \quad (\forall x \in D).
    \]

    Correspondingly, the necessary and sufficient condition for the function term series 
    \( \sum_{n=1}^\infty u_n(x) \) to converge uniformly on \( D \) is:
    \[
    \forall \varepsilon > 0, \exists N \in \mathbb{N}^*, \forall m > n > N: 
    \left| \sum_{i=n+1}^m u_i(x) \right| < \varepsilon \quad (\forall x \in D).
    \]
\end{theorem}

\begin{theorem}{Necessary and Sufficient Conditions for Uniform Convergence}
    Let \( \{ S_n(x) \} \) converge pointwise to \( S(x) \) on \( D \). 
    The necessary and sufficient conditions for \( S_n(x) \mathop{\rightrightarrows}\limits^{D} S(x) \) are:
    \begin{enumerate}
        \item  
            \[
            \lim_{n \to \infty} d(S_n, S) = \lim_{n \to \infty} \sup_{x \in D} |S_n(x) - S(x)| = 0.
            \]
        \item For any sequence \( \{ x_n \} \) where \( x_n \in D \), the following holds:
            \[
            \lim_{n \to \infty} \big(S_n(x_n) - S(x_n)\big) = 0.
            \]
    \end{enumerate}
\end{theorem}

With the concept of uniform convergence, the flaws of pointwise convergence can be remedied,
and the following properties can be established:
\begin{property}
    \begin{description}
        \item  [Continuity]
            Let \( f_n(x) \mathop{\rightrightarrows}\limits^{I \subset \mathbb{R}} f(x) \). 
            If \( f_n(x) \) is continuous at \( x_0 \in I \) for \( n = 1, 2, 3, \dots \), 
            then \( f(x) \) is also continuous at \( x_0 \). 

            In particular, if \( f_n(x) \in C(I) \), then \( f(x) \in C(I) \).

            \textbf{Termwise Limit} If \( \sum_{n=1}^\infty u_n(x) \mathop{\rightrightarrows}\limits^{I \subset \mathbb{R}} S(x) \) 
            and \( u_n(x) \in C(I) \), then the sum function \( S(x) \in C(I) \).
        \item  [Integrability]
            Let \( f_n(x) \mathop{\rightrightarrows}\limits^{[a, b]} f(x) \). 
            If \( f_n(x) \in R[a, b] \), then \( f(x) \in R[a, b] \), and:
            \[
            \lim_{n \to \infty} \int_a^b f_n(x) \, \mathrm{d}x = \int_a^b \lim_{n \to \infty} f_n(x) \, \mathrm{d}x = \int_a^b f(x) \, \mathrm{d}x.
            \]

            \textbf{Termwise Integration:} If \( \sum_{n=1}^\infty u_n(x) \mathop{\rightrightarrows}\limits^{[a, b]} S(x) \) 
            and \( u_n(x) \in R[a, b] \), then \( S(x) \in R[a, b] \).

        \item [Differentiability]
            Let \( f'_n(x) \mathop{\rightrightarrows}\limits^{[a, b]} \sigma(x) \). 
            If there exists \( x_0 \in [a, b] \) such that:\(\lim_{n \to \infty} f_n(x_0) = a\),
            then there exists a function \( f(x) \) such that 
            \[ 
            f_n(x) \mathop{\rightrightarrows}\limits^{[a, b]} f(x) \text{ and } f'(x) = \sigma(x).
            \]
            
            \textbf{Termwise Differentiation} If 
            \( \sum_{n=1}^\infty u'_n(x) \mathop{\rightrightarrows}\limits^{[a, b]} \sigma(x) \) 
            and there exists \( x_0 \in [a, b] \) such that: \(\sum_{n=1}^\infty u_n(x_0) \to a\),
            then there exists a function \( S(x) \) such that 
            \[ 
            \sum_{n=1}^\infty u_n(x) \mathop{\rightrightarrows}\limits^{[a, b]} S(x) \text{ and } S'(x) = \sigma(x).
            \]

            \textbf{Corollary} Obviously, if we add the condition \( f'_n(x) \in C[a, b] \), the conclusion still holds, 
            and the proof becomes simpler.
    \end{description}
\end{property}

\begin{note}
    Since continuity and differentiability are both local properties, 
    it suffices to have \textbf{internally closed uniform convergence} of \( (a, b) \) 
    to ensure that \( f(x) \) is continuous/differentiable.
\end{note}


\begin{leftbarTitle}{Quasi-Uniform Convergence}\end{leftbarTitle}
\begin{definition}{Quasi-Uniform Convergence}
    The sequence of functions \( \{ S_n(x) \} \) is said to \textbf{converge quasi-uniformly}
    on the interval \( [a, b] \) 
    if it converges pointwise to \( S(x) \) on \( [a, b] \), and the following condition is satisfied:
    \[
    \forall \varepsilon > 0, \forall N \in \mathbb{N}^*, \exists N_0 > N,
    \text{ s.t. } \forall x \in [a, b], \exists n_x \in [N, N_0] \; 
    (n_x \in \mathbb{N}^*): |S_{n_x}(x) - S(x)| < \varepsilon.
    \]    
\end{definition}


\section{Uniform Convergence Tests}
\begin{leftbarTitle}{Weierstrass Test (M-Test)}\end{leftbarTitle}
\begin{theorem}{Weierstrass Test (M-Test)}
    If there exists a convergent positive term series \( \sum_{n=1}^{\infty} a_n \) such that:
    \[
    |u_n(x)| \leqslant a_n, \quad \forall x \in E, n = 1, 2, 3, \dots
    \]
    then the function term series \( \sum_{n=1}^{\infty} u_n(x) \) converges uniformly on \( E \).

    The positive term series \( \sum_{n=1}^{\infty} a_n \) 
    is called a \textbf{majorant series} of \( \sum_{n=1}^{\infty} u_n(x) \).

    If replace the convergent positive term series \( \sum_{n=1}^{\infty} a_n \)
    with a uniform convergent series of functions \( \sum_{n=1}^{\infty} a_n(x) \),
    the conclusion still holds. 
\end{theorem}


\begin{leftbarTitle}{Abel-Dirichlet Test}\end{leftbarTitle}
\begin{theorem}{Abel-Dirichlet Test}
    If the series of functions \( \sum_{n=1}^{\infty} a_n(x) b_n(x) \) (\(x \in E\)) satisfies 
    at least one of the following two conditions, then it converges uniformly on \( E \).
    \begin{description}
        \item[Abel] \( \{ a_n(x_{0}) \} \) (\(\forall x_{0} \in E\)) is monotonic
            and the series of functions \( \{ a_n(x) \} \) is bounded uniformly on \( E \).
            Simultaneously, the series \( \sum_{n=1}^{\infty} b_n(x) \) converges uniformly on \( E \).
        \item[Dirichlet] \( \{ a_n(x_{0}) \} \) (\(\forall x_{0} \in E\)) is a monotonic and 
            and \( a_n(x) \to 0 \) uniformly convergent on \( E \) with limit \(0\).
            Simultaneously, the partial sums \( B_n(x) = \sum_{k=1}^{n} b_k(x) \) are uniformly bounded on \( E \).
    \end{description}
\end{theorem}

\begin{leftbarTitle}{Dini Theorem}\end{leftbarTitle}
\begin{theorem}{Dini Theorem}
    Let the sequence of functions \( \{ S_n(x) \} \) converges pointwise to \( S(x) \) on the closed interval \( [a, b] \),
    if
    \begin{enumerate}
        \item \( S_n(x) \in C[a, b] \) (\( n = 1, 2, 3, \dots \)); 
        \item \( S(x) \in C[a, b] \);
        \item \(\{ S_{n}(x_{0}) \}\) (\(\forall x_{0} \in [a, b]\)) is monotonic;
    \end{enumerate}
    then \( S_n(x) \mathop{\rightrightarrows}\limits^{[a, b]} S(x) \).
\end{theorem}

\begin{remark}
    Removing the condition of monotonicity, the Arzelà-Borel theorem (~\ref{thm:}) becomes the result of quasi-uniform convergence.
\end{remark}

\section{Special Cases}