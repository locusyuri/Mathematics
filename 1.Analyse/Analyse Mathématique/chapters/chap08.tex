\chapter{Numerical Series} % 数项级数
\section{Convergence of Numerical Series}

\section{Positive Term Series and Its Convergence Tests}
\begin{definition}{Positive Term Series}
    If all terms of the series \( \sum_{n=1}^{\infty} x_n \) are non-negative real numbers, 
    i.e., \( x_n \geqslant 0 \) (\( x_n > 0 \)), \( n = 1, 2, \dots \), 
    then this series is called a \textbf{positive term series} (or strictly positive term series).
\end{definition}

\begin{note}
    The positive term series converges if and only if the partial sums of the sequence are bounded. 
    If the partial sums are unbounded, the series must diverge to \( +\infty \).
\end{note}

\begin{leftbarTitle}{Comparison Test}\end{leftbarTitle}
\begin{theorem}{Comparison Test}
    Let \( \sum_{n=1}^{\infty} a_n \) and \( \sum_{n=1}^{\infty} b_n \) be positive term series. 
    If \( \exists N \in \mathbb{N}, \text{ s.t. } \forall n > N: a_n \leqslant b_n \), then:
    \begin{enumerate}
        \item If \( \sum_{n=1}^{\infty} b_n \) converges, then \( \sum_{n=1}^{\infty} a_n \) also converges.
        \item If \( \sum_{n=1}^{\infty} a_n \) diverges, then \( \sum_{n=1}^{\infty} b_n \) also diverges.
    \end{enumerate}

    \textbf{Limit Form}
    Let \( \sum_{n=1}^{\infty} a_n \) and \( \sum_{n=1}^{\infty} b_n \) be positive term series, 
    and suppose \( \lim_{n \to \infty} \frac{a_n}{b_n} \) exists. Then:

    \begin{enumerate}
        \item If \( 0 < l < +\infty \), \( \sum_{n=1}^{\infty} a_n \) and \( \sum_{n=1}^{\infty} b_n \) 
            have the same convergence or divergence behavior.
        \item If \( l = 0 \), \( \sum_{n=1}^{\infty} b_n \) converges, 
            then \( \sum_{n=1}^{\infty} a_n \) also converges.
        \item If \( l = +\infty \), \( \sum_{n=1}^{\infty} b_n \) diverges, 
            then \( \sum_{n=1}^{\infty} a_n \) also diverges.
    \end{enumerate}
\end{theorem}

\begin{theorem}
\begin{description}
    \item[Cauchy Test] Let \( \sum_{n=1}^{\infty} a_n \) be a positive term series.
        \begin{enumerate}
            \item If \( \exists q \in [0,1), \text{ s.t. } \sqrt[n]{a_n} \leqslant 
            q < 1 \quad (n \geqslant N, N \in \mathbb{N}) \), then the series converges.
            \item If \( \sqrt[n]{a_n} \geqslant 1 \) for infinitely many \( n \), then the series diverges.
        \end{enumerate}
        \textbf{Limit Form} Let \( \sum_{n=1}^{\infty} a_n \) be a positive term series, 
        and suppose \( \varlimsup_{n \to +\infty} \sqrt[n]{a_n} = r \). Then:
        \begin{enumerate}
            \item If \( 0 \leqslant r < 1 \), the series \( \sum_{n=1}^{\infty} a_n \) converges.
            \item If \( r > 1 \), the series \( \sum_{n=1}^{\infty} a_n \) diverges.
            \item If \( r = 1 \), the test fails.
        \end{enumerate}

    \item[D'Alembert Test] Let \( \sum_{n=1}^{\infty} a_n \) be a strictly positive term series.
        \begin{enumerate}
            \item If \( \exists q \in [0,1), \text{ s.t. } \frac{a_{n+1}}{a_n} \leqslant 
            q < 1 \quad (n \geqslant N, N \in \mathbb{N}) \), then the series converges.
            \item If \( \frac{a_{n+1}}{a_n} \geqslant 1 \quad (n \geqslant N, N \in \mathbb{N}) \), 
            then the series diverges.
        \end{enumerate}
        \textbf{Limit Form}
        Let \( \sum_{n=1}^{\infty} a_n \) be a strictly positive term series. Then:
        \begin{enumerate}
            \item If \( \varlimsup_{n \to +\infty} \frac{a_{n+1}}{a_n} = r \in (0,1) \), the series converges.
            \item If \( \varliminf_{n \to +\infty} \frac{a_{n+1}}{a_n} = r' > 1 \), the series diverges.
            \item If \( r = 1 \) or \( r' = 1 \), the test fails.
        \end{enumerate}
    
    \item[Raabe Test] Let \( \sum_{n=1}^{\infty} a_n \) be a strictly positive term series.
        \begin{enumerate}
            \item If \( \exists r > 1, \exists N_0 \in \mathbb{N} \text{ s.t. } 
                \forall n > N_0: n \left( \frac{a_n}{a_{n+1}} - 1 \right) \geqslant r \), 
                then the series converges.
            \item If \( \exists N_0 \in \mathbb{N}, \text{ s.t. } \forall n > N_0: 
                n \left( \frac{a_n}{a_{n+1}} - 1 \right) \leqslant 1 \), then the series diverges.
        \end{enumerate}
        \textbf{Limit Form}
        Let \( \sum_{n=1}^{\infty} a_n \) be a strictly positive term series. Then:
        \begin{enumerate}
            \item If \( \varliminf_{n \to +\infty} n \left( \frac{a_n}{a_{n+1}} - 1 \right) = l > 1 \), the series converges.
            \item If \( \varlimsup_{n \to +\infty} n \left( \frac{a_n}{a_{n+1}} - 1 \right) = l' < 1 \), the series diverges.
            \item If \( l = 1 \) or \( l' = 1 \), the test fails.
        \end{enumerate}

    \item[Bertrand Test] Let \( \sum_{n=1}^{\infty} a_n \) be a strictly positive term series.
        \begin{enumerate}
            \item If \( \varliminf_{n \to +\infty} \ln n \left[ n \left( \frac{a_n}{a_{n+1}} - 1 \right) \right] = l > 1 \), 
                the series converges.
            \item If \( \varlimsup_{n \to +\infty} \ln n \left[ n \left( \frac{a_n}{a_{n+1}} - 1 \right) \right] = l' < 1 \), 
                the series diverges.
            \item If \( l = 1 \) or \( l' = 1 \), the test fails.
        \end{enumerate}

    \item [Gauß Test] Let \( \sum_{n=1}^{\infty} a_n \) be a strictly positive term series, and suppose:
        \[
        \frac{a_n}{a_{n+1}} = 1 + \frac{1}{n} + \frac{\delta}{n \ln n} + o\left( \frac{1}{n \ln n} \right), \quad (n \to +\infty).
        \]
        Then:
        \begin{enumerate}
            \item If \( \delta > 1 \), the series converges.
            \item If \( \delta < 1 \), the series diverges.
            \item If \( \delta = 1 \), the criterion fails.
        \end{enumerate}

        \textbf{Generalized Form}
        Let \( \sum_{n=1}^{\infty} a_n \) be a strictly positive term series, and suppose:
        \[
        \frac{a_n}{a_{n+1}} = 1 + \frac{1}{n} + \frac{\delta_n}{n \ln n} + o\left( \frac{1}{n \ln n} \right), 
        \quad (n \to +\infty).
        \]
        If \( \lim_{n \to \infty} \delta_n = \delta \in \mathbb{R} \), then:
        \begin{enumerate}
            \item If \( \delta > 1 \), the series converges.
            \item If \( \delta < 1 \), the series diverges.
            \item If \( \delta = 1 \), the criterion fails.
        \end{enumerate}
\end{description}
\end{theorem}

\begin{note}
    The Bertrand test can be refined by considering series such as:
    \[
    \sum_{n=3}^{\infty} \frac{1}{n \ln n (\ln \ln n)^p}, 
    \quad \sum_{n=9}^{\infty} \frac{1}{n \ln n \ln \ln n (\ln \ln n)^p}, 
    \cdots
    \]
    These refinements are collectively known as the Bertrand test.
\end{note}

\begin{remark}
    All the aforementioned criteria are derived from the Comparison Criterion.
    \begin{itemize}
        \item By comparing positive term series with the geometric series (or equal ratio series), 
            the Cauchy Criterion and d'Alembert Criterion are derived.
        \item By comparing positive term series with the slower-converging series 
            \( \sum_{n=1}^{\infty} \frac{1}{n^\alpha} \) (\( \alpha > 1 \)), the Raabe Criterion is derived.
        \item By comparing positive term series with the even slower-converging series 
            \( \sum_{n=1}^{\infty} \frac{1}{n \ln^\alpha n} \) (\( \alpha > 1 \)), the Gauß Criterion is derived.
    \end{itemize}
    \textbf{General Observation}
    The slower the convergence of the series used for comparison, the more precise the derived criterion.
\end{remark}






\begin{leftbarTitle}{Integral Test}\end{leftbarTitle}
\begin{theorem}{Cauchy Integral Test}
    Let \( f(x) \) be defined on \( [a, +\infty) \), 
    where \( f(x) \geqslant 0 \), and \( f(x) \) is Riemann integrable on any finite interval \( [a, A] \).

    Consider a monotonic increasing sequence \( \{ a_n \} \) such that \( a = a_1 < a_2 < \dots < a_n < \dots \), and let:
    \[
    u_n = \int_{a_n}^{a_{n+1}} f(x) \, \mathrm{d}x.
    \]

    Then the improper integral \( \int_{a}^{+\infty} f(x) \, \mathrm{d}x \) 
    and the positive term series \( \sum_{n=1}^{\infty} u_n \) converge or diverge to \( +\infty \) simultaneously. 
    Moreover:
    \[
    \int_{a}^{+\infty} f(x) \, \mathrm{d}x 
    = \sum_{n=1}^{\infty} u_n 
    = \sum_{n=1}^{\infty} \int_{a_n}^{a_{n+1}} f(x) \, \mathrm{d}x.
    \]
\end{theorem}

\begin{leftbarTitle}{Other Tests}\end{leftbarTitle}
\begin{theorem}{Cauchy Condensation Test}
    Let \( \{ a_n \} \) be a monotonically decreasing sequence of positive numbers. 
    Then the positive term series \( \sum_{n=1}^{\infty} a_n \) converges if and only if the condensed series:
    \[
    \sum_{n=0}^{\infty} 2^n a_{2^n} = a_1 + 2a_2 + 4a_4 + \dots + 2^n a_{2^n} + \dots
    \]
    converges.
\end{theorem}


\section{General Term Series and Its Convergence Tests}
\begin{leftbarTitle}{Cauchy Convergence Criterion for Series}\end{leftbarTitle}
\begin{theorem}{Cauchy Convergence Criterion for Series}
    The necessary and sufficient condition for the convergence of the series \( \sum_{n=1}^{\infty} x_n \) is:
    \[
    \forall \varepsilon > 0, \exists N \in \mathbb{N}, \forall m, n > N : 
    \left| x_{n+1} + x_{n+2} + \cdots +x_{m} \right| = \left| \sum_{k=n+1}^{m} x_k \right| < \varepsilon.
    \]
\end{theorem}


\begin{leftbarTitle}{Alternative Series}\end{leftbarTitle}
\begin{definition}{Alternative Series}
    A series of the form:
    \[
    \sum_{n=1}^{\infty}x_{n} = 
    \sum_{n=1}^{\infty} (-1)^{n-1} u_n\quad (u_{n}>0),
    \]
    is called an \textbf{alternative series}.

    Moreover, if \( u_n \) is a monotonically decreasing sequence and \( \lim_{n \to \infty} u_n = 0 \), 
    then the series is called a \textbf{Leibniz series}.
\end{definition}

\begin{theorem}{Leibniz Test}
    Leibniz series converges.
\end{theorem}


\begin{leftbarTitle}{Abel-Dirichlet Test}\end{leftbarTitle}

\begin{theorem}{Abel Transform (Discrete Integration by Parts/Summation by Parts)}\label{thm:Abel Transform}
    Let \(\{a_n\}, \{b_n\}\) be two sequences, then for any \(n\in \mathbb{N}^{+}\),
    \[
        \sum_{k=1}^{n} a_k b_k = a_n B_n + \sum_{k=1}^{n-1} (a_{k+1} - a_{k})B_k,
    \]
    where \(B_n = \sum_{k=1}^{n} b_k\).
\end{theorem}

\begin{figure}[h]
    \centering
    \includegraphics[width=0.5\textwidth, angle=180]{img/AbelTransform.jpg}
\end{figure}

\begin{lemma}{Abel Lemma (Discrete Second Integral Mean Value Theorem)}
    Let \(\{a_n\}, \{b_n\}\) be two sequences, if \(\{a_n\}\) is a monotonic sequence 
    and \(\{B_k\} = \sum_{k=1}^{n} b_k\) is a bounded sequence with bound \(M\),
    then for any \(p\in \mathbb{N}^{+}\),
    \[
        \left| \sum_{k=1}^{p} a_k b_k \right| \leqslant M \left( |a_{1}| + 2|a_{p}| \right) .
    \]
\end{lemma}

\begin{theorem}{Abel-Dirichlet Test}
    The series \(\sum_{n=1}^{\infty} a_n b_n\) converges if one of the following two conditions is satisfied:
    \begin{description}
        \item[Abel] \(\{a_n\}\) is a bounded monotonic sequence and \(\sum_{n=1}^{\infty} b_n\) converges.
        \item[Dirichlet]  \(\{a_n\}\) is a monotonic sequence, \(\lim_{n \to \infty} a_n = 0\),
            and the partial sums \(B_n = \sum_{k=1}^{n} b_k\) are bounded.       
    \end{description}
\end{theorem}

\section{Absolute and Conditional Convergence of Series}
\begin{definition}{Absolute and Conditional Convergence of Series}
    If the series \( \sum_{n=1}^{\infty} |x_n| \) converges, 
    then the series \( \sum_{n=1}^{\infty} x_n \) is said to be \textbf{absolutely convergent}.

    If the series \( \sum_{n=1}^{\infty} x_n \) converges but is not absolutely convergent, 
    then the series \( \sum_{n=1}^{\infty} x_n \) is said to be \textbf{conditionally convergent}.
\end{definition}

\section{Comparison of Convergence Speed of Series}
The series \( \sum_{n=1}^{\infty} a_n \) is said to converge faster than the series \( \sum_{n=1}^{\infty} b_n \) if:
\[
\lim_{n \to \infty} \frac{a_n}{b_n} = 0.
\]

\begin{theorem}{Du Bois-Reymond Theorem}
    For a given convergent positive term series \( \sum_{n=1}^{\infty} a_n \), there always exists a convergent strictly positive term series \( \sum_{n=1}^{\infty} b_n \) such that:
    \[
    \lim_{n \to \infty} \frac{a_n}{b_n} = 0.
    \]
\end{theorem}

\begin{theorem}{Abel Theorem}
    For a given divergent positive term series \( \sum_{n=1}^{\infty} a_n \), there always exists a divergent positive term series \( \sum_{n=1}^{\infty} b_n \) such that:
    \[
    \lim_{n \to \infty} \frac{a_n}{b_n} = 0.
    \]
\end{theorem}

\begin{remark}
    The above two theorems imply that the slowest converging positive term series \underline{does not} exist.
\end{remark}



\section{Infinite Products}
\begin{leftbarTitle}{Infinite Products}\end{leftbarTitle}


\begin{leftbarTitle}{Two Formulas}\end{leftbarTitle}
\begin{theorem}{Wallis Formula}
    \[
    \lim_{n \to \infty} \frac{1}{2n+1} \left[ \frac{(2n)!!}{(2n-1)!!} \right]^{2}  = \frac{\pi}{2}.
    \]
    Equivalently (\(n\to +\infty\)),
    \begin{gather*}
        \frac{(2n)!!}{(2n-1)!!} \sim \sqrt{\pi n}, \\
        \frac{(n!)^{2}2^{2n}}{(2n)!} \sim \sqrt{\pi n}.
    \end{gather*}
\end{theorem}


\begin{theorem}{Stirling Formula}
    \[
    n! = \sqrt{2\pi n} \left( \frac{n}{e} \right)^n 
    \left( 1 + \frac{1}{12n} - \frac{1}{288n^2} + \frac{139}{51840n^3} - \frac{571}{2488320n^4} + \cdots 
    + \frac{B_{2n}}{2k(2k-1) n^{k}} + \cdots  \right),
    \]
    where \( B_{2k} \) are Bernoulli numbers of order \( 2k \).
    Simplified form:
    \[
    n! \sim \sqrt{2\pi n} \left( \frac{n}{e} \right)^{n} \quad (n \to +\infty),
    \]
    or
    \[
    n! = \sqrt{2\pi n} \left( \frac{n}{e} \right)^{n} e^{\theta_n}, \quad \frac{1}{12n+1} < \theta_n < \frac{1}{12n}.
    \]
\end{theorem}


\section{Special Series}
\begin{description}
    \item[Geometric Series] 
        \[
        \sum_{n=0}^{\infty} q^n = \frac{1}{1-q},
        \]
        it converges when \( |q| < 1 \), diverges otherwise.
    \item[Telescoping Series]
        \[
        \sum_{n=1}^{\infty} (a_n - a_{n+1}) = a_1 - \lim_{n \to \infty} a_{n+1},
        \]
        it converges when \( \lim_{n \to \infty} a_n \) exists, diverges otherwise.
    \item[\(p\)-Series/Hyperharmonic Series]
        \[
        \sum_{n=1}^{\infty} \frac{1}{n^p},
        \]
        it converges when \( p > 1 \), diverges otherwise.
    \item[\(q\)-Series]
        \[
        \sum_{n=1}^{\infty} \frac{1}{n (\ln n)^q},
        \]
        it converges when \( q > 1 \), diverges otherwise.
    \item[Generalized \(q\)-Series]
        \[
        \sum_{n=3}^{\infty} \frac{1}{n \ln n (\ln \ln n)\cdots (\ln^{(k-1)} n) (\ln^{(k)} n)^q},
        \]
        where \( \ln^{(k)} n \) denotes the \( k \)-th iterated logarithm,
        it converges when \( q > 1 \), diverges otherwise.        
\end{description}