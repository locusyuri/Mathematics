\chapter{Limits and Continuity in Euclidean Spaces} % Euclidean Spaces 上的极限与连续性

\section{Continuous Mappings}
\begin{leftbarTitle}{Continuous Mappings on Compact Sets}\end{leftbarTitle}

\begin{leftbarTitle}{Continuous Mappings on Connected Sets}\end{leftbarTitle}
\begin{definition}{Connected Set}
    Let \(S\) be a set of points in \(\mathbb{R}^n\). 
    If a continuous mapping 
    \[
        \gamma: [0, 1] \to \mathbb{R}^n
    \]
    satisfies that the range of \(\gamma([0, 1])\) lies entirely within \(S\), 
    we call \(\gamma\) a \textbf{path} in \(S\), 
    where \(\gamma(0)\) and \(\gamma(1)\) are referred to as the starting point and ending point of the path, respectively.  
    
    If for any two points \(\mathbf{x}, \mathbf{y} \in S\), 
    there exists a path in \(S\) with \(\mathbf{x}\) as the starting point and \(\mathbf{y}\) as the ending point, 
    \(S\) is called path-connected, or equivalently, \(S\) is called a \textbf{connected set}.  
    
    A connected open set is called an \textbf{(open) region}. The closure of an (open) region is referred to as a closed region.
\end{definition}

\begin{remark}
    Intuitively, this means that any two points in \(S\) can be connected 
    by a curve lying entirely within \(S\). 
    Clearly, a connected subset of \(\mathbb{R}\) is an interval, 
    and a connected subset of \(\mathbb{R}\) is compact if and only if it is a closed interval.
\end{remark}

