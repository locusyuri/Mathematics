\documentclass[11pt]{elegantbook}

\title{Équation Différentielle Ordinaire} % 这里放置书名
% \subtitle{Subtitle} % 这里放置副标题

\author{MonoCat} % 这里放置作者名
\date{July, 2025} % 这里放置日期
\version{0.1} % 这里放置版本号
% \institute{Elegant\LaTeX{} Program} % 这里放置机构名
% \bioinfo{Custom Key}{Custom Value} % 这里放置自定义信息

% \extrainfo{extra information} % 这里放置额外信息,将显示在最下方中央

\setcounter{tocdepth}{2} % 设置目录深度
\setcounter{secnumdepth}{2} % 设置章节编号深度


% \logo{logo-blue.png} % 这里放置封面logo,默认从figure目录下寻找
% \cover{LogiqueMathematique.png} % 这里放置封面图片,默认从figure目录下寻找

% modify the color in the middle of titlepage
\definecolor{customcolor}{RGB}{32,178,170} % 自定义颜色
\colorlet{coverlinecolor}{customcolor}
\usepackage{cprotect} % 保护命令参数不被 LaTeX 解析器过早处理,允许在某些特殊环境中使用脆弱命令(fragile commands)。
\usepackage{xeCJK} % 使用 xeCJK 包支持中文


% ===== 开始文档 =====
\begin{document}

\maketitle %生成文档的标题页,根据之前定义的标题信息(如标题、作者、日期等)自动创建一个格式化的标题页

% === 前言部分 ===
\frontmatter        % 开始前言,页码为 i, ii, iii...
\tableofcontents    % 目录 (页码: i, ii)
% \listoffigures      % 图表目录 (页码: iii)
% \listoftables       % 表格目录 (页码: iv)

\chapter{Preface}   % 前言章节(无编号,页码: v, vi...)
This is the preface of the book...

\textcolor{red}{\bfseries Alert} Throught this text, we focus exclusively on real-valued differential equations, 
where all quantities are assumed to be real unless stated otherwise. 
% \chapter{Acknowledgments}  % 致谢(无编号)
% I would like to thank...
% === 正文部分 ===
\mainmatter         % 开始正文,页码从 1 重新开始

\chapter{Introduction} % 这里放置章节标题
\section{Classification of Differential Equations} % 这里放置小节标题
An equation involving one dependent variable and its derivatives with respect to one or more independent variables 
is called a \textbf{differential equation}.
Differential equations can be classified according to the following criteria:
%% 
\begin{leftbarTitle}{Number of Independent Variables}\end{leftbarTitle}
An \textbf{ordinary differential equation(ODE)} is defined as an equation of the following form:
\begin{equation}\label{eq:plain ODE}
F\left( x, y, \frac{\mathrm{d}y}{\mathrm{d}x}, \frac{\mathrm{d}^2y}{\mathrm{d}x^2}, \ldots, \frac{\mathrm{d}^ny}{\mathrm{d}x^n} \right) = 0,\quad n \in \mathbb{N},
\end{equation}
or, using the prime notation for derivatives,
\begin{equation*}
F\left( x, y, y', y'', \ldots, y^{(n)} \right) = 0,\quad n \in \mathbb{N}.
\end{equation*}
If there are two or more independent variables, the equation is called a \textbf{partial differential equation(PDE)}.
%% 
\begin{leftbarTitle}{Order}\end{leftbarTitle}
The order of a differential equation is the order of the highest derivative present in the equation.
\begin{itemize}
    \item A first-order equation has the form $ F(x, y, y') = 0 $.
    \item A second-order equation has the form $ F(x, y, y', y'') = 0 $.
    \item Higher-order equations involve derivatives of order three or more.
\end{itemize}
%%
\begin{note}
    Crucially, the order tells you how many initial conditions are needed to find a unique solution.
\end{note}
%%
\begin{leftbarTitle}{Linearity}\end{leftbarTitle}
An $n$-th order differential equation is linear if it can be written in the form: 
\begin{equation*}
    a_n(x)y^{(n)} + a_{n-1}(x)y^{(n-1)} + \dots + a_1(x)y' + a_0(x)y = g(x)
\end{equation*}
where the coefficients $a_{i}(x)$ and the term $g(x)$ depend only on the independent variable $x$.
Otherwise, it is nonlinear.
%%
\begin{note}
    Specially, for the aforementioned equation, if $g(x) = 0$, it is called \textbf{homogeneous},
    and \textbf{non-homogeneous} otherwise.
\end{note}

\section{Solution to a Ordinary Differential Equation}
\begin{leftbarTitle}{Particular and General Solutions}\end{leftbarTitle}
Let $J$ be an interval in $\mathbb{R}$. 
A function $y=\phi(x)$ defined on the interval $J$ is called a solution to equation~\eqref{eq:plain ODE} if it satisfies: 
\[
F(x, \phi(x), \phi'(x), \phi''(x), \dots, \phi^{(n)}(x)) = 0 \quad x \in J. 
\]
The interval $J$ is then called the interval of existence of the solution $y = \phi(x)$.

Generally speaking, 
the solution to equation~\eqref{eq:plain ODE} contains one or more arbitrary constants, 
the determination of which depends on other conditions that the solution must satisfy. 
If a solution to a differential equation does not contain any arbitrary constants, 
it is called a \textbf{particular solution} of the differential equation.

Suppose $y = \phi(x; c_{1}, c_{2}, \cdots, c_{n})$ is a solution to equation~\eqref{eq:plain ODE}, 
where $c_{1}, c_{2}, \ldots, c_{n}$ are arbitrary constants. 
If $c_{1}, c_{2}, \ldots, c_{n}$ are mutually independent, 
then $y = \phi(x; c_{1}, c_{2}, \ldots, c_{n})$ is called the \textbf{general solution} to equation~\eqref{eq:plain ODE}. 
Here, "mutually independent" means that the Jacobian determinant is non-zero: 
\[
\det \frac{\partial(\phi, \phi', \dots, \phi^{(n-1)})}{\partial(c_1, c_2, \dots, c_n)} \neq 0, \quad x \in J.
\]
When all the arbitrary constants in the general solution are determined, 
one obtains a particular solution to the differential equation.

\begin{leftbarTitle}{Initial Conditions, Explicit and Implicit Solutions}\end{leftbarTitle}
Let $y = \phi(x)$ be a solution to equation~\eqref{eq:plain ODE} that also satisfies 
\begin{equation}\label{eq:initial conditions}
    \phi(x_0) = y_0, \quad \phi'(x_0) = y_0', \dots, \quad \phi^{(n-1)}(x_0) = y_0^{(n-1)}.
\end{equation}
The conditions~\eqref{eq:initial conditions} are called the initial conditions for equation~\eqref{eq:plain ODE}, 
and $y = \phi(x)$ is called the solution to equation~\eqref{eq:plain ODE} satisfying the initial conditions~\eqref{eq:initial conditions}.

A function $y = \phi(x)$ that turns the differential equation~\eqref{eq:plain ODE} into an identity is called an 
\textbf{(explicit) solution} to the equation. 
If a solution $y = \phi(x)$ to the differential equation~\eqref{eq:plain ODE} is determined 
by the relation $\Phi(x, y) = 0$, then $\Phi(x, y) = 0$ is called an \textbf{implicit solution} to the differential equation~\eqref{eq:plain ODE}. 
An implicit solution is also called an "integral". 

\begin{leftbarTitle}{Integral Curve and Direction Field}\end{leftbarTitle}
Consider the first-order differential equation:
\begin{equation}\label{eq:first order ODE}
    \frac{dy}{dx} = f(x,y),
\end{equation}
where $f$ is continuous in a planar region $G$. 
Suppose 
\[
y = \phi(x), \quad x \in J
\] 
is a solution to this equation, where $J \subset \mathbb{R}$ is an interval. 
Then the set of points in the plane 
\[
\Gamma = { (x,y) | y = \phi(x), x \in J }
\]
is a differentiable curve in the plane. 
This curve is called a solution curve or an \textbf{integral curve}.

Let $(x_0, y_0) \in \Gamma$. 
The slope of the tangent line to the curve $\Gamma$ at this point is 
\[
\phi'(x_{0}) = f(x_{0}, y_{0}).
\]
Therefore, the equation of the tangent line is 
\[
    y - y_0 = f(x_0, y_0)(x - x_0).
\]
This implies that even without knowing the explicit expression for $\phi$, 
we can obtain the slope and equation of the tangent line to the solution curve 
at a given point from equation~\eqref{eq:first order ODE}. 

\begin{remark}
    Note that in a small neighborhood of a point on a differentiable curve, 
    the tangent line can be seen as a first-order approximation of the curve. 
    Utilizing this viewpoint, one can obtain an approximate solution to the differential equation. 
    In fact, this is the fundamental idea behind Euler's method.
\end{remark}

At each point $P$ in the region $G$, 
we can draw a short line segment $l(P)$ with slope $f(P)$. 
We call $l(P)$ the line element of equation~\eqref{eq:first order ODE} at point $P$. 
The region $G$ together with the entire collection of these line elements is called 
the lineal \textbf{linear element field} or \textbf{direction field} for equation~\eqref{eq:first order ODE}.

\begin{theorem}
    A necessary and sufficient condition for a continuously differentiable curve 
    $\Gamma = \{ (x,y) | y = \psi(x), x \in J \}$ in the plane 
    to be an integral curve of equation~\eqref{eq:first order ODE} is that 
    for every point $(x, y)$ on the curve $\Gamma$, 
    its tangent line at that point coincides with the line element determined 
    by equation~\eqref{eq:first order ODE} at that point.
\end{theorem}

\begin{proof}
    The necessity follows from the preceding discussion. 
    We now prove the sufficiency. 
    For any point $(x, y) = (x, \psi(x))$ on the curve $\Gamma$, 
    the slope of the tangent line to $\Gamma$ at this point is $\psi'(x)$. 
    By the condition of the theorem, we have $\psi'(x) = f(x, y)$. 
    Since $(x, y)$ is an arbitrary point on the curve, it follows that $y = \psi(x)$ is a solution to equation~\eqref{eq:first order ODE}. 
\end{proof}

\chapter{First Order Equations}
\section{Exact Equations}
\begin{definition}{Exact Equations}
An equation of the form
\begin{equation}\label{eq:symmetric form}
    M(x,y) \, \mathrm{d}x + N(x,y) \, \mathrm{d}y = 0
\end{equation}
is called the symmetric form of a first-order differential equation.

If there exists a continuously differentiable function $u(x,y)$ such that
\[
\mathrm{d}u(x,y) = M(x,y) \, \mathrm{d}x + N(x,y) \, \mathrm{d}y,
\]
then equation~\eqref{eq:symmetric form} is said to be an \textbf{exact equation} or a \textbf{total differential equation}.

It follows that, when equation~\eqref{eq:symmetric form} is exact, it can be rewritten as
\[
\mathrm{d}u(x,y) = 0,
\]
which implies
\begin{equation}\label{general integral}
    u(x,y) = c,
\end{equation}
where $c$ is an arbitrary constant. Equation~\eqref{general integral} is called the \textbf{general integral} of equation~\eqref{eq:symmetric form}.
\end{definition}
\begin{remark}
    It should be noted that, strictly speaking, 
    equation~\eqref{eq:symmetric form} is not a differential equation. 
    However, expressing a first-order differential equation in the form of \eqref{eq:symmetric form} is extremely convenient for analysis.
    This formulation does not necessarily require $y$ to be expressed as a function of $x$. 
    For the sake of simplicity in description, we often refer to the symmetric form~\eqref{eq:symmetric form} as a differential equation, too.
\end{remark}

\begin{theorem}
    Let the functions \(M(x, y)\) and \(N(x, y)\) be continuous in a simply connected domain \(D \subset \mathbb{R}^2\), 
    and suppose their first-order partial derivatives \(\frac{\partial M}{\partial y}\) and \(\frac{\partial N}{\partial x}\) are also continuous. 
    Then a necessary and sufficient condition for equation~\eqref{eq:symmetric form} to be exact is
    \[
    \frac{\partial M}{\partial y} = \frac{\partial N}{\partial x}
    \]
    in the domain \(D\). 
    When this condition holds, for any \((x_0, y_0), (x, y) \in D\), a general integral of equation~\eqref{eq:symmetric form} is given by
    \[
    \int_{\gamma} M(x, y)dx + N(x, y)dy = c,
    \]
    where \(\gamma\) is any curve composed of finitely many smooth segments within \(D\) 
    connecting \((x_0, y_0)\) and \((x, y)\), and \(c\) is an arbitrary constant.
\end{theorem}
\begin{proof}
    
\end{proof}

\section{Separable Equations}
\begin{definition}{Separable Equations}
    If the functions \(M(x, y)\) and \(N(x, y)\) in Equation~\eqref{eq:symmetric form} can both be written as 
    the product of a function of \(x\) and a function of \(y\), that is,
    \[
    M(x, y) = M_1(x) M_2(y), \quad N(x, y) = N_1(x) N_2(y),
    \]
    then equation~\eqref{eq:symmetric form} is called a separable equation.

    When equation~\eqref{eq:symmetric form} is a separable equation, it can be written as
    \begin{equation}\label{eq:separable form}
        M_1(x) M_2(y) \, \mathrm{d}x + N_1(x) N_2(y) \, \mathrm{d}y = 0.  
    \end{equation}
\end{definition}

\section{Homogeneous Equations}
\begin{definition}\label{def:homogeneous equations}
    A first-order differential equation 
    \begin{equation*}
        M(x, y) \, \mathrm{d}x + N(x, y) \, \mathrm{d}y = 0
    \end{equation*}
    is called a \textbf{homogeneous equation} if both \(M\) and \(N\) are homogeneous functions
    \footnote{
        A function \(f(x, y)\) is called a homogeneous function of degree \(n\) 
        if it satisfies the condition \(f(tx, ty) = t^n f(x, y)\) for all \(t > 0\).
        }
    of the same degree \(n\).    
\end{definition}

\begin{example}
    A function \(f(x, y)\) is called a quasihomogeneous function of degree \(d\) with generalized weights if
    \[
    f(t^\alpha s x, t^\beta s y) = t^{ds} f(x, y),
    \]
    where \(t > 0\), \(\alpha\) and \(\beta\) are positive constants with \(\alpha + \beta = 1\), and \(s \in \mathbb{R}\). 
    Here, \(\alpha\) and \(\beta\) are called the weights of \(x\) and \(y\), respectively.
    Consider the differential equation
    \[
    M(x, y) \, \mathrm{d}x + N(x, y) \, \mathrm{d}y = 0,
    \]
    where \(M(x, y)\) and \(N(x, y)\) are quasihomogeneous functions of degree \(d_0\) and \(d_1\) 
    with weights \(\alpha\) and \(\beta\) for \(x\) and \(y\), respectively.
    Proposition: When \(d_0 = d_1 + \beta - \alpha\) the equation can be solved by elementary integration method.
\end{example}
\section{Linear Equations}
\begin{definition}{First-Order Linear Equations}
    A \textbf{first-order linear equation} is an equation of the form
    \begin{equation}\label{eq:first order linear equation}
        \frac{\mathrm{d}y}{\mathrm{d}x} + p(x) y = q(x),
    \end{equation}
    where \(p(x)\) and \(q(x)\) are continuous functions on the interval \((a, b)\).
    In Equation~\eqref{eq:first order linear equation}, when \(q(x) \equiv  0\), we obtain
    \begin{equation*}
        \frac{\mathrm{d}y}{\mathrm{d}x} + p(x) y = 0,
    \end{equation*}
    which is called a \textbf{first-order homogeneous linear equation} corresponding to Equation~\eqref{eq:first order linear equation}.
    Otherwise, it is called a first-order non-homogeneous linear equation.
\end{definition}
\begin{note}
    It should be noted that the definition of a homogeneous equation here \emph{differs} from that in the previous section.
\end{note}

\begin{definition}{Bernoulli's Equation}
    A first-order differential equation of the form
    \begin{equation*}
        \frac{\mathrm{d}y}{\mathrm{d}x} + p(x) y = q(x) y^n,\quad n \neq 0, 1,
    \end{equation*}
    where \(n\) is a real number and \(p(x)\) and \(q(x)\) are continuous functions on the interval \((a, b)\), 
    is called a \textbf{Bernoulli's equation}.
    
\end{definition}

\section{Integrating Factors}
\begin{definition}{Integrating Factors}
    An \textbf{integrating factor} for a first-order differential equation of the form
    \begin{equation}\label{eq:integrating factor equation}
        M(x, y) \, \mathrm{d}x + N(x, y) \, \mathrm{d}y = 0
    \end{equation}
    is a differentiable function \(\mu(x, y)\) such that when multiplied by the equation:
    \[
    \mu(x, y) M(x, y) \, \mathrm{d}x + \mu(x, y) N(x, y) \, \mathrm{d}y = 0,
    \]
    it becomes an exact equation.
    Id est, there exists a function \(\Phi(x, y)\) such that
    \[
       \mu(x, y) M(x, y) \, \mathrm{d}x + \mu(x, y) N(x, y) \, \mathrm{d}y = \mathrm{d}\Phi(x, y).
    \]
    If such functions \(\mu(x, y)\) and \(\Phi(x, y)\) exist, and \(\Phi(x, y)\) is smooth, then
    \[
    \frac{\partial(\mu M)}{\partial y} = \frac{\partial(\mu N)}{\partial x}\left( = \frac{\partial^2 \Phi}{\partial x \partial y} \right).
    \]
    In this case, \(\mu(x, y)\) is called an integrating factor for equation~\eqref{eq:integrating factor equation}.
\end{definition}


\section{Implicit Equations}
This section discusses the problem of solving the first-order implicit differential equations,
\begin{equation}\label{eq:implicit ODE}
F(x, y, y') = 0
\end{equation}
where \(F\) is a continuously differentiable function. 
A so-called implicit differential equation is one in which \(y'\) does not have an explicit solution, 
that is, the equation cannot be written in the form \(y' = f(x, y)\).

\begin{leftbarTitle}{Differentiation Method}\end{leftbarTitle}
Suppose that Equation~\eqref{eq:implicit ODE} can be solved for \(y\), that is,
\begin{equation}\label{eq:differentiation method}
    y = f(x, p),\quad p = \frac{dy}{dx},
\end{equation}
where \(f(x, p)\) is a continuously differentiable function.

Differentiating both sides of \(y = f(x, p)\) with respect to \(x\), we obtain
\[
p = \frac{dy}{dx} = \frac{\partial f}{\partial x} + \frac{\partial f}{\partial p} \frac{dp}{dx},
\]
that is,
\begin{equation*}
\frac{\partial f}{\partial p} \frac{dp}{dx} = p - \frac{\partial f}{\partial x}.
\end{equation*}

This is a first-order differential equation in the variables \(x\), \(p\), \(\frac{dp}{dx}\). 
If a solution \(p = p(x)\) can be found, then Equation~\eqref{eq:differentiation method} yields a solution
\[
y = f(x, p(x)).
\]
\begin{leftbarTitle}{Parametric Method}\end{leftbarTitle}
In general, Equation~\eqref{eq:implicit ODE} represents a surface in the \((x, y, p)\)-space. 
Therefore, the solution can be obtained using a parametric representation of the surface. 
Suppose the parametric form of the surface described by Equation~\eqref{eq:implicit ODE} is
\[
x = x(u, v),\quad y = y(u, v),\quad p = p(u, v) = y'.
\]
Note that
\[
dy = p \; \mathrm{d}x,
\]
thus we obtain
\[
y_u' \mathrm{d}u + y_v' \mathrm{d}v = p(u, v)(x_u' \mathrm{d}u + x_v' \mathrm{d}v).
\]
This is an explicit differential equation in the variables \(u\) and \(v\). Suppose it admits a solution
\[
v = v(u, c),
\]
where \(c\) is a constant, then Equation~\eqref{eq:implicit ODE} has a solution
\[
x = x(u, v(u, c)),\quad y = y(u, v(u, c)).
\]

\chapter{Existence and Uniqueness Theorem}


\begin{thebibliography}{99} 
\bibitem{en1} 作者, Title1, Journal1, Year1. \emph{ This is an example of a reference.}
\bibitem{en2} Author2, Title2, Journal2, Year2. \emph{ This is another example of a reference.}
\end{thebibliography}

\end{document}
