\chapter{Nonlinear Equations and Stability} % 非线性方程与稳定性
\section{The Phase Plane} % 相平面
We are concerned with systems of two simultaneous differential equations of the form:
\begin{equation}\label{eq:phase plane system}
    \begin{cases}
        \frac{\mathrm{d}x}{\mathrm{d}t} = F(x, y), \\
        \frac{\mathrm{d}y}{\mathrm{d}t} = G(x, y).
    \end{cases}
\end{equation}
Assume that the functions \(F\) and \(G\) are continuous and have continuous partial derivatives in
some domain \(D\) of the \(xy\)-plane. 

If \((x_{0}, y_{0})\) is a point in this domain, then by existence and uniqueness theorem
(\ref{thm:existence and uniqueness theorem for system of first-order linear equations})
there exists a unique solution \(x=x(t), y=y(t)\) of the system~\eqref{eq:phase plane system} 
satisfying the initial conditions
\[
x(t_{0}) = x_{0}, \quad y(t_{0}) = y_{0}.
\]
The solution is defined in some time interval \(I\) that contains the point \(t_{0}\).
Frequently, we will write the above initial value problem in the vector form
\begin{equation}\label{eq:vector form of phase plane system}
    \frac{\mathrm{d}\mathbf{x}}{\mathrm{d}t} = \mathbf{f}(\mathbf{x}),\quad
    \mathbf{x}(t_{0}) = \mathbf{x}^{0},
\end{equation}
where \(\mathbf{x} = \begin{pmatrix} x & y \end{pmatrix}^{\mathrm{T}}\),
\(\mathbf{f}(\mathbf{x}) = \begin{pmatrix} F(x, y) & G(x, y) \end{pmatrix}^{\mathrm{T}}\),
and \(\mathbf{x}^{0} = \begin{pmatrix} x_{0} & y_{0} \end{pmatrix}^{\mathrm{T}}\).

\begin{remark}
    Observe that the functions \(F\) and \(G\) in equations~\ref{eq:phase plane system} 
    do not depend on the independent variable \(t\), but only on the dependent variables \(x\) and \(y\). 
    A system with this property is said to be \textbf{autonomous}. 
\end{remark}

Value range \(D\) of \(\mathbf{x}\) is called the \textbf{phase space} of the system~\eqref{eq:vector form of phase plane system}.
Each solution \(\mathbf{x} = \mathbf{x}(t)\) of~\eqref{eq:vector form of phase plane system}
determines a curve in the phase space, called a \textbf{phase trajectory} or \textbf{orbit} of the system.
Drawing all the typical trajectories of the system (representing different initial conditions) on the same phase plane, 
with arrows indicating the direction of flow, creates a \textbf{phase portrait}. % 相图
In fact, the orbit of a solution \(\mathbf{x} = \mathbf{x}(t)\) on the phase plane
is just the projection of its integral curve in the three-dimensional space \((t, x, y)\)
onto the \(xy\)-plane.

The points, if any, where \(\mathbf{f}(\mathbf{x}) = \mathbf{0}\) are called 
\textbf{critical points} (or \textbf{equilibrium points}, or \textbf{singular points}) of the system.



\section{Ляпунов Stability} % Lyapunov 稳定性
\begin{leftbarTitle}{Lyapunov Stability}\end{leftbarTitle}

Consider the differential equations system:
\begin{equation}\label{eq:differential equations system Lyapunov}
    \frac{\mathrm{d}\mathbf{x}}{\mathrm{d}t} = \mathbf{f}(t, \mathbf{x}),
\end{equation}
where \(\mathbf{f}(t, \mathbf{x})\) is continuous for \(\mathbf{x} \in D \subset \mathbb{R}^{n}\) 
and \(t \in (-\infty, +\infty)\),
and satisfies the local Lipschitz condition with respect to \(\mathbf{x}\).

Let the unique solution to~\eqref{eq:differential equations system Lyapunov}
with initial condition \((t_{0}, \mathbf{x}_{0})\) (\(\mathbf{x}_{0}=\mathbf{x}(t_{0})\)) 
be denoted as \(\mathbf{x}= \mathbf{x}(t; t_{0}, \mathbf{x}_{0})\)
and \(\mathbf{x} = \boldsymbol{\phi}(t)\) be also a solution to~\eqref{eq:differential equations system Lyapunov}.

\begin{definition}{Stability in the Sense of Ляпунов}
    The solution \(\mathbf{x} = \boldsymbol{\phi}(t)\) to~\eqref{eq:differential equations system Lyapunov}
    is said to be \textbf{stable in the sense of Ляпунов} (simplified as stable),
    if for any \(\varepsilon > 0\) and \(t_{0}\geqslant 0\),
    there exists a \(\delta = \delta(\varepsilon, t_{0}) > 0\),
    such that when \(\|\mathbf{x}_{0} - \boldsymbol{\phi}(t_{0})\| < \delta\),
    it holds that
    \[
    \|\mathbf{x}(t; t_{0}, \mathbf{x}_{0}) - \boldsymbol{\phi}(t)\| < \varepsilon,
    \]
    for all \(t \geqslant  t_{0}\).

    If \(\mathbf{x} = \boldsymbol{\phi}(t)\) is stable, 
    and there exists \(\delta_{1}\in (0, \delta]\) such that when \(\|\mathbf{x}_{0} - \boldsymbol{\phi}(t_{0})\| < \delta_{1}\),
    it holds that
    \[
    \lim_{t \to +\infty} \|\mathbf{x}(t; t_{0}, \mathbf{x}_{0}) - \boldsymbol{\phi}(t)\| = 0,
    \]
    then \(\mathbf{x} = \boldsymbol{\phi}(t)\) is said to be \textbf{asymptotically stable}.

    If \(\mathbf{x} = \boldsymbol{\phi}(t)\) is asymptotically stable,
    and there exists \(\delta_{2}\in (0, \delta_{1}], \alpha, \beta\) such that 
    when \(\|\mathbf{x}_{0} - \boldsymbol{\phi}(t_{0})\| < \delta_{2}\),
    it holds that
    \[
    \|\mathbf{x}(t; t_{0}, \mathbf{x}_{0}) - \boldsymbol{\phi}(t)\| \leqslant 
    \alpha \|\mathbf{x}_{0} - \boldsymbol{\phi}(t_{0})\| e^{-\beta (t - t_{0})},
    \]
    for all \(t \geqslant t_{0}\),
    then \(\mathbf{x} = \boldsymbol{\phi}(t)\) is said to be \textbf{exponentially stable}.
\end{definition}

By substitution
\[
\mathbf{y} = \mathbf{x}(t; t_{0}, \mathbf{x}_{0}) - \boldsymbol{\phi}(t),
\]
we have
\begin{align*}
    \frac{\mathrm{d} \mathbf{y}}{\mathrm{d}t} = & \frac{\mathrm{d} \mathbf{x}}{\mathrm{d}t} - \frac{\mathrm{d} \boldsymbol{\phi}}{\mathrm{d}t} \\
    = & \mathbf{f}(t, \mathbf{x}) - \mathbf{f}(t, \boldsymbol{\phi}) \\
    = & \mathbf{f}(t, \mathbf{y} + \boldsymbol{\phi}) - \mathbf{f}(t, \boldsymbol{\phi}) \\
    := & \mathbf{F}(t, \mathbf{y}),
\end{align*}
then the system~\eqref{eq:differential equations system Lyapunov} can be transformed into:
\begin{equation}\label{eq:transformed system Lyapunov}
    \frac{\mathrm{d} \mathbf{y}}{\mathrm{d}t} = \mathbf{F}(t, \mathbf{y}).
\end{equation}
Then the stability of solution \(\mathbf{x} = \boldsymbol{\phi}(t; t_{0}, \mathbf{x}_{1})\)
to~\eqref{eq:differential equations system Lyapunov} is \emph{equivalent to the stability of the zero solution}
to~\eqref{eq:transformed system Lyapunov}.

Without loss of generality, we only discuss the stability of the zero solution \(\mathbf{x} = \mathbf{0}\)
to the autonomous system~\eqref{eq:differential equations system Lyapunov},
and assume that \(\mathbf{f}(t, \mathbf{0}) \equiv \mathbf{0}\).
The following definition is derived:
\begin{definition}{Stability of the Zero Solution}
    The zero solution \(\mathbf{x} = \mathbf{0}\) to~\eqref{eq:differential equations system Lyapunov}
    is said to be \textbf{stable} if for any \(\varepsilon > 0\) and \(t_{0} \geqslant 0\),
    there exists a \(\delta = \delta(\varepsilon, t_{0}) > 0\),
    such that when \(\|\mathbf{x}_{0}\| < \delta\),
    it holds that
    \[
    \|\mathbf{x}(t; t_{0}, \mathbf{x}_{0})\| < \varepsilon,
    \]
    for all \(t \geqslant t_{0}\).

    If \(\mathbf{x} = \mathbf{0}\) is stable, 
    and there exists \(\delta_{1}\in (0, \delta]\) such that when \(\|\mathbf{x}_{0}\| < \delta_{1}\),
    it holds that
    \[
    \lim_{t \to +\infty} \|\mathbf{x}(t; t_{0}, \mathbf{x}_{0})\| = 0,
    \]
    then \(\mathbf{x} = \mathbf{0}\) is said to be \textbf{asymptotically stable}.

    If \(\mathbf{x} = \mathbf{0}\) is asymptotically stable,
    and there exists \(\delta_{2}\in (0, \delta_{1}], \alpha, \beta\) such that 
    when \(\|\mathbf{x}_{0}\| < \delta_{2}\),
    it holds that
    \[
    \|\mathbf{x}(t; t_{0}, \mathbf{x}_{0})\| \leqslant 
    \alpha \|\mathbf{x}_{0}\| e^{-\beta (t - t_{0})},
    \]
    for all \(t \geqslant t_{0}\),
    then \(\mathbf{x} = \mathbf{0}\) is said to be \textbf{exponentially stable}.
\end{definition}

\begin{figure}[h]
    \centering
    \includegraphics[width=0.8\textwidth]{img/stable.png}
\end{figure}

\section{Ляпунов Methods} % Lyapunov 方法
Ляпунов established two methods for studying the stability of solutions to differential equations:
\begin{enumerate}[label=(\arabic*)]
    \item The first method is based on the series solutions of differential equations,
        which is often difficult to apply in practice.
    \item The second method, known as the Ляпунов second method,
        relies on constructing a special scalar function, called the Ляпунов function,
        to analyze the stability of solutions without explicitly solving the differential equations,
        which is also called the direct method of Ляпунов.
\end{enumerate}

\begin{equation}\label{eq:autonomous system Lyapunov}
    \frac{\mathrm{d}\mathbf{x}}{\mathrm{d}t} = \mathbf{f}(\mathbf{x}),\quad \mathbf{x}\in G \subset \mathbb{R}^{n},
\end{equation}
where \(\mathbf{f}(\mathbf{x})\) is continuous in \(G\)
and satisfies the local Lipschitz condition with respect to \(\mathbf{x}\),
and \(\mathbf{f}(\mathbf{0}) = \mathbf{0}\).

\begin{leftbarTitle}{Lyapunov First Method}\end{leftbarTitle} % Ляпунов 第一方法
\begin{theorem}{Ляпунов Linearization Theorem}
    For the autonomous system~\eqref{eq:autonomous system Lyapunov},
    let \(\mathbf{f}(\mathbf{x})\) be continuously differentiable in a neighborhood of \(\mathbf{x} = \mathbf{0}\),
    and let 
    \[
    J= \nabla \mathbf{f}(\mathbf{x})\mid_{\mathbf{x}=\mathbf{0}} = \left( \frac{\partial f_i}{\partial x_j} \right)_{n\times n}.
    \]
    Let \(\lambda_{1}, \lambda_{2}, \ldots, \lambda_{n}\) be all eigenvalues of matrix \(J\).
    Then for linearized system \(\dot{\mathbf{y}}=J\mathbf{y}\) it holds that:
    \begin{description}
        \item[Stability] if \(\mathrm{Re}(\lambda_i) < 0\) for all \(i=1,2,\ldots,n\),
            then the zero solution \(\mathbf{x} = \mathbf{0}\) is asymptotically stable;
        \item[Unstability] if there exists \(\mathrm{Re}(\lambda_k) > 0\),
            then the zero solution \(\mathbf{x} = \mathbf{0}\) is unstable.
        \item[Inconclusiveness] if there exists \(\mathrm{Re}(\lambda_m) = 0\),
            then the stability of the zero solution \(\mathbf{x} = \mathbf{0}\) 
            cannot be determined by the linearized system.
    \end{description}
\end{theorem}


\begin{leftbarTitle}{Lyapunov Second Method}\end{leftbarTitle} % Ляпунов 第二方法
\begin{definition}{Ляпунов Function}
    A scalar function \(V(\mathbf{x}): G \to \mathbb{R}\) is called a \textbf{Ляпунов function},
    if it satisfies the following conditions:
    \begin{itemize}
        \item \(V(\mathbf{0}) = 0\);
        \item \(V(\mathbf{x}), \nabla V(\mathbf{x})\) are continuous in \(G\).
    \end{itemize}
    In \(G_{1} \subseteq G\),
    If \(V(\mathbf{x}) > 0 (<0)\) for all \(\mathbf{x}\) except \(\mathbf{0}\),
    then \(V(\mathbf{x})\) is called a \textbf{positive/negative definite};
    if \(V(\mathbf{x}) \geqslant 0 (\leqslant 0)\) for all \(\mathbf{x}\),
    then \(V(\mathbf{x})\) is called a \textbf{positive/negative semidefinite};
    otherwise, it is called an \textbf{indefinite}.
\end{definition}


\begin{theorem}{Ляпунов Stability Theorem}
    For the autonomous system~\eqref{eq:autonomous system Lyapunov},
    let \(V(\mathbf{x})\) be a Ляпунов function in \(G\),
    and its total derivative along the trajectories of~\eqref{eq:autonomous system Lyapunov} is given by:
    \[
    \dot{V}(\mathbf{x}) = \nabla V(\mathbf{x}) \cdot \mathbf{f}(\mathbf{x}) 
    = \sum_{i=1}^{n}\frac{\partial V}{\partial x_i} f_i(\mathbf{x}).
    \]
    Then:
    \begin{description}
        \item[Stability Theorem] if \(V(\mathbf{x})\) is positive definite
            and \(\dot{V}(\mathbf{x})\) is negative semidefinite in \(G\),
            then the zero solution \(\mathbf{x} = \mathbf{0}\) is stable.
        \item[Asymptotically Stability Theorem] if \(V(\mathbf{x})\) is positive definite
            and \(\dot{V}(\mathbf{x})\) is negative definite in \(G\),
            then the zero solution \(\mathbf{x} = \mathbf{0}\) is asymptotically stable.
        \item[Unstability Theorem] if \(V(\mathbf{x})\) is positive definite
            and \(\dot{V}(\mathbf{x})\) is positive definite in \(G\),
            then the zero solution \(\mathbf{x} = \mathbf{0}\) is unstable.
    \end{description}
\end{theorem}

Here are some common Ляпунов functions: % 能量函数, 变量梯度, 二次型
\begin{description}
    \item[Energy Function] In many physical systems, the total energy (kinetic energy + potential energy) (\(E=T+U\))
        can serve as a Ляпунов function.
        For instance, for a simple pendulum without friction, \(\ddot{\theta} + \frac{g}{l} \sin \theta = 0\),
        the Ляпунов function can be chosen as
        \[
        V(\theta, \dot{\theta}) = T + U = \frac{1}{2} m l^{2} \dot{\theta}^{2} + mgl(1 - \cos \theta),
        \]
        where \(m\) is the mass of the pendulum bob, \(l\) is the length of the pendulum,
        and \(g\) is the acceleration due to gravity.
        Then by \(\dot{V}=\nabla V \cdot \mathbf{f}(\mathbf{x}) = 0\),
        we can conclude that the equilibrium point \((\theta, \dot{\theta}) = (0, 0)\) is stable but not asymptotically stable.
    \item[Variable Gradient] If there exists a scalar function \(V(\mathbf{x})\)
        such that \(\nabla V(\mathbf{x}) = \mathbf{g}(\mathbf{x})\),
        where \(\mathbf{g}(\mathbf{x})\) satisfies
        \begin{enumerate}
            \item \(\mathbf{g}(\mathbf{0}) = \mathbf{0}\);
            \item \(\mathbf{g}\) is field of gradients, i.e., \(\nabla \times \mathbf{g} = \mathbf{0}\) or equivalently 
                \(\frac{\partial g_i}{\partial x_j} = \frac{\partial g_j}{\partial x_i}\) for all \(i, j\);
            \item \(\dot{V}(\mathbf{x}) = \mathbf{g}(\mathbf{x}) \cdot \mathbf{f}(\mathbf{x}) \leqslant 0\) (or \(\geqslant 0\))
                for all \(\mathbf{x} \in G\).
        \end{enumerate}
        then \(V(\mathbf{x})\) can be used as a Ляпунов function.
    \item[Quadratic Form] For linear system \(\dot{\mathbf{x}}=\mathbf{A}x\), 
        let \(V(\mathbf{x})=\mathbf{x}^{\mathrm{T}} P \mathbf{x}\), where \(P\) is a symmetric positive definite matrix.
        Along the trajectories of the system, we have
        \[
        \dot{V}(\mathbf{x}) = \mathbf{x}^{\mathrm{T}} (A^{\mathrm{T}} P + P A) \mathbf{x}.
        \]
        If there exists a symmetric positive definite matrix \(Q\) such that
        \[
        A^{\mathrm{T}} P + P A = -Q,
        \]
        then \(V(\mathbf{x})\) is a Ляпунов function and the zero solution is asymptotically stable.
\end{description}


\begin{leftbarTitle}{Other Theorems}\end{leftbarTitle} % 其他定理

\begin{theorem}{LaSalle Invariance Principle} % LaSalle 不变定理
    For the autonomous system~\eqref{eq:autonomous system Lyapunov},
    let \(V(\mathbf{x})\) be a Ляпунов function in \(G\),
    and its total derivative along the trajectories of~\eqref{eq:autonomous system Lyapunov} is given by:
    \[
    \dot{V}(\mathbf{x}) = \nabla V(\mathbf{x}) \cdot \mathbf{f}(\mathbf{x}) 
    = \sum_{i=1}^{n}\frac{\partial V}{\partial x_i} f_i(\mathbf{x}).
    \]
    Let 
    \[
    S = \{ \mathbf{x} \in G \mid \dot{V}(\mathbf{x}) = 0 \},
    \]
    and let \(M\) be the largest invariant set in \(S\).
    Then the zero solution \(\mathbf{x} = \mathbf{0}\) is asymptotically stable
    if \(M = \{\mathbf{0}\}\).
\end{theorem}


\begin{theorem}{Chetaev Theorem} % Chetaev 定理
    For the autonomous system~\eqref{eq:autonomous system Lyapunov},
    let \(V(\mathbf{x})\) be a Ляпунов function in \(G\),
    and its total derivative along the trajectories of~\eqref{eq:autonomous system Lyapunov} is given by:
    \[
    \dot{V}(\mathbf{x}) = \nabla V(\mathbf{x}) \cdot \mathbf{f}(\mathbf{x}) 
    = \sum_{i=1}^{n}\frac{\partial V}{\partial x_i} f_i(\mathbf{x}).
    \]
    If there exists a domain \(U \subset G\) such that
    \begin{itemize}
        \item \(\mathbf{0} \in \partial U\) (the boundary of \(U\));
        \item \(V(\mathbf{x}) > 0\) for all \(\mathbf{x} \in U\);
        \item \(\dot{V}(\mathbf{x}) > 0\) for all \(\mathbf{x} \in U\);
    \end{itemize}
    then the zero solution \(\mathbf{x} = \mathbf{0}\) is unstable.
\end{theorem}

\section{Critical Points and Limit Cycles} % 平面奇点与极限环
\begin{leftbarTitle}{Classification of Critical Points}\end{leftbarTitle}
For plane system~\eqref{eq:phase plane system},
a critical point \((x_{0}, y_{0})\) is called a \textbf{primary critical point}
if \(\det A \neq 0\), where
\[
A = \begin{pmatrix}
\frac{\partial F}{\partial x} & \frac{\partial F}{\partial y} \\
\frac{\partial G}{\partial x} & \frac{\partial G}{\partial y}
\end{pmatrix}_{(x_{0}, y_{0})}.
\]
The classification of primary critical points is completely subject to the eigenvalues of matrix \(A\)
(Hartman-Grobman theorem). 



By the Jordan canonical form of matrix, \(A\sim J\), where there are three cases for \(J\):
\begin{description}
    \item[Distinct real eigenvalues] 
        \[
        J = \begin{pmatrix} \lambda_{1} & 0 \\ 0 & \lambda_{2} \end{pmatrix};
        \]
        In this case, the general solution of the linearized system is
        \[
        \mathbf{x}(t) = C_{1} e^{\lambda_{1} t} \boldsymbol{\xi}^{(1)} + C_{2} e^{\lambda_{2} t} \boldsymbol{\xi}^{(2)},
        \]
        if \(\lambda_{1}\lambda_{2} > 0\),
        then the critical point is a \textbf{node}
        (stable if \(\lambda_{1}, \lambda_{2} < 0\) (\textbf{nodal sink}~\ref{fig:stable node}),
        unstable if \(\lambda_{1}, \lambda_{2} > 0\)) (\textbf{nodal source}, the same pattern as~\ref{fig:stable node} 
        but arrows point outward); ;
        \begin{figure}[h]
            \centering
            \includegraphics[width=0.4\textwidth]{img/node.png}
            \caption{Trajectories in the phase plane when the origin is a stable node with \(\lambda_{1}<\lambda_{2} < 0\).
            The solid black and dashed black curves show the fundamental solutions \(\boldsymbol{\xi}^{(1)}e^{\lambda_1 t}\) 
            and \(\boldsymbol{\xi}^{(2)}e^{\lambda_2 t}\), respectively.}
            \label{fig:stable node}
        \end{figure}

        if \(\lambda_{1}\lambda_{2} < 0\),
        then the critical point is a \textbf{saddle point} (always unstable~\ref{fig:saddle point}).
        \begin{figure}[h]
            \centering
            \includegraphics[width=0.4\textwidth]{img/saddle.png}
            \caption{Trajectories in the phase plane when the origin is a saddle point with \(\lambda_{1}>0, \lambda_{2} < 0\).}
            \label{fig:saddle point}
        \end{figure}

    \item[Equal real eigenvalue] By the algebraic and geometric multiplicities of \(\lambda\),
        there are two subcases:
        \begin{description}
            \item[Diagonal]
                \[
                J = \begin{pmatrix} \lambda & 0 \\ 0 & \lambda \end{pmatrix};
                \]
                In this case, the general solution of the linearized system is
                \[
                \mathbf{x}(t) = e^{\lambda t} \left[ C_{1} \boldsymbol{\xi}^{(1)} + C_{2} \boldsymbol{\xi}^{(2)} \right].
                \]
                The critical point is called a \textbf{proper node} or \textbf{star node}
                (stable if \(\lambda < 0\)~\ref{fig:star node}, unstable if \(\lambda > 0\)).
                \begin{figure}[h]
                    \centering
                    \includegraphics[width=0.6\textwidth]{img/star.png}
                    \caption{(a) Trajectories in the phase plane when the origin is a stable star node with \(\lambda < 0\).
                    (b) and (c) show the corresponding component plots \(x(t)\) and \(y(t)\) versus \(t\).}
                    \label{fig:star node}
                \end{figure}
            \item[Non-diagonal] 
                \[
                J = \begin{pmatrix} \lambda & 1 \\ 0 & \lambda \end{pmatrix}.
                \]
                In this case, the general solution of the linearized system is
                \[
                \mathbf{x}(t) = e^{\lambda t} \left[ C_{1} \boldsymbol{\xi} + C_{2} \left( \boldsymbol{\xi}t 
                + \boldsymbol{\eta} \right) \right],
                \]
                where \(\boldsymbol{\xi}\) is the eigenvector corresponding to \(\lambda\),
                and \(\boldsymbol{\eta}\) is the generalized eigenvector satisfying
                \[
                (A - \lambda I) \boldsymbol{\eta} = \boldsymbol{\xi}.
                \]
                The critical point is called an \textbf{improper node} or \textbf{degenerate node}
                (stable if \(\lambda < 0\)~\ref{fig:improper node}, unstable if \(\lambda > 0\)).
                \begin{figure}[h]
                    \centering
                    \includegraphics[width=0.6\textwidth]{img/degenerated.png}
                    \caption{(a) The phase plane for an improper node with \(\lambda < 0\) and 
                    one independent eigenvector \(\boldsymbol{\xi}\).
                    (b) The phase plane for an improper node with the same \(\lambda\) and
                    eigenvector \(\boldsymbol{\xi}\) but a different generalized eigenvector \(\boldsymbol{\eta}\).}
                    \label{fig:improper node}
                \end{figure}
        \end{description}
    \item[Complex conjugate eigenvalues] By the real part of \(\lambda = \alpha \pm i \beta\),
        there are two subcases:
        \begin{description}
            \item[With non-zero real part]
                \[
                J = \begin{pmatrix} \alpha & \beta \\ -\beta & \alpha \end{pmatrix},\quad \beta > 0;
                \]
                In this case, the general solution of the linearized system is
                \[
                \mathbf{x}(t) = e^{\alpha t} \left[ C_{1} \begin{pmatrix} \cos \beta t \\ -\sin \beta t \end{pmatrix}
                + C_{2} \begin{pmatrix} \sin \beta t \\ \cos \beta t \end{pmatrix} \right].
                \]
                if \(\alpha < 0\),
                then the critical point is a \textbf{spiral sink} (stable~\ref{fig:spiral});
                if \(\alpha > 0\),
                then the critical point is a \textbf{spiral source} (unstable~\ref{fig:spiral}).
                \begin{figure}[h]
                    \centering
                    \includegraphics[width=0.6\textwidth]{img/spiral.png}
                    \caption{Trajectories in the phase plane for a linear system with eigenvalues \(\alpha \pm i \beta\),
                    where (a) \(\alpha < 0\) (spiral sink) and (b) \(\alpha > 0\) (spiral source).}
                    \label{fig:spiral}
                \end{figure}
            \item[Pure imaginary]
                \[
                J = \begin{pmatrix} 0 & \beta \\ -\beta & 0 \end{pmatrix},\quad \beta > 0;
                \]
                In this case, the general solution of the linearized system is
                \[
                \mathbf{x}(t) = C_{1} \begin{pmatrix} \cos \beta t \\ -\sin \beta t \end{pmatrix}
                + C_{2} \begin{pmatrix} \sin \beta t \\ \cos \beta t \end{pmatrix}.
                \]
                The critical point is called a \textbf{center} (always stable~\ref{fig:center}).
                \begin{figure}[h]
                    \centering
                    \includegraphics[width=0.6\textwidth]{img/center.png}
                    \caption{(a) Trajectories in the phase plane when the linear system has purely imaginary eigenvalues 
                    \(\pm i \beta\), representing a center.
                    (b) and (c) show the corresponding component plots \(x(t)\) and \(y(t)\) versus \(t\).}
                    \label{fig:center}
                \end{figure}
        \end{description}
\end{description}

The following table summarizes the classification of primary critical points:
\begin{table}[h]
    \centering
    \begin{tabular}{|c|c|c|c|}
        \hline
        Eigenvalues & Critical Point & Stability & Phase Portrait \\
        \hline
        \(\lambda_{1}, \lambda_{2} < 0\) real distinct & Nodal Sink & Asymptotically stable & \ref{fig:stable node} \\
        \hline
        \(\lambda_{1}, \lambda_{2} > 0\) real distinct & Nodal Source & Unstable & Opposite of \ref{fig:stable node} \\
        \hline
        \(\lambda_{1} > 0, \lambda_{2} < 0\) real distinct & Saddle Point & Unstable & \ref{fig:saddle point} \\
        \hline
        \(\lambda < 0\) real equal diagonal & Star Node & Asymptotically stable & \ref{fig:star node} \\
        \hline
        \(\lambda > 0\) real equal diagonal & Star Node & Unstable & Opposite of \ref{fig:star node} \\
        \hline
        \(\lambda < 0\) real equal non-diagonal & Improper Node & Asymptotically stable & \ref{fig:improper node} \\
        \hline
        \(\lambda > 0\) real equal non-diagonal & Improper Node & Unstable & Opposite of \ref{fig:improper node} \\
        \hline
        \(\alpha < 0 \pm i \beta\) complex conjugate & Spiral Sink & Asymptotically stable & \ref{fig:spiral}(a) \\
        \hline
        \(\alpha > 0 \pm i \beta\) complex conjugate & Spiral Source & Unstable & \ref{fig:spiral}(b) \\
        \hline
        \(\pm i \beta\) pure imaginary & Center & Stable & \ref{fig:center} \\
        \hline
    \end{tabular}
    \caption{Classification of Primary Critical Points}
    \label{table:classification of primary critical points}
\end{table}

\begin{leftbarTitle}{Limit Cycles}\end{leftbarTitle}
\begin{definition}{Limit Cycle}
    A \textbf{limit cycle} is a closed trajectory in the phase plane
    such that at least one other trajectory spirals towards it as \(t \to +\infty\) (stable limit cycle)
    or as \(t \to -\infty\) (unstable limit cycle).
\end{definition}

\begin{theorem}{Poincaré-Bendixson Theorem}
    For the plane system~\eqref{eq:phase plane system},
    let \(R\) be a closed bounded region in the phase plane,
    and suppose that:
    \begin{itemize}
        \item \(R\) contains no critical points of the system;
        \item A trajectory starts in \(R\) at some time \(t_{0}\),
            and remains in \(R\) for all \(t > t_{0}\).
    \end{itemize}
    Then the trajectory either is a closed orbit (a limit cycle),
    or approaches a closed orbit as \(t \to +\infty\).
\end{theorem}