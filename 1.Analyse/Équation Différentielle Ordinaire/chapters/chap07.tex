\chapter{Nonlinear Equations and Stability} % 非线性方程与稳定性
\section{The Phase Plane} % 相平面
We are concerned with systems of two simultaneous differential equations of the form:
\begin{equation}\label{eq:phase plane system}
    \begin{cases}
        \frac{\mathrm{d}x}{\mathrm{d}t} = F(x, y), \\
        \frac{\mathrm{d}y}{\mathrm{d}t} = G(x, y).
    \end{cases}
\end{equation}
Assume that the functions \(F\) and \(G\) are continuous and have continuous partial derivatives in
some domain \(D\) of the \(xy\)-plane. 

If \((x_{0}, y_{0})\) is a point in this domain, then by existence and uniqueness theorem
(\ref{thm:existence and uniqueness theorem for system of first-order linear equations})
there exists a unique solution \(x=x(t), y=y(t)\) of the system~\eqref{eq:phase plane system} 
satisfying the initial conditions
\[
x(t_{0}) = x_{0}, \quad y(t_{0}) = y_{0}.
\]
The solution is defined in some time interval \(I\) that contains the point \(t_{0}\).
Frequently, we will write the above initial value problem in the vector form
\begin{equation}\label{eq:vector form of phase plane system}
    \frac{\mathrm{d}\mathbf{x}}{\mathrm{d}t} = \mathbf{f}(\mathbf{x}),\quad
    \mathbf{x}(t_{0}) = \mathbf{x}^{0},
\end{equation}
where \(\mathbf{x} = \begin{pmatrix} x & y \end{pmatrix}^{\mathrm{T}}\),
\(\mathbf{f}(\mathbf{x}) = \begin{pmatrix} F(x, y) & G(x, y) \end{pmatrix}^{\mathrm{T}}\),
and \(\mathbf{x}^{0} = \begin{pmatrix} x_{0} & y_{0} \end{pmatrix}^{\mathrm{T}}\).

\begin{remark}
    Observe that the functions \(F\) and \(G\) in equations~\ref{eq:phase plane system} 
    do not depend on the independent variable \(t\), but only on the dependent variables \(x\) and \(y\). 
    A system with this property is said to be \textbf{autonomous}. 
\end{remark}

Value range \(D\) of \(\mathbf{x}\) is called the \textbf{phase space} of the system~\eqref{eq:vector form of phase plane system}.
Each solution \(\mathbf{x} = \mathbf{x}(t)\) of~\eqref{eq:vector form of phase plane system}
determines a curve in the phase space, called a \textbf{phase trajectory} or \textbf{orbit} of the system.
Drawing all the typical trajectories of the system (representing different initial conditions) on the same phase plane, 
with arrows indicating the direction of flow, creates a \textbf{phase portrait}. % 相图
In fact, the orbit of a solution \(\mathbf{x} = \mathbf{x}(t)\) on the phase plane
is just the projection of its integral curve in the three-dimensional space \((t, x, y)\)
onto the \(xy\)-plane.

The points, if any, where \(\mathbf{f}(\mathbf{x}) = \mathbf{0}\) are called 
\textbf{critical points} (or \textbf{equilibrium points}, or \textbf{singular points}) of the system.



\section{Ляпунов Stability} % Lyapunov 稳定性
\begin{leftbarTitle}{Lyapunov Stability}\end{leftbarTitle}

Consider the differential equations system:
\begin{equation}\label{eq:differential equations system Lyapunov}
    \frac{\mathrm{d}\mathbf{x}}{\mathrm{d}t} = \mathbf{f}(t, \mathbf{x}),
\end{equation}
where \(\mathbf{f}(t, \mathbf{x})\) is continuous for \(\mathbf{x} \in D \subset \mathbb{R}^{n}\) 
and \(t \in (-\infty, +\infty)\),
and satisfies the local Lipschitz condition with respect to \(\mathbf{x}\).

Let the unique solution to~\eqref{eq:differential equations system Lyapunov}
with initial condition \((t_{0}, \mathbf{x}_{0})\) (\(\mathbf{x}_{0}=\mathbf{x}(t_{0})\)) 
be denoted as \(\mathbf{x}= \mathbf{x}(t; t_{0}, \mathbf{x}_{0})\)
and \(\mathbf{x} = \boldsymbol{\phi}(t)\) be also a solution to~\eqref{eq:differential equations system Lyapunov}.

\begin{definition}{Stability in the Sense of Ляпунов}
    The solution \(\mathbf{x} = \boldsymbol{\phi}(t)\) to~\eqref{eq:differential equations system Lyapunov}
    is said to be \textbf{stable in the sense of Ляпунов} (simplified as stable),
    if for any \(\varepsilon > 0\) and \(t_{0}\geqslant 0\),
    there exists a \(\delta = \delta(\varepsilon, t_{0}) > 0\),
    such that when \(\|\mathbf{x}_{0} - \boldsymbol{\phi}(t_{0})\| < \delta\),
    it holds that
    \[
    \|\mathbf{x}(t; t_{0}, \mathbf{x}_{0}) - \boldsymbol{\phi}(t)\| < \varepsilon,
    \]
    for all \(t \geqslant  t_{0}\).

    If \(\mathbf{x} = \boldsymbol{\phi}(t)\) is stable, 
    and there exists \(\delta_{1}\in (0, \delta]\) such that when \(\|\mathbf{x}_{0} - \boldsymbol{\phi}(t_{0})\| < \delta_{1}\),
    it holds that
    \[
    \lim_{t \to +\infty} \|\mathbf{x}(t; t_{0}, \mathbf{x}_{0}) - \boldsymbol{\phi}(t)\| = 0,
    \]
    then \(\mathbf{x} = \boldsymbol{\phi}(t)\) is said to be \textbf{asymptotically stable}.

    If \(\mathbf{x} = \boldsymbol{\phi}(t)\) is asymptotically stable,
    and there exists \(\delta_{2}\in (0, \delta_{1}], \alpha, \beta\) such that 
    when \(\|\mathbf{x}_{0} - \boldsymbol{\phi}(t_{0})\| < \delta_{2}\),
    it holds that
    \[
    \|\mathbf{x}(t; t_{0}, \mathbf{x}_{0}) - \boldsymbol{\phi}(t)\| \leqslant 
    \alpha \|\mathbf{x}_{0} - \boldsymbol{\phi}(t_{0})\| e^{-\beta (t - t_{0})},
    \]
    for all \(t \geqslant t_{0}\),
    then \(\mathbf{x} = \boldsymbol{\phi}(t)\) is said to be \textbf{exponentially stable}.
\end{definition}

By substitution
\[
\mathbf{y} = \mathbf{x}(t; t_{0}, \mathbf{x}_{0}) - \boldsymbol{\phi}(t),
\]
we have
\begin{align*}
    \frac{\mathrm{d} \mathbf{y}}{\mathrm{d}t} = & \frac{\mathrm{d} \mathbf{x}}{\mathrm{d}t} - \frac{\mathrm{d} \boldsymbol{\phi}}{\mathrm{d}t} \\
    = & \mathbf{f}(t, \mathbf{x}) - \mathbf{f}(t, \boldsymbol{\phi}) \\
    = & \mathbf{f}(t, \mathbf{y} + \boldsymbol{\phi}) - \mathbf{f}(t, \boldsymbol{\phi}) \\
    := & \mathbf{F}(t, \mathbf{y}),
\end{align*}
then the system~\eqref{eq:differential equations system Lyapunov} can be transformed into:
\begin{equation}\label{eq:transformed system Lyapunov}
    \frac{\mathrm{d} \mathbf{y}}{\mathrm{d}t} = \mathbf{F}(t, \mathbf{y}).
\end{equation}
Then the stability of solution \(\mathbf{x} = \boldsymbol{\phi}(t; t_{0}, \mathbf{x}_{1})\)
to~\eqref{eq:differential equations system Lyapunov} is \underline{equivalent to the stability of the zero solution}
to~\eqref{eq:transformed system Lyapunov}.

Without loss of generality, we only discuss the stability of the zero solution \(\mathbf{x} = \mathbf{0}\)
to the autonomous system~\eqref{eq:differential equations system Lyapunov},
and assume that \(\mathbf{f}(t, \mathbf{0}) \equiv \mathbf{0}\).
The following definition is derived:
\begin{definition}{Stability of the Zero Solution}
    The zero solution \(\mathbf{x} = \mathbf{0}\) to~\eqref{eq:differential equations system Lyapunov}
    is said to be \textbf{stable} if for any \(\varepsilon > 0\) and \(t_{0} \geqslant 0\),
    there exists a \(\delta = \delta(\varepsilon, t_{0}) > 0\),
    such that when \(\|\mathbf{x}_{0}\| < \delta\),
    it holds that
    \[
    \|\mathbf{x}(t; t_{0}, \mathbf{x}_{0})\| < \varepsilon,
    \]
    for all \(t \geqslant t_{0}\).

    If \(\mathbf{x} = \mathbf{0}\) is stable, 
    and there exists \(\delta_{1}\in (0, \delta]\) such that when \(\|\mathbf{x}_{0}\| < \delta_{1}\),
    it holds that
    \[
    \lim_{t \to +\infty} \|\mathbf{x}(t; t_{0}, \mathbf{x}_{0})\| = 0,
    \]
    then \(\mathbf{x} = \mathbf{0}\) is said to be \textbf{asymptotically stable}.

    If \(\mathbf{x} = \mathbf{0}\) is asymptotically stable,
    and there exists \(\delta_{2}\in (0, \delta_{1}], \alpha, \beta\) such that 
    when \(\|\mathbf{x}_{0}\| < \delta_{2}\),
    it holds that
    \[
    \|\mathbf{x}(t; t_{0}, \mathbf{x}_{0})\| \leqslant 
    \alpha \|\mathbf{x}_{0}\| e^{-\beta (t - t_{0})},
    \]
    for all \(t \geqslant t_{0}\),
    then \(\mathbf{x} = \mathbf{0}\) is said to be \textbf{exponentially stable}.
\end{definition}

\begin{figure}[h]
    \centering
    \includegraphics[width=0.8\textwidth]{img/stable.png}
\end{figure}

\section{Ляпунов Second Method} % Lyapunov 第二方法
Ляпунов established two methods for studying the stability of solutions to differential equations:
\newline The first method is based on the series solutions of differential equations,
which is often difficult to apply in practice.
\newline The second method, known as the Ляпунов second method,
relies on constructing a special scalar function, called the Ляпунов function,
to analyze the stability of solutions without explicitly solving the differential equations,
which is also called the direct method of Ляпунов.

For convenience, we only consider autonomous system:
\begin{equation}\label{eq:autonomous system Lyapunov}
    \frac{\mathrm{d}\mathbf{x}}{\mathrm{d}t} = \mathbf{f}(\mathbf{x}),\quad \mathbf{x}\in G \subset \mathbb{R}^{n},
\end{equation}
where \(\mathbf{f}(\mathbf{x})\) is continuous in \(G\)
and satisfies the local Lipschitz condition with respect to \(\mathbf{x}\),
and \(\mathbf{f}(\mathbf{0}) = \mathbf{0}\).

\begin{definition}{Ляпунов Function}
    A scalar function \(V(\mathbf{x}): G \to \mathbb{R}\) is called a \textbf{Ляпунов function},
    if it satisfies the following conditions:
    \begin{itemize}
        \item \(V(\mathbf{0}) = 0\);
        \item \(V(\mathbf{x}), \nabla V(\mathbf{x})\) are continuous in \(G\).
    \end{itemize}
    In \(G_{1} \subseteq G\),
    If \(V(\mathbf{x}) > 0 (<0)\) for all \(\mathbf{x}\) except \(\mathbf{0}\),
    then \(V(\mathbf{x})\) is called a \textbf{positive/negative definite};
    if \(V(\mathbf{x}) \geqslant 0 (\leqslant 0)\) for all \(\mathbf{x}\),
    then \(V(\mathbf{x})\) is called a \textbf{positive/negative semidefinite};
    otherwise, it is called an \textbf{indefinite}.
\end{definition}


\begin{theorem}{Ляпунов Stability Theorem}
    For the autonomous system~\eqref{eq:autonomous system Lyapunov},
    let \(V(\mathbf{x})\) be a Ляпунов function in \(G\),
    and its total derivative along the trajectories of~\eqref{eq:autonomous system Lyapunov} is given by:
    \[
    \dot{V}(\mathbf{x}) = \nabla V(\mathbf{x}) \cdot \mathbf{f}(\mathbf{x}) 
    = \sum_{i=1}^{n}\frac{\partial V}{\partial x_i} f_i(\mathbf{x}).
    \]
    Then:
    \begin{description}
        \item[Stability Theorem] if \(V(\mathbf{x})\) is positive definite
            and \(\dot{V}(\mathbf{x})\) is negative semidefinite in \(G\),
            then the zero solution \(\mathbf{x} = \mathbf{0}\) is stable.
        \item[Asymptotically Stability Theorem] if \(V(\mathbf{x})\) is positive definite
            and \(\dot{V}(\mathbf{x})\) is negative definite in \(G\),
            then the zero solution \(\mathbf{x} = \mathbf{0}\) is asymptotically stable.
        \item[Unstability Theorem] if \(V(\mathbf{x})\) is positive definite
            and \(\dot{V}(\mathbf{x})\) is positive definite in \(G\),
            then the zero solution \(\mathbf{x} = \mathbf{0}\) is unstable.
    \end{description}
\end{theorem}

\section{Critical Points and Limit Cycles} % 平面奇点与极限环
For plane system~\eqref{eq:phase plane system},
a critical point \((x_{0}, y_{0})\) is called a \textbf{primary critical point}
if \(\det A \neq 0\), where
\[
A = \begin{pmatrix}
\frac{\partial F}{\partial x} & \frac{\partial F}{\partial y} \\
\frac{\partial G}{\partial x} & \frac{\partial G}{\partial y}
\end{pmatrix}_{(x_{0}, y_{0})}.
\]
The classification of primary critical points is completely subject to the eigenvalues of matrix \(A\)
(Hartman-Grobman theorem). 

For an eigenvalue \(\lambda\) of matrix \(A\), we are concerned with its three properties:
\begin{itemize}
    \item real part: \(\mathrm{Re}(\lambda)\), if \(\mathrm{Re}(\lambda) < 0 (>0)\), 
        then the critical point is attracting (repelling) along the corresponding eigenvector direction;
    \item imaginary part: \(\mathrm{Im}(\lambda)\);
    \item multiplicity: algebraic multiplicity and geometric multiplicity.
\end{itemize}

By the Jordan canonical form of matrix, \(A\sim J\), where
there are three cases for \(J\):
\begin{description}
    \item[Distinct real eigenvalues] 
        \[
        J = \begin{pmatrix} \lambda_{1} & 0 \\ 0 & \lambda_{2} \end{pmatrix};
        \]
        In this case, the general solution of the linearized system is
        \[
        \mathbf{x}(t) = C_{1} e^{\lambda_{1} t} \boldsymbol{\xi}^{(1)} + C_{2} e^{\lambda_{2} t} \boldsymbol{\xi}^{(2)},
        \]
        if \(\lambda_{1}\lambda_{2} > 0\),
        then the critical point is a \textbf{node}
        (stable if \(\lambda_{1}, \lambda_{2} < 0\) (\textbf{nodal sink}~\ref{fig:stable node}),
        unstable if \(\lambda_{1}, \lambda_{2} > 0\)) (\textbf{nodal source}, the same pattern as~\ref{fig:stable node} 
        but arrows point outward); ;
        \begin{figure}[h]
            \centering
            \includegraphics[width=0.6\textwidth]{img/node.png}
            \caption{Stable node.}
            \label{fig:stable node}
        \end{figure}

        if \(\lambda_{1}\lambda_{2} < 0\),
        then the critical point is a \textbf{saddle point} (always unstable).

    \item[Duplicate real eigenvalue]
        \[
        J = \begin{pmatrix} \lambda & 0 \\ 0 & \lambda \end{pmatrix} \text{ or }
        J = \begin{pmatrix} \lambda & 1 \\ 0 & \lambda \end{pmatrix};
        \]
        if \(J\) is diagonal,
        then the critical point is a \textbf{star node} (stable if \(\lambda < 0\),
        unstable if \(\lambda > 0\));
        if \(J\) is non-diagonal,
        then the critical point is an \textbf{improper node} (stable if \(\lambda < 0\),
        unstable if \(\lambda > 0\)).

    \item[Complex conjugate eigenvalues]
        \[
        J = \begin{pmatrix} \alpha & \beta \\ -\beta & \alpha \end{pmatrix},\quad \beta > 0;
        \]
        if \(\alpha = 0\),
        then the critical point is a \textbf{center} (always stable);
        if \(\alpha \neq 0\),
        then the critical point is a \textbf{focus} (stable if \(\alpha < 0\),
        unstable if \(\alpha > 0\)).
\end{description}
