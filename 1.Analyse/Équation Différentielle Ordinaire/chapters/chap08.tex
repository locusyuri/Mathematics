\chapter{First-Order Partial Differential Equations} % 一阶偏微分方程

\section{Characteristics Method} % 特征线解法
Consider the general form of a first-order quasi-linear PDE with two independent variables:
\begin{equation}\label{eq:general-quasi-linear-pde}
    a(x,y,u) \frac{\partial u}{\partial x} + b(x,y,u) \frac{\partial u}{\partial y} = c(x,y,u),
\end{equation}
where \(u=u(x,y)\) is the unknown function, and \(a\), \(b\), and \(c\) are given functions, with \(a\) and \(b\) not both zero.

The left side of the equation~\eqref{eq:general-quasi-linear-pde} can be regarded as the dot product of 
vector field \((a,b)\) and the gradient vector of \(u\), which is the directional derivative of \(u\) in the direction of \((a,b)\):
\[
    a \frac{\partial u}{\partial x} + b \frac{\partial u}{\partial y} = \nabla u \cdot (a,b).
\]
The equation indicates that at every point \((x, y, u)\), the directional derivative on \((a,b)\) is equal to \(c\). 
This leads us to introduce parameters \(t\), constructing curves in the space \((x(t), y(t), u(t))\), 
such that its tangent direction is perpendicular to \((a,b,c)\) are parallel. Such curves are called \textbf{characteristic curves}.

Along these characteristic curves, the following system of ordinary differential equations holds:
\[
\frac{\mathrm{d}x}{dt} = a(x,y,u), \quad
\frac{\mathrm{d}y}{dt} = b(x,y,u), \quad    
\frac{\mathrm{d}u}{dt} = c(x,y,u).
\]
This is a three-variable autonomous ODE system. 
If \((x, y)\) is regarded as a point on the independent variable plane, 
then the first two equations determine the projection of the characteristic curve on the plane, 
which is called the \textbf{characteristic baseline}.
Along these characteristic curves, ordinary PDE reduces to 
\[
\frac{\mathrm{d}u}{dt} = c(x,y,u),
\]
which is just the third equation of the characteristic system.
Therefore, solving the PDE reduces to solving the characteristic system of ODEs.


\section{First Integrals} % 首次积分

\section{First-Order Linear Homogeneous Partial Differential Equations} % 一阶线性齐次偏微分方程
In this section, we discuss the method of characteristics for 
solving first-order linear homogeneous partial differential equations.
Consider the general form of such an equation:
\begin{equation}\label{eq:first-order-linear-homogeneous-pde}
    X_{1}(x_1, x_2, \ldots, x_n) \frac{\partial z}{\partial x_1} +
    X_{2}(x_1, x_2, \ldots, x_n) \frac{\partial z}{\partial x_2} +
    \cdots +
    X_{n}(x_1, x_2, \ldots, x_n) \frac{\partial z}{\partial x_n} = 0,
\end{equation}
or simply,
\[
    \sum_{i=1}^{n} X_{i}(x_1, x_2, \ldots, x_n) \frac{\partial z}{\partial x_i} = 0.
\]


\begin{theorem}
    \(z = \varphi(x_{1}, x_{2}, \cdots, x_{n})\) is the solution of equation \eqref{eq:first-order-linear-homogeneous-pde}
    if and only if \(\varphi\) is the first integral of the characteristic system
    \[
        \frac{dx_{1}}{X_{1}} = \frac{dx_{2}}{X_{2}} = \cdots = \frac{dx_{n}}{X_{n}}.
    \]
\end{theorem}

\begin{theorem}
    If \(\varphi_{i}(x_{1}, x_{2}, \cdots, x_{n})\) \((i = 1, 2, \cdots, n-1)\) are \(n-1\) 
    independent first integrals of the characteristic system
    \[
        \frac{dx_{1}}{X_{1}} = \frac{dx_{2}}{X_{2}} = \cdots = \frac{dx_{n}}{X_{n}},
    \]
    then the general solution of equation \eqref{eq:first-order-linear-homogeneous-pde} is given by
    \[
        \Phi(\varphi_{1}, \varphi_{2}, \cdots, \varphi_{n-1}) = 0,
    \]
    where \(\Phi\) is an arbitrary function of \(n-1\) variables.
\end{theorem}

\section{First-Order Quasi-Linear Nonhomogeneous Partial Differential Equations} % 一阶拟线性非齐次偏微分方程

