\chapter{Existence and Uniqueness Theorem} % 存在唯一性定理
\section{Picard-Lindelöf Theorem}
\begin{theorem}{Bellman-Gronwall Inequality}
    Let \(f(x), g(x)\) be continuous functions on the interval \([a, b]\),
    \(g(x) \geqslant  0\), and \(c\) be a non-negative constant.
    If
    \[
    f(x) \leqslant  c + \int_{a}^{x} f(t) g(t) \, dt,
    \]
    then
    \[
    f(x) \leqslant  c \exp\left(\int_{a}^{x} g(t) \, dt\right).
    \]
\end{theorem}

For a Cauchy problem:
\begin{equation}\label{eq:Cauchy problem}
\begin{cases}
    \frac{\mathrm{d}y}{\mathrm{d}x} = f(x, y), \\
    y(x_0) = y_0,
\end{cases}
\end{equation}
give the existence and uniqueness theorem.

\begin{leftbarTitle}{Picard-Lindelöf Theorem}\end{leftbarTitle}
\begin{theorem}{Picard-Lindelöf Theorem}
    In the Cauchy problem~\eqref{eq:Cauchy problem},
    let \(D\) be a closed rectangle in the \(xy\)-plane: 
    \[
    D = [x_{0}-a, x_{0}+a] \times [y_{0}-b, y_{0}+b].
    \]
    If the function \(f(x, y)\) satisfies the following two conditions:
    \begin{enumerate}
        \item \(f(x, y)\) is continuous in \(D\).
        \item \(f(x, y)\) satisfies the Lipschitz condition with respect to \(y\) in \(D\), 
            i.e., there exists a constant \(L > 0\) such that for any \((x, y_1), (x, y_2) \in D\),
            \[
            |f(x, y_1) - f(x, y_2)| \leqslant  L |y_1 - y_2|.
            \]
    \end{enumerate}
    Then there exists a unique solution \(y = \varphi(x)\) (\(\varphi(x_0) = y_0\)) to the Cauchy problem~\eqref{eq:Cauchy problem}
    in the interval \([x_{0}-h, x_{0}+h]\), where
    \[
    h = \min\left\{a, \frac{b}{M}\right\}, M = \max_{(x, y) \in D} |f(x, y)|.
    \]
\end{theorem}


\begin{note}
    % Picard 迭代的主要过程如下:
    The main process of \textbf{Picard iteration} is as follows:
    \begin{enumerate}
        \item Convert the Cauchy problem~\eqref{eq:Cauchy problem} into an equivalent integral equation:
            \[
            y = y_0 + \int_{x_0}^{x} f(t, y) \, dt.
            \]
        \item Construct a sequence of approximate solutions \(\{y_n\}\) using the iteration formula:
            \[
            \varphi_{n+1} = y_0 + \int_{x_0}^{x} f(t, \varphi_n) \, dt, \quad n = 0, 1, 2, \ldots,
            \]
            with the initial approximation \(\varphi_0 = y_0\).
        \item Prove that the sequence \(\{\varphi_n\}\) converges uniformly to a function \(y = \varphi(x)\) 
            in the interval \([x_{0}-h, x_{0}+h]\).
            % 相邻两项的误差满足
            The error between two adjacent terms satisfies
            \[
            |\varphi_{n+1} - \varphi_n| \leqslant  \frac{M L^{n} }{(n+1)!}|x - x_0|^{n+1}.
            \]
        \item Show that the limit function \(y = \varphi(x)\) satisfies the integral equation and hence 
            is a solution to the Cauchy problem~\eqref{eq:Cauchy problem}.
        \item Finally, prove the uniqueness of the solution by assuming there are two solutions and 
            showing they must be identical using the Lipschitz condition.
    \end{enumerate}
\end{note}


\begin{leftbarTitle}{Peano Theorem and Osgood Theorem}\end{leftbarTitle}
In regard to the solutions for the Cauchy problem~\eqref{eq:Cauchy problem},
we have the following two theorems, which are weaker than the Picard-Lindelöf theorem:

\begin{definition}{Osgood Condition}
    Let \(f(x, y)\) be a continuous function in the region \(D\).
    If for any \((x, y_1), (x, y_2) \in D\), 
    \[
    |f(x, y_1) - f(x, y_2)| \leqslant  F(|y_1 - y_2|),
    \]
    where \(F(t) > 0\) (\(t > 0\)) is a continuous function, and
    \[
    \int_{0}^{\varepsilon} \frac{1}{F(t)} \, dt = +\infty, \quad \forall \varepsilon > 0,
    \]
    then \(f(x, y)\) is said to satisfy the \textbf{Osgood condition} with respect to \(y\) in \(D\).
\end{definition}

\begin{remark}
    If \(f(x, y)\) satisfies Lipschitz condition, then it also satisfies the Osgood condition.
    In fact, in this case, we can take \(F(t) = Lt\).
\end{remark}

\begin{theorem}{Peano Theorem}
    In the Cauchy problem~\eqref{eq:Cauchy problem},
    let \(D\) be a closed rectangle in the \(xy\)-plane: 
    \[
    D = [x_{0}-a, x_{0}+a] \times [y_{0}-b, y_{0}+b].
    \]
    If the function \(f(x, y)\) is continuous in \(D\),
    then there \emph{exists at least one solution} \(y = \varphi(x)\) (\(\varphi(x_0) = y_0\)) 
    to the Cauchy problem~\eqref{eq:Cauchy problem}
    in the interval \([x_{0}-h, x_{0}+h]\), where
    \[
    h = \min\left\{a, \frac{b}{M}\right\}, M = \max_{(x, y) \in D} |f(x, y)|.
    \]
\end{theorem}

\begin{theorem}{Osgood Theorem}
    In the Cauchy problem~\eqref{eq:Cauchy problem},
    let \(D\) be a closed rectangle in the \(xy\)-plane: 
    \[
    D = [x_{0}-a, x_{0}+a] \times [y_{0}-b, y_{0}+b]
    \].
    If the function \(f(x, y)\) satisfies the Osgood condition with respect to \(y\) in \(D\),
    then there exists a unique solution for any \((x_{0}, y_{0}) \in D\)
    to the Cauchy problem~\eqref{eq:Cauchy problem}
    in the interval \([x_{0}-h, x_{0}+h]\), where
    \[
    h = \min\left\{a, \frac{b}{M}\right\}, M = \max_{(x, y) \in D} |f(x, y)|.
    \]
\end{theorem}


\section{Continuation of the Solution}
\begin{leftbarTitle}{Uncontinuable Solutions}\end{leftbarTitle}
\begin{definition}{Uncontinuable Solutions}
    Let \(y = \phi(x)\) be a solution to the Cauchy problem~\eqref{eq:Cauchy problem} in the interval \(I_1\subset \mathbb{R}\).
    If there exists an another solution \(y =\tilde{\phi}(x)\) 
    in interval \(I_{2} \supsetneq I_{1}\) such that
    \[
    \tilde{\phi}(x)\mid_{I_1} \equiv \phi(x), 
    \]
    then \(y = \phi_{1}(x)\) is called \textbf{continuable},
    and \(y = \tilde{\phi}(x)\) is called a \textbf{continuation} of \(y = \phi_{1}(x)\).
    \newline If there does not exist such a solution \(y = \tilde{\phi}(x)\), 
    then \(y = \phi_{1}(x)\) is called \textbf{uncontinuable}, or \textbf{saturated}.
\end{definition}

\begin{theorem}
    In the Cauchy problem~\eqref{eq:Cauchy problem},
    let \(D\) be a \emph{bounded closed} rectangle in the \(xy\)-plane.
    
    Let the function \(f(x, y)\) be continuous in \(D\),
    and satisfies the \emph{local Lipschitz condition} with respect to \(y\) in \(D\),
    id est, for any point \((x', y') \in D\), there exists a neighborhood \(V((x', y')) \subset D\)
    and a constant \(L > 0\) such that for any \((x, y_1), (x, y_2) \in V((x', y'))\),
    \[
    |f(x, y_1) - f(x, y_2)| \leqslant  L |y_1 - y_2|.
    \]
    Then any solution \(y = \phi(x)\) passing through \((x_{0}, y_{0}) \in D\) 
    to the Cauchy problem~\eqref{eq:Cauchy problem} can be continued
    until it arbitrarily approaches the boundary of \(D\).
\end{theorem}

\begin{leftbarTitle}{Comparison Theorem}\end{leftbarTitle}
\begin{theorem}{Comparison Theorem}
    Let \(f(x, y)\) and \(g(x, y)\) be continuous functions in the region \(D\),
    and satisfy the Lipschitz condition with respect to \(y\) in \(D\).
    If for any \((x, y) \in D\),
    \[
    f(x, y) \leqslant  g(x, y),
    \]
    then the solutions \(y = \phi(x)\) and \(y = \psi(x)\) 
    passing through the same point \((x_{0}, y_{0}) \in D\)
    to the Cauchy problems
    \[
    \begin{cases}
        \frac{\mathrm{d}y}{\mathrm{d}x} = f(x, y), \\
        y(x_0) = y_0,
    \end{cases}
    \quad
    \begin{cases}
        \frac{\mathrm{d}y}{\mathrm{d}x} = g(x, y), \\
        y(x_0) = y_0,
    \end{cases}
    \]
    satisfy
    \[
    \phi(x) \leqslant  \psi(x)
    \]
    in their common interval of existence.
    
    The same conclusion holds if the above inequalities are replaced by strict inequalities.
\end{theorem}

\section{Singular Solutions and Envelopes}
\begin{definition}{Singular Solutions}
    In the differential equation \(F(x, y, y')=0\), let \(\Phi(x, y, c)\) be the general solution,
    where \(c\) is an arbitrary constant.

    If there exists an integral curve \(S\) such that
    \begin{enumerate}[label=(\roman*)]
        \item \(S\) satisfies the differential equation \(F(x, y, y')=0\),
        \item \(S\) is not a member of the family of curves represented by \(\Phi(x, y, c)\),
        \item at each point of \(S\), there are at least two distinct solutions of the equation passing through (
            thus destroying the uniqueness of the solution at that point
        ),
    \end{enumerate}
    then \(S\) is called a \textbf{singular solution} of the differential equation \(F(x, y, y')=0\).
\end{definition}

\begin{definition}{Envelope}
    Given a family of curves represented by \(\Phi(x, y, c) = 0\),
    if there exists a curve \(S\) such that at each point of \(S\),
    there is a member of the family tangent to \(S\),
    then \(S\) is called the \textbf{envelope} of the family of curves.
\end{definition}

\begin{caution}
    % 注意:奇解不一定是包络线(有可能是积分曲线族的奇点轨迹),包络线也不一定是奇解。
    A singular solution is not necessarily an envelope
    (it may be the locus of singular points of the family of integral curves),
    and an envelope is not necessarily a singular solution.
\end{caution}


\begin{theorem}{\(c\)-Test}
    Let \(\Phi(x, y, c) = 0\) be the general solution of the differential equation \(F(x, y, y')=0\).
    Then \(c\)-test curves obtained by eliminating \(c\) from the equations
    \[
    \Phi(x, y, c) = 0, \quad \frac{\partial \Phi}{\partial c} = 0.
    \]
    \(c\)-test curves is the \emph{candidate} for the envelope of the family of curves represented by \(\Phi(x, y, c) = 0\).

    If a \(c\)-test curve satisfies 
    \begin{enumerate}
        \item % 可解出连续可微的正则曲线, 即令x=varphi(c), y=psi(c), 则varphi(c), psi(c)连续可微且rank()=1, i.e., ^2 + ^2 !=0
            it can be expressed as a continuously differentiable regular curve, 
            i.e., let \(x = \varphi(c), y = \psi(c)\),
            where \(\varphi(c), \psi(c)\) are continuously differentiable and \(\operatorname{rank}(\varphi', \psi') = 1\), 
            or equivalently, \(\varphi'^{2} + \psi'^{2} \neq 0\);
        \item % 梯度不为零, 即令G(x,y)=Phi(x,y,c), 则grad G !=0
            the gradient is non-zero, i.e., \(\nabla \Phi = (\Phi_{x}, \Phi_{y}) \neq 0\);
    \end{enumerate}
    then it is just the envelope of the family of curves,
    and also a singular solution of the differential equation \(F(x, y, y')=0\).
\end{theorem}


\begin{theorem}{\(p\)-test}
    For the differential equation \(F(x, y, y')=0\) (\(p=y'\)),
    \(p\)-test curves are obtained by eliminating \(p\) from the equations
    \[
    F(x, y, p) = 0, \quad \frac{\partial F}{\partial p} = 0.
    \]

    If a \(p\)-test curve satisfies 
    \begin{enumerate}
        \item % 消去p得到的函数y=psi(x)是方程的解 
            the function \(y = \psi(x)\) obtained by eliminating \(p\) is a solution of the equation;
        \item \(F_{y}\neq 0, F_{pp}\neq 0\);
    \end{enumerate}
    then it is a singular solution of the differential equation \(F(x, y, y')=0\).
\end{theorem}

\section{Dependency of Solutions on Initial Values}
For the Cauchy problem~\eqref{eq:Cauchy problem},
let the solution passing through \((x_0, y_0)\) be denoted as \(y = \varphi(x; x_0, y_0)\).

\begin{definition}{Continuous Dependence on Initial Values}
    If for any \(\varepsilon > 0\), there exists a \(\delta > 0\) such that
    whenever \(|x_0 - x_1| < \delta\) and \(|y_0 - y_1| < \delta\),
    the solutions \(y = \varphi(x; x_0, y_0)\) and \(y = \varphi(x; x_1, y_1)\) satisfy
    \[
    |\varphi(x; x_0, y_0) - \varphi(x; x_1, y_1)| < \varepsilon,
    \]
    then the solution \(y = \varphi(x; x_0, y_0)\) is said to depend continuously on the initial values
    \((x_0, y_0)\).
\end{definition}

\begin{theorem}
    In the Cauchy problem~\eqref{eq:Cauchy problem},
    let \(D\) be a closed rectangle in the \(xy\)-plane: 
    \[
    D = [x_{0}-a, x_{0}+a] \times [y_{0}-b, y_{0}+b].
    \]
    If the function \(f(x, y)\) and its partial derivative \(f_{y}(x, y)\)
    are continuous in \(D\),
    then the solution \(y = \varphi(x; x_0, y_0)\) (as the function of \(x, x_{0}, y_{0}\))
    is \(C^{1}\) in the existence domain.

    Moreover,
    \[
    \frac{\partial \varphi}{\partial y_0} = \exp\left(\int_{x_0}^{x} f_{y}(t, \varphi(t; x_0, y_0)) \, dt\right),
    \frac{\partial \varphi}{\partial x_0} = -f(x_0, y_0) \exp\left(\int_{x_0}^{x} f_{y}(t, \varphi(t; x_0, y_0)) \, dt\right).
    \]
\end{theorem}

