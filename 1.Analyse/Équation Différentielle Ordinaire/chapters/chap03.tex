\chapter{Existence and Uniqueness Theorem} % 存在唯一性定理
\section{Picard-Lindelöf Theorem}
\begin{theorem}{Bellman-Gronwall Inequality}
    Let \(f(x), g(x)\) be continuous functions on the interval \([a, b]\),
    \(g(x) \geqslant  0\), and \(c\) be a non-negative constant.
    If
    \[
    f(x) \leqslant  c + \int_{a}^{x} f(t) g(t) \, dt,
    \]
    then
    \[
    f(x) \leqslant  c \exp\left(\int_{a}^{x} g(t) \, dt\right).
    \]
\end{theorem}

For a Cauchy problem:
\begin{equation}\label{eq:Cauchy problem}
\begin{cases}
    \frac{\mathrm{d}y}{\mathrm{d}x} = f(x, y), \\
    y(x_0) = y_0,
\end{cases}
\end{equation}
give the existence and uniqueness theorem.

\begin{leftbarTitle}{Picard-Lindelöf Theorem}\end{leftbarTitle}
\begin{theorem}{Picard-Lindelöf Theorem}
    In the Cauchy problem~\eqref{eq:Cauchy problem},
    let \(D\) be a closed rectangle in the \(xy\)-plane: 
    \[
    D = [x_{0}-a, x_{0}+a] \times [y_{0}-b, y_{0}+b].
    \]
    If the function \(f(x, y)\) satisfies the following two conditions:
    \begin{enumerate}
        \item \(f(x, y)\) is continuous in \(D\).
        \item \(f(x, y)\) satisfies the Lipschitz condition with respect to \(y\) in \(D\), 
            i.e., there exists a constant \(L > 0\) such that for any \((x, y_1), (x, y_2) \in D\),
            \[
            |f(x, y_1) - f(x, y_2)| \leqslant  L |y_1 - y_2|.
            \]
    \end{enumerate}
    Then there exists a unique solution \(y = \varphi(x)\) (\(\varphi(x_0) = y_0\)) to the Cauchy problem~\eqref{eq:Cauchy problem}
    in the interval \([x_{0}-h, x_{0}+h]\), where
    \[
    h = \min\left\{a, \frac{b}{M}\right\}, M = \max_{(x, y) \in D} |f(x, y)|.
    \]
\end{theorem}

\begin{proposition}
    
\end{proposition}

\begin{leftbarTitle}{Peano Theorem and Osgood Theorem}\end{leftbarTitle}
In regard to the solutions for the Cauchy problem~\eqref{eq:Cauchy problem},
we have the following two theorems, which are weaker than the Picard-Lindelöf theorem:

\begin{definition}{Osgood Condition}
    Let \(f(x, y)\) be a continuous function in the region \(D\).
    If for any \((x, y_1), (x, y_2) \in D\), 
    \[
    |f(x, y_1) - f(x, y_2)| \leqslant  F(|y_1 - y_2|),
    \]
    where \(F(t) > 0\) (\(t > 0\)) is a continuous function, and
    \[
    \int_{0}^{\varepsilon} \frac{1}{F(t)} \, dt = +\infty, \quad \forall \varepsilon > 0,
    \]
    then \(f(x, y)\) is said to satisfy the \textbf{Osgood condition} with respect to \(y\) in \(D\).
\end{definition}

\begin{remark}
    If \(f(x, y)\) satisfies Lipschitz condition, then it also satisfies the Osgood condition.
    In fact, in this case, we can take \(F(t) = Lt\).
\end{remark}

\begin{theorem}{Peano Theorem}
    In the Cauchy problem~\eqref{eq:Cauchy problem},
    let \(D\) be a closed rectangle in the \(xy\)-plane: 
    \[
    D = [x_{0}-a, x_{0}+a] \times [y_{0}-b, y_{0}+b]
    \].
    If the function \(f(x, y)\) is continuous in \(D\),
    then there \underline{exists} at least one solution \(y = \varphi(x)\) (\(\varphi(x_0) = y_0\)) 
    to the Cauchy problem~\eqref{eq:Cauchy problem}
    in the interval \([x_{0}-h, x_{0}+h]\), where
    \[
    h = \min\left\{a, \frac{b}{M}\right\}, M = \max_{(x, y) \in D} |f(x, y)|.
    \]
\end{theorem}

\begin{theorem}{Osgood Theorem}
    In the Cauchy problem~\eqref{eq:Cauchy problem},
    let \(D\) be a closed rectangle in the \(xy\)-plane: 
    \[
    D = [x_{0}-a, x_{0}+a] \times [y_{0}-b, y_{0}+b]
    \].
    If the function \(f(x, y)\) satisfies the Osgood condition with respect to \(y\) in \(D\),
    then there exists a unique solution for any \((x_{0}, y_{0}) \in D\)
    to the Cauchy problem~\eqref{eq:Cauchy problem}
    in the interval \([x_{0}-h, x_{0}+h]\), where
    \[
    h = \min\left\{a, \frac{b}{M}\right\}, M = \max_{(x, y) \in D} |f(x, y)|.
    \]
\end{theorem}


\section{Continuation of the Solution}
\begin{leftbarTitle}{Uncontinuable Solutions}\end{leftbarTitle}
\begin{definition}{Uncontinuable Solutions}
    Let \(y = \varphi(x)\) be a solution to the Cauchy problem~\eqref{eq:Cauchy problem} in the interval \(I_1\subset \mathbb{R}\).
    If there exists an another solution \(y = \varphi_{2}(x)\) to the Cauchy problem~\eqref{eq:Cauchy problem} 
    in any interval \(I_{2} \supsetneq I_{1}\) such that
    \[
    \varphi_{2}(x) \equiv \varphi(x), \quad x \in I_{1},
    \]
    then \(y = \varphi_{1}(x)\) is called \textbf{continuable},
    and \(y = \varphi_{2}(x)\) is called a \textbf{continuation} of \(y = \varphi_{1}(x)\).
    If there does not exist such a solution \(y = \varphi_{2}(x)\), 
    then \(y = \varphi_{1}(x)\) is called \textbf{uncontinuable}, or \textbf{saturated}.
\end{definition}

\begin{theorem}
    In the Cauchy problem~\eqref{eq:Cauchy problem},
    let \(D\) be a \underline{bounded} closed rectangle in the \(xy\)-plane.
    If the function \(f(x, y)\) is continuous in \(D\),
    and satisfies the local Lipschitz condition with respect to \(y\) in \(D\),
    then any solution \(y = \varphi(x)\) passing through \((x_{0}, y_{0}) \in D\) 
    to the Cauchy problem~\eqref{eq:Cauchy problem} can be continued
    until it arbitrarily approaches the boundary of \(D\).
\end{theorem}

\begin{leftbarTitle}{Comparison Theorem}\end{leftbarTitle}

\section{Singular Solutions and Envelopes}

\section{Dependency of Solutions on Initial Data}
