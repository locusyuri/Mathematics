\documentclass[11pt]{../../TexTemplate/elegantbook} % 这里是文档类,默认使用 elegantbook
\title{Analyse de Fourier} % 这里放置书名
% \subtitle{Subtitle} % 这里放置副标题

\author{CatMono} % 这里放置作者名
\date{November, 2025} % 这里放置日期
\version{0.1} % 这里放置版本号
% \institute{Elegant\LaTeX{} Program} % 这里放置机构名
% \bioinfo{Custom Key}{Custom Value} % 这里放置自定义信息

% \extrainfo{extra information} % 这里放置额外信息,将显示在最下方中央

\setcounter{tocdepth}{2} % 设置目录深度
\setcounter{secnumdepth}{2} % 设置章节编号深度


% \logo{logo-blue.png} % 这里放置封面logo,默认从figure目录下寻找
% \cover{LogiqueMathematique.png} % 这里放置封面图片,默认从figure目录下寻找

% modify the color in the middle of titlepage
\definecolor{customcolor}{RGB}{32,178,170} % 自定义颜色
\colorlet{coverlinecolor}{customcolor}
\usepackage{cprotect} % 保护命令参数不被 LaTeX 解析器过早处理,允许在某些特殊环境中使用脆弱命令(fragile commands)。
\usepackage{xeCJK} % 使用 xeCJK 包支持中文
\usepackage{amsmath} % 使用 amsmath 包支持数学公式

% ===== 开始文档 =====
\begin{document}

\maketitle %生成文档的标题页,根据之前定义的标题信息(如标题、作者、日期等)自动创建一个格式化的标题页

% === 前言部分 ===
\frontmatter        % 开始前言,页码为 i, ii, iii...
\tableofcontents    % 目录 (页码: i, ii)
% \listoffigures      % 图表目录 (页码: iii)
% \listoftables       % 表格目录 (页码: iv)

\chapter{Preface}   % 前言章节(无编号,页码: v, vi...)
This is the preface of the book...

% \chapter{Acknowledgments}  % 致谢(无编号)
% I would like to thank...
% === 正文部分 ===
\mainmatter         % 开始正文,页码从 1 重新开始

\chapter{Fourier Series} % 这里放置章节标题
\section{Fourier Expansion}



\chapter{Convergence of Fourier Series} % 傅里叶级数的收敛性
\section{Mean Convergence} % 均方收敛

\begin{lemma}{Riemann-Lebesgue Lemma}
    Let \( f(x) \in R[a, b] \), \( g(x) \) has a period \( T \) and \( g(x) \in R[0, T] \),
    then:
    \[
    \lim_{p \to +\infty}\int_{a}^{b} f(x)g(px) \, \mathrm{d}x 
    = \int_{a}^{b} f(x) \, \mathrm{d}x \cdot \frac{1}{T} \int_{0}^{T} g(t) \, \mathrm{d}t.
    \]
    A special case is when \( g(x) = \sin x \) or \( g(x) = \cos x \), then:
    \[
    \lim_{p \to +\infty}\int_{a}^{b} f(x)\sin(px) \, \mathrm{d}x 
    = \int_{a}^{b} f(x)\cos(px) \, \mathrm{d}x = 0.
    \]
\end{lemma}



\section{Pointwise Convergence} % 点值收敛

\chapter{Fourier Transform on \(\mathbb{R}\)} % R1 上的傅里叶变换

\chapter{Fourier Transform on \(\mathbb{R}^n\)} % Rn 上的傅里叶变换

\chapter{Finite Fourier Analysis} % 有限傅里叶分析

\chapter{Dirichlet Theorem} % Dirichlet 定理


\begin{thebibliography}{99} 
\bibitem{en1} Elias M. Stein, Rami Shakarchi. \emph{Fourier Analysis: An Introduction}. Princeton University Press, 2016.
\bibitem{en2} Author2, Title2, Journal2, Year2. \emph{ This is another example of a reference.}
\end{thebibliography}

\end{document}