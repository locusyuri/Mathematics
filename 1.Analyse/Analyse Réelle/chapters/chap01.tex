\chapter{Lebesgue Measure} % Lebesgue 测度
\begin{definition}{Measurable Space} % 可测空间
    A \textbf{measurable space} is a pair \((X, \Sigma_{X})\) where \(X\) is a set 
    and \(\Sigma_{X}\) is a \(\sigma\)-algebra of \(X\).
\end{definition}


\begin{definition}{Measure} % 测度
    Let \((X, \Sigma_{X})\) be a measurable space. A \textbf{measure} on \((X, \Sigma_{X})\) 
    is a function \(\mu : \Sigma_{X} \to [0, +\infty]\) that satisfies the following properties:
    \begin{enumerate}[label=(\roman*)]
        \item \(\mu(\emptyset) = 0\) (Null empty set);
        \item (Countable Additivity/\(\sigma\)-additivity) For any countable collection \(\{A_i\}_{i=1}^{\infty}\) 
            of pairwise disjoint sets in \(\Sigma_{X}\),
            \[
                \mu\left( \bigcup_{i=1}^{\infty} A_i \right) = \sum_{i=1}^{\infty} \mu(A_i).
            \]
    \end{enumerate}
\end{definition}

\begin{definition}{Measure Space} % 测度空间
    A \textbf{measure space} is a triple \((X, \Sigma_{X}, \mu)\) where \((X, \Sigma_{X})\) 
    is a measurable space and \(\mu\) is a measure on \((X, \Sigma_{X})\).
\end{definition}

\begin{caution}
    % 注意区分可测空间与测度空间: 可测空间只是一个集合与其 σ-代数的配对, 而测度空间则在此基础上增加了测度的概念.
    Note the distinction between measurable space and measure space:
    A measurable space is simply a pair of a set and its \(\sigma\)-algebra,
    while a measure space adds the concept of measure on top of that.
\end{caution}

\section{Outer Measure and Measurable Sets} % 外测度与可测集
\begin{leftbarTitle}{Outer Measure}\end{leftbarTitle}
\begin{definition}{Lebesgue Outer Measure} % Lebesgue 外测度
    Let \(E\subseteq \mathbb{R^{n}}\). If \(\{ I_{k} \}\) is a countable collection of open rectangles
    such that \(E \subseteq \bigcup_{k} I_{k}\), then \(\{ I_{k} \}\) is called a \textbf{\(L\)-cover} of \(E\).

    The \textbf{Lebesgue outer measure} (simplified outer measure) of \(E\) is defined as
    \[
        m^{*}(E) = \inf \left\{ \sum_{k} |I_{k}| : \{ I_{k} \} \text{ is an } L\text{-cover of } E \right\},
    \]
    where \(|I_{k}|\) denotes the volume of the rectangle \(I_{k}\).

    If for any \(L\)-cover \(\{ I_{k} \}\) of \(E\), 
    \[
        \sum_{k} |I_{k}| = +\infty,
    \]
    we define \(m^{*}(E) = +\infty\). Otherwise, \(m^{*}(E)\) is a non-negative real number.
\end{definition}

% 外测度为 0 的集合称为零测集.
Sets with outer measure zero are called \textbf{null sets}.

\begin{remark}
    % 现代实分析教学中, Lebesgue 外测度是定义 Lebesgue 测度的基础工具. 虽然也存在 Lebesgue 内测度的概念, 但其使用较少.
    % 内测度在有限测度空间或有拓扑结构时可辅助直观(如计算勒贝格测度),但因其依赖外部条件、无法自然定义σ-代数、且缺乏一般可测性判据,
    % 在基础理论中被Carathéodory方法取代。
    In modern real analysis education, Lebesgue outer measure is the foundational tool for defining Lebesgue measure.
    Although the concept of Lebesgue inner measure also exists, it is used less frequently.
    
    Inner measure can aid intuition in finite measure spaces or those with topological structures 
    (e.g., calculating Lebesgue measure),
    but due to its dependence on external conditions, inability to naturally define \(\sigma\)-algebras, 
    and lack of general measurability criteria,
    it has been supplanted by the Carathéodory method in foundational theory.
\end{remark}

\begin{property}
    \begin{description}
        \item[Non-negativity] For any \(E \subseteq \mathbb{R^{n}}\), \(m^{*}(E) \geq 0\).
        \item[Monotonicity] If \(A \subseteq B \subseteq \mathbb{R^{n}}\), then 
            \(m^{*}(A) \leq m^{*}(B)\).
        \item[Countable Subadditivity] For any countable collection \(\{ A_{i} \}_{i=1}^{\infty}\) 
            of subsets of \(\mathbb{R^{n}}\),
            \[
                m^{*}\left( \bigcup_{i=1}^{\infty} A_{i} \right) \leq \sum_{i=1}^{\infty} m^{*}(A_{i}).
            \]
    \end{description}
\end{property}


\begin{theorem}{Translation Invariance} % 平移不变性
    For any \(E \subseteq \mathbb{R^{n}}\) and any vector \(x_{0} \in \mathbb{R^{n}}\),
    \[
        m^{*}(E + x_{0}) = m^{*}(E),
    \]
    where \(E + x = \{ x + x_{0} : x \in E \}\).
\end{theorem}

\begin{leftbarTitle}{Carathéodory Measurability Criterion}\end{leftbarTitle} % Caratheodory 可测性准则
\begin{theorem}{Carathéodory Measurability Criterion} % Caratheodory 可测性准则
    A set \(E \subseteq \mathbb{R^{n}}\) is \textbf{Carathéodory measurable} (or simply \textbf{measurable}) 
    if for every set \(T \subseteq \mathbb{R^{n}}\),
    \[
        m^{*}(T) = m^{*}(T \cap E) + m^{*}(T \cap E^{c}).
    \]

    % 可测集的全体成为可测集类, 记作 \(\mathscr{M}\). 则 \((\mathbb{R^{n}}, \mathcal{M}, m)\) 构成测度空间, 其中 \(m\) 是 Lebesgue 测度.
    The collection of all measurable sets forms a \(\sigma\)-algebra, denoted by \(\mathscr{M}\).
    Thus, \((\mathbb{R^{n}}, \mathscr{M}, m)\) constitutes a measure space, where \(m\) is the Lebesgue measure.
\end{theorem}

\begin{definition}{Continuity of Measure for Increasing Sequences} % 递增可测集列的测度运算
    Let \(\{ E_{i} \}_{i=1}^{\infty}\) be a sequence of measurable sets such that 
    \(E_{1} \subseteq E_{2} \subseteq E_{3} \subseteq \cdots\).
    Then,
    \[
        m\left( \bigcup_{i=1}^{\infty} E_{i} \right) = \lim_{i \to \infty} m(E_{i}).
    \]

    Similarly, for a sequence of measurable sets \(\{ F_{i} \}_{i=1}^{\infty}\) such that
    \(F_{1} \supseteq F_{2} \supseteq F_{3} \supseteq \cdots\) and \(m(F_{1}) < +\infty\),
    \[
        m\left( \bigcap_{i=1}^{\infty} F_{i} \right) = \lim_{i \to \infty} m(F_{i}).
    \]
\end{definition}

\section{Measurable Sets and Borel Sets} % 可测集与 Borel 集
\begin{lemma}{Carathéodory Lemma}
    Let \(G\neq \mathbb{R}^{n}\) be an open set, \(E \subset G\). 
    Let 
    \[
    E_{k} = \{ x \in E : d(x, G^{c}) \geq 1/k \}, \quad k=1,2,\ldots
    \]
    Then each \(E_{k}\) is closed and
    \[
        m^{*}(E) = \lim_{k \to \infty} m^{*}(E_{k}).
    \]
\end{lemma}
