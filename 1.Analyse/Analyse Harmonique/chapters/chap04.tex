\chapter{Fourier Transform} % 傅里叶变换
\section{Introduction to Fourier Transform} % 傅里叶变换引入
\begin{leftbarTitle}{From Periodic to Non-Periodic Functions}\end{leftbarTitle} % 从周期函数到非周期函数
Until now, we have only discussed Fourier series for periodic functions.
To extend the idea of Fourier series to non-periodic functions,
we consider the limit as the period \(T \to \infty\).
In this limit, the discrete frequencies of the Fourier series become continuous, 
leading to the definition of the Fourier transform.

For a function \(f(x)\) with period \(T\), its Fourier series representation is given by
\[
f_{T}(x) = \sum_{n=-\infty}^{\infty} \left[ \frac{1}{2T} \int_{-T}^{T} f(x) e^{-i \frac{\pi n}{T} x} \, \mathrm{d}x \right]
e^{i \frac{\pi n}{T} x}.
\]
Define the frequency variable \(\omega_{n} = \frac{n}{2T}\), 
and the frequency increment \(\Delta \omega = \omega_{n+1}- \omega_{n} = \frac{1}{2T}\).
Then we can rewrite the Fourier series as
\[
f_{T}(x) = \sum_{n=-\infty}^{\infty} \left[ \int_{-T}^{T} f(x) e^{-i 2 \pi \omega_{n} x} \, \mathrm{d}x \right] 
e^{i 2 \pi \omega_{n} x} \Delta \omega.
\]
Let \(T\to \infty\), we have \(\Delta \omega \to 0\) and the sum becomes an integral:
\[
f(x) = \int_{-\infty}^{\infty} \left[ \int_{-\infty}^{\infty} f(t) e^{-i 2 \pi \omega t} \, \mathrm{d}t \right] 
e^{i 2 \pi \omega x} \, \mathrm{d}\omega.
\]
Naturally, we define the Fourier transform and its inverse as follows.
\begin{definition}{Fourier Transform and Inverse Fourier Transform} % 傅里叶变换与反变换的定义
    The \textbf{Fourier transform} of a function \(f(x)\) is defined as
    \[
    \hat{f}(\omega) = \mathcal{F}[f](\omega) = \int_{-\infty}^{\infty} f(x) e^{-i \omega x} \, \mathrm{d}x.
    \]
    The \textbf{inverse Fourier transform} is given by
    \[
    f(x) = \mathcal{F}^{-1}[\hat{f}](x) = \frac{1}{2\pi} \int_{-\infty}^{\infty} \hat{f}(\omega) e^{i \omega x} \, \mathrm{d}\omega.
    \]
\end{definition}
\begin{remark}
    The definition of \(\omega_{n}\) above leads to the difference of \(2\pi\) in the exponent compared to the classical definition.
\end{remark}

\begin{leftbarTitle}{Poisson Summation Formula}\end{leftbarTitle} % Poisson 求和公式

\section{Schwartz Space} % 速降空间: Schwartz Space
\begin{definition}{Schwartz Space} % 速降空间的定义
    The \textbf{Schwartz space} \(\mathcal{S}(\mathbb{R})\) is the set of 
    all infinitely differentiable functions \(f: \mathbb{R} \to \mathbb{C}\)
    such that for every pair of non-negative integers \(m, n\),
    \[
    \sup_{x \in \mathbb{R}} |x^{m} f^{(n)}(x)| < \infty.
    \]
    In other words, functions in \(\mathcal{S}(\mathbb{R})\) and 
    all their derivatives decay faster than any polynomial as \(|x| \to \infty\).
\end{definition}

\section{Basic Properties} % 傅里叶变换的基本性质
\begin{property}
    \begin{description}
        \item[Linearity] For any functions \(f, g \in \mathcal{S}(\mathbb{R})\) and scalars \(a, b \in \mathbb{C}\),
            \[
            \mathcal{F}[a f + b g] = a \mathcal{F}[f] + b \mathcal{F}[g].
            \]
        \item[Translation] For any function \(f \in \mathcal{S}(\mathbb{R})\) and \(x_{0} \in \mathbb{R}\),
            \[
            \mathcal{F}[f(x - x_{0})](\omega) = e^{-i \omega x_{0}} \hat{f}(\omega).
            \]
        \item[Scaling] For any function \(f \in \mathcal{S}(\mathbb{R})\) and \(a \in \mathbb{R} \setminus \{0\}\),
            \[
            \mathcal{F}[f(ax)](\omega) = \frac{1}{|a|} \hat{f}\left( \frac{\omega}{a} \right).
            \]
        \item[Differentiation] For any function \(f \in \mathcal{S}(\mathbb{R})\),
            \[
            \mathcal{F}\left[ \frac{\mathrm{d}^{n} f}{\mathrm{d} x^{n}} \right](\omega) = (i \omega)^{n} \hat{f}(\omega).
            \]
        \item[Integration] 
    \end{description}
\end{property}



\begin{leftbarTitle}{Convolution}\end{leftbarTitle} % 多种形式的卷积

\section{Fourier Inversion Theorem} % Fourier 反演定理

\section{The Dichotomy} % $L^1$ 与 $L^2$ 理论的二分 (The Dichotomy)

\section{Heisenberg's Uncertainty Principle} % 海森堡不确定性原理

\begin{theorem}{Heisenberg's Uncertainty Principle} % 海森堡不确定性原理
    For any function \(f \in \mathcal{S}(\mathbb{R})\),
    \[
    \left( \int_{-\infty}^{\infty} x^{2} |f(x)|^{2} \, \mathrm{d}x \right)
    \left( \int_{-\infty}^{\infty} \omega^{2} |\hat{f}(\omega)|^{2} \, \mathrm{d}\omega \right)
    \geq \frac{1}{4} \left( \int_{-\infty}^{\infty} |f(x)|^{2} \, \mathrm{d}x \right)^{2}.
    \]
    Equality holds if and only if \(f(x)\) is a Gaussian function of the form
    \(f(x) = A e^{-a x^{2}}\) for some constants \(A \in \mathbb{C}\) and \(a > 0\).
\end{theorem}

\section{Laplace Transform} % 拉普拉斯变换
Fourier transform requires the function to be integrable over the entire real line,
which may not hold for functions that grow exponentially.
To handle such functions, we introduce the Laplace transform, defined as follows.
\begin{definition}{Laplace Transform} % 拉普拉斯变换
    The \textbf{Laplace transform} of a function \(f(t)\) defined for \(t \geq 0\) is given by
    \[
    \mathcal{L}[f](s) = \int_{0}^{\infty} f(t) e^{-s t} \, \mathrm{d}t,
    \]
    where \(s\) is a complex variable.
\end{definition}

\begin{theorem}{Existence of Laplace Transform} % 拉普拉斯变换存在性定理
    If there exist constants \(M, s_{0} > 0\) such that \(|f(t)| \leq M e^{s_{0} t}\) for all \(t \geq 0\),
    then the Laplace transform \(\mathcal{L}[f](s)\) exists for all \(s\) with \(\mathrm{Re}(s) > s_{0}\).

    
\end{theorem}

\begin{property}
    \begin{description}
        \item[Linearity] For any functions \(f, g\) and scalars \(a, b\),
            \[
            \mathcal{L}[a f + b g](s) = a \mathcal{L}[f](s) + b \mathcal{L}[g](s).
            \]
        \item[Derivative] For any function \(f\) with appropriate growth conditions,
            \[
            \mathcal{L}\left[ \frac{\mathrm{d} f}{\mathrm{d} t} \right](s) = s \mathcal{L}[f](s) - f(0).
            \]
            Or more generally,
            \[
            \mathcal{L}\left[ \frac{\mathrm{d}^{n} f}{\mathrm{d} t^{n}} \right](s) = 
            s^{n} \mathcal{L}[f](s) - \sum_{k=0}^{n-1} s^{n-1-k} f^{(k)}(0).
            \]
    \end{description}
\end{property}

Some common Laplace transforms are listed in the following table.
\begin{table}[h]
    \centering
    \begin{tabular}{cc}
        \toprule
        \(f(t)\) & \(\mathcal{L}[f](s)\) \\
        \toprule
        \(1\) & \(\frac{1}{s}\) \\ 
        \(t^{n}\) & \(\frac{n!}{s^{n+1}}\) \\ 
        \(e^{a t}\) & \(\frac{1}{s - a}\) \\ 
        \(t^{n} e^{a t}\) & \(\frac{n!}{(s - a)^{n+1}}\) \\ 
        \(\sin(\omega t)\) & \(\frac{\omega}{s^{2} + \omega^{2}}\) \\ 
        \(\cos(\omega t)\) & \(\frac{s}{s^{2} + \omega^{2}}\) \\ 
        \(\sinh(\omega t)\) & \(\frac{\omega}{s^{2} - \omega^{2}}\) \\ 
        \(\cosh(\omega t)\) & \(\frac{s}{s^{2} - \omega^{2}}\) \\ 
        \(t^{n}e^{a t} \sin(\omega t)\) & \(\) \\ 
        \(t^{n}e^{a t} \cos(\omega t)\) & \(\) \\
        \bottomrule
    \end{tabular}
    \caption{Common Laplace Transforms}
    \label{tab:laplace transforms}
\end{table}

\section{Fast Fourier Transform} % 快速 Fourier 变换