\chapter{Cesàro Summation} % Cesàro 求和法
\section{Cesàro Summation and Fejér Kernel} % Cesàro 求和与 Fejér 核
Until now, when we discuss the convergence of series \(\sum_{n=1}^{\infty}a_{n} \), 
the convergence of partial sums \( S_{N} = \sum_{n=1}^{N} a_{n} \) as \( N \to \infty \) is considered de facto.
The definition is put forward by Cauchy in 1821.
However, there are other ways to define the convergence of series.
Here we introduce Cesàro summation, which is a method to assign sums to some divergent series\footnote{
    By Cauchy proposition (see \emph{Analyse Mathématique - Section 2.1: Convergent Sequences}),
    if \(\lim_{n \to \infty} x_n = l\), then \(\lim_{n \to \infty} \frac{x_{1}+x_{2}+ \cdots +x_{n}}{n} = l\).
    That is, the Cesàro sum of a convergent series is equal to its Cauchy sum.
}.
\begin{definition}{Cesàro Summation}
    A series \( \sum_{n=1}^{\infty} a_{n} \) is said to be Cesàro summable to \( S \) 
    if the arithmetical average \(\sigma_{k}\) of its partial sums converges to \( S \), i.e.,
    \[
    \lim_{k \to \infty} \sigma_{k} = S,
    \]
    where
    \[
    \sigma_{k} = \frac{S_{1} + S_{2} + \cdots + S_{k}}{k}.
    \]
\end{definition}

Due to 
\[
S_{N}(f; x) = \frac{1}{2\pi} \int_{-\pi}^{\pi} f(x - t) D_{N}(t) \, \mathrm{d}t,
\]
then
\begin{align*}
    \sigma_{N}(f; x) &= \frac{1}{N} \sum_{k=0}^{N-1} S_{k}(f; x) \\
    &= \frac{1}{N} \sum_{k=0}^{N-1} \frac{1}{2\pi} \int_{-\pi}^{\pi} f(x - t) D_{k}(t) \, \mathrm{d}t \\
    &= \frac{1}{2\pi} \int_{-\pi}^{\pi} f(x - t) \left( \frac{1}{N} \sum_{k=0}^{N-1} D_{k}(t) \right) \mathrm{d}t \\
    &= \frac{1}{2\pi} \int_{-\pi}^{\pi} f(x - t) F_{N}(t) \, \mathrm{d}t,
\end{align*}
where 
\[
F_{N}(t) = \frac{1}{N} \sum_{k=0}^{N-1} D_{k}(t) = \frac{1}{N} \left( \frac{\sin \frac{N t}{2}}{\sin \frac{t}{2}} \right)^{2},
\]
is called the \textbf{Fejér kernel}.

\begin{figure}[h]
    \centering
    \includegraphics[width=0.6\textwidth]{img/Fejér_kernels.png}
    \caption{Fejér kernels \( F_{N}(t) \) for \( N = 2, 4, 6, 8, 10 \).}
    \label{fig:Fejér kernels}
\end{figure}
Fejér kernel has the following excellent properties:
\begin{property}
    \begin{description}
        \item[Positivity] For any integer \( N \) and real number \( t \), \( F_{N}(t) \geqslant  0 \).
            This property significantly distinguishes Fejér kernel from Dirichlet kernel.
        \item[Normalization] For any integer \( N \),
        \[
            \frac{1}{2\pi} \int_{-\pi}^{\pi} F_{N}(t) \, \mathrm{d}t = 1.
        \]
        \item[Concentration] For any \( \delta \in (0, \pi) \),
        \[
            \lim_{N \to \infty} \int_{\delta}^{\pi} F_{N}(t) \, \mathrm{d}t = 0.
        \]
        \item[Bounded] Its \(L^{1}\) norm is bounded:
        \[
        \frac{1}{2\pi} \int_{-\pi}^{\pi} \left| F_{N}(t) \right| \, \mathrm{d}t = 1.
        \]
        Distinct from Dirichlet kernel, whose \(L^{1}\) norm grows logarithmically with \(N\):
        \[
        \frac{1}{2\pi} \int_{-\pi}^{\pi} \left| D_{N}(t) \right| \, \mathrm{d}t \sim \frac{4}{\pi^{2}} \log N.
        \]
    \end{description}
\end{property}

\begin{table}[h]
\centering
\begin{tabular}{cll}
\toprule
\textbf{Property}       & \textbf{Dirichlet Kernel ($D_N$)}                & \textbf{Fejér Kernel ($K_N$)}        \\ 
\toprule
Corresponding operation & Partial sum $S_N$ (truncation)                  & Cesàro sum $\sigma_N$ (averaging)                      \\ 
\hline
Formula feature         & $\frac{\sin\left((N+\frac{1}{2})x\right)}{2\sin\left(\frac{x}{2}\right)}$ (1st-order) 
& $\frac{1}{N}\left(\frac{\sin\left(\frac{(N+1)x}{2}\right)}{\sin\left(\frac{x}{2}\right)}\right)^2$ (square) \\ 
\hline
Positivity             & Violent oscillations (positive \& negative)     & Non-negative everywhere ($\geqslant  0$)                      \\ 
\hline
$L^1$ Norm              & $\ln N \to \infty$ (divergent)                  & $= 1$ (bounded)                                       \\ 
\hline
If \(f\in C\)    & May diverge            & Uniform convergence          \\ 
\hline
Role                    & Projection operator                             & Approximation operator                                \\ 
\bottomrule
\end{tabular}
\end{table}


% 有着这样一个"好核", 我们能发展出以下定理:
With such a "good kernel", we can develop the following theorem:
\begin{theorem}{Fejér Theorem}
    If \( f(x) \) is a continuous function defined on \(\mathbb{T}\),
    then its Cesàro means \( \sigma_{N}(f; x) \) converge uniformly to \( f(x) \) on \(\mathbb{T}\), i.e.,
    \[
        \lim_{N \to \infty} \sup_{x \in \mathbb{T}} \left| \sigma_{N}(f; x) - f(x) \right| = 0.
    \]

    The generalized version is:
    if \(f(x)\in \mathcal{R}[-\pi, \pi]\), and \(f\) has left and right limits at point \(x_{0}\in [-\pi, \pi]\),
    then its Cesàro means \(\sigma_{N}(f; x_{0})\) converge to the average of the left and right limits, i.e.,
    \[
        \lim_{N \to \infty} \sigma_{N}(f; x_{0}) = \frac{f(x_{0}^{+}) + f(x_{0}^{-})}{2}.
    \]
\end{theorem}

\begin{remark}
    It means that as long as \(f\) is integrable or absolutely integrable,
    and has left and right limits at point \(x_{0}\),
    then its Fourier series converges in Cesàro sense at \(x_{0}\).

    Compared with various convergence tests of Fourier series in previous chapter,
    Fejér theorem is brief and to the point.
\end{remark}

With Fejér theorem, we can obtain that 
the trigonometric polynomials are dense in the continuous function space $C(\mathbb{T})$.
and the Weierstrass second approximation theorem\footnote{
    The Weierstrass second approximation theorem states that
    any continuous function defined on a closed interval can be uniformly approximated
    by polynomials to any desired degree of accuracy.
    Refer to \emph{Analyse Mathématique - Section 10.3: Smooth Appropriation of Functions} for details.
}
can be proved easily.


\section{Equidistribution} % 等分布问题