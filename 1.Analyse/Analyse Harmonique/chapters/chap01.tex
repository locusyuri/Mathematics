\chapter{Classical Fourier Series} % 古典傅里叶级数
In this chapter, we will explore the Fourier series in such function space:
\begin{description}
    \item[Set and Field] The linear space we are working on is the set of all integrable 
        (in the \underline{Riemann sense})\footnote{
            For common integral, it should be Riemann integral;
            for defective integral, it should be \underline{absolute Riemann integral}.
            For convenience, we just say Riemann integral in this context.
        } 
        complex-valued periodic functions defined on \([-\pi, \pi]\)\footnote{
            It can be also defined on interval \([-T, T]\),
            but we choose \([- \pi, \pi]\) for simplicity.
        }, equipped with the usual addition and scalar multiplication of functions.
        We denote it as \( \mathcal{R}[-\pi, \pi] \) that is a infinite-dimensional linear space.
        The field of scalars is the set of complex numbers \( \mathbb{C} \).
    \item[Inner Product] For any two functions \( f(x), g(x) \) in this space, we define their inner product as:
        \[
        \langle f, g \rangle = \frac{1}{2\pi} \int_{-\pi}^{\pi} f(x) \overline{g(x)} \, \mathrm{d}x,
        \]
        where \(\frac{1}{2\pi}\) is a normalization factor.
    \item[Norm] The norm induced by this inner product is given by:
        \[
        \| f \| = \sqrt{\langle f, f \rangle} = \left( \frac{1}{2\pi} \int_{-\pi}^{\pi} |f(x)|^2 \, \mathrm{d}x \right)^{\frac{1}{2}}.
        \]
\end{description}
In fact, we often assume that the functions are always piecewise continuous or piecewise smooth on \([-\pi, \pi]\),
which is the most common case in engineering.

\begin{leftbarTitle}{Function Defined on the Unit Circle}\end{leftbarTitle} % 引入圆周上的函数
For a periodic function \( f(x): \mathbb{R}\to \mathbb{C} \) with period \( 2\pi \),
we can explore it from the perspective of complex exponential functions on the unit circle in the complex plane.
Let 
\[
\mathbb{T} = \{ z \in \mathbb{C} : |z| = 1 \},
\]
which is one-dimensional torus, also known as the unit circle in the complex plane.

For any \( \theta \in \mathbb{R} \), we can define:
\[
f(\theta) = F(e^{i\theta}),
\]
where \( F: \mathbb{T} \to \mathbb{C} \) is a \textbf{function defined on the unit circle}.
Thus, we can study the periodic function \( f(x) \) by analyzing the function \( F(z) \) on the unit circle \( \mathbb{T} \).
From the perspective of algebra, the set of all such functions \( F(z) \) forms a function space over the unit circle,
which is \underline{isomorphic} to the space of periodic functions \( f(x) \) with period \( 2\pi \).

By introducing \(\mathbb{T}\) that is a compact manifold without boundary in fact,
we can not only eliminate the hassles of endpoints but also simplify many discussions.
Furthermore, since \(\mathbb{T}\) is a multiplicative group of complex numbers,
% 我们能更好地理解傅里叶级数的本质: 紧致阿贝尔群上的对偶理论
we can better understand the essence of Fourier series: the duality theory on compact Abelian groups.



\section{Fourier Coefficients} % 傅里叶系数
\begin{theorem}
    \[
    \mathcal{E} = \{ e^{inx} : n \in \mathbb{Z} \}
    \]
    is an orthonormal basis of the inner product space \( \mathcal{R}[-\pi, \pi] \).

    In real form,
    \[
    \{ 1, \cos x, \sin x, \cos 2x, \sin 2x, \ldots \}
    \]
    is also an orthogonal basis\footnote{
        Note that this set is orthogonal but not orthonormal,
        for
        \[
        \langle 1, 1 \rangle = 1, \quad \langle \cos nx, \cos nx \rangle = \langle \sin nx, 
        \sin nx \rangle = \frac{1}{2}, \quad n \in \mathbb{N}.
        \] 
        To make it orthonormal, each function should be normalized by the appropriate factor.
    } of the inner product space \( \mathcal{R}[-\pi, \pi] \).
\end{theorem}

\begin{definition}
    The Fourier coefficients \(\hat{f}(n)\) of a function \( f(x) \in \mathcal{R}[-\pi, \pi] \) is the 
    projection of \( f(x) \) onto the basis function \( e^{inx} \):
    \[
    \hat{f}(n) = \langle f, e^{inx} \rangle = \frac{1}{2\pi} \int_{-\pi}^{\pi} f(x) e^{-inx} \, \mathrm{d}x, \quad n \in \mathbb{Z},
    \]
    that is called Euler-Fourier formula.

    Hence, the Fourier series of \( f(x) \) is given by:
    \[
    f(x) \sim \sum_{n=-\infty}^{+\infty} \hat{f}(n) e^{inx},
    \]
    or in real form:
    \[
    f(x) \sim \frac{a_0}{2} + \sum_{n=1}^{+\infty} \left[ a_n \cos(nx) + b_n \sin(nx) \right],
    \]
    where 
    \begin{align*}
    &a_{0} = \frac{1}{\pi} \int_{-\pi}^{\pi} f(x) \, \mathrm{d}x, \\
    &a_{n} = \frac{1}{\pi} \int_{-\pi}^{\pi} f(x) \cos(nx) \, \mathrm{d}x, \\ 
    &b_{n} = \frac{1}{\pi} \int_{-\pi}^{\pi} f(x) \sin(nx) \, \mathrm{d}x, \quad n = 1, 2, \ldots
    \end{align*}
    and the symbol "\(\sim\)" indicates that the right-hand side is the Fourier series representation of \( f(x) \).
\end{definition}
\begin{note}
    Utilizing Euler formula:
    \[
    \cos nx = \frac{e^{inx} + e^{-inx}}{2}, \quad \sin nx = \frac{e^{inx} - e^{-inx}}{2i},
    \]
    we can easily derive the relationship between Fourier coefficients in complex form and real form:
    \[
    \hat{f}(0) = \frac{a_0}{2}, \quad \hat{f}(n) = \frac{a_n - i b_n}{2}, 
    \quad \hat{f}(-n) = \frac{a_n + i b_n}{2}, \quad n = 1, 2, \ldots
    \]
    \[
    a_{0} = 2\hat{f}(0), \quad a_{n} = \hat{f}(n) + \hat{f}(-n) = 2\mathrm{Re} \hat{f}(n), \quad
    b_{n} = i \left[ \hat{f}(n) - \hat{f}(-n) \right] = -2\mathrm{Im} \hat{f}(n), \quad n = 1, 2, \ldots
    \]
\end{note}

% 扩展到任一周期
It can be easily extended to any periodic function with period \( 2T \) by the substitution \( x = \frac{\pi}{T} t \):
\[
f(x) \sim \sum_{n=-\infty}^{+\infty} \hat{f}(n) e^{i n \frac{\pi}{T} x},
\]
or in real form:
\[
f(x) \sim \frac{a_0}{2} + 
\sum_{n=1}^{+\infty} \left[ a_n \cos\left(n \frac{\pi}{T} x\right) + b_n \sin\left(n \frac{\pi}{T} x\right) \right].
\]

\vspace{0.7cm}
% 正弦级数与余弦级数
When \( f(x) \) is an even function, all sine terms vanish, and the Fourier series reduces to a cosine series:
\[
f(x) \sim \frac{a_0}{2} + \sum_{n=1}^{+\infty} a_n \cos(nx).
\]
When \( f(x) \) is an odd function, all cosine terms vanish, and the Fourier series reduces to a sine series:
\[
f(x) \sim \sum_{n=1}^{+\infty} b_n \sin(nx).
\]

\section{The Dirichlet Kernel} % 狄利克雷核

\begin{leftbarTitle}{Dirichlet Kernel}\end{leftbarTitle}
For partial sum of the first \( N \) terms of the Fourier series of \( f(x) \):
\[
S_{N}(f; x) = \sum_{n=-N}^{N} \hat{f}(n) e^{inx} = \frac{a_{0}}{2} + \sum_{n=1}^{N} \left[ a_n \cos(nx) + b_n \sin(nx) \right],
\]
in order to study its convergence, we can transform it into integral form.
By Euler-Fourier formula, we have:
\begin{align*}
    S_{N}(f; x) 
    & = \sum_{n=-N}^{N} \hat{f}(n) e^{inx} \\
    & = \sum_{n=-N}^{N} \left( \frac{1}{2\pi} \int_{-\pi}^{\pi} f(t) e^{-int} \, \mathrm{d}t \right) e^{inx} \\
    & = \frac{1}{2\pi} \int_{-\pi}^{\pi} f(t) \left( \sum_{n=-N}^{N} e^{in(x - t)} \right) \, \mathrm{d}t \\
    & = \frac{1}{2\pi} \int_{-\pi}^{\pi} f(t) D_{N}(x - t) \, \mathrm{d}t,
\end{align*}
where 
\[
D_{N}(x)=\sum_{n=-N}^{N} e^{inx} = \sum_{n=1}^{N} 2\cos(nx) + 1 = 
\frac{\sin\left( \frac{2N+1}{2} x \right)}{\sin\left( \frac{x}{2} \right)},
\]
is called the \textbf{Dirichlet kernel}.

Dirichlet kernel possesses the following important properties:

\begin{property}
    \begin{description}
        \item[Evenness] \[ D_{N}(-x) = D_{N}(x). \]
        \item[Normalization] 
            \[
            \frac{1}{2\pi} \int_{-\pi}^{\pi} D_{N}(x) \, \mathrm{d}x = 1.
            \]
    \end{description}
\end{property}

\begin{figure}[h]
    \centering
    \includegraphics[width=0.6\textwidth]{img/Dirichlet_kernels.png}
    \caption{Dirichlet kernels for various values of \( N \).}
    \label{fig:Dirichlet_kernels}
\end{figure}
However, \( D_{N}(x) \) is like water waves, with both positive and negative values. 
This means that during convolution (weighted averaging), positive and negative offsets may lead to extremely unstable results.
For example, for integral mean of the absolute value of the Dirichlet kernel,
which is called the \textbf{Lebesgue constant}:
\[
L_{n} := \frac{1}{2\pi} \int_{-\pi}^{\pi} |D_{N}(x)| \, \mathrm{d}x \approx \frac{4}{\pi^{2}} \ln N, \quad (N \to +\infty).
\]
It is precisely because the absolute integral of \(D_{N}(x)\) tends to infinity 
that it is a "bad kernel function". 
It amplifies errors, causing \underline{the Fourier series of a continuous function to potentially diverge}.

\vspace{0.7cm}
With the help of convolution theorem, we have:
\begin{align*}
    S_{N}(f; x) &= \frac{1}{2\pi} \int_{-\pi}^{\pi} f(t) D_{N}(x - t) \, \mathrm{d}t\\
    & \xlongequal{\text{Let } u = t - x} \frac{1}{2\pi} \int_{-\pi}^{\pi} f(x + u) D_{N}(-u) \, \mathrm{d}u \\
    & \xlongequal{D_{N}(-u) = D_{N}(u)} \frac{1}{2\pi} \int_{-\pi}^{\pi} f(x + u) D_{N}(u) \, \mathrm{d}u \\
    & \xlongequal{\text{Divide by } 2} \frac{1}{2\pi} \int_{0}^{\pi} \left[ f(x + u) + f(x - u) \right] D_{N}(u) \, \mathrm{d}u.
\end{align*}
Then the convergence of \( S_{N}(f; x) \) can be analyzed through the properties of the last integral 
that is called the \textbf{Dirichlet integral}.

Since the normalization property of Dirichlet kernel, we can analyze the difference between 
\( S_{N}(f; x) \) and any a function \( \sigma(x) \):
\[
S_{N}(f; x) - \sigma(x) = \frac{1}{2\pi} \int_{0}^{\pi} \left[ f(x + u) + f(x - u) - 2\sigma(x) \right] D_{N}(u) \, \mathrm{d}u.
\]
Denote \(\varphi_{\sigma}(u, x) = f(x + u) + f(x - u) - 2\sigma(x)\),
then the convergence of \( S_{N}(f; x) \) to \( \sigma(x) \) is equivalent to:
\[
\lim_{N \to +\infty} \int_{0}^{\pi} \varphi_{\sigma}(u, x) D_{N}(u) \, \mathrm{d}u = 0.
\]

\begin{leftbarTitle}{Convolution}\end{leftbarTitle}
\begin{definition}{Convolution} % 卷积
    For two functions \( f(x), g(x) \) defined on \(\mathbb{R}\),
    their convolution \( f * g \) is defined as:
    \[
    (f * g)(x) = \int_{-\infty}^{+\infty} f(t) g(x - t) \, \mathrm{d}t.
    \]

    Specially, if the functions are periodically defined on a finite interval \(\mathbb{T}\) with period \( 2\pi \),
    then the convolution is defined as:
    \[
    (f * g)(x) = \frac{1}{2\pi} \int_{-\pi}^{\pi} f(t) g(x - t) \, \mathrm{d}t.
    \]
    Here, \(\frac{1}{2\pi}\) is a normalization factor.
\end{definition}

\begin{remark}
    % 从物理直观的角度, 卷积是一种“加权平均”或“滤波”。$g(t)$ 是权重函数(核),它在 $x$ 点附近这种“滑动窗口”内对 $f$ 进行取样平均。
    From a physically intuitive perspective, convolution is a form of "weighted averaging" or "filtering".
    Here, \( g(t) \) serves as the weight function (kernel), 
    which samples and averages \( f \) within a "sliding window" around the point \( x \).
\end{remark}

\begin{property}
    \begin{description}
        \item[Commutativity] \( f * g = g * f \). % 交换律
        \item[Associativity] \( f * (g * h) = (f * g) * h \). % 结合律
        \item[Distributivity] \( f * (g + h) = f * g + f * h \). % 分配律
        \item[Translation Invariance] \( (T_a f) * g = T_a (f * g) \), where \( (T_a f)(x) = f(x - a) \). % 平移不变性
    \end{description}
\end{property}

\vspace{0.7cm}
With the definition of convolution, we can rewrite the partial sum of Fourier series as:
\[
S_{N}(f; x) = \frac{1}{2\pi} \int_{-\pi}^{\pi} f(t) D_{N}(x - t) \, \mathrm{d}t = (f * D_{N})(x).
\]
Actually, this is a special case of convolution theorem,
and we have the following general conclusion:

\begin{theorem}{Convolution Theorem}
    Under suitable conditions the Fourier coefficients of a convolution of two functions (or signals) 
    is the product of their Fourier coefficients,
    \[
    \widehat{f * g}(n) = \hat{f}(n) \cdot \hat{g}(n).
    \]

    In other words, the convolution in one domain corresponds to the product in another domain, 
    for example, the convolution in the time domain corresponds to the product in the frequency domain.
\end{theorem}

\begin{leftbarTitle}{Localization Theorem}\end{leftbarTitle}
First, we need the following important lemma:
\begin{lemma}{Riemann-Lebesgue Lemma}\label{lem:Riemann-Lebesgue}
    Let \( f(x) \in R[a, b] \), \( g(x) \) has a period \( T \) and \( g(x) \in R[0, T] \),
    then:
    \[
    \lim_{p \to +\infty}\int_{a}^{b} f(x)g(px) \, \mathrm{d}x 
    = \int_{a}^{b} f(x) \, \mathrm{d}x \cdot \frac{1}{T} \int_{0}^{T} g(t) \, \mathrm{d}t.
    \]
    A special case is when \( g(x) = \sin x \) or \( g(x) = \cos x \), then:
    \[
    \lim_{p \to +\infty}\int_{a}^{b} f(x)\sin(px) \, \mathrm{d}x 
    = \int_{a}^{b} f(x)\cos(px) \, \mathrm{d}x = 0.
    \]
\end{lemma}
\begin{proof}
    
    \noindent{\color{violet!80}\textbf{Special case.}}{\color{violet!80}}
    Prove for \( g(x) = \sin x \), the case for \( g(x) = \cos x \) is similar.
    
    If \(f(x)\in B[a,b]\), i.e., \(f(x)\) is integrable in the common Riemann sense on \([a,b]\).
    Then there exists \(M>0\) such that \(|f(x)|\leqslant  M\) for all \(x\in[a,b]\).
    Denote \(n=[\sqrt{p}]\), then when \(p\to +\infty\), we have \(n\to +\infty\).
    \newline Divide the interval \([a,b]\) into \(n\) subintervals of equal length:
    \[
        a = x_0 < x_1 < x_2 < \cdots < x_n = b,
    \]
    and let \(\omega_{i}\) be the oscillation of \(f(x)\) on the \(i\)-th subinterval \([x_{i-1}, x_i]\).
    \newline By the integrability theory, 
    \[
    \lim_{n \to \infty}\sum_{i=1}^{n} \omega_i\Delta x_{i} = 0.
    \]
    And we have: 
    \[
    \left| \int_{x_{i-1}}^{x_{i}}\sin(px) \, \mathrm{d}x \right|< \frac{2}{p},\quad 
    \left| \sin(px) \right|\leqslant 1.
    \]
    Then we can estimate:
    \begin{align*}
        \left| \int_{a}^{b} f(x)\sin(px) \, \mathrm{d}x \right| 
        & = \left| \sum_{i=1}^{n} \int_{x_{i-1}}^{x_{i}} f(x)\sin(px) \, \mathrm{d}x \right| \\
        & \leqslant \left| \sum_{i=1}^{n} \int_{x_{i-1}}^{x_{i}} \left( f(x)-f(x_{i}) \right)\sin(px) \, \mathrm{d}x \right| 
        + \left| \sum_{i=1}^{n} \int_{x_{i-1}}^{x_{i}} f(x_{i})\sin(px) \, \mathrm{d}x \right| \\
        & \leqslant \sum_{i=1}^{n} \omega_i \Delta x_i 
        + M\sum_{i=1}^{n}\left| \int_{x_{i-1}}^{x_{i}} \sin(px)  \, \mathrm{d}x \right| \\
        & \leqslant \sum_{i=1}^{n} \omega_i \Delta x_i + M \cdot n \cdot \frac{2}{p} \to 0, \quad (p \to +\infty).
    \end{align*}
    Thus, \(\lim_{p \to \infty}\int_{a}^{b} f(x)\sin(px) \, \mathrm{d}x = 0\).

    If \(f(x)\not\in B[a,b]\), i.e., \(f(x)\) is absolutely integrable in the improper Riemann sense on \([a,b]\).
    Without loss of generality, assume that \(f(x)\) is defective at point \(b\).
    Then 
    \[
    \forall \varepsilon > 0, \exists \delta > 0, \forall \eta \in (0, \delta): 
    \int_{b-\eta}^{b} |f(x)| \, \mathrm{d}x < \frac{\varepsilon}{2}.
    \]
    Fix such \(\eta\), then \(f(x)\in R[a,b-\eta]\).
    According to the previous discussion, there exists \(P>0\), such that when \(p>P\):
    \[
    \left| \int_{a}^{b-\eta} f(x)\sin(px) \, \mathrm{d}x \right| < \frac{\varepsilon}{2}.
    \]
    Then we have:
    \begin{align*}
        \left| \int_{a}^{b} f(x)\sin(px) \, \mathrm{d}x \right| 
        & \leqslant \left| \int_{a}^{b-\eta} f(x)\sin(px) \, \mathrm{d}x \right| 
        + \left| \int_{b-\eta}^{b} f(x)\sin(px) \, \mathrm{d}x \right| \\
        & < \frac{\varepsilon}{2} + \int_{b-\eta}^{b} |f(x)| \, \mathrm{d}x < \varepsilon.
    \end{align*}
    Thus, \(\lim_{p \to \infty}\int_{a}^{b} f(x)\sin(px) \, \mathrm{d}x = 0\).

    In summary, regardless of whether \(f(x)\) is integrable in the common Riemann sense or 
    absolutely integrable in the improper Riemann sense, 
    we have proved the special case of Riemann-Lebesgue Lemma.
\end{proof}

Then we can state Riemann's Localization Theorem:
\begin{theorem}{Riemann's Localization Theorem}
    The convergence or divergence of the Fourier series of a function \( f(x)\in  \mathcal{R}[-\pi, \pi]\) 
    at a given point \( x \) depends only on the behavior of \( f(x) \) in 
    an arbitrarily small neighborhood of \( x \).
\end{theorem}
\begin{proof}
    For any given \(\delta > 0\), since \(f(x)\in \mathcal{R}[-\pi, \pi]\),
    \(\frac{f(x+u)+f(x-u)}{\sin \frac{u}{2}}\in \mathcal{R}[\delta, \pi]\).
    \newline Then by Riemann-Lebesgue lemma (\ref{lem:Riemann-Lebesgue}), we have:
    \[
    \lim_{N \to \infty} \int_{\delta}^{\pi} 
    \left[ f(x + u) + f(x - u) \right] \frac{\sin \left( N + \frac{1}{2} \right) u}{\sin \frac{u}{2}} \, \mathrm{d}u = 0.
    \]
    Thus, divided the integral interval of \(S_{N}(f; x)\) into \([0, \delta]\) and \([\delta, \pi]\), that is:
    \[
    S_{N}(f; x) = \frac{1}{2\pi} \int_{0}^{\delta} \left[ f(x + u) + f(x - u) \right] D_{N}(u) \, \mathrm{d}u
    + \frac{1}{2\pi} \int_{\delta}^{\pi} 
    \left[ f(x + u) + f(x - u) \right] \frac{\sin \left( N + \frac{1}{2} \right) u}{\sin \frac{u}{2}} \, \mathrm{d}u.
    \]
    When \(N \to +\infty\), the second term tends to zero, i.e.,
    the convergence of \(S_{N}(f; x)\) only depends on the first term:
    \[
    \lim_{N \to +\infty} S_{N}(f; x) = \lim_{N \to +\infty} \frac{1}{2\pi} \int_{0}^{\delta}
    \left[ f(x + u) + f(x - u) \right] D_{N}(u) \, \mathrm{d}u.
    \]
\end{proof}

\vspace{0.7cm}
Since the oscillation of \(D_{N}(x)\) is so severe that it causes poor convergence, is there a way to "smooth it out"?
In fact, we can use \textbf{Cesàro summation} and \textbf{Fejér kernel} to achieve this goal, 
which will be discussed in the next chapter.

\section{Pointwise Convergence Tests} % 逐点收敛性判别法
In this section, we will discuss several important convergence tests from coarse to fine for Fourier series.

\begin{definition}{Hölder condition}
    There exists a constant \( L > 0 \) and \( \alpha \in (0, 1] \),
    such that for all sufficiently small \( \delta \):
    \[
    |f(x\pm u) - f(x)| \leqslant  L u^\alpha, \quad 0 < u < \delta,
    \]
    then \(f\) satisfies \(\alpha\)-order \textbf{Hölder condition} at point \( x \),
    denoted as \(f \in \text{Lip}_\alpha(x)\).
    When \(\alpha = 1\), it is called \textbf{Lipschitz condition}.
\end{definition}

\begin{lemma}{Dirichlet's Lemma}
    Let \(f(x)\) be monotonic on \([0, \delta]\), then:
    \[
    \lim_{p \to \infty} \int_{0}^{\delta} \frac{f(u)-f(0+)}{u} \sin(pu) \, \mathrm{d}u = 0.
    \]
\end{lemma}
\begin{proof}
    Without loss of generality, assume that \(f(x)\) is increasing on \([0, \delta]\),
    then for any \( \varepsilon \), there exists a \( \eta \in (0, \delta) \), such that:
    \[
    0 \leqslant f(u) - f(0+) < \varepsilon, \quad u \in (0, \eta].
    \]
    Divide the integral into two parts:
    \begin{align*}
        & \int_{0}^{\delta} \frac{f(u)-f(0+)}{u} \sin(pu) \, \mathrm{d}u  \\
        =& \int_{0}^{\eta} \frac{f(u)-f(0+)}{u} \sin(pu) \, \mathrm{d}u 
        + \int_{\eta}^{\delta} \frac{f(u)-f(0+)}{u} \sin(pu) \, \mathrm{d}u.
    \end{align*}

    For the first term, by integral second mean value theorem,
    there exists \(\xi \in [0, \eta]\),
    \begin{align*}
        \left| \int_{0}^{\eta} \frac{f(u)-f(0+)}{u} \sin(pu) \, \mathrm{d}u \right| 
        &=
        [f(\eta) - f(0+)] \left| \int_{\xi}^{\eta} \frac{\sin (pu)}{u} \,  \mathrm{d}u \right| \\
        & \leqslant 
        \varepsilon \left| \int_{\xi}^{\eta} \frac{\sin (pu)}{u} \,  \mathrm{d}u \right| \\
        & = 
        \left| \int_{p\xi}^{p\eta} \frac{\sin u}{u} \,  \mathrm{d}u \right| \varepsilon
    \end{align*}
    Since 
    \[
    \int_{0}^{+\infty} \frac{\sin u}{u} \,  \mathrm{d}u = \frac{\pi}{2},
    \]
    there exists a constant \(M > 0\), such that:
    \[
    \left| \int_{p\xi}^{p\eta} \frac{\sin u}{u} \,  \mathrm{d}u \right| < M,
    \]
    that is,
    \[
    \left| \int_{0}^{\eta} \frac{f(u)-f(0+)}{u} \sin(pu) \, \mathrm{d}u \right| < M \varepsilon.
    \]

    For the second term, since \(\frac{f(u)-f(0+)}{u}\in \mathcal{R}[\eta,\delta]\),
    by Riemann-Lebesgue lemma (\ref{lem:Riemann-Lebesgue}), 
    there exists a \(P > 0\), such that when \(p > P\):
    \[
    \left| \int_{\eta}^{\delta} \frac{f(u)-f(0+)}{u} \sin(pu) \, \mathrm{d}u \right| < \varepsilon.
    \]

    In summary, the conclusion holds.
\end{proof}
\begin{note}
    \begin{itemize}
        \item There is an equivalent form of Dirichlet's lemma:
        \[
        \lim_{p \to \infty} \int_{0}^{\delta} f(u) \frac{\sin (pu)}{u} \, \mathrm{d}u = \frac{\pi}{2} f(0+).
        \] 
        \item If \(f(x)\) is a piecewise monotonic bounded function,
        then Dirichlet's lemma still holds.
    \end{itemize}
\end{note}

\begin{theorem} 
    Let \( f(x) \in \mathcal{R}[-\pi, \pi] \), and satisfies one of the following conditions,
    then the Fourier series of \( f(x) \) converges to \( \frac{f(x+)+f(x-)}{2} \) at point \( x \):
    \begin{description}
        \item[Lipschitz's Test] If \(f \in \text{Lip}_\alpha(x)\).

            Since the condition is not easy to verify directly, we can use the following sufficient condition:
            the two quasi-uniliteral derivatives of \(f\) at point \(x\) exist, i.e.,
            \[
            \lim_{h \to 0^+} \frac{f(x\pm h) - f(x\pm )}{h} 
            \]
            exist finitely.

        \item[Dini's Test] There exists a \( \delta > 0 \), such that:
            \[
            \int_{0}^{\delta} \frac{|f(x + u) + f(x - u) - 2S|}{u} \, \mathrm{d}u < +\infty,
            \]
            where \( S = \frac{f(x+) + f(x-)}{2} \).

        \item[Dirichlet-Jordan Test] If \( f(x) \) is of bounded variation on some neighborhood of point \( x \),
            i.e., there exists a \( \delta > 0 \), such that \( f \in BV(x - \delta, x + \delta) \).
    \end{description}
\end{theorem}


\begin{example}\label{ex:1}
    Let \(f(x)\) be a \(2\pi\)-periodic function defined as:
    \[
    f(x) = \begin{cases}
        x, & x \in [-\pi, \pi), \\
        -\pi, & x =\pi.
    \end{cases}
    \]
    Find its Fourier series and \(1-\frac{1}{3}+\frac{1}{5}-\frac{1}{7}+\cdots\).
\end{example}
\begin{solution}
    By Euler-Fourier formula, we have:
    \begin{align*}
        & a_0 = \frac{1}{\pi} \int_{-\pi}^{\pi} f(x) \, \mathrm{d}x = 0, \\
        & a_n = \frac{1}{\pi} \int_{-\pi}^{\pi} f(x) \cos(nx) \, \mathrm{d}x 
        = \frac{1}{\pi} \left[ \left.x \cdot \frac{\sin(nx)}{n} \right|_{-\pi}^{\pi} 
        - \int_{-\pi}^{\pi} \frac{\sin(nx)}{n} \, \mathrm{d}x \right] = 0, \\
        & b_n = \frac{1}{\pi} \int_{-\pi}^{\pi} f(x) \sin(nx) \, \mathrm{d}x 
        = \frac{1}{\pi} \left[ \left.-x \cdot \frac{\cos(nx)}{n} \right|_{-\pi}^{\pi} + \int_{-\pi}^{\pi} \frac{\cos(nx)}{n} \, \mathrm{d}x \right] 
        = \frac{2(-1)^{n+1}}{n}.
    \end{align*}
    Thus, the Fourier series of \(f(x)\) is:
    \[
    f(x) \sim \sum_{n=1}^{+\infty} (-1)^{n+1}\frac{2}{n} \sin(nx).
    \]
    Since \(f(x)\in BV[-\pi, \pi]\), by Dirichlet-Jordan test,
    its Fourier series converges to \(f(x)\) at every continuous point \(x\) 
    and to \(\frac{f(x+)+f(x-)}{2}\) at every discontinuous point \(x\).
    \newline Then
    \[
    x = f(x) = \sum_{n=1}^{+\infty} (-1)^{n+1}\frac{2}{n} \sin(nx), \quad x \in (-\pi, \pi).
    \]
    Furthermore,
    \begin{align*}
        \tilde{f}(x) = 
        &\begin{cases} f(x), &x\neq 2k\pi+\pi,  \\ \frac{f(\pi+)+f(\pi-)}{2}, &x= 2k\pi+\pi, \end{cases} \\
        &= \begin{cases} f(x), & x \in (2k\pi - \pi, 2k\pi + \pi), \\ \frac{-\pi+\pi}{2}=0, & x = 2k\pi + \pi, \end{cases}\\
        & = \sum_{n=1}^{+\infty} (-1)^{n+1}\frac{2}{n} \sin(nx),
    \end{align*}
    where \(k\in \mathbb{Z}, x\in \mathbb{R}\).
    \newline Let \(x = \frac{\pi}{2}\), then we have:
    \[
    \frac{\pi}{2} = \sum_{n=1}^{+\infty} (-1)^{n+1}\frac{2}{n} \sin\left( n \cdot \frac{\pi}{2} \right),
    \]
    \[
    \frac{\pi}{4} = \sum_{n=1}^{+\infty} (-1)^{n+1} \frac{1}{n} \sin\left( n \cdot \frac{\pi}{2} \right)=
    \sum_{n=1}^{+\infty} (-1)^{2n+2} \frac{1}{2n+1} \sin\left( (2n+1) \cdot \frac{\pi}{2} \right) = 
    1 - \frac{1}{3} + \frac{1}{5} - \frac{1}{7} + \cdots. 
    \]
\end{solution}


\begin{leftbarTitle}{Gibbs Phenomenon}\end{leftbarTitle} % 吉布斯现象
When a function \( f(x) \) has jump discontinuities,
its Fourier series converges to the midpoint of the jump at the discontinuity point.
However, near the discontinuity, the Fourier series exhibits oscillations that overshoot and undershoot
the function's actual values.
This phenomenon is known as the \textbf{Gibbs phenomenon}.

For example, for a square wave function \(f(x)\) with period \( 2\pi \):
\[
f(x) = \begin{cases}
    -1, & x \in [-\pi, 0) \\
    1, & x \in (0, \pi],
\end{cases} \qquad
f(x) \sim \frac{4}{\pi} \sum_{n=0}^{+\infty} \frac{\sin((2n+1)x)}{2n+1}.
\]
Using Dirichlet-Jordan test, \(f(x)\in BV\),
then we can show that the Fourier series of \(f(x)\) converges to:
\[
\begin{cases}
    f(x), & x \text{ is continuous} \\
    \frac{f(x+) + f(x-)}{2}, & x \text{ is discontinuous}.
\end{cases}
\]
Regarding the partial sum of the \(N\)-term Fourier series \(S_{N}(f; x)\),
it will exhibit significant overshoot near the discontinuity points (Fig.~\ref{fig:gibbs}). 
Even if more sine terms are used, this approximation error will only converge to a limit of 
approximately \(9\%\) of the jump height, 
although the infinite Fourier series will eventually converge almost everywhere.
\begin{figure}[h]
    \centering
    \includegraphics[width=0.4\textwidth]{img/Gibbs.png}
    \caption{Gibbs phenomenon of square wave near a jump discontinuity.}
    \label{fig:gibbs}
\end{figure}

% 这个现象在信号处理中是一个重要的考虑因素,因为它会导致高频噪声和振铃效应。
This phenomenon is an important consideration in signal processing,
as it can cause high-frequency noise and ringing effects.

\section{Properties of Fourier series} % Fourier 级数的性质
By Riemann-Lebesgue lemma (\ref{lem:Riemann-Lebesgue}), 
we have the following important conclusion of Fourier coefficients directly:
\begin{proposition}
    Let \( f(x) \in \mathcal{R}[-\pi, \pi] \),
    then its Fourier coefficients satisfy:
    \[
    \lim_{|n| \to +\infty} \hat{f}(n) = 0.
    \]

    In real form, it is equivalent to:
    \[
    \lim_{n \to +\infty} a_n = 0, \quad \lim_{n \to +\infty} b_n = 0.
    \]
\end{proposition}
And we give a theoretical property, used as a fallback option.
\begin{theorem}{Uniqueness Theorem}
    If \( f(x), g(x) \in \mathcal{R}[-\pi, \pi] \) have the same Fourier coefficients,
    i.e., \( \hat{f}(n) = \hat{g}(n) \) for all \( n \in \mathbb{Z} \),
    then \( f(x) = g(x) \) almost everywhere on \([- \pi, \pi]\).
\end{theorem}

\begin{leftbarTitle}{Analytical Properties}\end{leftbarTitle} % 分析性质
\begin{theorem}{Termwise Integration}
    If \( f(x) \in \mathcal{R}[-\pi, \pi] \) with Fourier series:
    \[
    f(x) \sim \frac{a_0}{2} + \sum_{n=1}^{+\infty} \left[ a_n \cos(nx) + b_n \sin(nx) \right],
    \]
    then its integral also has a Fourier series obtained by termwise integration:
    \[
    \int_{c}^{x} f(t) \, \mathrm{d}t = \int_{c}^{x} \frac{a_0}{2} \, \mathrm{d}t + \sum_{n=1}^{+\infty} 
    \int_{c}^{x} \left[ a_n \cos(nt) + b_n \sin(nt) \right] \, \mathrm{d}t, \quad c, x \in [-\pi, \pi].
    \]
\end{theorem}

\begin{note}
    Note that after termwise integration, the resulting Fourier series converges on \([- \pi, \pi]\)
    (\(\sim\) to \(=\)).
    This is because integration smooths out the function, reducing oscillations and improving convergence behavior.
\end{note}

\begin{proof}
    Here, we only prove the case that \(f(x)\) has finitely many discontinuities of the first kind on \([- \pi, \pi]\).
\end{proof}


\begin{theorem}{Termwise Differentiation}
    If \( f(x) \in \mathcal{R}[-\pi, \pi] \) with Fourier series:
    \[
    f(x) \sim \frac{a_0}{2} + \sum_{n=1}^{+\infty} \left[ a_n \cos(nx) + b_n \sin(nx) \right],
    \]
    and if \( f(x) \) is piecewise smooth on \([- \pi, \pi]\),
    then its derivative also has a Fourier series obtained by termwise differentiation:
    \[
    f'(x) \sim \sum_{n=1}^{+\infty} \left[ -n a_n \sin(nx) + n b_n \cos(nx) \right].
    \]
\end{theorem}

\begin{leftbarTitle}{Smoothness and Decay Rate}\end{leftbarTitle}
The smoother and less angular a function is, 
its Fourier coefficients decay more rapidly (with fewer high-frequency components). 
The rougher a function is (with jumps), 
the more high-frequency components it has (and the slower the coefficients decay).

% 光滑性与衰减速度的表格
\begin{table}[h]
    \centering
    \begin{tabular}{cc}
        \toprule
        \textbf{Function Smoothness} & \textbf{Fourier Coefficient Decay Rate}\\
        \toprule
        \(f\in \mathcal{R}[-\pi, \pi]\) & \(\hat{f}(n)\to 0\) (slowest) \\
        \hline
        \(f\) has jump discontinuities & \( \hat{f}(n) \sim O\left( \frac{1}{n} \right) \) \\
        \hline
        \(f\) is continuous but not differentiable & \( \hat{f}(n) \sim O\left( \frac{1}{n^2} \right) \) \\
        \hline
        \(f\in C^{k}\) & \( \hat{f}(n) \sim O\left( \frac{1}{n^{k+1}} \right) \) \\
        \hline
        Analytic Function & Exponential Decay (\(O(e^{-c|n|})\)) \\
        \bottomrule
    \end{tabular}
\end{table}

