\chapter{Classical Fourier Series} % 古典傅里叶级数
In this chapter, we will explore the Fourier series in such function space:
\begin{description}
    \item[Set and Field] The linear space we are working on is the set of all integrable 
        (in the \underline{Riemann sense})\footnote{
            For common integral, it should be Riemann integral;
            for defective integral, it should be \underline{absolute Riemann integral}.
            For convenience, we just say Riemann integral in this context.
        } 
        complex-valued periodic functions defined on \([-\pi, \pi]\)\footnote{
            It can be also defined on interval \([-T, T]\),
            but we choose \([- \pi, \pi]\) for simplicity.
        }, equipped with the usual addition and scalar multiplication of functions.
        We denote it as \( \mathcal{R}[-\pi, \pi] \) that is a infinite-dimensional linear space.
        The field of scalars is the set of complex numbers \( \mathbb{C} \).
    \item[Inner Product] For any two functions \( f(x), g(x) \) in this space, we define their inner product as:
        \[
        \langle f, g \rangle = \frac{1}{2\pi} \int_{-\pi}^{\pi} f(x) \overline{g(x)} \, \mathrm{d}x,
        \]
        where \(\frac{1}{2\pi}\) is a normalization factor.
    \item[Norm] The norm induced by this inner product is given by:
        \[
        \| f \| = \sqrt{\langle f, f \rangle} = \left( \frac{1}{2\pi} \int_{-\pi}^{\pi} |f(x)|^2 \, \mathrm{d}x \right)^{\frac{1}{2}}.
        \]
\end{description}
In fact, we often assume that the functions are always piecewise continuous or piecewise smooth on \([-\pi, \pi]\),
which is the most common case in engineering.

\begin{leftbarTitle}{Function Defined on the Unit Circle}\end{leftbarTitle} % 引入圆周上的函数
For a periodic function \( f(x): \mathbb{R}\to \mathbb{C} \) with period \( 2\pi \),
we can explore it from the perspective of complex exponential functions on the unit circle in the complex plane.
Let 
\[
\mathbb{T} = \{ z \in \mathbb{C} : |z| = 1 \},
\]
which is one-dimensional torus, also known as the unit circle in the complex plane.

For any \( \theta \in \mathbb{R} \), we can define:
\[
f(\theta) = F(e^{i\theta}),
\]
where \( F: \mathbb{T} \to \mathbb{C} \) is a \textbf{function defined on the unit circle}.
Thus, we can study the periodic function \( f(x) \) by analyzing the function \( F(z) \) on the unit circle \( \mathbb{T} \).
From the perspective of algebra, the set of all such functions \( F(z) \) forms a function space over the unit circle,
which is \underline{isomorphic} to the space of periodic functions \( f(x) \) with period \( 2\pi \).

By introducing \(\mathbb{T}\) that is a compact manifold without boundary in fact,
we can not only eliminate the hassles of endpoints but also simplify many discussions.
Furthermore, since \(\mathbb{T}\) is a multiplicative group of complex numbers,
% 我们能更好地理解傅里叶级数的本质: 紧致阿贝尔群上的对偶理论
we can better understand the essence of Fourier series: the duality theory on compact Abelian groups.



\section{Fourier Coefficients} % 傅里叶系数
\begin{theorem}
    \[
    \mathcal{E} = \{ e^{inx} : n \in \mathbb{Z} \}
    \]
    or in real form:
    \[
    \{ 1, \cos x, \sin x, \cos 2x, \sin 2x, \ldots \}
    \]
    is an orthonormal basis of the inner product space \( \mathcal{R}[-\pi, \pi] \).
\end{theorem}

\begin{definition}
    The Fourier coefficients \(\hat{f}(n)\) of a function \( f(x) \in \mathcal{R}[-\pi, \pi] \) is the 
    projection of \( f(x) \) onto the basis function \( e^{inx} \):
    \[
    \hat{f}(n) = \langle f, e^{inx} \rangle = \frac{1}{2\pi} \int_{-\pi}^{\pi} f(x) e^{-inx} \, \mathrm{d}x, \quad n \in \mathbb{Z},
    \]
    that is called Euler-Fourier formula.

    Hence, the Fourier series of \( f(x) \) is given by:
    \[
    f(x) \sim \sum_{n=-\infty}^{+\infty} \hat{f}(n) e^{inx},
    \]
    or in real form:
    \[
    f(x) \sim \frac{a_0}{2} + \sum_{n=1}^{+\infty} \left[ a_n \cos(nx) + b_n \sin(nx) \right],
    \]
    where 
    \begin{align*}
    &a_{0} = \frac{1}{\pi} \int_{-\pi}^{\pi} f(x) \, \mathrm{d}x, \\
    &a_{n} = \frac{1}{\pi} \int_{-\pi}^{\pi} f(x) \cos(nx) \, \mathrm{d}x, \\ 
    &b_{n} = \frac{1}{\pi} \int_{-\pi}^{\pi} f(x) \sin(nx) \, \mathrm{d}x, \quad n = 1, 2, \ldots
    \end{align*}
    and the symbol "\(\sim\)" indicates that the right-hand side is the Fourier series representation of \( f(x) \).
\end{definition}
% 扩展到任一周期
It can be easily extended to any periodic function with period \( 2T \) by the substitution \( x = \frac{\pi}{T} t \):
\[
f(x) \sim \sum_{n=-\infty}^{+\infty} \hat{f}(n) e^{i n \frac{\pi}{T} x},
\]
or in real form:
\[
f(x) \sim \frac{a_0}{2} + 
\sum_{n=1}^{+\infty} \left[ a_n \cos\left(n \frac{\pi}{T} x\right) + b_n \sin\left(n \frac{\pi}{T} x\right) \right].
\]

\vspace{0.7cm}
% 正弦级数与余弦级数
When \( f(x) \) is an even function, all sine terms vanish, and the Fourier series reduces to a cosine series:
\[
f(x) \sim \frac{a_0}{2} + \sum_{n=1}^{+\infty} a_n \cos(nx).
\]
When \( f(x) \) is an odd function, all cosine terms vanish, and the Fourier series reduces to a sine series:
\[
f(x) \sim \sum_{n=1}^{+\infty} b_n \sin(nx).
\]

\section{The Dirichlet Kernel} % 狄利克雷核
\begin{leftbarTitle}{Convolution}\end{leftbarTitle}
\begin{definition}{Convolution} % 卷积
    For two functions \( f(x), g(x) \) defined on \(\mathbb{T}\),
    their convolution \( f * g \) is defined as:
    \[
    (f * g)(x) = \int_{-\infty}^{+\infty} f(t) g(x - t) \, \mathrm{d}t.
    \]
\end{definition}

\begin{remark}
    % 从物理直观的角度, 卷积是一种“加权平均”或“滤波”。$g(t)$ 是权重函数(核),它在 $x$ 点附近这种“滑动窗口”内对 $f$ 进行取样平均。
    From a physically intuitive perspective, convolution is a form of "weighted averaging" or "filtering".
    Here, \( g(t) \) serves as the weight function (kernel), 
    which samples and averages \( f \) within a "sliding window" around the point \( x \).
\end{remark}

\begin{property}
    \begin{description}
        \item[Symmetry] \( f * g = g * f \).
    \end{description}
\end{property}

\begin{theorem}{Convolution Theorem}
    Under suitable conditions the Fourier transform of a convolution of two functions (or signals) 
    is the product of their Fourier transforms,
    \[
    \widehat{f * g}(n) = \hat{f}(n) \cdot \hat{g}(n).
    \]

    In other words, the convolution in one domain corresponds to the product in another domain, 
    for example, the convolution in the time domain corresponds to the product in the frequency domain.
\end{theorem}

\begin{leftbarTitle}{Dirichlet Kernel}\end{leftbarTitle}
For partial sum of the first \( N \) terms of the Fourier series of \( f(x) \):
\[
S_{N}(f; x) = \sum_{n=-N}^{N} \hat{f}(n) e^{inx} = \frac{a_{0}}{2} + \sum_{n=1}^{N} \left[ a_n \cos(nx) + b_n \sin(nx) \right],
\]
in order to study its convergence, we can transform it into integral form by convolution theorem:
\[
S_{N}(f; x) = f * D_{N},
\]
where 
\[
D_{N}(x)=\sum_{n=-N}^{N} e^{inx} = \sum_{n=1}^{N} 2\cos(nx) + 1 = 
\frac{\sin\left( \frac{2N+1}{2} x \right)}{\sin\left( \frac{x}{2} \right)},
\]
is called the \textbf{Dirichlet kernel}.

Dirichlet kernel possesses the following important properties:
\begin{description}
    \item[Evenness] \( D_{N}(-x) = D_{N}(x) \).
    \item[Normalization] 
    \[
    \frac{1}{2\pi} \int_{-\pi}^{\pi} D_{N}(x) \, \mathrm{d}x = 1.
    \]
\end{description}
\begin{figure}[h]
    \centering
    \includegraphics[width=0.6\textwidth]{img/Dirichlet_kernels.png}
    \caption{Dirichlet kernels for various values of \( N \).}
    \label{fig:Dirichlet_kernels}
\end{figure}
However, \( D_{N}(x) \) is like water waves, with both positive and negative values. 
This means that during convolution (weighted averaging), positive and negative offsets may lead to extremely unstable results.
For example, for integral mean of the absolute value of the Dirichlet kernel,
which is called the \textbf{Lebesgue constant}:
\[
L_{n} := \frac{1}{2\pi} \int_{-\pi}^{\pi} |D_{N}(x)| \, \mathrm{d}x \approx \frac{4}{\pi^{2}} \ln N, \quad (N \to +\infty).
\]
It is precisely because the absolute integral of \(D_{N}(x)\) tends to infinity 
that it is a "bad kernel function". 
It amplifies errors, causing \underline{the Fourier series of a continuous function to potentially diverge}.

\vspace{0.7cm}
With the help of convolution theorem, we have:
\begin{align*}
    S_{N}(f; x) &= \frac{1}{2\pi} \int_{-\pi}^{\pi} f(t) D_{N}(x - t) \, \mathrm{d}t\\
    & \xlongequal{\text{Let } u = t - x} \frac{1}{2\pi} \int_{-\pi}^{\pi} f(x + u) D_{N}(-u) \, \mathrm{d}u \\
    & \xlongequal{D_{N}(-u) = D_{N}(u)} \frac{1}{2\pi} \int_{-\pi}^{\pi} f(x + u) D_{N}(u) \, \mathrm{d}u \\
    & \xlongequal{\text{Divide by } 2} \frac{1}{2\pi} \int_{0}^{\pi} \left[ f(x + u) + f(x - u) \right] D_{N}(u) \, \mathrm{d}u.
\end{align*}
Then the convergence of \( S_{N}(f; x) \) can be analyzed through the properties of the last integral 
that is called the \textbf{Dirichlet integral}.

Since the normalization property of Dirichlet kernel, we can analyze the difference between 
\( S_{N}(f; x) \) and any a function \( \sigma(x) \):
\[
S_{N}(f; x) - \sigma(x) = \frac{1}{2\pi} \int_{0}^{\pi} \left[ f(x + u) + f(x - u) - 2\sigma(x) \right] D_{N}(u) \, \mathrm{d}u.
\]
Denote \(\varphi_{\sigma}(u, x) = f(x + u) + f(x - u) - 2\sigma(x)\),
then the convergence of \( S_{N}(f; x) \) to \( \sigma(x) \) is equivalent to:
\[
\lim_{N \to +\infty} \int_{0}^{\pi} \varphi_{\sigma}(u, x) D_{N}(u) \, \mathrm{d}u = 0.
\]

\begin{leftbarTitle}{Localization Theorem}\end{leftbarTitle}
First, we need the following important lemma:
\begin{lemma}{Riemann-Lebesgue Lemma}
    Let \( f(x) \in R[a, b] \), \( g(x) \) has a period \( T \) and \( g(x) \in R[0, T] \),
    then:
    \[
    \lim_{p \to +\infty}\int_{a}^{b} f(x)g(px) \, \mathrm{d}x 
    = \int_{a}^{b} f(x) \, \mathrm{d}x \cdot \frac{1}{T} \int_{0}^{T} g(t) \, \mathrm{d}t.
    \]
    A special case is when \( g(x) = \sin x \) or \( g(x) = \cos x \), then:
    \[
    \lim_{p \to +\infty}\int_{a}^{b} f(x)\sin(px) \, \mathrm{d}x 
    = \int_{a}^{b} f(x)\cos(px) \, \mathrm{d}x = 0.
    \]
\end{lemma}
\begin{proof}
    
    \noindent{\color{violet!80}\textbf{Special case.}}{\color{violet!80}}
    Prove for \( g(x) = \sin x \), the case for \( g(x) = \cos x \) is similar.
    
    If \(f(x)\in B[a,b]\), i.e., \(f(x)\) is integrable in the common Riemann sense on \([a,b]\).
    Then there exists \(M>0\) such that \(|f(x)|\leqslant  M\) for all \(x\in[a,b]\).
    Denote \(n=[\sqrt{p}]\), then when \(p\to +\infty\), we have \(n\to +\infty\).
    \newline Divide the interval \([a,b]\) into \(n\) subintervals of equal length:
    \[
        a = x_0 < x_1 < x_2 < \cdots < x_n = b,
    \]
    and let \(\omega_{i}\) be the oscillation of \(f(x)\) on the \(i\)-th subinterval \([x_{i-1}, x_i]\).
    \newline By the integrability theory, 
    \[
    \lim_{n \to \infty}\sum_{i=1}^{n} \omega_i\Delta x_{i} = 0.
    \]
    And we have: 
    \[
    \left| \int_{x_{i-1}}^{x_{i}}\sin(px) \, \mathrm{d}x \right|< \frac{2}{p},\quad 
    \left| \sin(px) \right|\leqslant 1.
    \]
    Then we can estimate:
    \begin{align*}
        \left| \int_{a}^{b} f(x)\sin(px) \, \mathrm{d}x \right| 
        & = \left| \sum_{i=1}^{n} \int_{x_{i-1}}^{x_{i}} f(x)\sin(px) \, \mathrm{d}x \right| \\
        & \leqslant \left| \sum_{i=1}^{n} \int_{x_{i-1}}^{x_{i}} \left( f(x)-f(x_{i}) \right)\sin(px) \, \mathrm{d}x \right| 
        + \left| \sum_{i=1}^{n} \int_{x_{i-1}}^{x_{i}} f(x_{i})\sin(px) \, \mathrm{d}x \right| \\
        & \leqslant \sum_{i=1}^{n} \omega_i \Delta x_i 
        + \sum_{i=1}^{n} \int_{x_{i-1}}^{x_{i}} \left| f(x_{i}) \right| \left| \sin(px) \right|   \, \mathrm{d}x \\
        & \leqslant \sum_{i=1}^{n} \omega_i \Delta x_i + M \cdot n \cdot \frac{2}{p} \to 0, \quad (p \to +\infty).
    \end{align*}
    Thus, \(\lim_{p \to \infty}\int_{a}^{b} f(x)\sin(px) \, \mathrm{d}x = 0\).

    If \(f(x)\not\in B[a,b]\), i.e., \(f(x)\) is absolutely integrable in the improper Riemann sense on \([a,b]\).
    Without loss of generality, assume that \(f(x)\) is defective at point \(b\).
    Then 
    \[
    \forall \varepsilon > 0, \exists \delta > 0, \forall \eta \in (0, \delta): 
    \int_{b-\eta}^{b} |f(x)| \, \mathrm{d}x < \frac{\varepsilon}{2}.
    \]
    Fix such \(\eta\), then \(f(x)\in R[a,b-\eta]\).
    According to the previous discussion, there exists \(P>0\), such that when \(p>P\):
    \[
    \left| \int_{a}^{b-\eta} f(x)\sin(px) \, \mathrm{d}x \right| < \frac{\varepsilon}{2}.
    \]
    Then we have:
    \begin{align*}
        \left| \int_{a}^{b} f(x)\sin(px) \, \mathrm{d}x \right| 
        & \leqslant \left| \int_{a}^{b-\eta} f(x)\sin(px) \, \mathrm{d}x \right| 
        + \left| \int_{b-\eta}^{b} f(x)\sin(px) \, \mathrm{d}x \right| \\
        & < \frac{\varepsilon}{2} + \int_{b-\eta}^{b} |f(x)| \, \mathrm{d}x < \varepsilon.
    \end{align*}
    Thus, \(\lim_{p \to \infty}\int_{a}^{b} f(x)\sin(px) \, \mathrm{d}x = 0\).

    In summary, regardless of whether \(f(x)\) is integrable in the common Riemann sense or 
    absolutely integrable in the improper Riemann sense, 
    we have proved the special case of Riemann-Lebesgue Lemma.
\end{proof}

Then we can state Riemann's Localization Theorem:
\begin{theorem}{Riemann's Localization Theorem}
    The convergence or divergence of the Fourier series of a function \( f(x) \) 
    at a given point \( x \) depends only on the behavior of \( f(x) \) in 
    an arbitrarily small neighborhood of \( x \).
\end{theorem}
\begin{proof}
    
\end{proof}

\vspace{0.7cm}
Since the oscillation of \(D_{N}(x)\) is so severe that it causes poor convergence, is there a way to "smooth it out"?
In fact, we can use \textbf{Cesàro summation} and \textbf{Fejér kernel} to achieve this goal, 
which will be discussed in the next chapter.

\section{Pointwise Convergence Tests} % 逐点收敛性判别法
In this section, we will discuss several important convergence tests from coarse to fine for Fourier series.

\begin{definition}{Bounded Variation} % 有界变差
    A function \( f(x) \) is said to be of bounded variation on the interval \([a, b]\),
    if there exists a constant \( M > 0 \), such that for any partition \( P = \{ x_0, x_1, \ldots, x_n \} \) of \([a, b]\):
    \[
    V_{a}^{b}(f) = \sum_{i=1}^{n} |f(x_i) - f(x_{i-1})| \leq M.
    \]
    The quantity \( V_{a}^{b}(f) \) is called the total variation of \( f(x) \) on \([a, b]\).
    We denote this as \( f \in BV[a, b] \).
\end{definition}

\begin{definition}{Hölder condition}
    There exists a constant \( L > 0 \) and \( \alpha \in (0, 1] \),
    such that for all sufficiently small \( \delta \):
    \[
    |f(x\pm u) - f(x)| \leqslant  L u^\alpha, \quad 0 < u < \delta,
    \]
    then \(f\) satisfies \(\alpha\)-order \textbf{Hölder condition} at point \( x \),
    denoted as \(f \in \text{Lip}_\alpha(x)\).
    When \(\alpha = 1\), it is called \textbf{Lipschitz condition}.
\end{definition}

\begin{theorem} 
    Let \( f(x) \in \mathcal{R}[-\pi, \pi] \), and satisfies one of the following conditions,
    then the Fourier series of \( f(x) \) converges to \( \frac{f(x+)+f(x-)}{2} \) at every point \( x \):
    \begin{description}
        \item[Lipschitz's Test] If \(f \in \text{Lip}_\alpha(x)\).

        \item[Dini's Test] There exists a \( \delta > 0 \), such that:
            \[
            \int_{0}^{\delta} \frac{|f(x + u) + f(x - u) - 2S|}{u} \, \mathrm{d}u < +\infty,
            \]
            where \( S = \frac{f(x+) + f(x-)}{2} \).

        \item[Dirichlet-Jordan Test] If \( f(x) \) is of bounded variation in a neighborhood of point \( x \),
            i.e., there exists a \( \delta > 0 \), such that \( f \in BV(x - \delta, x + \delta) \).
    \end{description}
\end{theorem}

\section{Analytical properties of Fourier series} % Fourier 级数的分析性质
