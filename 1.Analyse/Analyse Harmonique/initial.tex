\documentclass[11pt]{../../TexTemplate/elegantbook} % 这里是文档类,默认使用 elegantbook
\title{Analyse Harmonique} % 调和分析
% \subtitle{Subtitle} % 这里放置副标题

\author{CatMono} % 这里放置作者名
\date{November, 2025} % 这里放置日期
\version{0.1} % 这里放置版本号
% \institute{Elegant\LaTeX{} Program} % 这里放置机构名
% \bioinfo{Custom Key}{Custom Value} % 这里放置自定义信息

% \extrainfo{extra information} % 这里放置额外信息,将显示在最下方中央

\setcounter{tocdepth}{2} % 设置目录深度
\setcounter{secnumdepth}{2} % 设置章节编号深度


% \logo{logo-blue.png} % 这里放置封面logo,默认从figure目录下寻找
% \cover{LogiqueMathematique.png} % 这里放置封面图片,默认从figure目录下寻找

% modify the color in the middle of titlepage
\definecolor{customcolor}{RGB}{32,178,170} % 自定义颜色
\colorlet{coverlinecolor}{customcolor}
\usepackage{cprotect} % 保护命令参数不被 LaTeX 解析器过早处理,允许在某些特殊环境中使用脆弱命令(fragile commands)。
\usepackage{xeCJK} % 使用 xeCJK 包支持中文
\usepackage{amsmath} % 使用 amsmath 包支持数学公式

% ===== 开始文档 =====
\begin{document}

\maketitle %生成文档的标题页,根据之前定义的标题信息(如标题、作者、日期等)自动创建一个格式化的标题页

% === 前言部分 ===
\frontmatter        % 开始前言,页码为 i, ii, iii...
\tableofcontents    % 目录 (页码: i, ii)
% \listoffigures      % 图表目录 (页码: iii)
% \listoftables       % 表格目录 (页码: iv)

\chapter{Preface}   % 前言章节(无编号,页码: v, vi...)
This is the preface of the book...

% \chapter{Acknowledgments}  % 致谢(无编号)
% I would like to thank...
% === 正文部分 ===
\mainmatter         % 开始正文,页码从 1 重新开始

\chapter{Classical Fourier Series} % 古典傅里叶级数
In this chapter, we will explore the Fourier series in such function space:
\begin{description}
    \item[Set and Field] The linear space we are working on is the set of all integrable 
        (in the \underline{Riemann sense})\footnote{
            For common integral, it should be Riemann integral;
            for defective integral, it should be \underline{absolute Riemann integral}.
            For convenience, we just say Riemann integral in this context.
        } 
        complex-valued periodic functions defined on \([-\pi, \pi]\)\footnote{
            It can be also defined on interval \([-T, T]\),
            but we choose \([- \pi, \pi]\) for simplicity.
        }, equipped with the usual addition and scalar multiplication of functions.
        We denote it as \( \mathcal{R}[-\pi, \pi] \) that is a infinite-dimensional linear space.
        The field of scalars is the set of complex numbers \( \mathbb{C} \).
    \item[Inner Product] For any two functions \( f(x), g(x) \) in this space, we define their inner product as:
        \[
        \langle f, g \rangle = \frac{1}{2\pi} \int_{-\pi}^{\pi} f(x) \overline{g(x)} \, \mathrm{d}x,
        \]
        where \(\frac{1}{2\pi}\) is a normalization factor.
    \item[Norm] The norm induced by this inner product is given by:
        \[
        \| f \| = \sqrt{\langle f, f \rangle} = \left( \frac{1}{2\pi} \int_{-\pi}^{\pi} |f(x)|^2 \, \mathrm{d}x \right)^{\frac{1}{2}}.
        \]
\end{description}
In fact, we often assume that the functions are always piecewise continuous or piecewise smooth on \([-\pi, \pi]\),
which is the most common case in engineering.

\begin{leftbarTitle}{Function Defined on the Unit Circle}\end{leftbarTitle} % 引入圆周上的函数
For a periodic function \( f(x): \mathbb{R}\to \mathbb{C} \) with period \( 2\pi \),
we can explore it from the perspective of complex exponential functions on the unit circle in the complex plane.
Let 
\[
\mathbb{T} = \{ z \in \mathbb{C} : |z| = 1 \},
\]
which is one-dimensional torus, also known as the unit circle in the complex plane.

For any \( \theta \in \mathbb{R} \), we can define:
\[
f(\theta) = F(e^{i\theta}),
\]
where \( F: \mathbb{T} \to \mathbb{C} \) is a \textbf{function defined on the unit circle}.
Thus, we can study the periodic function \( f(x) \) by analyzing the function \( F(z) \) on the unit circle \( \mathbb{T} \).
From the perspective of algebra, the set of all such functions \( F(z) \) forms a function space over the unit circle,
which is \underline{isomorphic} to the space of periodic functions \( f(x) \) with period \( 2\pi \).

By introducing \(\mathbb{T}\) that is a compact manifold without boundary in fact,
we can not only eliminate the hassles of endpoints but also simplify many discussions.
Furthermore, since \(\mathbb{T}\) is a multiplicative group of complex numbers,
% 我们能更好地理解傅里叶级数的本质: 紧致阿贝尔群上的对偶理论
we can better understand the essence of Fourier series: the duality theory on compact Abelian groups.



\section{Fourier Coefficients} % 傅里叶系数
\begin{theorem}
    \[
    \mathcal{E} = \{ e^{inx} : n \in \mathbb{Z} \}
    \]
    or in real form:
    \[
    \{ 1, \cos x, \sin x, \cos 2x, \sin 2x, \ldots \}
    \]
    is an orthonormal basis of the inner product space \( \mathcal{R}[-\pi, \pi] \).
\end{theorem}

\begin{definition}
    The Fourier coefficients \(\hat{f}(n)\) of a function \( f(x) \in \mathcal{R}[-\pi, \pi] \) is the 
    projection of \( f(x) \) onto the basis function \( e^{inx} \):
    \[
    \hat{f}(n) = \langle f, e^{inx} \rangle = \frac{1}{2\pi} \int_{-\pi}^{\pi} f(x) e^{-inx} \, \mathrm{d}x, \quad n \in \mathbb{Z}.
    \]

    Hence, the Fourier series of \( f(x) \) is given by:
    \[
    f(x) \sim \sum_{n=-\infty}^{+\infty} \hat{f}(n) e^{inx},
    \]
    or in real form:
    \[
    f(x) \sim \frac{a_0}{2} + \sum_{n=1}^{+\infty} \left[ a_n \cos(nx) + b_n \sin(nx) \right],
    \]
    where 
    \begin{align*}
    &a_{0} = \frac{1}{\pi} \int_{-\pi}^{\pi} f(x) \, \mathrm{d}x, \\
    &a_{n} = \frac{1}{\pi} \int_{-\pi}^{\pi} f(x) \cos(nx) \, \mathrm{d}x, \\ 
    &b_{n} = \frac{1}{\pi} \int_{-\pi}^{\pi} f(x) \sin(nx) \, \mathrm{d}x, \quad n = 1, 2, \ldots
    \end{align*}
    and the symbol "\(\sim\)" indicates that the right-hand side is the Fourier series representation of \( f(x) \).
\end{definition}
% 扩展到任一周期
It can be easily extended to any periodic function with period \( 2T \) by the substitution \( x = \frac{\pi}{T} t \):
\[
f(x) \sim \sum_{n=-\infty}^{+\infty} \hat{f}(n) e^{i n \frac{\pi}{T} x},
\]
or in real form:
\[
f(x) \sim \frac{a_0}{2} + 
\sum_{n=1}^{+\infty} \left[ a_n \cos\left(n \frac{\pi}{T} x\right) + b_n \sin\left(n \frac{\pi}{T} x\right) \right].
\]

\vspace{0.7cm}
% 正弦级数与余弦级数
When \( f(x) \) is an even function, all sine terms vanish, and the Fourier series reduces to a cosine series:
\[
f(x) \sim \frac{a_0}{2} + \sum_{n=1}^{+\infty} a_n \cos(nx).
\]
When \( f(x) \) is an odd function, all cosine terms vanish, and the Fourier series reduces to a sine series:
\[
f(x) \sim \sum_{n=1}^{+\infty} b_n \sin(nx).
\]

\section{Convergence of Fourier Series} % 傅里叶级数的收敛性


\begin{lemma}{Riemann-Lebesgue Lemma}
    Let \( f(x) \in R[a, b] \), \( g(x) \) has a period \( T \) and \( g(x) \in R[0, T] \),
    then:
    \[
    \lim_{p \to +\infty}\int_{a}^{b} f(x)g(px) \, \mathrm{d}x 
    = \int_{a}^{b} f(x) \, \mathrm{d}x \cdot \frac{1}{T} \int_{0}^{T} g(t) \, \mathrm{d}t.
    \]
    A special case is when \( g(x) = \sin x \) or \( g(x) = \cos x \), then:
    \[
    \lim_{p \to +\infty}\int_{a}^{b} f(x)\sin(px) \, \mathrm{d}x 
    = \int_{a}^{b} f(x)\cos(px) \, \mathrm{d}x = 0.
    \]
\end{lemma}


\chapter{Cesàro Summation} % Cesàro 求和法

\chapter{Modern Fourier Series} % 现代傅里叶级数(Lebesgue 傅里叶级数)

\chapter{Fourier Transform} % 傅里叶变换

\chapter{Sobolev Spaces} % Sobolev 空间





\begin{thebibliography}{99} 
\bibitem{en1} Elias M. Stein, Rami Shakarchi. \emph{Fourier Analysis: An Introduction}. Princeton University Press, 2016.
\bibitem{ch1} 于品海. \emph{Fourier 分析}.
\bibitem{en2} Author2, Title2, Journal2, Year2. \emph{ This is another example of a reference.}
\end{thebibliography}

\end{document}