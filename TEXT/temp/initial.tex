\documentclass[11pt]{../../TexTemplate/elegantbook} % 这里是文档类,默认使用 elegantbook
\title{BookName} % 这里放置书名
\subtitle{Subtitle} % 这里放置副标题

\author{CatMono} % 这里放置作者名
\date{July, 2025} % 这里放置日期
\version{0.1} % 这里放置版本号
% \institute{Elegant\LaTeX{} Program} % 这里放置机构名
% \bioinfo{Custom Key}{Custom Value} % 这里放置自定义信息

% \extrainfo{extra information} % 这里放置额外信息,将显示在最下方中央

\setcounter{tocdepth}{2} % 设置目录深度
\setcounter{secnumdepth}{2} % 设置章节编号深度


% \logo{logo-blue.png} % 这里放置封面logo,默认从figure目录下寻找
% \cover{LogiqueMathematique.png} % 这里放置封面图片,默认从figure目录下寻找

% modify the color in the middle of titlepage
\definecolor{customcolor}{RGB}{32,178,170} % 自定义颜色
\colorlet{coverlinecolor}{customcolor}
\usepackage{cprotect} % 保护命令参数不被 LaTeX 解析器过早处理,允许在某些特殊环境中使用脆弱命令(fragile commands)。
\usepackage{xeCJK} % 使用 xeCJK 包支持中文
\usepackage{amsmath} % 使用 amsmath 包支持数学公式

% ===== 开始文档 =====
\begin{document}

\maketitle %生成文档的标题页,根据之前定义的标题信息(如标题、作者、日期等)自动创建一个格式化的标题页

% === 前言部分 ===
\frontmatter        % 开始前言,页码为 i, ii, iii...
\tableofcontents    % 目录 (页码: i, ii)
% \listoffigures      % 图表目录 (页码: iii)
% \listoftables       % 表格目录 (页码: iv)

\chapter{Preface}   % 前言章节(无编号,页码: v, vi...)
This is the preface of the book...

% \chapter{Acknowledgments}  % 致谢(无编号)
% I would like to thank...
% === 正文部分 ===
\mainmatter         % 开始正文,页码从 1 重新开始
\chapter{Chapter Title} % 这里放置章节标题
\section{Null}
求极限:
\[
\lim_{n \to \infty} n\left( \frac{1}{n^{2}+n} + \frac{1}{n^{2}+2n} + \cdots + \frac{1}{n^{2}+nn} \right) 
\]
\begin{solution}
    设
    \[
    S_{n} = n\left( \frac{1}{n^{2}+n} + \frac{1}{n^{2}+2n} + \cdots + \frac{1}{n^{2}+nn} \right) 
    = \sum_{k=1}^{n} \frac{1}{n+k} = \sum_{k=1}^{n} \frac{1}{n\left( 1+\frac{k}{n} \right)}.
    \]
    当 \( n \to \infty \) 时,\( S_{n} \) 可看作在区间 \( [0, 1] \) 上的 Riemann 和:
    \[
    S_{n} = \sum_{k=1}^{n} \frac{1}{1+\frac{k}{n}} \cdot \frac{1}{n}.
    \]
    因此,
    \[
    \lim_{n \to \infty} S_{n} = \int_{0}^{1} \frac{1}{1+x} \, \mathrm{d}x = \left. \ln(1+x) \right|_{0}^{1} = \ln 2.
    \]
    故所求极限为 \( \ln 2 \)。
\end{solution}

\section{Null}
求函数极限:
\[
\lim_{x \to 0} \frac{\tan x- \sin x}{\sin^{3} x}
\]
\begin{solution}
    由Taylor公式可得:
    \[
    \tan x = x + \frac{x^{3}}{3} + o(x^{3}), \quad \sin x = x - \frac{x^{3}}{6} + o(x^{3}).
    \]
    因此,
    \[
    \tan x - \sin x = \frac{x^{3}}{2} + o(x^{3}).
    \]
    又因为
    \[
    \sin^{3} x = \left( x - \frac{x^{3}}{6} + o(x^{3}) \right)^{3} = x^{3} + o(x^{3}),
    \]
    故
    \[
    \lim_{x \to 0} \frac{\tan x - \sin x}{\sin^{3} x} = \lim_{x \to 0} \frac{\frac{x^{3}}{2} + o(x^{3})}{x^{3} + o(x^{3})} = \frac{1}{2}.
    \]
\end{solution}
\section{Null}
求函数极限:
\[
\lim_{x \to 1} x^{\frac{3}{x-1}}
\]
\begin{solution}
    设 \( y = x^{\frac{3}{x-1}} \),则
    \[
    \ln y = \frac{3}{x-1} \ln x.
    \]
    当 \( x \to 1 \) 时,\( \ln x \to 0 \),因此可使用洛必达法则:
    \[
    \lim_{x \to 1} \ln y = \lim_{x \to 1} \frac{3 \ln x}{x-1} = \lim_{x \to 1} \frac{3 \cdot \frac{1}{x}}{1} = 3.
    \]
    故
    \[
    \lim_{x \to 1} y = e^{3}.
    \]
    (连续性保证了指数函数的极限)
\end{solution}

\section{Null}
设
\[
f(x) = \begin{cases} x^{2}, & x\leqslant 1, \\ ax+b, & x>1, \end{cases}
\]
为了使 \( f(x) \) 在 \( x=1 \) 处可导,求 \( a \) 和 \( b \) 的值。
\begin{solution}
    由导数极限定理,\( f(x) \) 在 \( x=1 \) 处可导的充分条件是:
    \[
    \lim_{x \to 1^{-}} f(x) = \lim_{x \to 1^{+}} f(x),
    \]
    且
    \[
    \lim_{x \to 1^{-}} f'(x) = \lim_{x \to 1^{+}} f'(x).
    \]
    (\(f(x)\)在\(\mathring{U}(1)\)的连续性与可导性已经由初等函数保证)
    \newline 计算得:
    \[
    \lim_{x \to 1^{-}} f(x) = 1, \quad \lim_{x \to 1^{+}} f(x) = a + b,
    \]
    故有
    \[
    a + b = 1. \quad (1)
    \]
    又
    \[
    f'(x) = \begin{cases} 2x, & x<1, \\ a, & x>1, \end{cases}
    \]
    因此
    \[
    \lim_{x \to 1^{-}} f'(x) = 2, \quad \lim_{x \to 1^{+}} f'(x) = a,
    \]
    故有
    \[
    a = 2. \quad (2)
    \]
    联立方程 (1) 和 (2),解得:
    \[
    a = 2, \quad b = -1.
    \]
\end{solution}

\section{Null}
求函数
\[
f(x) = \begin{cases} \frac{\sin x}{x(x-1)}, & x \neq 0,1, \\ -1, & x = 0,1, \end{cases}
\]
的间断点,并判断其类型。
\begin{solution}
    由题意可知,函数 \( f(x) \) 在 \( x=0 \) 和 \( x=1 \) 处可能存在间断点。
    \newline 计算得:
    \[
    \lim_{x \to 0} f(x) = \lim_{x \to 0} \frac{\sin x}{x(x-1)} = \lim_{x \to 0} \frac{1}{x-1} = -1,
    \]
    故函数在 \( x=0 \) 处连续。
    \newline 计算得:
    \[
    \lim_{x \to 1^{+}} f(x) = \lim_{x \to 1^{+}} \frac{\sin x}{x(x-1)} = +\infty.
    \]
    所以 \( f(x) \) 在 \( x=1 \) 处有第二类间断点。
\end{solution}

\section{Null}
求导:
\[
y= (\sin x)^{\tan x}
\]

\begin{solution}
    设 \( y = (\sin x)^{\tan x} \),则
    \[
    \ln y = \tan x \cdot \ln(\sin x).
    \]
    对两边求导,得
    \[
    \frac{1}{y} \frac{\mathrm{d}y}{\mathrm{d}x} = \sec^{2} x \cdot \ln(\sin x) + \tan x \cdot \cot x.
    \]
    因此,
    \[
    \frac{\mathrm{d}y}{\mathrm{d}x} = (\sin x)^{\tan x} \left( \sec^{2} x \cdot \ln(\sin x) + \tan x \cdot \cot x \right).
    \]
\end{solution}

\section{Null}
求由方程
\[
\sin(xy) + \ln(y-x) = x
\]
所确定的隐函数 \( y = y(x) \) 在 \( x=0 \) 处的导数 \( \left.\frac{\mathrm{d}y}{\mathrm{d}x} \right|_{x=0} \)。
\begin{solution}
    对方程两边关于 \( x \) 求导,得
    \[
    y \cos(xy) + x \cos(xy) \frac{\mathrm{d}y}{\mathrm{d}x} + \frac{1}{y-x} \left( \frac{\mathrm{d}y}{\mathrm{d}x} - 1 \right) = 1.
    \]
    当 \( x=0 \) 时,代入方程得 \( y(0) = 1 \)。
    \newline 将 \( x=0 \) 和 \( y=1 \) 代入导数方程,得
    \[
    1 \cdot \cos(0) + 0 + \frac{1}{1-0} \left( \frac{\mathrm{d}y}{\mathrm{d}x} - 1 \right) = 1,
    \]
    即
    \[
    1 + \left( \frac{\mathrm{d}y}{\mathrm{d}x} - 1 \right) = 1.
    \]
    解得
    \[
    \left. \frac{\mathrm{d}y}{\mathrm{d}x} \right|_{x=0} = 1.
    \]
\end{solution}

\section{Null}
求 \(n\) 阶导数:
\[
y = \frac{1-x}{1+x}.
\]
\begin{solution}
    设 \( y = \frac{1-x}{1+x} = 1 - \frac{2x}{1+x} \),则
    \[
    \frac{\mathrm{d}y}{\mathrm{d}x} = -2 \cdot \frac{(1+x) - x}{(1+x)^{2}} = -\frac{2}{(1+x)^{2}}.
    \]
    继续求导,得
    \[
    \frac{\mathrm{d}^{2}y}{\mathrm{d}x^{2}} = 4 \cdot \frac{1}{(1+x)^{3}},
    \]
    \[
    \frac{\mathrm{d}^{3}y}{\mathrm{d}x^{3}} = -12 \cdot \frac{1}{(1+x)^{4}},
    \]
    由此可见,导数的符号交替出现,且分母的指数随着阶数增加而增加。归纳可得:
    \[
    \frac{\mathrm{d}^{n}y}{\mathrm{d}x^{n}} = (-1)^{n} \cdot 2n! \cdot \frac{1}{(1+x)^{n+1}}.
    \]
\end{solution}

\section{Null}
设
\[
\begin{cases} x = \ln(1+t^{2}), \\ y=t-\arctan(t), \end{cases}
\]
求 \(\frac{\mathrm{d}y}{\mathrm{d}x}, \frac{\mathrm{d}^{2}y}{\mathrm{d}x^{2}}\)。

\begin{solution}
    由链式法则可得:
    \[
    \frac{\mathrm{d}y}{\mathrm{d}x} = \frac{\frac{\mathrm{d}y}{\mathrm{d}t}}{\frac{\mathrm{d}x}{\mathrm{d}t}} = \frac{1 - \frac{1}{1+t^{2}}}{\frac{2t}{1+t^{2}}} = \frac{t^{2}}{2t} = \frac{t}{2}.
    \]
    继续求导,得
    \[
    \frac{\mathrm{d}^{2}y}{\mathrm{d}x^{2}} = \frac{\mathrm{d}}{\mathrm{d}x} \left( \frac{t}{2} \right) = \frac{\frac{\mathrm{d}}{\mathrm{d}t} \left( \frac{t}{2} \right)}{\frac{\mathrm{d}x}{\mathrm{d}t}} = \frac{\frac{1}{2}}{\frac{2t}{1+t^{2}}} = \frac{1+t^{2}}{4t}.
    \]
\end{solution}

\section{Null}
设函数 \(f(x)\) 在区间 \([0,2a]\) 上连续,且满足 \(f(0)=f(2a)\),证明存在 \(\xi \in [0,a]\),使得
\[
f(\xi) = f(\xi + a).
\]

\begin{proof}
    设函数 \(g(x) = f(x) - f(x+a)\),则 \(g(x)\) 在区间 \([0,a]\) 上连续,且
    \[
    g(0) = f(0) - f(a), \quad g(a) = f(a) - f(2a) = f(a) - f(0).
    \]
    因此,
    \[
    g(0) + g(a) = 0.
    \]
    若 \(g(0) = 0\),则 \(f(0) = f(a)\),取 \(\xi = 0\) 即可。
    \newline 若 \(g(a) = 0\),则 \(f(a) = f(2a)\),取 \(\xi = a\) 即可。
    \newline 若 \(g(0) \neq 0\) 且 \(g(a) \neq 0\),则 \(g(0)\) 和 \(g(a)\) 异号。由闭区间上连续函数的介值定理,存在 \(\xi \in (0,a)\),使得
    \[
    g(\xi) = 0,
    \]
    即
    \[
    f(\xi) = f(\xi + a).
    \]
\end{proof}


\begin{thebibliography}{99} 
\bibitem{en1} 作者, Title1, Journal1, Year1. \emph{ This is an example of a reference.}
\bibitem{en2} Author2, Title2, Journal2, Year2. \emph{ This is another example of a reference.}
\end{thebibliography}

\end{document}