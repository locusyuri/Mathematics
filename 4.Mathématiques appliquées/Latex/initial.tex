\documentclass[11pt]{elegantbook}

\title{Tutoriel \LaTeX} % 这里放置书名
% \subtitle{Subtitle} % 这里放置副标题

\author{CatMono} % 这里放置作者名
\date{July, 2025} % 这里放置日期
\version{0.1} % 这里放置版本号
% \institute{Elegant\LaTeX{} Program} % 这里放置机构名
% \bioinfo{Custom Key}{Custom Value} % 这里放置自定义信息

% \extrainfo{extra information} % 这里放置额外信息,将显示在最下方中央

\setcounter{tocdepth}{2} % 设置目录深度
\setcounter{secnumdepth}{2} % 设置章节编号深度


% \logo{logo-blue.png} % 这里放置封面logo,默认从figure目录下寻找
% \cover{LogiqueMathematique.png} % 这里放置封面图片,默认从figure目录下寻找

% modify the color in the middle of titlepage
\definecolor{customcolor}{RGB}{32,178,170} % 自定义颜色
\colorlet{coverlinecolor}{customcolor}
\usepackage{cprotect} % 保护命令参数不被 LaTeX 解析器过早处理,允许在某些特殊环境中使用脆弱命令(fragile commands)。
\usepackage{xeCJK} % 使用 xeCJK 包支持中文
\usepackage{multirow} % 使用 multirow 包支持表格中跨行合并单元格
\usepackage{diagbox} % 使用 diagbox 包支持表格中斜线分割单元格
\usepackage{pdflscape} % 使用 pdflscape 包支持横向页面布局


% ===== 开始文档 =====
\begin{document}

\maketitle %生成文档的标题页,根据之前定义的标题信息(如标题、作者、日期等)自动创建一个格式化的标题页

% === 前言部分 ===
\frontmatter        % 开始前言,页码为 i, ii, iii...
\tableofcontents    % 目录 (页码: i, ii)
% \listoffigures      % 图表目录 (页码: iii)
% \listoftables       % 表格目录 (页码: iv)

% \chapter{Preface}   % 前言章节(无编号,页码: v, vi...)
% This is the preface of the book...

% \chapter{Acknowledgments}  % 致谢(无编号)
% I would like to thank...
% === 正文部分 ===
\mainmatter         % 开始正文,页码从 1 重新开始

\chapter{Document Structure} % 这里放置章节标题

\chapter{Typesetting and Formatting}

\section{Text Formatting}

\begin{leftbarTitle}{Font Size}\end{leftbarTitle} 
\tiny  tiny \(<\)
\scriptsize  scriptsize \(<\)
\footnotesize  footnotesize \(<\)
\small  small \(<\)
\normalsize  normalsize \(<\)
\large  large \(<\)
\Large  Large \(<\)
\LARGE  LARGE \(<\)
\huge  huge \(<\)
\Huge Huge 

\normalsize % 恢复默认大小

\begin{note}
    \begin{enumerate}
        \item Use these commands directly in the \textbf{main text, tables, or formulas} to change the font size.
        \item Each command affects the text that follows it, until another font size command is encountered or the environment ends.
        \item \lstinline|\normalsize| is the default font size and can be used to restore the normal size.
    \end{enumerate}
\end{note}



\chapter{Mathematical Formulas and Environments}


\section{Multi-line and complex math environments}
Math mode in~\LaTeX~is divided into inline math mode (such as \lstinline|$...$,\(...\)| , 
suitable for inserting short formulas within the text) and display math mode 
(such as \lstinline|$$...$$,\[...\],displaymath| or environments like \lstinline|equation|, 
used for independently centered and longer or more important formulas). 
It is recommended to use \lstinline|\(...\)| and \lstinline|\[...\]|, and not to use \lstinline|$$...$$|.

A few common environments for multi-line and complex formulas are listed below.

\begin{tabular}{|c|c|c|c|c|}
\hline
\diagbox{}{} & \textbf{Environment} & \textbf{Alignment} & \textbf{Usage} & \textbf{Remark} \\
\hline
\multirow{4}{*}{\rotatebox{90}{Standalone}} 
    & align      & \checkmark & Multiline equation alignment &{\scriptsize Each line numbered \textbf{independently}} \\
\cline{2-5}
    & gather     &            & {\scriptsize Centered arrangement of multiline equations}         &  \\
\cline{2-5}
    & multline   &            & Long equations, line breaks& {\scriptsize First line left-aligned, last line right-aligned} \\
\cline{2-5}
    & equation   &            & {\scriptsize Multi-line alignment for a \textbf{single formula}}& Overall numbering \\
\hline
\multirow{2}{*}{\rotatebox{90}{Sub}} 
    & aligned    & \checkmark & {\scriptsize Local alignment (inside another math env)} & Used in equation, align, gather \\
\cline{2-5}
    & cases      & \checkmark & Piecewise functions        & One \& each case  \\
\cline{2-5}
    & split      & \checkmark & Split a long equation       & Used within equation \\
\hline
\end{tabular}

\begin{note}
    \begin{enumerate}
        \item \lstinline|equation*,align*,gather*,multline*| are used for unnumbered equations.
        \item Since \lstinline|equation|, \lstinline|split|, and \lstinline|aligned| are used for multi-line alignment of a single equation, 
        and \lstinline|cases| is for an entire piecewise function, \textbf{none of them support page breaks}.
        \item All of these environments come from the amsmath package; 
        environments such as \lstinline|eqnarray| and \lstinline|array| are not recommended for use.
        \item Sub-environments must be used within the standalone environments.
    \end{enumerate}
\end{note}



\chapter{Figures and Tables}

\chapter{Citations and Indexing}

\chapter{Packages and Customization}

\chapter{Templates}

\chapter{Appendices}

\begin{thebibliography}{99} 
\bibitem{en1} 作者, Title1, Journal1, Year1. \emph{ This is an example of a reference.}
\bibitem{en2} Author2, Title2, Journal2, Year2. \emph{ This is another example of a reference.}
\end{thebibliography}

\end{document}