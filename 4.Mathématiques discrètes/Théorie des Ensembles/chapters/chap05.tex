\chapter{Cardinals} % 基数
\section{Cardinality} % 基数
\begin{leftbarTitle}{Equinumerosity and Cardinality}\end{leftbarTitle} % 等势与基数
\begin{definition}{Equinumerosity and Cardinality} % 等势与基数
    Two sets \(A\) and \(B\) are said to be \textbf{equinumerous} (or have the same cardinality), 
    denoted by \(|A| = |B|\), if there exists a bijection \(f : A \to B\).
    
    The \textbf{cardinality} of a set \(A\) is the least ordinal \(\kappa\) such that \(|A| = |\kappa|\).
\end{definition}


\begin{definition}{Aleph Numbers} % 阿列夫数
    The \textbf{aleph numbers} are a sequence of cardinal numbers defined as follows:
    \begin{itemize}
        \item \(\aleph_0\) is the cardinality of the set of natural numbers \(\mathbb{N}\).
        \item For any ordinal \(\alpha\), \(\aleph_{\alpha + 1}\) is the least cardinal number greater than \(\aleph_\alpha\).
        \item For any limit ordinal \(\lambda\), \(\aleph_\lambda = \sup\{\aleph_\beta \mid \beta < \lambda\}\).
    \end{itemize}
\end{definition}

\begin{theorem}{Cantor-Bernstein-Schröder Theorem} % 康托-伯恩斯坦-施罗德定理
    If there exist injections \(f : A \to B\) and \(g : B \to A\), 
    then there exists a bijection \(h : A \to B\). In particular, \(|A| = |B|\).
\end{theorem}

\begin{leftbarTitle}{Countable and Uncountable Sets}\end{leftbarTitle} % 可数集与不可数集

\begin{theorem}
\end{theorem}

\begin{leftbarTitle}{Continuum Hypothesis}\end{leftbarTitle} % 连续统假设

\section{Cardinal Arithmetic} % 基数运算


\section{The Canonical Well-Ordering of \(\alpha\times \alpha\)} % 序数乘积的典范良序

\section{Cofinality} % 共终性