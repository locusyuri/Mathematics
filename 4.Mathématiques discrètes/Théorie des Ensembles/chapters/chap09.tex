\chapter{Borel and Analytic Sets} % Borel 集与分析集
\section{Borel Sets} % Borel 集
\begin{definition}{Algebra of Sets} % 集合代数
    Let \(S\) be a non-empty set. A non-empty collection \(\mathcal{S}\) of subsets of \(S\) 
    (i.e., \(\mathcal{S} \subseteq \mathscr{P}(S)\)) is called an 
    \textbf{algebra of sets} on \(S\) if it satisfies the following properties:
    \begin{enumerate}[label=(\roman*)]
        \item \(S\in \mathcal{S}\).
        \item If \(A \in \mathcal{S}\), then \(S \setminus A \in \mathcal{S}\)\footnote{
            % 结合第(i)条, 这意味着空集 \(\emptyset\) 也属于集合代数.
            Combining with item (i), this means that the empty set \(\emptyset\) also belongs to the algebra of sets.
            % \( \emptyset, S \in \mathcal{S} \) 这种说法也是常见的, 尤其是在拓扑学中.
            The notation \( \emptyset, S \in \mathcal{S} \) is also common, especially in topology.
        }.
        \item If \(A_1, A_2, \ldots, A_n \in \mathcal{S}\), then \(\bigcup_{i=1}^n A_i \in \mathcal{S}\).
    \end{enumerate}
    From these properties, it follows that an algebra of sets is also closed under finite intersections.
\end{definition}

\begin{definition}{\(\sigma\)-Algebra of Sets} % σ-代数
    Let \(S\) be a non-empty set. A non-empty collection \(\mathcal{S}\) of subsets of \(S\) 
    (i.e., \(\mathcal{S} \subseteq \mathscr{P}(S)\)) is called a 
    \(\sigma\)-\textbf{algebra of sets} on \(S\) if it satisfies the following properties:
    \begin{enumerate}[label=(\roman*)]
        \item \(S\in \mathcal{S}\).
        \item If \(A \in \mathcal{S}\), then \(S \setminus A \in \mathcal{S}\).
        \item If \(A_1, A_2, \ldots \in \mathcal{S}\), then \(\bigcup_{i=1}^{\infty} A_i \in \mathcal{S}\).
    \end{enumerate}
    From these properties, it follows that a \(\sigma\)-algebra of sets is also closed under countable intersections.
\end{definition}

\begin{remark}
    \(\sigma\)-algebra is the \(\sigma\)-completion of algebra of sets.
    It inherits all the properties of algebra.
    Additionally, it requires closure under countable union (and hence countable intersection),
    which is the core of handling limit processes in analysis (such as interchange of integration and limits).
\end{remark}

\begin{definition}{Borel Set} % Borel 集
    The \textbf{Borel \(\sigma\)-algebra} on Euclidean space \(\mathbb{R}^n\), denoted by \(\mathcal{B}(\mathbb{R}^n)\), 
    is the smallest \(\sigma\)-algebra containing all open sets in \(\mathbb{R}^n\).
    Sets in \(\mathcal{B}(\mathbb{R}^n)\) are called \textbf{Borel sets}.
    
    Similarly, for any topological space \((X, \tau)\),
    the \textbf{Borel \(\sigma\)-algebra} on \(X\), denoted by \(\mathcal{B}(X)\),
    is the smallest \(\sigma\)-algebra containing all open sets in \(X\).
\end{definition}