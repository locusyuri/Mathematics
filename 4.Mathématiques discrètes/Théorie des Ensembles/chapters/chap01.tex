\chapter{Naive Set Theory} % 朴素集合论
\section{Sets and Their Operations} % 集合及其运算


\begin{definition}{Power Set} % 幂集
    Let \(X\) be a set. 
    The \textbf{power set} of \(X\), denoted by \(\mathscr{P}(X)\), 
    is defined as the set of all subsets of \(X\):
    \[
        \mathscr{P}(X) = \{A \mid A \subseteq X\}.
    \]
    
\end{definition}

\section{Relations and Mappings} % 关系与映射
\begin{leftbarTitle}{Relations}\end{leftbarTitle}
\begin{definition}{Cartesian Product} % 笛卡尔积
    Let \(X\) and \(Y\) be two sets. 
    The \textbf{Cartesian product} (or direct product) of \(X\) and \(Y\), denoted by \(X \times Y\), 
    is defined as the set of all ordered pairs \((x, y)\) where \(x \in X\) and \(y \in Y\):
    \[
        X \times Y = \{(x, y) \mid x \in X, y \in Y\}.
    \]
    The Cartesian product can be extended to finitely many sets.
    \newline The Cartesian product of \(X\) and itself \(n\) times is denoted by \(X^n\):
\end{definition}

\begin{definition}{Relation} % 关系
    Let \(X\) and \(Y\) be two sets. 
    A \textbf{relation} \(R\) from \(X\) to \(Y\) is a subset of the Cartesian product \(X \times Y\):
    \[
        R \subseteq X \times Y.
    \]
    If \((x, y) \in R\), we say that \(x\) is related to \(y\) by the relation \(R\), denoted by \(xRy\).

    If \(A \subseteq X\), then the subset of \(Y\) defined by
    \[
        R(A) = \{y \in Y \mid \exists x \in A, (x, y) \in R\}
    \]
    is called the \textbf{image} of \(A\) under the relation \(R\).
    \(R(X)\) is called the \textbf{range} of the relation \(R\).
\end{definition}
There are several special types of relations:
\begin{description}
    \item[Empty relation] The empty set \(\emptyset\) is a relation from \(X\) to \(Y\).
    \item[Total relation] The Cartesian product \(X \times Y\) is a relation from \(X\) to \(Y\).
    \item[Identity relation] The relation \(I_X = \{(x, x) \mid x \in X\}\) is called the \textbf{identity relation} on \(X\).
\end{description}


When studying binary relations, we often focus on whether they have some special properties. 
For a binary relation \(R\) on a set \(X\), we define the following special properties:
\begin{description}
    \item[Reflexive] \((\forall  x \in X)\, xRx\).
    \item[Irreflexive] \((\forall x \in X)\, \neg xRx\).
    \item[Symmetric] \((\forall x, y \in X)\, (xRy \Leftrightarrow yRx)\).
    \item[Antisymmetric] \((\forall x, y \in X)\, (xRy \wedge yRx) \implies x = y\).
    \item[Transitive] \((\forall x, y, z \in X)\, (xRy \wedge yRz) \implies xRz\).
    \item[Connected (Total)] \((\forall x, y \in X)\, x\neq y \implies (xRy \vee yRx)\).
    \item[Well-founded] \((\exists x\in X \neq \emptyset)\,(\forall y \in X\setminus \{x\})\, \neg(yRx)\).
    \item[Transitive of incomparability] 
        \((\forall x, y, z \in X)\, (\neg(xRy \vee yRx) \wedge \neg(yRz \vee zRy)) \implies \neg(xRz \vee zRx)\).
\end{description}

Then we can define the equivalence relations based on these properties:
\begin{definition}{Equivalence Relation} % 等价关系
    A binary relation \(R\) on a set \(X\) is called an \textbf{equivalence relation} 
    if it is reflexive, symmetric, and transitive.
\end{definition}


\begin{leftbarTitle}{Mappings}\end{leftbarTitle}
\begin{definition}{Mapping (Function)} % 映射
    A \textbf{mapping} (or function) \(f\) from a set \(X\) to a set \(Y\) is a relation 
    such that for every \(x \in X\), there exists a unique \(y \in Y\) such that \((x, y) \in f\).
    We denote this by \(f: X \to Y\) and write \(f(x) = y\).

    The set \(X\) is called the \textbf{domain} of \(f\), and the set \(Y\) is called the \textbf{codomain} of \(f\).
    \newline The set \(f(X) = \{f(x) \mid x \in X\}\) is called the \textbf{image} of \(f\).
\end{definition}
There are several special types of mappings:
\begin{description}
    \item[Identity mapping] The mapping \(\mathrm{id}_X: X \to X\) defined by \(\mathrm{id}_X(x) = x\) for all \(x \in X\) 
        is called the \textbf{identity mapping} on \(X\).
    \item[Constant mapping] A mapping \(f: X \to Y\) is called a \textbf{constant mapping} 
        if there exists a fixed element \(y_0 \in Y\) such that \(f(x) = y_0\) for all \(x \in X\).
\end{description}
Mappings can be classified based on their behavior:
\begin{description}
    \item[Injective (One-to-One):] A mapping \(f: X \to Y\) is \textbf{injective} 
        if for every \(x_1, x_2 \in X\), \(f(x_1) = f(x_2)\) implies \(x_1 = x_2\).
    \item[Surjective (Onto):] A mapping \(f: X \to Y\) is \textbf{surjective} 
        if for every \(y \in Y\), there exists an \(x \in X\) such that \(f(x) = y\).
    \item[Bijective:] A mapping \(f: X \to Y\) is \textbf{bijective} 
        if it is both injective and surjective.
\end{description}

For \(A\subseteq X\), let
\[
\chi_A(x) =
\begin{cases} 
    1, & x\in A \\
    0, & x \notin A ,
\end{cases}
\]
be the \textbf{characteristic function} of set \(A\).


\begin{definition}{Inverse Mapping and Composition Mappings} % 逆映射和复合映射
    Let \(f: X \to Y\) be a bijective mapping. 
    The \textbf{inverse mapping} of \(f\), denoted by \(f^{-1}: Y \to X\), 
    is defined by \(f^{-1}(y) = x\) if and only if \(f(x) = y\).
    
    Let \(f: X \to Y\) and \(g: Y \to Z\) be two mappings. 
    The \textbf{composition mapping} of \(f\) and \(g\), denoted by \(g \circ f: X \to Z\), 
    is defined by \((g \circ f)(x) = g(f(x))\) for all \(x \in X\).
\end{definition}

\begin{definition}{Restriction and Extension} % 限制与延拓
    Let \(f: X \to Y\) be a mapping, and let \(A \subseteq X\).
    The \textbf{restriction} of \(f\) to \(A\), denoted by \(f|_A\), 
    is the mapping from \(A\) to \(Y\) defined by \(f|_A(x) = f(x)\) for all \(x \in A\).

    Conversely, if \(g: A \to Y\) is a mapping and \(A \subseteq X\),
    an \textbf{extension} of \(g\) to \(X\) is a mapping \(f: X \to Y\) such that \(f|_A = g\).
\end{definition}


