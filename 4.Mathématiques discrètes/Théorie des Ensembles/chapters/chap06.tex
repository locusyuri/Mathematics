\chapter{Real Numbers} % 实数
\section{Construction of Real Numbers and the Cardinality of the Continuum} % 实数构造与连续统基数

\section{Point Sets in Euclidean Space} % Euclid 空间中的点集
In this section, we explore the point sets in Euclidean space.
Furthermore, these concepts can be generalized to metric spaces and topological spaces.

\begin{definition}{Diameter and Bounded Set} % 直径与有界集
    Let \(A\) be a subset of the Euclidean space \(\mathbb{R}^n\). 
    The \textbf{diameter} of set \(A\) is defined as
    \[
        \mathrm{diam}(A) = \sup \{d(x, y) \mid x, y \in A\},
    \]
    where \(d(x, y)\) denotes the Euclidean distance between points \(x\) and \(y\).
    
    A set \(A\) is called \textbf{bounded} if there exists a real number \(M > 0\) such that
    \[
        d(x, y) < M, \quad \forall x, y \in A.
    \]
    
    Let \(x_{0}\in \mathbb{R}^{n}, \delta>0\), the set
    \[
        B\left(x_{0}, \delta\right)=\left\{x \in \mathbb{R}^{n} \mid d\left(x, x_{0}\right)<\delta\right\}
    \]
    is called the \textbf{open ball} (or \textbf{neighborhood}) with center \(x_{0}\) and radius \(\delta\)\footnote{
        It can be also denoted as \(N(x_0, \delta)\) or \(U(x_0, \delta)\).
        % 当delta不需要被强调时, 也可以简写为 \(B(x_0)\).
        When \(\delta\) does not need to be emphasized, it can also be abbreviated as \(B(x_0)\).
    }.
    Similarly, the closed ball can be defined as
    \[
        \bar{B}\left(x_{0}, \delta\right)=\left\{x \in \mathbb{R}^{n} \mid d\left(x, x_{0}\right) \leq \delta\right\}.
    \]

    Let \(a_{i},b_{i}\) (\(i=1,2,\ldots,n\)) be real numbers with \(a_{i} < b_{i}\), the set
    \[
        \prod_{i=1}^n [a_i, b_i] = \{(x_1, x_2, \ldots, x_n) \in 
        \mathbb{R}^n \mid a_i \leq x_i \leq b_i \text{ for all } i=1,2,\ldots,n \}
    \]
    is called a \textbf{rectangle} (or \textbf{box}) in \(\mathbb{R}^n\).
    If all the edge lengths are equal, i.e., \(b_i - a_i = c\) for some constant \(c > 0\) and for all \(i\), 
    then the rectangle is called a \textbf{cube} with side length \(c\).
    Similarly, we can define the open rectangle (or open box) as
    \[
        \prod_{i=1}^n (a_i, b_i) = \{(x_1, x_2, \ldots, x_n) \in 
        \mathbb{R}^n \mid a_i < x_i < b_i \text{ for all } i=1,2,\ldots,n \}.
    \]
    Rectangles are often denoted by \(I, J, \ldots\) and their volumes by \(|I|, |J|, \ldots\).
\end{definition}

\begin{definition}{Limit}
    Let \(\{x_k\}\) be a sequence in \(\mathbb{R}^n\) and \(x \in \mathbb{R}^n\). 
    We say that \(\{x_k\}\) \textbf{converges} to \(x\), or \(x\) is the \textbf{limit} of the sequence \(\{x_k\}\), 
    if for every \(\varepsilon > 0\), there exists a natural number \(N\) such that
    \[
        d(x_k, x) < \varepsilon, \quad \forall k > N.
    \]
    In this case, we write
    \[
        \lim_{k \to \infty} x_k = x.
    \]
\end{definition}

\begin{leftbarTitle}{Classification of Points}\end{leftbarTitle} % 点的分类
\begin{definition}{Classification of Points} % 点的分类
    Let \(E\) be a subset of the Euclidean space \(\mathbb{R}^n\).
    Points in \(\mathbb{R}^n\) can be classified based on their relationship to set \(E\):
    \begin{description}
        \item[Interior Point] A point \(x \in E\) is called an \textbf{interior point} of set \(E\) 
            if there exists \(U(x)\) such that \(U(x) \subseteq E\).
        \item[Exterior Point] A point \(x \in \mathbb{R}^n \setminus E\) is called an \textbf{exterior point} of set \(E\) 
            if there exists \(U(x)\) such that \(U(x) \subseteq \mathbb{R}^n \setminus E \), or equivalently, 
            \(U(x) \cap E = \emptyset\).
        \item[Boundary Point] A point \(x \in \mathbb{R}^n\) is called a \textbf{boundary point} of set \(E\) 
            if for every \(U(x)\), the set \(U(x)\) contains points in both \(E\) and \(\mathbb{R}^n \setminus E\).
        \item[Accumulation Point (Limit Point)] A point \(x \in \mathbb{R}^n\) is called an 
            \textbf{accumulation point} (or \textbf{limit point}) of set \(E\) if for every \(U(x)\), 
            the set \(U(x)\) contains at least one point of \(E\) different from \(x\)\footnote{
                Obviously, only infinite sets can have accumulation points.
                % 事实上, 在这里, 邻域中包含一点(相异)与包含无穷多点是等价的.
                In fact, here, containing at least one (distinct) point in the neighborhood is equivalent to 
                containing infinitely many points.
            }.
        \item[Isolated Point] A point \(x \in E\) is called an \textbf{isolated point} of set \(E\) 
            if \(x\) is not an accumulation point of \(E\), i.e., there exists \(U(x)\) such that
            \(U(x) \cap E = \{x\}\).
    \end{description}
\end{definition}

% 任何点 \(x \in \mathbb{R}^n\) 都可以被唯一地分类为下列三类之一:
Any point \(x \in \mathbb{R}^n\) can be uniquely classified into one of the following three categories:
\[
\begin{cases}
\text{Interior Point} & \text{if } \exists U(x) \subseteq E; \\
\text{Boundary Point} & \text{if } \forall U(x) \cap (\mathbb{R}^n \setminus E) \neq \emptyset; \\
\text{Exterior Point} & \text{if } \exists U(x) \subseteq \mathbb{R}^n \setminus E; \\
\end{cases}
\]
% 或是唯一地分类为下列三类之一:
Or it can be uniquely classified into one of the following three categories:
\[
\begin{cases}
\text{Accumulation Point} & \text{if } \forall U(x) \cap (E \setminus \{x\}) \neq \emptyset; \\
\text{Isolated Point} & \text{if } \exists U(x) \cap E = \{x\}; \\
\text{Exterior Point} & \text{if } \exists U(x) \cap E = \emptyset; \\
\end{cases}
\]

\begin{definition}
    Let \(E\) be a subset of the Euclidean space \(\mathbb{R}^n\).
    \begin{description}
        \item[Derived Set] The \textbf{derived set} of \(E\), denoted by \(E'\), 
            is the set of all accumulation points of \(E\).
        \item[Interior] The \textbf{interior} of set \(E\), denoted by \(\mathrm{int}(E)\), or \(\mathring{E}\), 
            is the set of all interior points of \(E\).
        \item[Boundary] The \textbf{boundary} of set \(E\), denoted by \(\partial E\), 
            is the set of all boundary points of \(E\), or equivalently, \(\partial E = \bar{E} \setminus \mathring{E}\).
        \item[Closure] The \textbf{closure} of set \(E\), denoted by \(\bar{E}\), 
            is the union of \(E\) and its accumulation points, i.e., \(\bar{E} = E \cup E'\).
    \end{description}
\end{definition}

\begin{property}
    \begin{itemize}
        \item \(\left( \mathring{E} \right)^{c}  = \overline{E^c},\quad \left( \overline{E} \right)^{c} = \mathring{E^c}\);
        \item Let \(A \subseteq B\), then \(A' \subseteq B'\), \(\mathring{A} \subseteq \mathring{B}\) 
            and \(\overline{A} \subseteq \overline{B}\);
        \item \((A \cup B)' = A' \cup B'\).
    \end{itemize}
\end{property}



\begin{note}
    % 度量空间中, 可以给出聚点的另一定义, 即: 点 \(x\) 是集合 \(E\) 的聚点, 当且仅当它是某一\(E\)中点列的极限.
    In a metric space, an alternative definition of accumulation point can be given:
    A point \(x\) is an accumulation point of set \(E\) if and only if it is the limit of some sequence of points in \(E\).
\end{note}

\begin{remark}
    % 通过把欧氏距离替换为一般的度量 \(d\),上述所有定义都可以自然地推广到一般的度量空间 \((X, d)\) 中。
    By replacing the Euclidean distance with a general metric \(d\), 
    all the above definitions can be naturally extended to a general metric space \((X, d)\).

    % 通过把度量 \(d\) 替换为一般的拓扑结构中的开集族,上述所有定义都可以推广到一般的拓扑空间 \((X, \tau)\) 中。
    By replacing the metric \(d\) with the family of open sets in a general topological structure,
    all the above definitions can be extended to a general topological space \((X, \tau)\).
\end{remark}

\begin{leftbarTitle}{Open and Closed Sets}\end{leftbarTitle} % 开集与闭集
\begin{definition}{Classification of Point Sets} % 点集的分类
    Let \(E\) be a subset of the Euclidean space \(\mathbb{R}^n\).
    Point sets can be classified:
    \begin{description}
        \item[Closed Set] A set \(E\) is called a \textbf{closed set} if it contains all its accumulation points.
        \item[Open Set] A set \(E\) is called an \textbf{open set} if every point in \(E\) is an interior point of \(E\).
        \item[Compact Set] A set \(E\) is called a \textbf{compact set} if every open cover of \(E\) has a finite subcover,
            or equivalently, if \(E\) is closed and bounded (Heine-Borel Theorem).
        \item[Perfect Set] A set \(E\) is called a \textbf{perfect set} if it is closed and has no isolated points, 
            i.e., every point in \(E\) is an accumulation point of \(E\), or equivalently, \(E=E'\).
    \end{description}
\end{definition}

\begin{note}
    % 度量空间中, 可以给出闭集的另一定义, 即: 集合 \(E\) 是闭集, 当且仅当它包含其所有点列极限.
    In a metric space, an alternative definition of closed set can be given:
    A set \(E\) is closed if and only if it contains all its sequential limits.
    % 这是由于度量空间满足第一可数公理, 序列收敛与拓扑闭包等价.
    (This is because metric spaces satisfy the first countability axiom,
    and sequential convergence is equivalent to topological closure.)
    % 实际上, 在度量空间中, 闭集与序列闭集是等价的.
    In fact, in a metric space, closed sets and sequentially closed sets are equivalent.
    
    % 而在拓扑空间中, 开集与闭集的定义需要依赖于拓扑结构, 闭集一定是序列闭集, 反之不真.
    However, in a topological space, the definitions of open and closed sets depend on the topological structure,
    and closed sets are always sequentially closed, but the converse is not true.
\end{note}

\begin{theorem}{Open Set Construction Theorem} % 开集的构造定理
    % R^1直线上任一个非空开集可以表示成至多可数个互不相交的开区间的并集.
    % 进一步, R^n 中的任一非空开集都可以表示成可数个开矩形的并集.
\end{theorem}

