\chapter{Recurrence Relations and Generating Functions} % 递归关系与生成函数
\section{Recurrence Relations} % 递归关系
\textbf{Recurrence relations} are equations that define sequences recursively,
expressing each term as a function of its preceding terms, with the form of:
\[
a_{n} = f(a_{n-1}, a_{n-2}, \ldots, a_{n-k}), \quad n \geq k,
\]
where \( k \) is the order of the recurrence relation, and \( f \) is a function that combines the previous \( k \) terms.

Recurrence relations can be classified into several types:
\begin{description}
    \item[Homogeneous vs. Non-homogeneous] A recurrence relation is homogeneous if it can be expressed solely in terms of 
        previous terms of the sequence.
        If it includes additional functions or constants, it is non-homogeneous.
    \item[Linear vs. Non-linear] A recurrence relation is linear if each term is a linear combination of previous terms:
        \[
        a_{n} = c_1 a_{n-1} + c_2 a_{n-2} + \cdots + c_k a_{n-k}.
        \]
        If it involves products or other non-linear operations, it is non-linear.
    \item[Constant Coefficients vs. Variable Coefficients] A recurrence relation has constant coefficients if 
        the coefficients in the linear combination are constants. If they vary with \( n \), it has variable coefficients.
\end{description}

\begin{leftbarTitle}{Methods for Solving Recurrence Relations}\end{leftbarTitle} % 递推关系的解法


\begin{leftbarTitle}{Common Recurrence Relations}\end{leftbarTitle} % 常见递推关系
\begin{example}
    Define the \textbf{Fibonacci sequence} \( \{F_n\} \) by the recurrence relation:
    \[
    F_{0} = 0, \quad F_{1} = 1, \quad F_{n} = F_{n-1} + F_{n-2}, \quad n \geq 2,
    \]
    which is a linear homogeneous recurrence relation with constant coefficients.
\end{example}

\begin{example}
    The \textbf{tower of Hanoi} problem (Figure~\ref{fig:Hanoi}) is a classic example of a problem 
    that can be solved using a recurrence relation.
    Move \(n\) plates from A to C, using B as an auxiliary, 
    with the condition that only one plate can be moved at a time and a larger plate cannot be placed on a smaller one.
    \begin{figure}[h]
        \centering
        \includegraphics[width=0.8\textwidth]{img/Hanoi.png}
        \caption{Tower of Hanoi problem}
        \label{fig:Hanoi}
    \end{figure}
    \newline It can be described by the recurrence relation:
    \[
    T(n) = 2T(n-1) + 1, \quad T(1) = 1,
    \]
    where \( T(n) \) represents the minimum number of moves required to transfer \( n \) disks from one peg to another.
\end{example}




\section{Generating Functions} % 生成函数
\begin{definition}
    The \textbf{ordinary generating function} (OGF) of a sequence \( \{a_n\} \) is defined as the formal power series:
    \[
    G(a_n; x) = \sum_{n=0}^{\infty} a_n x^n,
    \]
    where \( x \) is an indeterminate.

    The \textbf{exponential generating function} (EGF) of a sequence \( \{a_n\} \) is defined as:
    \[
    E(a_n; x) = \sum_{n=0}^{\infty} \frac{a_n}{n!} x^n.
    \]

    The \textbf{Dirichlet generating function} (DGF) of a sequence \( \{a_n\} \) is defined as:
    \[
    D(a_n; s) = \sum_{n=1}^{\infty} \frac{a_n}{n^s},
    \]
    where \( s \) is a complex variable.
\end{definition}

\begin{leftbarTitle}{Solving Recurrence Relations Using Generating Functions}\end{leftbarTitle} % 生成函数解递推关系

\begin{leftbarTitle}{Integer Partitions}\end{leftbarTitle} % 整数拆分