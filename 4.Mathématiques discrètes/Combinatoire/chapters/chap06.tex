\chapter{Special Counting Sequences} % 特殊计数序列
\section{Catalan Numbers} % Catalan 数
\begin{definition}{Catalan Numbers}
    The \( n \)-th Catalan number \( C_n \) is given by:
    \[
    C_n = \frac{1}{n+1} \binom{2n}{n} = \frac{(2n)!}{(n+1)!n!}=\binom{2n}{n} - \binom{2n}{n+1}.
    \]
    The first ten Catalan numbers are:
    \begin{gather*}
    C_0 = 1, \quad C_1 = 1, \quad C_2 = 2, \quad C_3 = 5, \quad C_4 = 14, \\
    C_5 = 42, \quad C_6 = 132, \quad C_7 = 429, \quad C_8 = 1430, \quad C_9 = 4862.
    \end{gather*}
\end{definition}
\begin{property}
    The Catalan numbers satisfy multiple recurrence relations:
    \begin{enumerate}
        \item  
            \[
            C_{n} = \sum_{i=0}^{n-1} C_i C_{n-1-i} \quad (n \geq 1), \quad C_0 = 1.
            \]
            This recurrence relation reflects the self-similarity of Catalan numbers.
        \item 
            \[
            C_{n} = \frac{2(2n-1)}{n+1} C_{n-1} \quad (n \geq 1), \quad C_0 = 1.
            \]
            This recurrence relation can be derived from the closed-form expression of Catalan numbers.
        \item Let \( G(x) = \sum_{n=0}^{\infty} C_n x^n \) be the generating function of Catalan numbers. 
            Then \( G(x) \) satisfies the functional equation:
            \[
            G(x) = 1 + x G(x)^2,
            \]
            id est,
            \[
            G(x) = \frac{1 - \sqrt{1-4x}}{2x}.
            \]
            This functional equation can be used to derive the closed-form expression of Catalan numbers using 
            the Lagrange inversion formula.
    \end{enumerate}
\end{property}

\vspace{0.7cm}
Catalan numbers is the answer to many combinatorial problems, 
\begin{description}
    \item[Ballot problem] There is an \( n \times n \) grid graph, 
        with the bottom-left corner at \((0, 0)\) and the top-right corner at \((n, n)\). 
        Starting from the bottom-left corner, and \emph{moving only right or up one unit at each step}, 
        the total number of paths to reach the top-right corner 
        without going above the diagonal \(y=x\) (but allowing touching it) is denoted as \(C_n\).
    \item[Dyck path counting problem] A Dyck path of semilength \(n\) is a lattice path from \((0, 0)\) to \((2n, 0)\) 
        that never dips below the \(x\)-axis and consists of steps \((1, 1)\) (up step) and \((1, -1)\) (down step). 
        The number of Dyck paths of semilength \(n\) is \(C_n\).
    \item[Counting non-intersecting chords in a circle] There are \(2n\) points on a circle. 
        The number of ways to pair these points with \(n\) chords 
        such that no two chords intersect is the Catalan number \(C_n\).
    \item[Triangulation counting problem] The number of ways to divide a convex \((n+2)\)-sided region 
        into triangular regions without intersecting diagonals is \(C_n\).
    \item[Binary Tree Counting Problem] The number of structurally different binary trees with \(n\)
        nodes is \(C_n\). 
        Equivalently, the number of structurally different full binary trees with \(n\) non-leaf nodes is \(C_n\).
    \item[Counting problem of parenthesis sequences] The number of valid parenthesis sequences 
        consisting of \(n\) pairs of parentheses is \(C_n\).
    \item[Stack popping sequence counting problem] The push sequence of a stack (of infinite size) is \(1, 2, \ldots, n\), 
        and the number of valid popping sequences is \(C_n\).
    \item[Sequence Counting Problem] The number of sequences \(a_1, a_2, \ldots, a_{2n}\) 
        consisting of \(n\) \(+1\)'s and \(n\) \(-1\)'s such that 
        the partial sums satisfy \(a_1 + a_2 + \ldots + a_k \geq 0\) (\(k = 1, 2, 3, \ldots, 2n\)) is \(C_n\).
\end{description}


\section{Stirling Numbers} % Stirling 数
