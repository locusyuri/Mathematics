\chapter{Basic Counting Principles} % 基本计数原理


\section{Addition and Multiplication Principles} % 加法与乘法原理

\section{Bijection Principle} % 一一对应原理


\section{Permutations and Combinations} % 排列与组合
\begin{definition}{Permutation and Combination}
    Let \( n \) be a non-negative integer, and \( k \) be an integer such that \( 0 \leq k \leq n \). 
    The number of ways to choose \( k \) elements from a set of \( n \) distinct elements 
    and arrange them in a specific order is called the number of permutations of \( n \) elements taken \( k \) at a time, 
    denoted as \( P(n, k) \) (or \(_nP_k\), \(P_n^k\) or \(A_n^k\)). 
    It is given by the formula:
    \[
    P(n, k) = \frac{n!}{(n-k)!}.
    \]
    The number of ways to choose \( k \) elements from a set of \( n \) distinct elements 
    without regard to the order of selection is called the number of combinations of \( n \) elements taken \( k \) at a time, 
    denoted as \( C(n, k) \) (or \(_nC_k\), \(C_n^k\) or \(\binom{n}{k}\)). 
    It is given by the formula:
    \[
    C(n, k) = \frac{n!}{k!(n-k)!} = \binom{n}{k}.
    \]
\end{definition}

\begin{property}
    The following properties hold for permutations and combinations:
    \begin{enumerate}
        \item \( A_{n}^{0} = 1 \) and \( A_{n}^{n} = n! \).
        \item \( C_{n}^{0} = 1 \) and \( C_{n}^{n} = 1 \).
        \item \( C_{n}^{k} = C_{n}^{n-k} \).
        \item \( A_{n}^{k} = k! \cdot C_{n}^{k} \).
        \item \(C_{n}^{k}=C_{n-1}^{k-1}+C_{n-1}^{k}\) (\textbf{Pascal's triangle/YangHui's triangle})\footnote{
            This property can also be understood that to choose \(k\) elements from \(n+1\), 
            you can first take one element \(A\): 
            \begin{enumerate}
                \item  The number of ways that include \(A\) is \(\mathrm{C}_{n}^{k-1}\); 
                \item  The number of ways that does not include \(A\) is \(\mathrm{C}_{n}^{k}\).
            \end{enumerate}
        }.
        % 在Pascal's triangle中,每个元素等于其左上方和右上方的两个元素之和,如图\ref{fig:pascal_triangle}所示。
        \newline In Pascal's triangle, each element is equal to the sum of the two elements directly above it, 
        as shown in Figure \ref{fig:pascal_triangle}.
        \begin{figure}[h]
            \centering
            \includegraphics[width=0.8\textwidth]{img/pascal_triangle.png}
            \caption{Pascal's triangle (YangHui's triangle).}
            \label{fig:pascal_triangle}
        \end{figure}
        \item \((a+b)^{n}=\sum_{k=0}^{n} C_{n}^{k} a^{k} b^{n-k}\) (\textbf{Binomial theorem}).
        % 因此我们可以得到Pascal's triangle和Binomial Theorem之间的关系,如图\ref{fig:pascal_and_binomial}所示。
        \newline Therefore, we can see the relationship between Pascal's triangle and the Binomial theorem, 
        as shown in Figure \ref{fig:pascal_and_binomial}.
        % 其中\(C_{n}^{k}\)是Pascal's triangle中第\(n\)行第\(k\)列的元素。
        Here, \(C_{n}^{k}\) is the element in the \(n\)-th row and \(k\)-th column of Pascal's triangle.
        \begin{figure}[h]
            \centering
            \includegraphics[width=0.8\textwidth]{img/pascal_and_binomial.png}
            \caption{Pascal's triangle and Binomial theorem.}
            \label{fig:pascal_and_binomial}
        \end{figure}
    \end{enumerate}
\end{property}
