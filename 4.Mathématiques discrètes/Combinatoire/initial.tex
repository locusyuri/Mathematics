\documentclass[11pt]{../../TexTemplate/elegantbook} % 这里是文档类,默认使用 elegantbook
\title{Combinatoire} % 组合数学
% \subtitle{Subtitle} % 这里放置副标题

\author{CatMono} % 这里放置作者名
\date{November, 2025} % 这里放置日期
\version{0.1} % 这里放置版本号
% \institute{Elegant\LaTeX{} Program} % 这里放置机构名
% \bioinfo{Custom Key}{Custom Value} % 这里放置自定义信息

% \extrainfo{extra information} % 这里放置额外信息,将显示在最下方中央

\setcounter{tocdepth}{2} % 设置目录深度
\setcounter{secnumdepth}{2} % 设置章节编号深度


% \logo{logo-blue.png} % 这里放置封面logo,默认从figure目录下寻找
% \cover{LogiqueMathematique.png} % 这里放置封面图片,默认从figure目录下寻找

% modify the color in the middle of titlepage
\definecolor{customcolor}{RGB}{32,178,170} % 自定义颜色
\colorlet{coverlinecolor}{customcolor}
\usepackage{cprotect} % 保护命令参数不被 LaTeX 解析器过早处理,允许在某些特殊环境中使用脆弱命令(fragile commands)。
\usepackage{xeCJK} % 使用 xeCJK 包支持中文
\usepackage{amsmath} % 使用 amsmath 包支持数学公式

% ===== 开始文档 =====
\begin{document}

\maketitle %生成文档的标题页,根据之前定义的标题信息(如标题、作者、日期等)自动创建一个格式化的标题页

% === 前言部分 ===
\frontmatter        % 开始前言,页码为 i, ii, iii...
\tableofcontents    % 目录 (页码: i, ii)
% \listoffigures      % 图表目录 (页码: iii)
% \listoftables       % 表格目录 (页码: iv)

\chapter{Preface}   % 前言章节(无编号,页码: v, vi...)
This is the preface of the book...

% \chapter{Acknowledgments}  % 致谢(无编号)
% I would like to thank...
% === 正文部分 ===
\mainmatter         % 开始正文,页码从 1 重新开始

\chapter{Permutations and Combinations} % 排列与组合

\chapter{Basic Counting Principles} % 基本计数原理
\section{Pigeonhole Principle} % 鸽巢原理

\chapter{Binomial Coefficients} % 二项式系数

\chapter{Inclusion-Exclusion Principle} % 容斥原理
\chapter{Recurrence Relations and Generating Functions} % 生成函数

\chapter{Special Counting Sequences} % 特殊计数序列
\section{Catalan Numbers} % Catalan 数
\begin{definition}{Catalan Numbers}
    The \( n \)-th Catalan number \( C_n \) is given by:
    \[
    C_n = \frac{1}{n+1} \binom{2n}{n} = \frac{(2n)!}{(n+1)!n!}.
    \]
    The first ten Catalan numbers are:
    \begin{gather*}
    C_0 = 1, \quad C_1 = 1, \quad C_2 = 2, \quad C_3 = 5, \quad C_4 = 14, \\
    C_5 = 42, \quad C_6 = 132, \quad C_7 = 429, \quad C_8 = 1430, \quad C_9 = 4862.
    \end{gather*}
\end{definition}

Catalan numbers is the answer to many combinatorial problems, 
\begin{description}
    \item[Path counting problem] There is an \( n \times n \) grid graph, 
        with the bottom-left corner at \((0, 0)\) and the top-right corner at \((n, n)\). 
        Starting from the bottom-left corner, and moving only right or up one unit at each step, 
        the total number of paths to reach the top-right corner 
        without going above the diagonal \(y=x\) (but allowing touching it) is denoted as \(C_n\).
    \item[Counting non-intersecting chords in a circle] There are \(2n\) points on a circle. 
        The number of ways to pair these points with \(n\) chords 
        such that no two chords intersect is the Catalan number \(C_n\).
    \item[Triangulation counting problem] The number of ways to divide a convex \((n+2)\)-sided region 
        into triangular regions without intersecting diagonals is \(C_n\).
    \item[Binary Tree Counting Problem] The number of structurally different binary trees with \(n\)
        nodes is \(C_n\). 
        Equivalently, the number of structurally different full binary trees with \(n\) non-leaf nodes is \(C_n\).
    \item[Counting problem of parenthesis sequences] The number of valid parenthesis sequences 
        consisting of \(n\) pairs of parentheses is \(C_n\).
    \item[Stack popping sequence counting problem] The push sequence of a stack (of infinite size) is \(1, 2, \ldots, n\), 
        and the number of valid popping sequences is \(C_n\).
    \item[Sequence Counting Problem] The number of sequences \(a_1, a_2, \ldots, a_{2n}\) 
        consisting of \(n\) \(+1\)'s and \(n\) \(-1\)'s such that 
        the partial sums satisfy \(a_1 + a_2 + \ldots + a_k \geq 0\) (\(k = 1, 2, 3, \ldots, 2n\)) is \(C_n\).
\end{description}


\section{Stirling Numbers} % Stirling 数


\chapter{Extremal Principle} % 极值原理


\chapter{Ramsey Theory} % Ramsey 理论

\chapter{Design Theory} % 组合设计

\chapter{Pólya Counting} % Pólya 计数




\begin{thebibliography}{99} 
\bibitem{en1} Elias M. Stein, Rami Shakarchi. \emph{Fourier Analysis: An Introduction}. Princeton University Press, 2016.
\bibitem{en2} Author2, Title2, Journal2, Year2. \emph{ This is another example of a reference.}
\end{thebibliography}

\end{document}