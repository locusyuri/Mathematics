\documentclass[11pt]{../../TexTemplate/elegantbook} % 这里是文档类,默认使用 elegantbook
\title{Combinatoire} % 组合数学
% \subtitle{Subtitle} % 这里放置副标题

\author{CatMono} % 这里放置作者名
\date{November, 2025} % 这里放置日期
\version{0.1} % 这里放置版本号
% \institute{Elegant\LaTeX{} Program} % 这里放置机构名
% \bioinfo{Custom Key}{Custom Value} % 这里放置自定义信息

% \extrainfo{extra information} % 这里放置额外信息,将显示在最下方中央

\setcounter{tocdepth}{2} % 设置目录深度
\setcounter{secnumdepth}{2} % 设置章节编号深度


% \logo{logo-blue.png} % 这里放置封面logo,默认从figure目录下寻找
% \cover{LogiqueMathematique.png} % 这里放置封面图片,默认从figure目录下寻找

% modify the color in the middle of titlepage
\definecolor{customcolor}{RGB}{32,178,170} % 自定义颜色
\colorlet{coverlinecolor}{customcolor}
\usepackage{cprotect} % 保护命令参数不被 LaTeX 解析器过早处理,允许在某些特殊环境中使用脆弱命令(fragile commands)。
\usepackage{xeCJK} % 使用 xeCJK 包支持中文
\usepackage{amsmath} % 使用 amsmath 包支持数学公式

% ===== 开始文档 =====
\begin{document}

\maketitle %生成文档的标题页,根据之前定义的标题信息(如标题、作者、日期等)自动创建一个格式化的标题页

% === 前言部分 ===
\frontmatter        % 开始前言,页码为 i, ii, iii...
\tableofcontents    % 目录 (页码: i, ii)
% \listoffigures      % 图表目录 (页码: iii)
% \listoftables       % 表格目录 (页码: iv)

\chapter{Preface}   % 前言章节(无编号,页码: v, vi...)
This is the preface of the book...

% \chapter{Acknowledgments}  % 致谢(无编号)
% I would like to thank...
% === 正文部分 ===
\mainmatter         % 开始正文,页码从 1 重新开始


% --- Part 1: 基础计数工具 (The Foundation) ---
\chapter{Integers}  % 整数
\section{Divisibility and Prime Numbers} % 整除性与素数
Let\footnote{
    Sometimes, natural numbers refer to the set of positive integers excluding zero, i.e., \(\mathbb{N}^{+}=\{1,2,3,\ldots\}\).
} 
\[
\mathbb{N}=\{0,1,2,3,\ldots\},\quad \mathbb{N}^{+}=\{1,2,3,\ldots\},\quad \mathbb{Z}=\{\ldots,-2,-1,0,1,2,\ldots\}.
\]
% 对于一个实数, 引入高斯符号

\begin{definition}{Gauß Symbols}
    For a real number \(x\), the floor function (greatest integer function) is defined as:
    \[
    \lfloor x \rfloor = \max\{n \in \mathbb{Z} \mid n \leq x\}.
    \]
    Similarly, the ceiling function (least integer function) is defined as:
    \[
    \lceil x \rceil = \min\{n \in \mathbb{Z} \mid n \geq x\}.
    \]
\end{definition}
\begin{property}
    \begin{enumerate}
        \item For any \(m \in \mathbb{N}^{+}\), there is \textbf{Hermite's identity}:
            \[
            \left\lfloor mx \right\rfloor = \left\lfloor x \right\rfloor + \left\lfloor x+ \frac{1}{m} \right\rfloor 
            + \cdots + \left\lfloor x+ \frac{m-1}{m} \right\rfloor.
            \]
            \[
            \left\lceil mx \right\rceil = \left\lceil x \right\rceil + \left\lceil x - \frac{1}{m} \right\rceil 
            + \cdots + \left\lceil x - \frac{m-1}{m} \right\rceil.
            \]
        \item 
    \end{enumerate}
\end{property}


\begin{theorem}{Euclidean Algorithm}
    For any integers \(a\) and \(b\) with \(b > 0\), there exist unique integers \(q\) and \(r\) such that
    \[
    a = bq + r, \quad 0 \leq r < b.
    \]
    \(r\) is called the remainder of \(a\) divided by \(b\), denoted as \(r = a \bmod b\).

    If \(r = 0\), then \(b\) divides \(a\), denoted as \(b \mid a\); 
    otherwise, \(b\) does not divide \(a\), denoted as \(b \nmid a\).
    In other words, \(b \mid a\) if and only if there exists an integer \(k\) such that \(a = bk\).

    If \(a=kb\) and \(b\neq a, b\neq 1\), then \(b\) is called a proper divisor of \(a\).
\end{theorem}

\begin{property}
    If \(b\neq 0, c\neq 0\), then
    \begin{enumerate}
        \item If \(b\mid a, c\mid b\), then \(c\mid a\).
        \item If \(b\mid a\), then \(bc \mid ac\).
        \item If \(c \mid d, c\mid e\), then \(c \mid (md + ne)\), for any integers \(m, n\).
    \end{enumerate}
\end{property}

\section{Carry System} % 进位制
Carry system (or positional numeral system) is a method of representing numbers using a radix (or base) \(r\) (\(r \geq 2\)).
In base \(r\), any non-negative integer \(N\) can be expressed as:
\[
N = a_k r^k + a_{k-1} r^{k-1} + \cdots + a_1 r + a_0 = \sum_{i=0}^{k} a_i r^i =: (a_k a_{k-1} \cdots a_1 a_0)_r,
\]
where \(a_i\) are the digits satisfying \(0 \leq a_i < r\) and \(a_k \neq 0\).

% 扩展到小数
This can be extended to decimal fractions as:
\[
N = a_k r^k + a_{k-1} r^{k-1} + \cdots + a_1 r + a_0 + a_{-1} r^{-1} + a_{-2} r^{-2} + \cdots = \sum_{i=-m}^{k} a_i r^i =: 
(a_k a_{k-1} \cdots a_1 a_0 . a_{-1} a_{-2} \cdots a_{-m})_r,
\]
where \(m\) is a positive integer.

\begin{leftbarTitle}{Radix Conversion}\end{leftbarTitle} % 进制转换
% 十进制与 r 进制的转换
Here are \emph{methods for converting between decimal and base \(r\)}:
\begin{itemize}
    \item Decimal to base \(r\):
        \begin{enumerate}
            \item For the integer part, repeatedly divide by \(r\) and record the remainders.
            \item For the fractional part, repeatedly multiply by \(r\) and record the integer parts.
            \item Combine the results to form the base \(r\) representation.
        \end{enumerate}
    \item Base \(r\) to decimal:
        \begin{enumerate}
            \item For the integer part, multiply each digit by \(r\) raised to its position power and sum them.
            \item For the fractional part, multiply each digit by \(r\) raised to its negative position power and sum them.
            \item Combine both sums to get the decimal representation.
        \end{enumerate}
\end{itemize}

\begin{example}
    \begin{enumerate}
        \item Convert decimal \(45.625\) to binary.
        \item Convert binary \((1101.101)_2\) to decimal.
    \end{enumerate}
\end{example}

\begin{solution}
\begin{enumerate}
    \item For integer part \(45\):
        \begin{align*}
            &45 \div 2 = 22 \text{ remainder } 1 \\
            &22 \div 2 = 11 \text{ remainder } 0 \\
            &11 \div 2 = 5 \text{ remainder } 1 \\
            &5 \div 2 = 2 \text{ remainder } 1 \\
            &2 \div 2 = 1 \text{ remainder } 0 \\
            &1 \div 2 = 0 \text{ remainder } 1 \\
        \end{align*}
        Reading remainders from bottom to top gives \(101101\).
        \newline For fractional part \(0.625\):
        \begin{align*}
            &0.625 \times 2 = 1.25 \quad (\text{integer part } 1) \\
            &0.25 \times 2 = 0.5 \quad (\text{integer part } 0) \\
            &0.5 \times 2 = 1.0 \quad (\text{integer part } 1)
        \end{align*}
        Reading integer parts gives \(101\).
        \newline Combining both parts, we get \(45.625_{10} = (101101.101)_2\).
    \item Since  
        \(
        1 \times 2^3 + 1 \times 2^2 + 0 \times 2^1 + 1 \times 2^0 + 1 \times 2^{-1} + 0 \times 2^{-2} + 1 \times 2^{-3} 
        = 8 + 4 + 0 + 1 + 0.5 + 0 + 0.125 = 13.625;
        \)
        thus, \((1101.101)_2 = 13.625_{10}\).
\end{enumerate}
\end{solution}

\begin{leftbarTitle}{Generalized Carry System}\end{leftbarTitle} % 广义进制系统

\begin{leftbarTitle}{Balanced Ternary}\end{leftbarTitle} % 平衡三进制
\textbf{Balanced ternary} (symmetric ternary) is a non-standard positional numeral system 
that uses three digits: \(-1\), \(0\), and \(1\).
Since \(-1\) is not a standard digit, it is often represented by the symbol \(Z\) or \(\bar{1}\).
The weight calculation is the same as standard ternary, with the weight of the \(i\)-th digit being \(3^{i}\).

\begin{theorem}{Uniqueness of Balanced Ternary Representation}
    Every integer can be uniquely represented in balanced ternary.
\end{theorem}

\begin{proposition}
    % 对于负数, 只需将对应整数的每一位取反即可
    For negative numbers, simply negate each digit of the corresponding positive integer's balanced ternary representation.
\end{proposition}

\vspace{0.7cm}
Here are \emph{methods for converting between decimal and balanced ternary}:
\begin{itemize}
    \item Decimal to balanced ternary:
        \begin{enumerate}
            \item Repeatedly divide the number by \(3\), recording the remainders.
            \item If a remainder is \(2\), replace it with \(-1\) (or \(Z\)) and increment the quotient by \(1\).
            \item Continue until the quotient is \(0\).
            \item Read the remainders from bottom to top to form the balanced ternary representation.
        \end{enumerate}
    \item Balanced ternary to decimal:
        \begin{enumerate}
            \item Multiply each digit by \(3\) raised to its position power and sum them.
            \item For digits equal to \(-1\) (or \(Z\)), treat them as \(-1\) in the calculation.
        \end{enumerate}
\end{itemize}


\begin{example}
    \begin{enumerate}
        \item Convert decimal \(64\) to balanced ternary. 
        \item Convert balanced ternary \(1Z0Z1\) to decimal.
    \end{enumerate}
\end{example}

\begin{solution}
\begin{enumerate}
    \item For integer part \(64\):
        \begin{align*}
            &64 \div 3 = 21 \text{ remainder } 1 \\
            &21 \div 3 = 7 \text{ remainder } 0 \\
            &7 \div 3 = 2 \text{ remainder } 1 \\
            &2 \div 3 = 0 \text{ remainder } 2 \quad (\text{replace } 2 \text{ with } Z, \text{ increment quotient to } 1) \\
            &1 \div 3 = 0 \text{ remainder } 1
        \end{align*}
        Reading remainders from bottom to top gives \(1Z0Z1\).
        Thus, \(64_{10} = (1Z0Z1)_{3b}\).
    \item Since  
        \(
        1 \times 3^4 + (-1) \times 3^3 + 0 \times 3^2 + (-1) \times 3^1 + 1 \times 3^0 
        = 81 - 27 + 0 - 3 + 1 = 52;
        \)
        thus, \((1Z0Z1)_{3b} = 52_{10}\).
\end{enumerate}
\end{solution}

\section{Greatest Common Divisor and Least Common Multiple} % 最大公约数与最小公倍数

\section{Fundamental Theorem of Arithmetic} % 算数基本定理 % 基本计数原理
% \chapter{Permutations and Combinations} % 排列与组合 % 排列与组合      已合并入第一章
\chapter{Binomial Coefficients} % 二项式系数 % 二项式系数

% --- Part 2: 进阶计数方法 (Advanced Counting) ---
\chapter{Recurrence Relations and Generating Functions} % 递归关系与生成函数
 % 递推关系与生成函数
\chapter{Fourier Transform} % 傅里叶变换

\section{Laplace Transform} % 拉普拉斯变换 % 容斥原理
\chapter{Real Numbers} % 实数
\section{Construction of Real Numbers and the Cardinality of the Continuum} % 实数构造与连续统基数

\section{Point Sets in Euclidean Space} % Euclid 空间中的点集
In this section, we explore the point sets in Euclidean space.
Furthermore, these concepts can be generalized to metric spaces and topological spaces.

\begin{definition}{Diameter and Bounded Set} % 直径与有界集
    Let \(A\) be a subset of the Euclidean space \(\mathbb{R}^n\). 
    The \textbf{diameter} of set \(A\) is defined as
    \[
        \mathrm{diam}(A) = \sup \{d(x, y) \mid x, y \in A\},
    \]
    where \(d(x, y)\) denotes the Euclidean distance between points \(x\) and \(y\).
    
    A set \(A\) is called \textbf{bounded} if there exists a real number \(M > 0\) such that
    \[
        d(x, y) < M, \quad \forall x, y \in A.
    \]
    
    Let \(x_{0}\in \mathbb{R}^{n}, \delta>0\), the set
    \[
        B\left(x_{0}, \delta\right)=\left\{x \in \mathbb{R}^{n} \mid d\left(x, x_{0}\right)<\delta\right\}
    \]
    is called the \textbf{open ball} (or \textbf{neighborhood}) with center \(x_{0}\) and radius \(\delta\)\footnote{
        It can be also denoted as \(N(x_0, \delta)\) or \(U(x_0, \delta)\).
        % 当delta不需要被强调时, 也可以简写为 \(B(x_0)\).
        When \(\delta\) does not need to be emphasized, it can also be abbreviated as \(B(x_0)\).
    }.
    Similarly, the closed ball can be defined as
    \[
        \bar{B}\left(x_{0}, \delta\right)=\left\{x \in \mathbb{R}^{n} \mid d\left(x, x_{0}\right) \leq \delta\right\}.
    \]

    Let \(a_{i},b_{i}\) (\(i=1,2,\ldots,n\)) be real numbers with \(a_{i} < b_{i}\), the set
    \[
        \prod_{i=1}^n [a_i, b_i] = \{(x_1, x_2, \ldots, x_n) \in 
        \mathbb{R}^n \mid a_i \leq x_i \leq b_i \text{ for all } i=1,2,\ldots,n \}
    \]
    is called a \textbf{rectangle} (or \textbf{box}) in \(\mathbb{R}^n\).
    If all the edge lengths are equal, i.e., \(b_i - a_i = c\) for some constant \(c > 0\) and for all \(i\), 
    then the rectangle is called a \textbf{cube} with side length \(c\).
    Similarly, we can define the open rectangle (or open box) as
    \[
        \prod_{i=1}^n (a_i, b_i) = \{(x_1, x_2, \ldots, x_n) \in 
        \mathbb{R}^n \mid a_i < x_i < b_i \text{ for all } i=1,2,\ldots,n \}.
    \]
    Rectangles are often denoted by \(I, J, \ldots\) and their volumes by \(|I|, |J|, \ldots\).
\end{definition}

\begin{definition}{Limit}
    Let \(\{x_k\}\) be a sequence in \(\mathbb{R}^n\) and \(x \in \mathbb{R}^n\). 
    We say that \(\{x_k\}\) \textbf{converges} to \(x\), or \(x\) is the \textbf{limit} of the sequence \(\{x_k\}\), 
    if for every \(\varepsilon > 0\), there exists a natural number \(N\) such that
    \[
        d(x_k, x) < \varepsilon, \quad \forall k > N.
    \]
    In this case, we write
    \[
        \lim_{k \to \infty} x_k = x.
    \]
\end{definition}

\begin{leftbarTitle}{Classification of Points}\end{leftbarTitle} % 点的分类
\begin{definition}{Classification of Points} % 点的分类
    Let \(E\) be a subset of the Euclidean space \(\mathbb{R}^n\).
    Points in \(\mathbb{R}^n\) can be classified based on their relationship to set \(E\):
    \begin{description}
        \item[Interior Point] A point \(x \in E\) is called an \textbf{interior point} of set \(E\) 
            if there exists \(U(x)\) such that \(U(x) \subseteq E\).
        \item[Exterior Point] A point \(x \in \mathbb{R}^n \setminus E\) is called an \textbf{exterior point} of set \(E\) 
            if there exists \(U(x)\) such that \(U(x) \subseteq \mathbb{R}^n \setminus E \), or equivalently, 
            \(U(x) \cap E = \emptyset\).
        \item[Boundary Point] A point \(x \in \mathbb{R}^n\) is called a \textbf{boundary point} of set \(E\) 
            if for every \(U(x)\), the set \(U(x)\) contains points in both \(E\) and \(\mathbb{R}^n \setminus E\).
        \item[Accumulation Point (Limit Point)] A point \(x \in \mathbb{R}^n\) is called an 
            \textbf{accumulation point} (or \textbf{limit point}) of set \(E\) if for every \(U(x)\), 
            the set \(U(x)\) contains at least one point of \(E\) different from \(x\)\footnote{
                Obviously, only infinite sets can have accumulation points.
                % 事实上, 在这里, 邻域中包含一点(相异)与包含无穷多点是等价的.
                In fact, here, containing at least one (distinct) point in the neighborhood is equivalent to 
                containing infinitely many points.
            }.
        \item[Isolated Point] A point \(x \in E\) is called an \textbf{isolated point} of set \(E\) 
            if \(x\) is not an accumulation point of \(E\), i.e., there exists \(U(x)\) such that
            \(U(x) \cap E = \{x\}\).
    \end{description}
\end{definition}

% 任何点 \(x \in \mathbb{R}^n\) 都可以被唯一地分类为下列三类之一:
Any point \(x \in \mathbb{R}^n\) can be uniquely classified into one of the following three categories:
\[
\begin{cases}
\text{Interior Point} & \text{if } \exists U(x) \subseteq E; \\
\text{Boundary Point} & \text{if } \forall U(x) \cap (\mathbb{R}^n \setminus E) \neq \emptyset; \\
\text{Exterior Point} & \text{if } \exists U(x) \subseteq \mathbb{R}^n \setminus E; \\
\end{cases}
\]
% 或是唯一地分类为下列三类之一:
Or it can be uniquely classified into one of the following three categories:
\[
\begin{cases}
\text{Accumulation Point} & \text{if } \forall U(x) \cap (E \setminus \{x\}) \neq \emptyset; \\
\text{Isolated Point} & \text{if } \exists U(x) \cap E = \{x\}; \\
\text{Exterior Point} & \text{if } \exists U(x) \cap E = \emptyset; \\
\end{cases}
\]

\begin{definition}
    Let \(E\) be a subset of the Euclidean space \(\mathbb{R}^n\).
    \begin{description}
        \item[Derived Set] The \textbf{derived set} of \(E\), denoted by \(E'\), 
            is the set of all accumulation points of \(E\).
        \item[Interior] The \textbf{interior} of set \(E\), denoted by \(\mathrm{int}(E)\), or \(\mathring{E}\), 
            is the set of all interior points of \(E\).
        \item[Boundary] The \textbf{boundary} of set \(E\), denoted by \(\partial E\), 
            is the set of all boundary points of \(E\), or equivalently, \(\partial E = \bar{E} \setminus \mathring{E}\).
        \item[Closure] The \textbf{closure} of set \(E\), denoted by \(\bar{E}\), 
            is the union of \(E\) and its accumulation points, i.e., \(\bar{E} = E \cup E'\).
    \end{description}
\end{definition}

\begin{property}
    \begin{itemize}
        \item \(\left( \mathring{E} \right)^{c}  = \overline{E^c},\quad \left( \overline{E} \right)^{c} = \mathring{E^c}\);
        \item Let \(A \subseteq B\), then \(A' \subseteq B'\), \(\mathring{A} \subseteq \mathring{B}\) 
            and \(\overline{A} \subseteq \overline{B}\);
        \item \((A \cup B)' = A' \cup B'\).
    \end{itemize}
\end{property}



\begin{note}
    % 度量空间中, 可以给出聚点的另一定义, 即: 点 \(x\) 是集合 \(E\) 的聚点, 当且仅当它是某一\(E\)中点列的极限.
    In a metric space, an alternative definition of accumulation point can be given:
    A point \(x\) is an accumulation point of set \(E\) if and only if it is the limit of some sequence of points in \(E\).
\end{note}

\begin{remark}
    % 通过把欧氏距离替换为一般的度量 \(d\),上述所有定义都可以自然地推广到一般的度量空间 \((X, d)\) 中。
    By replacing the Euclidean distance with a general metric \(d\), 
    all the above definitions can be naturally extended to a general metric space \((X, d)\).

    % 通过把度量 \(d\) 替换为一般的拓扑结构中的开集族,上述所有定义都可以推广到一般的拓扑空间 \((X, \tau)\) 中。
    By replacing the metric \(d\) with the family of open sets in a general topological structure,
    all the above definitions can be extended to a general topological space \((X, \tau)\).
\end{remark}

\begin{leftbarTitle}{Open and Closed Sets}\end{leftbarTitle} % 开集与闭集
\begin{definition}{Classification of Point Sets} % 点集的分类
    Let \(E\) be a subset of the Euclidean space \(\mathbb{R}^n\).
    Point sets can be classified:
    \begin{description}
        \item[Closed Set] A set \(E\) is called a \textbf{closed set} if it contains all its accumulation points.
        \item[Open Set] A set \(E\) is called an \textbf{open set} if every point in \(E\) is an interior point of \(E\).
        \item[Compact Set] A set \(E\) is called a \textbf{compact set} if every open cover of \(E\) has a finite subcover,
            or equivalently, if \(E\) is closed and bounded (Heine-Borel Theorem).
        \item[Perfect Set] A set \(E\) is called a \textbf{perfect set} if it is closed and has no isolated points, 
            i.e., every point in \(E\) is an accumulation point of \(E\), or equivalently, \(E=E'\).
    \end{description}
\end{definition}

\begin{note}
    % 度量空间中, 可以给出闭集的另一定义, 即: 集合 \(E\) 是闭集, 当且仅当它包含其所有点列极限.
    In a metric space, an alternative definition of closed set can be given:
    A set \(E\) is closed if and only if it contains all its sequential limits.
    % 这是由于度量空间满足第一可数公理, 序列收敛与拓扑闭包等价.
    (This is because metric spaces satisfy the first countability axiom,
    and sequential convergence is equivalent to topological closure.)
    % 实际上, 在度量空间中, 闭集与序列闭集是等价的.
    In fact, in a metric space, closed sets and sequentially closed sets are equivalent.
    
    % 而在拓扑空间中, 开集与闭集的定义需要依赖于拓扑结构, 闭集一定是序列闭集, 反之不真.
    However, in a topological space, the definitions of open and closed sets depend on the topological structure,
    and closed sets are always sequentially closed, but the converse is not true.
\end{note}

\begin{theorem}{Open Set Construction Theorem} % 开集的构造定理
    % R^1直线上任一个非空开集可以表示成至多可数个互不相交的开区间的并集.
    % 进一步, R^n 中的任一非空开集都可以表示成可数个开矩形的并集.
\end{theorem}

 % 特殊计数序列

% --- Part 3: 存在性与极值 (Existence & Structure) ---
\chapter{Diophantine Equations} % 不定方程 % 鸽巢原理
\chapter{Numerical Series} % 数项级数
\section{Convergence of Numerical Series}

\section{Positive Term Series and Its Convergence Tests}
\begin{definition}{Positive Term Series}
    If all terms of the series \( \sum_{n=1}^{\infty} x_n \) are non-negative real numbers, 
    i.e., \( x_n \geqslant 0 \) (\( x_n > 0 \)), \( n = 1, 2, \dots \), 
    then this series is called a \textbf{positive term series} (or strictly positive term series).
\end{definition}

\begin{note}
    The positive term series converges if and only if the partial sums of the sequence are bounded. 
    If the partial sums are unbounded, the series must diverge to \( +\infty \).
\end{note}

\begin{leftbarTitle}{Comparison Test}\end{leftbarTitle}
\begin{theorem}{Comparison Test}
    Let \( \sum_{n=1}^{\infty} a_n \) and \( \sum_{n=1}^{\infty} b_n \) be positive term series. 
    If \( \exists N \in \mathbb{N}, \text{ s.t. } \forall n > N: a_n \leqslant b_n \), then:
    \begin{enumerate}
        \item If \( \sum_{n=1}^{\infty} b_n \) converges, then \( \sum_{n=1}^{\infty} a_n \) also converges.
        \item If \( \sum_{n=1}^{\infty} a_n \) diverges, then \( \sum_{n=1}^{\infty} b_n \) also diverges.
    \end{enumerate}

    \textbf{Limit Form}
    Let \( \sum_{n=1}^{\infty} a_n \) and \( \sum_{n=1}^{\infty} b_n \) be positive term series, 
    and suppose \( \lim_{n \to \infty} \frac{a_n}{b_n} \) exists. Then:

    \begin{enumerate}
        \item If \( 0 < l < +\infty \), \( \sum_{n=1}^{\infty} a_n \) and \( \sum_{n=1}^{\infty} b_n \) 
            have the same convergence or divergence behavior.
        \item If \( l = 0 \), \( \sum_{n=1}^{\infty} b_n \) converges, 
            then \( \sum_{n=1}^{\infty} a_n \) also converges.
        \item If \( l = +\infty \), \( \sum_{n=1}^{\infty} b_n \) diverges, 
            then \( \sum_{n=1}^{\infty} a_n \) also diverges.
    \end{enumerate}
\end{theorem}

\begin{theorem}
\begin{description}
    \item[Cauchy Test] Let \( \sum_{n=1}^{\infty} a_n \) be a positive term series.
        \begin{enumerate}
            \item If \( \exists q \in [0,1), \text{ s.t. } \sqrt[n]{a_n} \leqslant 
            q < 1 \quad (n \geqslant N, N \in \mathbb{N}) \), then the series converges.
            \item If \( \sqrt[n]{a_n} \geqslant 1 \) for infinitely many \( n \), then the series diverges.
        \end{enumerate}
        \textbf{Limit Form} Let \( \sum_{n=1}^{\infty} a_n \) be a positive term series, 
        and suppose \( \varlimsup_{n \to +\infty} \sqrt[n]{a_n} = r \). Then:
        \begin{enumerate}
            \item If \( 0 \leqslant r < 1 \), the series \( \sum_{n=1}^{\infty} a_n \) converges.
            \item If \( r > 1 \), the series \( \sum_{n=1}^{\infty} a_n \) diverges.
            \item If \( r = 1 \), the test fails.
        \end{enumerate}

    \item[D'Alembert Test] Let \( \sum_{n=1}^{\infty} a_n \) be a strictly positive term series.
        \begin{enumerate}
            \item If \( \exists q \in [0,1), \text{ s.t. } \frac{a_{n+1}}{a_n} \leqslant 
            q < 1 \quad (n \geqslant N, N \in \mathbb{N}) \), then the series converges.
            \item If \( \frac{a_{n+1}}{a_n} \geqslant 1 \quad (n \geqslant N, N \in \mathbb{N}) \), 
            then the series diverges.
        \end{enumerate}
        \textbf{Limit Form}
        Let \( \sum_{n=1}^{\infty} a_n \) be a strictly positive term series. Then:
        \begin{enumerate}
            \item If \( \varlimsup_{n \to +\infty} \frac{a_{n+1}}{a_n} = r \in (0,1) \), the series converges.
            \item If \( \varliminf_{n \to +\infty} \frac{a_{n+1}}{a_n} = r' > 1 \), the series diverges.
            \item If \( r = 1 \) or \( r' = 1 \), the test fails.
        \end{enumerate}
    
    \item[Raabe Test] Let \( \sum_{n=1}^{\infty} a_n \) be a strictly positive term series.
        \begin{enumerate}
            \item If \( \exists r > 1, \exists N_0 \in \mathbb{N} \text{ s.t. } 
                \forall n > N_0: n \left( \frac{a_n}{a_{n+1}} - 1 \right) \geqslant r \), 
                then the series converges.
            \item If \( \exists N_0 \in \mathbb{N}, \text{ s.t. } \forall n > N_0: 
                n \left( \frac{a_n}{a_{n+1}} - 1 \right) \leqslant 1 \), then the series diverges.
        \end{enumerate}
        \textbf{Limit Form}
        Let \( \sum_{n=1}^{\infty} a_n \) be a strictly positive term series. Then:
        \begin{enumerate}
            \item If \( \varliminf_{n \to +\infty} n \left( \frac{a_n}{a_{n+1}} - 1 \right) = l > 1 \), the series converges.
            \item If \( \varlimsup_{n \to +\infty} n \left( \frac{a_n}{a_{n+1}} - 1 \right) = l' < 1 \), the series diverges.
            \item If \( l = 1 \) or \( l' = 1 \), the test fails.
        \end{enumerate}

    \item[Bertrand Test] Let \( \sum_{n=1}^{\infty} a_n \) be a strictly positive term series.
        \begin{enumerate}
            \item If \( \varliminf_{n \to +\infty} \ln n \left[ n \left( \frac{a_n}{a_{n+1}} - 1 \right) \right] = l > 1 \), 
                the series converges.
            \item If \( \varlimsup_{n \to +\infty} \ln n \left[ n \left( \frac{a_n}{a_{n+1}} - 1 \right) \right] = l' < 1 \), 
                the series diverges.
            \item If \( l = 1 \) or \( l' = 1 \), the test fails.
        \end{enumerate}

    \item [Gauß Test] Let \( \sum_{n=1}^{\infty} a_n \) be a strictly positive term series, and suppose:
        \[
        \frac{a_n}{a_{n+1}} = 1 + \frac{1}{n} + \frac{\delta}{n \ln n} + o\left( \frac{1}{n \ln n} \right), \quad (n \to +\infty).
        \]
        Then:
        \begin{enumerate}
            \item If \( \delta > 1 \), the series converges.
            \item If \( \delta < 1 \), the series diverges.
            \item If \( \delta = 1 \), the criterion fails.
        \end{enumerate}

        \textbf{Generalized Form}
        Let \( \sum_{n=1}^{\infty} a_n \) be a strictly positive term series, and suppose:
        \[
        \frac{a_n}{a_{n+1}} = 1 + \frac{1}{n} + \frac{\delta_n}{n \ln n} + o\left( \frac{1}{n \ln n} \right), 
        \quad (n \to +\infty).
        \]
        If \( \lim_{n \to \infty} \delta_n = \delta \in \mathbb{R} \), then:
        \begin{enumerate}
            \item If \( \delta > 1 \), the series converges.
            \item If \( \delta < 1 \), the series diverges.
            \item If \( \delta = 1 \), the criterion fails.
        \end{enumerate}
\end{description}
\end{theorem}

\begin{note}
    The Bertrand test can be refined by considering series such as:
    \[
    \sum_{n=3}^{\infty} \frac{1}{n \ln n (\ln \ln n)^p}, 
    \quad \sum_{n=9}^{\infty} \frac{1}{n \ln n \ln \ln n (\ln \ln n)^p}, 
    \cdots
    \]
    These refinements are collectively known as the Bertrand test.
\end{note}

\begin{remark}
    All the aforementioned criteria are derived from the Comparison Criterion.
    \begin{itemize}
        \item By comparing positive term series with the geometric series (or equal ratio series), 
            the Cauchy Criterion and d'Alembert Criterion are derived.
        \item By comparing positive term series with the slower-converging series 
            \( \sum_{n=1}^{\infty} \frac{1}{n^\alpha} \) (\( \alpha > 1 \)), the Raabe Criterion is derived.
        \item By comparing positive term series with the even slower-converging series 
            \( \sum_{n=1}^{\infty} \frac{1}{n \ln^\alpha n} \) (\( \alpha > 1 \)), the Gauß Criterion is derived.
    \end{itemize}
    \textbf{General Observation}
    The slower the convergence of the series used for comparison, the more precise the derived criterion.
\end{remark}






\begin{leftbarTitle}{Integral Test}\end{leftbarTitle}
\begin{theorem}{Cauchy Integral Test}
    Let \( f(x) \) be defined on \( [a, +\infty) \), 
    where \( f(x) \geqslant 0 \), and \( f(x) \) is Riemann integrable on any finite interval \( [a, A] \).

    Consider a monotonic increasing sequence \( \{ a_n \} \) such that \( a = a_1 < a_2 < \dots < a_n < \dots \), and let:
    \[
    u_n = \int_{a_n}^{a_{n+1}} f(x) \, \mathrm{d}x.
    \]

    Then the improper integral \( \int_{a}^{+\infty} f(x) \, \mathrm{d}x \) 
    and the positive term series \( \sum_{n=1}^{\infty} u_n \) converge or diverge to \( +\infty \) simultaneously. 
    Moreover:
    \[
    \int_{a}^{+\infty} f(x) \, \mathrm{d}x 
    = \sum_{n=1}^{\infty} u_n 
    = \sum_{n=1}^{\infty} \int_{a_n}^{a_{n+1}} f(x) \, \mathrm{d}x.
    \]
\end{theorem}

\begin{leftbarTitle}{Other Tests}\end{leftbarTitle}
\begin{theorem}{Cauchy Condensation Test}
    Let \( \{ a_n \} \) be a monotonically decreasing sequence of positive numbers. 
    Then the positive term series \( \sum_{n=1}^{\infty} a_n \) converges if and only if the condensed series:
    \[
    \sum_{n=0}^{\infty} 2^n a_{2^n} = a_1 + 2a_2 + 4a_4 + \dots + 2^n a_{2^n} + \dots
    \]
    converges.
\end{theorem}


\section{General Term Series and Its Convergence Tests}
\begin{leftbarTitle}{Cauchy Convergence Criterion for Series}\end{leftbarTitle}
\begin{theorem}{Cauchy Convergence Criterion for Series}
    The necessary and sufficient condition for the convergence of the series \( \sum_{n=1}^{\infty} x_n \) is:
    \[
    \forall \varepsilon > 0, \exists N \in \mathbb{N}, \forall m, n > N : 
    \left| x_{n+1} + x_{n+2} + \cdots +x_{m} \right| = \left| \sum_{k=n+1}^{m} x_k \right| < \varepsilon.
    \]
\end{theorem}


\begin{leftbarTitle}{Alternative Series}\end{leftbarTitle}
\begin{definition}{Alternative Series}
    A series of the form:
    \[
    \sum_{n=1}^{\infty}x_{n} = 
    \sum_{n=1}^{\infty} (-1)^{n-1} u_n\quad (u_{n}>0),
    \]
    is called an \textbf{alternative series}.

    Moreover, if \( u_n \) is a monotonically decreasing sequence and \( \lim_{n \to \infty} u_n = 0 \), 
    then the series is called a \textbf{Leibniz series}.
\end{definition}

\begin{theorem}{Leibniz Test}
    Leibniz series converges.
\end{theorem}


\begin{leftbarTitle}{Abel-Dirichlet Test}\end{leftbarTitle}

\begin{theorem}{Abel Transform (Discrete Integration by Parts/Summation by Parts)}\label{thm:Abel Transform}
    Let \(\{a_n\}, \{b_n\}\) be two sequences, then for any \(n\in \mathbb{N}^{+}\),
    \[
        \sum_{k=1}^{n} a_k b_k = a_n B_n + \sum_{k=1}^{n-1} (a_{k+1} - a_{k})B_k,
    \]
    where \(B_n = \sum_{k=1}^{n} b_k\).
\end{theorem}

\begin{figure}[h]
    \centering
    \includegraphics[width=0.5\textwidth, angle=180]{img/AbelTransform.jpg}
\end{figure}

\begin{lemma}{Abel Lemma (Discrete Second Integral Mean Value Theorem)}
    Let \(\{a_n\}, \{b_n\}\) be two sequences, if \(\{a_n\}\) is a monotonic sequence 
    and \(\{B_k\} = \sum_{k=1}^{n} b_k\) is a bounded sequence with bound \(M\),
    then for any \(p\in \mathbb{N}^{+}\),
    \[
        \left| \sum_{k=1}^{p} a_k b_k \right| \leqslant M \left( |a_{1}| + 2|a_{p}| \right) .
    \]
\end{lemma}

\begin{theorem}{Abel-Dirichlet Test}
    The series \(\sum_{n=1}^{\infty} a_n b_n\) converges if one of the following two conditions is satisfied:
    \begin{description}
        \item[Abel] \(\{a_n\}\) is a bounded monotonic sequence and \(\sum_{n=1}^{\infty} b_n\) converges.
        \item[Dirichlet]  \(\{a_n\}\) is a monotonic sequence, \(\lim_{n \to \infty} a_n = 0\),
            and the partial sums \(B_n = \sum_{k=1}^{n} b_k\) are bounded.       
    \end{description}
\end{theorem}

\section{Absolute and Conditional Convergence of Series}
\begin{definition}{Absolute and Conditional Convergence of Series}
    If the series \( \sum_{n=1}^{\infty} |x_n| \) converges, 
    then the series \( \sum_{n=1}^{\infty} x_n \) is said to be \textbf{absolutely convergent}.

    If the series \( \sum_{n=1}^{\infty} x_n \) converges but is not absolutely convergent, 
    then the series \( \sum_{n=1}^{\infty} x_n \) is said to be \textbf{conditionally convergent}.
\end{definition}

\section{Comparison of Convergence Speed of Series}
The series \( \sum_{n=1}^{\infty} a_n \) is said to converge faster than the series \( \sum_{n=1}^{\infty} b_n \) if:
\[
\lim_{n \to \infty} \frac{a_n}{b_n} = 0.
\]

\begin{theorem}{Du Bois-Reymond Theorem}
    For a given convergent positive term series \( \sum_{n=1}^{\infty} a_n \), there always exists a convergent strictly positive term series \( \sum_{n=1}^{\infty} b_n \) such that:
    \[
    \lim_{n \to \infty} \frac{a_n}{b_n} = 0.
    \]
\end{theorem}

\begin{theorem}{Abel Theorem}
    For a given divergent positive term series \( \sum_{n=1}^{\infty} a_n \), there always exists a divergent positive term series \( \sum_{n=1}^{\infty} b_n \) such that:
    \[
    \lim_{n \to \infty} \frac{a_n}{b_n} = 0.
    \]
\end{theorem}

\begin{remark}
    The above two theorems imply that the slowest converging positive term series \underline{does not} exist.
\end{remark}



\section{Infinite Products}
\begin{leftbarTitle}{Infinite Products}\end{leftbarTitle}


\begin{leftbarTitle}{Two Formulas}\end{leftbarTitle}
\begin{theorem}{Wallis Formula}
    \[
    \lim_{n \to \infty} \frac{1}{2n+1} \left[ \frac{(2n)!!}{(2n-1)!!} \right]^{2}  = \frac{\pi}{2}.
    \]
    Equivalently (\(n\to +\infty\)),
    \begin{gather*}
        \frac{(2n)!!}{(2n-1)!!} \sim \sqrt{\pi n}, \\
        \frac{(n!)^{2}2^{2n}}{(2n)!} \sim \sqrt{\pi n}.
    \end{gather*}
\end{theorem}


\begin{theorem}{Stirling Formula}
    \[
    n! = \sqrt{2\pi n} \left( \frac{n}{e} \right)^n 
    \left( 1 + \frac{1}{12n} - \frac{1}{288n^2} + \frac{139}{51840n^3} - \frac{571}{2488320n^4} + \cdots 
    + \frac{B_{2n}}{2k(2k-1) n^{k}} + \cdots  \right),
    \]
    where \( B_{2k} \) are Bernoulli numbers of order \( 2k \).
    Simplified form:
    \[
    n! \sim \sqrt{2\pi n} \left( \frac{n}{e} \right)^{n} \quad (n \to +\infty),
    \]
    or
    \[
    n! = \sqrt{2\pi n} \left( \frac{n}{e} \right)^{n} e^{\theta_n}, \quad \frac{1}{12n+1} < \theta_n < \frac{1}{12n}.
    \]
\end{theorem}


\section{Special Series}
\begin{description}
    \item[Geometric Series] 
        \[
        \sum_{n=0}^{\infty} q^n = \frac{1}{1-q},
        \]
        it converges when \( |q| < 1 \), diverges otherwise.
    \item[Telescoping Series]
        \[
        \sum_{n=1}^{\infty} (a_n - a_{n+1}) = a_1 - \lim_{n \to \infty} a_{n+1},
        \]
        it converges when \( \lim_{n \to \infty} a_n \) exists, diverges otherwise.
    \item[\(p\)-Series/Hyperharmonic Series]
        \[
        \sum_{n=1}^{\infty} \frac{1}{n^p},
        \]
        it converges when \( p > 1 \), diverges otherwise.
    \item[\(q\)-Series]
        \[
        \sum_{n=1}^{\infty} \frac{1}{n (\ln n)^q},
        \]
        it converges when \( q > 1 \), diverges otherwise.
    \item[Generalized \(q\)-Series]
        \[
        \sum_{n=3}^{\infty} \frac{1}{n \ln n (\ln \ln n)\cdots (\ln^{(k-1)} n) (\ln^{(k)} n)^q},
        \]
        where \( \ln^{(k)} n \) denotes the \( k \)-th iterated logarithm,
        it converges when \( q > 1 \), diverges otherwise.        
\end{description} % 极值原理
\chapter{Quadratic Forms} % 二次型
\section{Quadratic Forms and Their Standard Forms}
\begin{definition}{Quadratic Form}
    Let \(P\) be a number field, a quadratic homogeneous polynomial in \( n \) variables over \( P \)\footnote{
        That is, the coefficients of the polynomial belong to the field \( P \).
    }:
    \begin{align*}
        f( x_{1}, x_{2}, \cdots, x_{n}) &= \sum_{i=1}^{n} \sum_{j=1}^{n} a_{ij} x_{i} x_{j} \\
    &= a_{11}x_{1}^{2} + 2a_{12}x_{1}x_{2} + \cdots + 2a_{1n}x_{1}x_{n} + a_{22}x_{2}^{2} + \cdots + 2a_{2n}x_{2}x_{n} + \cdots + a_{nn}x_{n}^{2},
    \end{align*}
    is called a \textbf{quadratic form} in \( n \) variables over field \( P \).

    It can be expressed in matrix form as:
    \[
    f( x_{1}, x_{2}, \cdots, x_{n}) = X^{\mathrm{T}} A X,
    \]
    where 
    \[
    X = \begin{pmatrix}
        x_{1} \\
        x_{2} \\
        \vdots \\
        x_{n}
    \end{pmatrix}, \quad
    A = (a_{ij})_{n \times n}, \quad a_{ij} = a_{ji} \quad (1 \leqslant i, j \leqslant n).
    \]
\end{definition}
It is easy to verify that the matrix \( A \) of a quadratic form is symmetric.
\begin{note}
    In fact, for any square matrix \( B \)\footnote{
        For a skew-symmetric matrix \( S \) (\( S^{T} = -S \)),
        \[
        X^{T} S X = - (X^{T} S X)^{T} = - X^{T} S^{T} X = - X^{T} S X \implies X^{T} S X = 0.
        \]
    }, we have:
    \begin{align*}
        X^{T}BX &= X^{T}\left( \frac{B + B^{T}}{2} \right)X + X^{T}\left( \frac{B - B^{T}}{2} \right)X \\
        &= X^{T}\left( \frac{B + B^{T}}{2} \right)X + 0 \\
        &= X^{T}\left( \frac{B + B^{T}}{2} \right)X.
    \end{align*}
    It shows that any quadratic form can be represented by a symmetric matrix.
\end{note}




\section{Canonical Forms}
\section{Definite Quadratic Forms}
\begin{definition}{Positive Definite Quadratic Form}
    A real quadratic form \( f( x_{1}, x_{2}, \cdots, x_{n})=X^{\mathrm{T}}AX \) is called \textbf{positive definite} if:
    \[
    f( x_{1}, x_{2}, \cdots, x_{n}) > 0, \quad \forall X \neq 0.
    \]
    And \( A \) is called a \textbf{positive definite matrix}.
\end{definition}

\begin{theorem}{Sufficient and Necessary Condition for Positive Definiteness}
    A real quadratic form \( f( x_{1}, x_{2}, \cdots, x_{n})=X^{\mathrm{T}}AX \) is positive definite if and only if:
    \begin{enumerate}
        \item The positive inertia index of \( f \) is \( n \);
        \item \(A\) is congruent to the identity matrix \( E \);
        \item All eigenvalues of \( A \) are positive;
        \item All leading principal minors\footnote{
            The leading principal minors of a matrix are the determinants of 
            the top-left \( k \times k \) submatrices for \( k = 1, 2, \ldots, n \).
        } of \( A \) are positive.
    \end{enumerate}
    
\end{theorem} % Ramsey 理论

% --- Part 4: 结构与代数 (Structure & Algebra) ---
\chapter{Inner Product Spaces} % 内积空间
\section{Bilinear Forms}
\begin{definition}{Bilinear Form}
    Let \( V \) be a linear space over field \( F \).
    A function \( f: V \times V \to F \) is called a \textbf{bilinear form} on \( V \) if:
    \begin{enumerate}
        \item For any fixed \( \beta \in V \), the function \( f(\cdot, \beta): V \to F \) defined by
            \( f(\alpha, \beta) \) is a linear function on \( V \);
        \item For any fixed \( \alpha \in V \), the function \( f(\alpha, \cdot): V \to F \) defined by
            \( f(\alpha, \beta) \) is a linear function on \( V \).
    \end{enumerate}
    
\end{definition}

\section{Real Inner Product Spaces}
\begin{definition}{Real Inner Product Space}
    A \textbf{real inner product space} is a real linear space \( V \) 
    equipped with a function \( (\cdot, \cdot): V \times V \to \mathbb{R} \) 
    satisfying the following properties:
    \begin{description}
        \item [Positivity] \( (\alpha, \alpha) \geq 0, \quad \forall \alpha \in V\),
            and \( (\alpha, \alpha) = 0 \) if and only if \( \alpha = 0 \);
        \item [Symmetry] \( (\alpha, \beta) = (\beta, \alpha), \quad \forall \alpha, \beta \in V\);
        \item [Linearity in the First Argument] 
            \( (k_1\alpha_1 + k_2\alpha_2, \beta) = k_1(\alpha_1, \beta) + k_2(\alpha_2, \beta), 
            \quad \forall k_1, k_2 \in \mathbb{R}, 
            \alpha_1, \alpha_2, \beta \in V. \)
    \end{description}
    The function \( (\cdot, \cdot) \) is called the (real) \textbf{inner product}\footnote{
        The inner product can be also defined as a positive-definite bilinear form.
    } on \( V \).

    Real inner product spaces with finite dimensions are called \textbf{Euclidean spaces}. 
\end{definition}


\begin{definition}{Normed Linear Space}
    A real \textbf{normed linear space} is a real linear space \( V \) 
    equipped with a function \( \| \cdot \|: V \to \mathbb{R} \) 
    satisfying the following properties\footnote{
        Similarly, the definition of norm can be given in complex linear spaces.
    }:
    \begin{description}
        \item [Positivity] \( \| \alpha \| \geq 0, \quad \forall \alpha \in V\),
            and \( \| \alpha \| = 0 \) if and only if \( \alpha = 0 \);
        \item [Homogeneity] \( \| k\alpha \| = |k| \|\alpha\|, 
            \quad \forall k \in \mathbb{R}, \alpha \in V; \)
        \item [Triangle Inequality] \( \| \alpha + \beta \| \leq \|\alpha\| + \|\beta\|, 
            \quad \forall \alpha, \beta \in V. \)
    \end{description}
    The function \( \| \cdot \| \) is called the (vector) \textbf{norm}\footnote{
        If replace positivity with semi-positivity in the above definition,
        i.e., \( \| \alpha \| \geq 0, \quad \forall \alpha \in V\),
        then we get the definition of \textbf{semi-norm}.
    } on \( V \).
\end{definition}

In Euclidean spaces, the norm can be induced by the inner product:
\[
\| \alpha \| = (\alpha, \alpha)^{\frac{1}{2}}, \quad \forall \alpha \in V,
\]
which is called the \textbf{Euclidean norm}.

\begin{remark}
    The definition of Euclidean space can be also derived from normed linear space
    or metric space.
\end{remark}
\vspace{0.7cm}
\begin{theorem}{Cauchy-Буняко́вский-Schwarz Inequality}
    Let \( V \) be an inner product space\footnote{
        It \emph{does not} require in real inner product spaces.
    }.
    For any vectors \( \alpha, \beta \in V \), the following inequality holds:
    \[
    |(\alpha, \beta)| \leq \|\alpha\| \|\beta\|,
    \]
    with equality if and only if \( \alpha \) and \( \beta \) are linearly dependent.
\end{theorem}

\begin{note}
    \begin{enumerate}
        \item In linear space \(\mathbb{R}^{n}\), for vectors \( \alpha, \beta \in \mathbb{R}^{n} \), 
            define the inner product as:
            \[
            (\alpha, \beta) = \alpha^{\mathrm{T}} \beta = \sum_{i=1}^{n} x_{i} y_{i},
            \]
            then \(\mathbb{R}^{n}\) forms a Euclidean space, called the \textbf{standard Euclidean space}.
            We still denote it as \(\mathbb{R}^{n}\) without confusion.

            In this case, Cauchy-Буняко́вский-Schwarz inequality becomes:
            \[
            \left| \sum_{i=1}^{n} x_{i} y_{i} \right| \leqslant
            \left( \sum_{i=1}^{n} x_{i}^{2} \right)^{\frac{1}{2}}
            \left( \sum_{i=1}^{n} y_{i}^{2} \right)^{\frac{1}{2}}.
            \]

        \item In \(C[a, b]\), which is the linear space of continuous real-valued functions on interval \([a, b]\),
            for functions \( f(x), g(x) \in C[a, b] \),
            define the inner product as:
            \[
            (f, g) = \int_{a}^{b} f(x) g(x) \, \mathrm{d}x,
            \]
            then \(C[a, b]\) forms a Euclidean space.

            In this case, Cauchy-Буняко́вский-Schwarz inequality becomes:
            \[
            \left| \int_{a}^{b} f(x) g(x) \, \mathrm{d}x \right| \leqslant
            \left( \int_{a}^{b} f^{2}(x) \, \mathrm{d}x \right)^{\frac{1}{2}}
            \left( \int_{a}^{b} g^{2}(x) \, \mathrm{d}x \right)^{\frac{1}{2}}.
            \]
    \end{enumerate}
\end{note}

\vspace{0.7cm}
Some concepts in real inner product spaces can be derived:
\begin{description}
    \item[Distance] The distance between two vectors \( \alpha, \beta \in V \) is defined as:
    \[
    d(\alpha, \beta) = \| \alpha - \beta \|
    \]
    \item[Angle] The angle \( \theta \) between two non-zero vectors \( \alpha, \beta \in V \) is defined using the inner product:
    \[
    \cos \theta = \frac{(\alpha, \beta)}{\|\alpha\| \|\beta\|}, 0 \leq \theta \leq \pi.
    \]
    \item[Orthogonality] Two vectors \( \alpha, \beta \in V \) are said to be orthogonal if:
    \[
    (\alpha, \beta) = 0,
    \]
    denoted as \( \alpha \perp \beta \).
\end{description}

\section{Metric Matrices and Orthonormal Bases} % 度量矩阵与标准正交基
\begin{leftbarTitle}{Metric Matrices}\end{leftbarTitle}
\begin{definition}{Gram Matrix}
    Let \( V \) be a \(n\)-dimensional Euclidean space, 
    and let \( \{\varepsilon_1, \varepsilon_2, \dots, \varepsilon_n\} \) be a basis of \( V \).
    The matrix
    \[
    G = ((\varepsilon_i, \varepsilon_j))_{n \times n} =
    \begin{pmatrix}
        (\varepsilon_1, \varepsilon_1) & (\varepsilon_1, \varepsilon_2) & \cdots & (\varepsilon_1, \varepsilon_n) \\
        (\varepsilon_2, \varepsilon_1) & (\varepsilon_2, \varepsilon_2) & \cdots & (\varepsilon_2, \varepsilon_n) \\
        \vdots & \vdots & \ddots & \vdots \\
        (\varepsilon_n, \varepsilon_1) & (\varepsilon_n, \varepsilon_2) & \cdots & (\varepsilon_n, \varepsilon_n)
    \end{pmatrix}
    \]
    is called the \textbf{Gram matrix} (or metric matrix) of the inner product on \( V \)
    under the basis \( \{\varepsilon_1, \varepsilon_2, \dots, \varepsilon_n\} \).
\end{definition}

For all vectors
\begin{gather*}
    \alpha = x_{1}\varepsilon_{1}+x_{2}\varepsilon_{2}+\cdots+x_{n}\varepsilon_{n},  \\
    \beta = y_{1}\varepsilon_{1}+y_{2}\varepsilon_{2}+\cdots+y_{n}\varepsilon_{n} \in V ,
\end{gather*}
since \(( \alpha, \beta)=\sum_{i=1}^{n} \sum_{j=1}^{n}  (\varepsilon_i, \varepsilon_j)x_{i}y_{j} \), 
it can be expressed in matrix form as:
\[
( \alpha, \beta) = X^{\mathrm{T}} G Y, \quad
X = \begin{pmatrix} x_1 \\ x_2 \\ \vdots \\ x_n \end{pmatrix}, \quad
Y = \begin{pmatrix} y_1 \\ y_2 \\ \vdots \\ y_n \end{pmatrix}.
\]

\begin{property}
    \begin{enumerate}
        \item Gram matrix \( G \) is symmetric; 
        \item \( G \) is positive semi-definite;
        \item In \(V\), Gram matrices under different bases are congruent;
        \item \(|G|\geqslant 0\), equality holds if and only if the basis is linearly dependent.
    \end{enumerate}
\end{property}

\begin{leftbarTitle}{Orthonormal Bases}\end{leftbarTitle}
\begin{definition}{Orthonormal Vector Set and Orthonormal Basis}
    Let \( V \) be a \(n\)-dimensional real inner product space.
    A set of non-zero vectors \( \{ \varepsilon_1, \varepsilon_2, \dots, \varepsilon_n \} \) in \( V \) 
    is called an \textbf{orthonormal vector set} if they are pairwise orthogonal.

    A orthonormal vector set with \( n \) vectors is called a \textbf{orthonormal basis} of \( V \).
    Orthonormal basis made up of unit vectors is called a \textbf{orthonormal basis} of \( V \).
\end{definition}

\begin{property}
    In \(n\)-dimensional Euclidean space, 
    \begin{enumerate}
        \item Orthonormal vector set is linearly independent;
        \item A set of basis is orthonormal basis if and only if its Gram matrix is the identity matrix;
        \item Standard orthonormal basis always exists.
        \item Any orthonormal vector set can be extended to a orthonormal basis.
    \end{enumerate}
\end{property}
In \(n\)-dimensional Euclidean space, let \( \{ \varepsilon_1, \varepsilon_2, \dots, \varepsilon_n \} \) 
be a orthonormal basis, then
\begin{itemize}
    \item for all \(\alpha \in V:\alpha = \sum_{i=1}^{n} (\alpha, \varepsilon_i) \varepsilon_i\), 
        which is called the \textbf{Fourier expansion} of vector \( \alpha \) under the basis;
    \item if the coordinates of \(\alpha, \beta\) under this basis are
        \[
        X = \begin{pmatrix} x_1 \\ x_2 \\ \vdots \\ x_n \end{pmatrix}, \quad
        Y = \begin{pmatrix} y_1 \\ y_2 \\ \vdots \\ y_n \end{pmatrix},
        \]
        then
        \[
        (\alpha, \beta) = X^{\mathrm{T}} Y = \sum_{i=1}^{n} x_i y_i.
        \]
\end{itemize}

\begin{definition}{Orthonormal Matrix}
    A square matrix \( T \in \mathbb{R}^{n \times n} \) is called an \textbf{orthonormal matrix} if:
    \[
    T^{\mathrm{T}} T = T T^{\mathrm{T}} = E,
    \]
    where \( E \) is the identity matrix of order \( n \).
\end{definition}
Let \( V \) be a \(n\)-dimensional Euclidean space,
\(\varepsilon_{1}, \varepsilon_{2},\cdots, \varepsilon_{n}\) and \(\eta_{1}, \eta_{2},\cdots, \eta_{n}\)
be orthonormal bases of \( V \), and let \( T \) be the transition matrix between them,
i.e., 
\[
(\eta_{1}, \eta_{2}, \cdots, \eta_{n}) = (\varepsilon_{1}, \varepsilon_{2}, \cdots, \varepsilon_{n}) T.
\]
Then 
\begin{enumerate}
    \item \(T\) is an orthonormal matrix; 
    \item \(T\) is upper triangular matrix;
\end{enumerate}

\begin{theorem}
    Let \(\varepsilon_{1}, \varepsilon_{2},\cdots, \varepsilon_{n}\) be a orthonormal basis of 
    \(n\)-dimensional Euclidean space \( V \), and
    \[
    (\eta_{1}, \eta_{2}, \cdots, \eta_{n}) = (\varepsilon_{1}, \varepsilon_{2}, \cdots, \varepsilon_{n}) T.
    \]
    Then \( \eta_{1}, \eta_{2},\cdots, \eta_{n} \) is a orthonormal basis of \( V \)
    if and only if \( T \) is an orthonormal matrix.
\end{theorem}


\begin{leftbarTitle}{Gram-Schmidt Process}\end{leftbarTitle}
\begin{theorem}
    For any basis \( \{ \varepsilon_1, \varepsilon_2, \dots, \varepsilon_n \} \) of \(n\)-dimensional Euclidean space \( V \),
    there exists a orthonormal basis \( \{ \eta_1, \eta_2, \dots, \eta_n \} \) such that:
    \[
    \langle \varepsilon_1, \varepsilon_2, \dots, \varepsilon_k \rangle
    = \langle \eta_1, \eta_2, \dots, \eta_k \rangle, \quad k = 1, 2, \dots, n.
    \]
\end{theorem}

\begin{proof}
    
\end{proof}

\vspace{0.7cm}
\begin{example}
    Let \(n\geqslant 2\), \(A\) is an \(n\)-order real symmetric matrix.
    \(\alpha=(a_{1},a_{2},\ldots,a_{n})^{\mathrm{T}}, \beta=(b_{1},b_{2},\ldots,b_{n})^{\mathrm{T}}\),
    which are two eigenvectors of \(A\) corresponding to different eigenvalues \(\lambda_{1}, \lambda_{2}\) respectively.
    Let 
    \[
    B = \begin{pmatrix} 
        a_{1}+b_{1} & a_{1}+b_{2} & \cdots & a_{1}+b_{n} \\
        a_{2}+b_{1} & a_{2}+b_{2} & \cdots & a_{2}+b_{n} \\
        \vdots & \vdots & \ddots & \vdots \\ 
        a_{n}+b_{1} & a_{n}+b_{2} & \cdots & a_{n}+b_{n} 
    \end{pmatrix},
    \]
    find all eigenvalues of matrix \(B\).
\end{example}

\begin{solution}
    Let \(\xi_{1}=\frac{\alpha}{\|\alpha\|}, \xi_{2}=\frac{\beta}{\|\beta\|}\),
    and \(S_{a}=\sum_{i=1}^{n}a_{i}, S_{b}=\sum_{i=1}^{n}b_{i}\).
    \newline It is easy to see that \(\xi_{1}, \xi_{2}\) are orthonormal unit vectors.
    \newline Extend \(\{\xi_{1}, \xi_{2}\}\) to an orthonormal basis \(\{\xi_{1}, \xi_{2}, \ldots, \xi_{n}\}\) of \(\mathbb{R}^{n}\).
    \newline Let \(P=(\xi_{1}, \xi_{2}, \ldots, \xi_{n})\), then \(P\) is an orthonormal matrix.
    \newline \(B\) can be expressed as: 
    \[
    B = \alpha e^{\mathrm{T}} + e \beta^{\mathrm{T}},
    \]
    where \(e=(1,1,\ldots,1)^{\mathrm{T}}\).
    \newline Therefore,
    \[
    P^{-1}BP = P^{\mathrm{T}}BP = \begin{pmatrix} 
        \xi_{1}^{\mathrm{T}} \\ \xi_{2}^{\mathrm{T}} \\ \vdots \\ \xi_{n}^{\mathrm{T}} 
    \end{pmatrix} 
    \begin{pmatrix} 
        B \xi_{1} & B \xi_{2} & \cdots & B \xi_{n}
    \end{pmatrix}.
    \]
    Since 
    \[
    B\xi_{1} = (\alpha e^{\mathrm{T}} + e \beta^{\mathrm{T}}) \frac{a_{1}}{\|\alpha\|} = 
    \frac{\alpha(e^{\mathrm{T}}\alpha)}{\|\alpha\|} + \frac{e(\beta^{\mathrm{T}}\alpha)}{\|\alpha\|} =
    \frac{S_{a}}{\|\alpha\|} \alpha + 0 = S_{a} \|\alpha\| \xi_{1},
    \]
    and similarly, \(B\xi_{2} = S_{b} \|\beta\| \xi_{2}\).
    For \(k\geqslant 3\), we have
    \[
    B\xi_{k} = (\alpha e^{\mathrm{T}} + e \beta^{\mathrm{T}}) \xi_{k} = 
    \alpha (e^{\mathrm{T}} \xi_{k}) + e (\beta^{\mathrm{T}} \xi_{k}) = 0 + 0 = 0.
    \]
    Then 
    \[
    P^{\mathrm{T}}BP = \begin{pmatrix} 
        S_{a}  & 0 & 0 & \cdots & 0 \\
        0 & S_{b} & 0 & \cdots & 0 \\
        0 & 0 & 0 & \cdots & 0 \\
        \vdots & \vdots & \vdots & \ddots & \vdots \\
        0 & 0 & 0 & \cdots & 0
    \end{pmatrix}.
    \]
    Therefore, the eigenvalues of matrix \(B\) are:
    \[
    \lambda_{1} = S_{a}, \quad \lambda_{2} = S_{b}, \quad \lambda_{3} = 0, \quad \ldots, \quad \lambda_{n} = 0.
    \]
\end{solution}

\section{Isomorphism of Real Inner Product Spaces}

\section{Orthogonal Completion and Orthogonal Projection}
\begin{leftbarTitle}{Orthogonal Completion}\end{leftbarTitle}
\begin{definition}{Orthogonal Complement}
    Let \( V \) be a real inner product space, and let \( W \) be a subspace of \( V \).
    The set
    \[
    W^{\perp} = \{ \alpha \in V \mid (\alpha, \beta) = 0, \quad \forall \beta \in W \}
    \]
    is called the \textbf{orthogonal complement}\footnote{
        Another equivalent definition is:
        \[
        W^{\perp} \perp W \text{ and } V = W + W^{\perp}.
        \]
    } of \( W \) in \( V \),
    and \( W^{\perp} \) is also a subspace of \( V \), called the \textbf{orthogonal subspace} of \( W \) in \( V \).
\end{definition}
\begin{property}
    \begin{enumerate}
        \item There exists a unique orthogonal complement \( W^{\perp} \) of \( W \) in Euclidean space \( V \);
        \item \(W \oplus W^{\perp} = V\);
        \item \(\left( W^{\perp} \right)^{\perp} = W\)
        \item \(\left( V_{1}+V_{2} \right)^{\perp} = V_{1}^{\perp} \cap V_{2}^{\perp}, \quad 
            \left( V_{1} \cap V_{2} \right)^{\perp} = V_{1}^{\perp} + V_{2}^{\perp} \),
            where \( V_{1}, V_{2} \) are subspaces of \( V \).
    \end{enumerate}
\end{property}


\begin{leftbarTitle}{Least Squares Method}\end{leftbarTitle}
\begin{definition}{Orthogonal Projection}
    Let \( V \) be a real inner product space, 
    and let \( W \) be a subspace of \( V \).
    For any vector \( \alpha \in V \),
    if there exists a vector \( \beta \in W \) such that:
    \[
    \alpha - \beta \in W^{\perp},
    \]
    then \( \beta \) is called the \textbf{orthogonal projection} of \( \alpha \) onto \( W \),
    denoted as \( \beta = \operatorname{proj}_{W} \alpha \).
\end{definition}

% 最佳逼近元
\begin{definition}{Best Approximation Element}
    Let \( V \) be a real inner product space, 
    and let \( W \) be a subspace of \( V \).
    For any vector \( \alpha \in V \),
    if there exists a vector \( \beta \in W \) such that:
    \[
    \| \alpha - \beta \| = \min_{\gamma \in W} \| \alpha - \gamma \|,
    \]
    then \( \beta \) is called the \textbf{best approximation element} of \( \alpha \) in \( W \).
\end{definition}

\begin{property}
    In Euclidean space \( V \), let \( W \) be an \(n\)-dimensional subspace of \( V \).
    Take an orthogonal basis \( \{ \varepsilon_1, \varepsilon_2, \dots, \varepsilon_n \} \) of \( W \).
    \begin{enumerate}
        \item For any vector \( \alpha \in V \),
            the orthogonal projection of \( \alpha \) onto \( W \) exists and is unique,
            and it is also the best approximation element of \( \alpha \) in \( W \);
        \item The orthogonal projection of \( \alpha \) onto \( W \) can be expressed as:
            \[
            \operatorname{proj}_{W} \alpha = 
            \sum_{i=1}^{n} \frac{(\alpha, \varepsilon_i)}{(\varepsilon_i, \varepsilon_i)} \varepsilon_i
            =: \sum_{i=1}^{n} c_{i} \varepsilon_i.
            \]
        \item The remainder of the orthogonal projection satisfies:
            \[
            \| \alpha - \operatorname{proj}_{W} \alpha \|^{2} = \| \alpha \|^{2} - \| \operatorname{proj}_{W} \alpha \|^{2}
            = \| \alpha \|^{2} - \sum_{i=1}^{n} c_{i}^{2} \|\varepsilon_i\|^{2} .
            \]
    \end{enumerate}
\end{property}

\section{Orthogonal Transformations and Symmetric Transformations}
\begin{leftbarTitle}{Orthogonal Transformations}\end{leftbarTitle}
\begin{leftbarTitle}{Symmetric Transformations}\end{leftbarTitle}
\begin{definition}{Symmetric Transformation}
    Let \( V \) be a real inner product space, 
    and let \( \mathcal{A}\in \operatorname{Hom}(V) \).
    If
    \[
    (\mathcal{A}\alpha, \beta) = (\alpha, \mathcal{A}\beta), \quad \forall \alpha, \beta \in V,
    \]
    then \( \mathcal{A} \) is called a \textbf{symmetric transformation} on \( V \).
\end{definition}
Obviously, symmetric transformations are linear transformations.

\begin{lemma}
    \begin{enumerate}
        \item If \( A \) is real symmetric matrix, then \( A \) has and only has \( n \) real eigenvalues (counting multiplicities). 
        \item Let \(V\) be a \(n\)-dimensional Euclidean space, 
            then any linear transformation \( \mathcal{A}: V \to V \) is symmetric if and only if
            there exists a orthonormal basis of \( V \) such that 
            the matrix representation of \( \mathcal{A} \) under this basis is a symmetric matrix.
        \item If \( \mathcal{A} \) is a symmetric transformation on a real inner product space \( V \),
            then if \(W\) is an invariant subspace of \( V \) under \( \mathcal{A} \),
            then its orthogonal complement \( W^{\perp} \) is also an invariant subspace of \( V \) under \( \mathcal{A} \).
        \item If \(A\) is a symmetric transformation on a Euclidean space \( V \),
            then eigenvectors corresponding to distinct eigenvalues are definitely orthogonal.
    \end{enumerate}
\end{lemma}

\begin{theorem}{Orthogonal Diagonalization of Symmetric Transformations}
    Let \( V \) be a \( n \)-dimensional Euclidean space, 
    and let \( \mathcal{A}: V \to V \) be a symmetric transformation on \( V \).
    Then there exists a orthonormal basis of \( V \) such that 
    the matrix representation of \( \mathcal{A} \) under this basis is a diagonal matrix.
    
    From the perspective of matrices,
    let \( A \in \mathbb{R}^{n \times n} \) be a real symmetric matrix.
    Then there exists an orthogonal matrix \( T \) such that:
    \[
    T^{-1} A T = T^{\mathrm{T}} A T = \Lambda = 
        \begin{pmatrix}
        \lambda_1 & 0 & \cdots & 0 \\
        0 & \lambda_2 & \cdots & 0 \\
        \vdots & \vdots & \ddots & \vdots \\
        0 & 0 & \cdots & \lambda_n
        \end{pmatrix},
    \]
    where \( \lambda_1, \lambda_2, \dots, \lambda_n \) are the eigenvalues of \( A \).
\end{theorem}

\begin{note}
    According to the theorem above,
    for all real quadratic forms \( f(x_1, x_2, \dots, x_n) = X^{\mathrm{T}} A X \),
    there exists an orthogonal transformation \( X = T Y \) such that:
    \[
    f(x_1, x_2, \dots, x_n) = Y^{\mathrm{T}} \Lambda Y = \sum_{i=1}^{n} \lambda_i y_i^2.
    \]
\end{note}

\section{Unitary Spaces and Normal Transformations}
\begin{leftbarTitle}{Unitary Spaces}\end{leftbarTitle}
\begin{definition}{Complex Inner Product Space}
    A \textbf{complex inner product space} is a complex linear space \( V \) 
    equipped with a function \( (\cdot, \cdot): V \times V \to \mathbb{C} \) 
    satisfying the following properties:
    \begin{description}
        \item [Positivity] \( (\alpha, \alpha) \geq 0, \quad \forall \alpha \in V\),
            and \( (\alpha, \alpha) = 0 \) if and only if \( \alpha = 0 \);
        \item [Conjugate Symmetry (Hermitian)] \( (\alpha, \beta) = \overline{(\beta, \alpha)}, \quad \forall \alpha, \beta \in V\);
        \item [Linearity in the First Argument] 
            \( (k_1\alpha_1 + k_2\alpha_2, \beta) = k_1(\alpha_1, \beta) + k_2(\alpha_2, \beta), 
            \quad \forall k_1, k_2 \in \mathbb{C}, 
            \alpha_1, \alpha_2, \beta \in V. \)
    \end{description}
    The function \( (\cdot, \cdot) \) is called the (complex) \textbf{inner product} on \( V \).

    Complex inner product spaces with finite dimensions are called \textbf{unitary spaces}.
\end{definition}
The norm in unitary spaces can be also induced by the inner product:
\[
\| \alpha \| = (\alpha, \alpha)^{\frac{1}{2}}, \quad \forall \alpha \in V.
\]




\begin{leftbarTitle}{Normal Transformation and Diagonalization}\end{leftbarTitle}
Define the conjugate transpose (Hermitian adjoint) of a complex matrix \( A \in \mathbb{C}^{m \times n} \) as:
\[
A^{\dagger} = \overline{A}^{\mathrm{T}}.
\]
Similarly, for a linear transformation \( \mathcal{A} \in \operatorname{Hom}(V) \) on a complex inner product space \( V \),
the \textbf{adjoint transformation} \( \mathcal{A}^{\dagger} \) of \( \mathcal{A} \) is defined as: % 伴随变换
\[
(\mathcal{A}\alpha, \beta) = (\alpha, \mathcal{A}^{\dagger}\beta), \quad \forall \alpha, \beta \in V.
\]

\begin{definition}{Normal Transformation and Normal Matrix}
    Let \( V \) be a complex inner product space, 
    and let \( \mathcal{A}\in \operatorname{Hom}(V) \).
    If
    \[
    \mathcal{A} \mathcal{A}^{\dagger} = \mathcal{A}^{\dagger} \mathcal{A},
    \]
    then \( \mathcal{A} \) is called a \textbf{normal transformation} on \( V \).

    A square matrix \( A \in \mathbb{C}^{n \times n} \) is called a \textbf{normal matrix} if:
    \[
    A A^{\dagger} = A^{\dagger} A.
    \]
\end{definition}


\begin{theorem}
    Let \(\mathcal{A}\) be a normal transformation on a unitary space \( V \).
    Then there exists an orthonormal basis of \( V \) such that
    the matrix representation of \( \mathcal{A} \) under this basis is a diagonal matrix.

    From the perspective of matrices,
    let \( A \in \mathbb{C}^{n \times n} \) be a normal matrix.
    Then there exists a unitary matrix \( U \) such that:
    \[
    U^{-1} A U = U^{\dagger} A U = \Lambda = 
        \begin{pmatrix}
        \lambda_1 & 0 & \cdots & 0 \\
        0 & \lambda_2 & \cdots & 0 \\
        \vdots & \vdots & \ddots & \vdots \\
        0 & 0 & \cdots & \lambda_n
        \end{pmatrix},
    \]
    where \( \lambda_1, \lambda_2, \dots, \lambda_n \) are the eigenvalues of \( A \).
\end{theorem}

\begin{leftbarTitle}{Special Normal Transformation}\end{leftbarTitle}
\begin{definition}{Unitary Transformation and Unitary Matrix}
    Let \( V \) be a complex inner product space, 
    and let \( \mathcal{A}\in \operatorname{Hom}(V) \).
    If
    \[
    \mathcal{A}^{\dagger} = \mathcal{A}^{-1}, \quad \text{or} \quad
    (\mathcal{A}\alpha, \mathcal{A}\beta) = (\alpha, \beta), \quad \forall \alpha, \beta \in V,
    \]
    then \( \mathcal{A} \) is called a \textbf{unitary transformation} on \( V \).

    A square matrix \( A \in \mathbb{C}^{n \times n} \) is called a \textbf{unitary matrix} if:
    \[
    A^{\dagger} = A^{-1},\quad \text{or} \quad A A^{\dagger} = A^{\dagger} A = E.
    \]
\end{definition}
Obviously, unitary transformations are generalizations of orthogonal transformations in real inner product spaces.


\begin{definition}{Hermitian Transformation and Hermitian Matrix}
    Let \( V \) be a complex inner product space, 
    and let \( \mathcal{A}\in \operatorname{Hom}(V) \).
    If
    \[
    (\mathcal{A}\alpha, \beta) = (\alpha, \mathcal{A}\beta), \quad \forall \alpha, \beta \in V,
    \]
    then \( \mathcal{A} \) is called a \textbf{Hermitian transformation} on \( V \).

    A square matrix \( A \in \mathbb{C}^{n \times n} \) is called a \textbf{Hermitian matrix} if:
    \[
    A^{\dagger} = A.
    \]
\end{definition}
Obviously, Hermitian transformations are generalizations of symmetric transformations in real inner product spaces.







\section{Symplectic Spaces} % 组合设计
\chapter{Pólya Counting} % Pólya 计数
 % Pólya 计数




\begin{thebibliography}{99} 
\bibitem{en1} Elias M. Stein, Rami Shakarchi. \emph{Fourier Analysis: An Introduction}. Princeton University Press, 2016.
\bibitem{en2} Author2, Title2, Journal2, Year2. \emph{ This is another example of a reference.}
\end{thebibliography}

\end{document}