\documentclass[11pt]{elegantbook}

\title{BookName} % 这里放置书名
\subtitle{Subtitle} % 这里放置副标题

\author{CatMono} % 这里放置作者名
\date{August, 2025} % 这里放置日期
\version{0.1} % 这里放置版本号
\institute{Elegant\LaTeX{} Program} % 这里放置机构名
\bioinfo{Custom Key}{Custom Value} % 这里放置自定义信息

\extrainfo{extra information} % 这里放置额外信息,将显示在最下方中央

\setcounter{tocdepth}{3} % 设置目录深度
\setcounter{secnumdepth}{3} % 设置章节编号深度


% \logo{} % 这里放置封面logo,默认从figure目录下寻找
% \cover{} % 这里放置封面图片,默认从figure目录下寻找

% modify the color in the middle of titlepage
\definecolor{customcolor}{RGB}{32,178,170} % 自定义颜色
\colorlet{coverlinecolor}{customcolor}
\usepackage{cprotect} % 保护命令参数不被 LaTeX 解析器过早处理,允许在某些特殊环境中使用脆弱命令(fragile commands)。
\usepackage{xeCJK} % 使用 xeCJK 包支持中文


% ===== 开始文档 =====
\begin{document}

\maketitle %生成文档的标题页,根据之前定义的标题信息(如标题、作者、日期等)自动创建一个格式化的标题页

% === 前言部分 ===
\frontmatter        % 开始前言,页码为 i, ii, iii...
\tableofcontents    % 目录 (页码: i, ii)
% \listoffigures      % 图表目录 (页码: iii)
% \listoftables       % 表格目录 (页码: iv)

\chapter{Preface}   % 前言章节(无编号,页码: v, vi...)
This is the preface of the book...

% \chapter{Acknowledgments}  % 致谢(无编号)
% I would like to thank...
% === 正文部分 ===
\mainmatter         % 开始正文,页码从 1 重新开始

\chapter{The Propositional Calculus} % 这里放置章节标题
\section{Connectives and Truth Tables} % 这里放置小节标题
A proposition is a statement whose truth value can be determined 
(represented by 1 for true and 0 for false\footnote{
The set of logical truth values can be represented in various ways,
such as \(\{T, F\}\), \(\{\top, \bot\}\), or \(\{\text{True}, \text{False}\}\).
\(\{ 0,1 \}\) is applied in this book for simplicity.
}), 
and a proposition is either true or false (within the framework of classical binary logic). 
We use lowercase letters such as \(p,q,r\cdots\), to denote propositions.

There are five commonly used propositional connectives:
\begin{leftbarTitle}{Negation}\end{leftbarTitle}
If \(p\) is a proposition, then the \textbf{negation} of \(p\) is denoted by \(\neg p\), 
which is true if and only if \(p\) is false.
Its truth table is as follows:
\begin{center}
\begin{tabular}{cc}
\toprule
$p$ & $\neg p$ \\ % 表头横线
\midrule
0 & 1  \\ % 第一行
1 & 0  \\ % 第二行
\bottomrule
\end{tabular}
\end{center}

\begin{leftbarTitle}{Conjunction}\end{leftbarTitle}
The \textbf{conjunction} of two propositions \(p\) and \(q\) is denoted by \(p \land q\),
which is true if and only if both \(p\) and \(q\) are true.
Its truth table is as follows:
\begin{center}
\begin{tabular}{ccc}
\toprule
$p$ & $q$ & $p \land q$ \\ % 表头横线
\midrule
0 & 0 & 0 \\ % 第一行
0 & 1 & 0 \\ % 第二行
1 & 0 & 0 \\ % 第三行
1 & 1 & 1 \\ % 第四行
\bottomrule
\end{tabular}
\end{center}

\begin{leftbarTitle}{Disjunction}\end{leftbarTitle}
The \textbf{disjunction} of two propositions \(p\) and \(q\) is denoted by \(p \lor q\),
which is true if and only if at least one of \(p\) or \(q\) is true.
Its truth table is as follows:
\begin{center}
\begin{tabular}{ccc}
\toprule
$p$ & $q$ & $p \lor q$ \\ % 表头横线
\midrule
0 & 0 & 0 \\ % 第一行
0 & 1 & 1 \\ % 第二行
1 & 0 & 1 \\ % 第三行
1 & 1 & 1 \\ % 第四行
\bottomrule
\end{tabular}
\end{center}

\begin{leftbarTitle}{Implication}\end{leftbarTitle}
Proposition \(p\) \textbf{implies} \(q\), namely "if \(p\), then \(q\)", is denoted by \(p \to q\),
which is false if and only if \(p\) is true and \(q\) is false.
Its truth table is as follows:
\begin{center}
\begin{tabular}{ccc}
\toprule
$p$ & $q$ & $p \to q$ \\ % 表头横线
\midrule
0 & 0 & 1 \\ % 第一行
0 & 1 & 1 \\ % 第二行
1 & 0 & 0 \\ % 第三行
1 & 1 & 1 \\ % 第四行
\bottomrule
\end{tabular}
\end{center}

\begin{leftbarTitle}{Biconditional}\end{leftbarTitle}
The \textbf{biconditional} of two propositions \(p\) and \(q\) is denoted by \(p \leftrightarrow q\),
which is true if and only if \(p\) and \(q\) have the same truth value.
Its truth table is as follows:
\begin{center}
\begin{tabular}{ccc}
\toprule
$p$ & $q$ & $p \leftrightarrow q$ \\ % 表头横线
\midrule
0 & 0 & 1 \\ % 第一行
0 & 1 & 0 \\ % 第二行
1 & 0 & 0 \\ % 第三行
1 & 1 & 1 \\ % 第四行
\bottomrule
\end{tabular}
\end{center}

\section{Axiom System}
\subsection{Propositional Calculus}
Propositional calculus (also called \textbf{Zero-Order Logic}) is a formal system \(\mathcal{L}\left( A, \Omega, Z, I \right) \),
whose formulas are constructed as follows:
\begin{description}
    \item[\({A}\)]  The infinite set consisting of propositional variables.
    \item[\({\Omega}\)]  The infinite set consisting of logical connectives\footnote{
        It is divided into the following mutually disjoint subsets:
        \[
        \Omega = \Omega_{0} \cup \Omega_{1} \cup \Omega_{2} \cup \cdots \cup \Omega_{m},
        \]
        where \(\Omega_{j}\) is the set of \(j\)-ary logical connectives(operators).
        In general, \(\Omega_{0}=\{ 0,1 \}\), \(\Omega_{1}=\{ \neg \}\), 
        \(\Omega_{2}=\{ \land, \lor, \to, \leftrightarrow \}\).
    }.
    \item[\({Z}\)]  The infinite set consisting of inference rules.
    \item[\({I}\)]  The infinite set consisting of axioms (start point).
\end{description}
Parentheses, namely "(" and ")", are commonly used as auxiliaries to facilitate the construction of formulas.

\begin{definition}{Well-formed Formula (WFF)}
    The language of \(\mathcal{L}\), denoted by \(L(A)\), is also called the set of well-formed formulas
    A \textbf{well-formed formula} (short for WFF or formula) is a finite sequence of symbols from \({A}\) and \({\Omega}\) 
    that is constructed recursively according to the following rules:
    \begin{enumerate}
        \item Any element of \({A}\) is a formula.
        \item If \(p_{1},p_{2},\cdots p_{j}\) are formulas, \(f\in \Omega_{j}\), then \((fp_{1}p_{2}\cdots p_{j})\) is also a formula.
        \item Every formula is generated by a finite number of applications of Rules 1 and 2.
    \end{enumerate}
\end{definition}
\(L(A)\) can be layered as follows:
TODO;


Although axiomatic proof has been used since the famous Ancient Greek textbook, Euclid's \textit{Elements of Geometry}, 
in propositional logic it dates back to Gottlob Frege's 1879 \textit{Begriffsschrift}.
Frege's system used only implication and negation as connectives.
It had six axioms (\(p,q,r\in A\)):

\begin{tabular}{@{}l l@{}}
$p \to (q \to p)$ & \quad (Law of Affirming the Consequent) \\
$(p \to (q \to r)) \to ((p \to q) \to (p \to r))$ & \quad (Law of Distribution of Implication) \\
\((p\to (q\to r))\to (q\to (p\to r))\) & \quad (Law of Permutation) \\
$(p \to q) \to (\neg q \to \neg p)$ & \quad (Law of Transposition) \\
\(\neg \neg p \to p\) & \quad (Law of Double Negation Elimination) \\
\(p \to \neg \neg p\) & \quad (Law of Double Negation Introduction) \\
\end{tabular}

These were used by Frege together with modus ponens and a rule of substitution (which was used but never precisely stated) 
to yield a complete and consistent axiomatization of classical truth-functional propositional logic.

Jan \L{}ukasiewicz showed that, in Frege's system, 
"the third axiom is superfluous since it can be derived from the preceding two axioms, 
and that the last three axioms can be replaced by the single sentence 
\(C\)\(C\)\(N\)\(p\)\(N\)\(q\)\(\)q\(p\) or \(\to \to \neg p \neg q \to qp\)
with his Polish notation\footnote{
    Polish notation, also known as prefix notation, 
    is a way of writing mathematical expressions in which operators precede their operands. 
    It allows expressions to be written without parentheses to indicate order of operations 
    and is commonly used in logic and computer science.
    For example, the expression \((5-6)\times 7\) in Polish notation is written as \(\times - 5 6 7\).
},
namely \((\neg p \to \neg q) \to (q \to p)\).
It is a simplified version of Frege's system,
which is also known as the \L{}ukasiewicz axiom system. 


\begin{definition}{Jan \L{}ukasiewicz Axiom System}
    \textbf{\L{}ukasiewicz axiom system} is defined as \(\mathcal{L} = \left( A, \Omega, Z, I \right)\),
    where:
    \begin{itemize}
        \item \({A}\) contains sufficiently many propositional variables for discussion.
        \item \({\Omega}=\Omega_{1}\cup \Omega_{2}\) is complete,
        where \(\Omega_{1}=\{ \neg \}\) and \(\Omega_{2}=\{ \to \}\).
        \item \({Z}\) contains a single inference rule: Modus Ponens (MP),
        which states that if \(p\) and \(p \to q\) are both formulas, then \(q\) is also a formula.
        \item \({I}\) contains 3 axioms (\(p,q,r\in A\)):
        
            \begin{tabular}{@{}l l l@{}}
            (L1) & $p \to (q \to p)$ & \quad (Law of Affirming the Consequent) \\
            (L2) & $(p \to (q \to r)) \to ((p \to q) \to (p \to r))$ & \quad (Law of Distribution of Implication) \\
            (L3) & $(\neg p \to \neg q) \to (q \to p)$ & \quad (Law of Contraposition) \\
            \end{tabular}
    \end{itemize}
\end{definition}

\subsection{Propositional Algebra}


\subsection{Proof}

\begin{definition}{Proof}
    Let \(\Gamma \subseteq L(X)\), \(p \in L(X)\). 
    When we say that "the formula \(p\) is provable from the set of formulas \(\Gamma\)", 
    we mean that there exists a finite sequence of formulas \(p_1, \dots, p_n\) from \(L(X)\), 
    with the last term \(p_n = p\), such that each \(p_k\) \((k = 1, \dots, n)\) satisfies one of the following:
    \begin{itemize}
        \item \(p_k\) is an element of \(\Gamma\).
        \item \(p_k\) is an axiom of the system.
        \item There exist \(i, j < k\) such that \(p_j = p_i \to p_k\) (MP).
    \end{itemize}
    Any finite sequence \(p_1, \dots, p_n\) possessing the above properties 
    is called a "proof of \(p\) from \(\Gamma\)".
\end{definition}

If a formula \(p\) is provable from a set of formulas \(\Gamma\), 
we write \(\Gamma \vdash p\), or equivalently \(\Gamma \vdash_{\mathcal{L}} p\). 
In this case, the formulas in \(\Gamma\) are called "assumptions," 
and \(p\) is called a syntactic consequence of \(\Gamma\).

If \(\emptyset\vdash p\), then \(p\) is called a "theorem" (interior theorem) of \(\mathcal{L}\), 
denoted \(\vdash p\). 
A proof of \(p\) from \(\emptyset\) in \(\mathcal{L}\) is a sequence \(p_1, \dots, p_n\), 
abbreviated as a proof of \(p\) in \(\mathcal{L}\).

In a proof, when \(p_{j} = p_i \to p_k\) \((i, j < k)\), 
we say that \(p_k\) is obtained from \(p_i\) and \(p_i \to p_k\) 
by using the rule of MP.

\begin{definition}{Consistent Set of Formulas}
    If, for any formula \(q\), neither \(\Gamma \vdash q\) nor \(\Gamma \vdash \neg q\) hold simultaneously, 
    then the set of formulas \(\Gamma\) is called a consistent set of formulas; 
    otherwise, \(\Gamma\) is called an inconsistent set of formulas.
    
\end{definition}

\begin{proposition}
    If \(\Gamma\) is a consistent set of formulas, then for any formula \(p\) in \(L(A)\), 
    we have \(\Gamma \nvdash \neg p\).
\end{proposition}

\begin{proposition}
    \begin{tabular}{@{}l l l@{}}
    1) & \(\vdash p\to p\) & \quad (Law of Identity) \\
    2) & \(\vdash \neg q \to (q\to p)\) & \quad (Law of Denying the Antecedent) \\
    3) & {\footnotesize\(\left( \left( x_{1}\to (x_{2}\to x_{3}) \right)\to (x_{1}\to x_{2})  \right)
    \to \left( \left( x_{1}\to (x_{2}\to x_{3}) \right)\to (x_{1}\to x_{3})  \right)  \) }
    & \quad (12312\(\to\)12313) \\
    \end{tabular}
\end{proposition}

\begin{proof}
    
\end{proof}

\subsection{Deduction Theorems}
\begin{theorem}{Deduction Theorem}
    \[
    \Gamma \cup \{ p \} \vdash q \iff \Gamma \vdash p \to q
    \]    
\end{theorem}

\begin{proof}
    
\end{proof}

\begin{corollary}{Hypothetical Syllogism (HS)}
    \[
    \{ p\to q,  q\to r \} \vdash p \to r
    \]
\end{corollary}

\begin{proposition}
    \begin{tabular}{@{}l l l@{}}
    1) & \(\vdash (p \to q) \to (\neg q \to \neg p)\) & \quad (Law of Transition) \\
    2) & \(\vdash p\to \neg\neg p\) & \quad (Law of Double Negation Introduction) \\
    3) & \(\vdash ((p\to q)\to p)\to p\)    & \quad (Peirce's Law) \\
    4) & \(\vdash \neg(p\to q)\to (q\to p)\) & \quad (Overall Negation Reversal) \\
    \end{tabular}
\end{proposition}

\subsection{Law of Contradiction and Absurdum}
\begin{theorem}{Law of Contradiction}
    \[
    \left.
    \begin{array}{l}
    \Gamma \cup \{\neg p\} \vdash q \\
    \Gamma \cup \{\neg p\} \vdash \neg q
    \end{array}
    \right\}
    \implies \Gamma \vdash p.
    \]
\end{theorem}

\begin{theorem}{Law of Abusrdum}
    \[
    \left.
    \begin{array}{l}
    \Gamma \cup \{p\} \vdash q \\
    \Gamma \cup \{p\} \vdash \neg q
    \end{array}
    \right\}
    \implies \Gamma \vdash \neg p.
    \]
\end{theorem}

\begin{corollary}{Law of Double Negation Introduction and Elimination}
    \begin{enumerate}
        \item \(\vdash p \to \neg\neg p\) 
        \item \(\vdash \neg\neg p \to p\) 
    \end{enumerate}
\end{corollary}

\begin{proposition}
    \begin{tabular}{@{}l l l@{}}
    1) & \(\vdash (p \to \neg q) \to (q \to \neg p)\) & \quad (Negation Reversal) \\
    2) & \(\vdash (\neg p\to q)\to (\neg q\to p)\) & \quad (Negation Reversal) \\
    3) & \(\vdash \neg(p\to q)\to \neg q\) & \quad (Overall Negation) \\
    4) & \(\vdash \neg(p\to q)\to p\) & \quad (Overall Negation) \\
    \end{tabular}
\end{proposition}

\subsection{Other Connectives}
In \({\mathcal{L}}\), we can also define three other binary operators:
\begin{align*}
&p \lor q = \neg p \to q, \\
&p \land q = \neg (p \to \neg q), \\
&p \leftrightarrow q = (p \to q) \land (q \to p).
\end{align*}

\section{Semantics of Propositional Calculus}


\chapter{First-Order Logic}

\chapter{Second-Order Logic}


\begin{thebibliography}{99} 
\bibitem{en1} 作者, Title1, Journal1, Year1. \emph{ This is an example of a reference.}
\bibitem{en2} Author2, Title2, Journal2, Year2. \emph{ This is another example of a reference.}
\end{thebibliography}

\end{document}