\documentclass[11pt]{elegantbook}

\title{BookName} % 这里放置书名
\subtitle{Subtitle} % 这里放置副标题

\author{AuthorName} % 这里放置作者名
\date{July, 2025} % 这里放置日期
\version{0.1} % 这里放置版本号
\institute{Elegant\LaTeX{} Program} % 这里放置机构名
\bioinfo{Custom Key}{Custom Value} % 这里放置自定义信息

\extrainfo{extra information} % 这里放置额外信息,将显示在最下方中央

\setcounter{tocdepth}{3} % 设置目录深度
\setcounter{secnumdepth}{3} % 设置章节编号深度


\logo{logo-blue.png} % 这里放置封面logo,默认从figure目录下寻找
\cover{LogiqueMathematique.png} % 这里放置封面图片,默认从figure目录下寻找

% modify the color in the middle of titlepage
\definecolor{customcolor}{RGB}{32,178,170} % 自定义颜色
\colorlet{coverlinecolor}{customcolor}
\usepackage{cprotect} % 保护命令参数不被 LaTeX 解析器过早处理,允许在某些特殊环境中使用脆弱命令(fragile commands)。
\usepackage{xeCJK} % 使用 xeCJK 包支持中文


% ===== 开始文档 =====
\begin{document}

\maketitle %生成文档的标题页,根据之前定义的标题信息(如标题、作者、日期等)自动创建一个格式化的标题页

% === 前言部分 ===
\frontmatter        % 开始前言,页码为 i, ii, iii...
\tableofcontents    % 目录 (页码: i, ii)
% \listoffigures      % 图表目录 (页码: iii)
% \listoftables       % 表格目录 (页码: iv)

\chapter{Preface}   % 前言章节(无编号,页码: v, vi...)
This is the preface of the book...

\chapter{Acknowledgments}  % 致谢(无编号)
I would like to thank...
% === 正文部分 ===
\mainmatter         % 开始正文,页码从 1 重新开始

\chapter{Chapter Title} % 这里放置章节标题
\section{Section Title} % 这里放置小节标题
\subsection{Subsection Title} % 这里放置子小节标题


\begin{thebibliography}{99} 
\bibitem{en1} 作者, Title1, Journal1, Year1. \emph{ This is an example of a reference.}
\bibitem{en2} Author2, Title2, Journal2, Year2. \emph{ This is another example of a reference.}
\end{thebibliography}

\end{document}