\documentclass[11pt]{elegantbook}

\title{BookName} % 这里放置书名
\subtitle{Subtitle} % 这里放置副标题

\author{CatMono} % 这里放置作者名
\date{August, 2025} % 这里放置日期
\version{0.1} % 这里放置版本号
\institute{Elegant\LaTeX{} Program} % 这里放置机构名
\bioinfo{Custom Key}{Custom Value} % 这里放置自定义信息

\extrainfo{extra information} % 这里放置额外信息,将显示在最下方中央

\setcounter{tocdepth}{3} % 设置目录深度
\setcounter{secnumdepth}{3} % 设置章节编号深度


% \logo{} % 这里放置封面logo,默认从figure目录下寻找
% \cover{} % 这里放置封面图片,默认从figure目录下寻找

% modify the color in the middle of titlepage
\definecolor{customcolor}{RGB}{32,178,170} % 自定义颜色
\colorlet{coverlinecolor}{customcolor}
\usepackage{cprotect} % 保护命令参数不被 LaTeX 解析器过早处理,允许在某些特殊环境中使用脆弱命令(fragile commands)。
\usepackage{xeCJK} % 使用 xeCJK 包支持中文


% ===== 开始文档 =====
\begin{document}

\maketitle %生成文档的标题页,根据之前定义的标题信息(如标题、作者、日期等)自动创建一个格式化的标题页

% === 前言部分 ===
\frontmatter        % 开始前言,页码为 i, ii, iii...
\tableofcontents    % 目录 (页码: i, ii)
% \listoffigures      % 图表目录 (页码: iii)
% \listoftables       % 表格目录 (页码: iv)

\chapter{Preface}   % 前言章节(无编号,页码: v, vi...)
This is the preface of the book...

% \chapter{Acknowledgments}  % 致谢(无编号)
% I would like to thank...
% === 正文部分 ===
\mainmatter         % 开始正文,页码从 1 重新开始

\chapter{The Propositional Calculus} % 这里放置章节标题
\section{Propositional Connectives and Truth Tables} % 这里放置小节标题
A proposition is a statement whose truth value can be determined 
(represented by 1 for true and 0 for false\footnote{
The set of logical truth values can be represented in various ways,
such as \(\{T, F\}\), \(\{\top, \bot\}\), or \(\{\text{True}, \text{False}\}\).
\(\{ 0,1 \}\) is applied in this book for simplicity.
}), 
and a proposition is either true or false (within the framework of classical binary logic). 
We use lowercase letters such as \(p,q,r\cdots\), to denote propositions.

There are five commonly used propositional connectives:
\begin{leftbarTitle}{Negation}\end{leftbarTitle}
If \(p\) is a proposition, then the \textbf{negation} of \(p\) is denoted by \(\neg p\), 
which is true if and only if \(p\) is false.
Its truth table is as follows:
\begin{center}
\begin{tabular}{cc}
\toprule
$p$ & $\neg p$ \\ % 表头横线
\midrule
0 & 1  \\ % 第一行
1 & 0  \\ % 第二行
\bottomrule
\end{tabular}
\end{center}

\begin{leftbarTitle}{Conjunction}\end{leftbarTitle}
The \textbf{conjunction} of two propositions \(p\) and \(q\) is denoted by \(p \land q\),
which is true if and only if both \(p\) and \(q\) are true.
Its truth table is as follows:
\begin{center}
\begin{tabular}{ccc}
\toprule
$p$ & $q$ & $p \land q$ \\ % 表头横线
\midrule
0 & 0 & 0 \\ % 第一行
0 & 1 & 0 \\ % 第二行
1 & 0 & 0 \\ % 第三行
1 & 1 & 1 \\ % 第四行
\bottomrule
\end{tabular}
\end{center}

\begin{leftbarTitle}{Disjunction}\end{leftbarTitle}
The \textbf{disjunction} of two propositions \(p\) and \(q\) is denoted by \(p \lor q\),
which is true if and only if at least one of \(p\) or \(q\) is true.
Its truth table is as follows:
\begin{center}
\begin{tabular}{ccc}
\toprule
$p$ & $q$ & $p \lor q$ \\ % 表头横线
\midrule
0 & 0 & 0 \\ % 第一行
0 & 1 & 1 \\ % 第二行
1 & 0 & 1 \\ % 第三行
1 & 1 & 1 \\ % 第四行
\bottomrule
\end{tabular}
\end{center}

\begin{leftbarTitle}{Implication}\end{leftbarTitle}
Proposition \(p\) \textbf{implies} \(q\), namely "if \(p\), then \(q\)", is denoted by \(p \to q\),
which is false if and only if \(p\) is true and \(q\) is false.
Its truth table is as follows:
\begin{center}
\begin{tabular}{ccc}
\toprule
$p$ & $q$ & $p \to q$ \\ % 表头横线
\midrule
0 & 0 & 1 \\ % 第一行
0 & 1 & 1 \\ % 第二行
1 & 0 & 0 \\ % 第三行
1 & 1 & 1 \\ % 第四行
\bottomrule
\end{tabular}
\end{center}

\begin{leftbarTitle}{Biconditional}\end{leftbarTitle}
The \textbf{biconditional} of two propositions \(p\) and \(q\) is denoted by \(p \leftrightarrow q\),
which is true if and only if \(p\) and \(q\) have the same truth value.
Its truth table is as follows:
\begin{center}
\begin{tabular}{ccc}
\toprule
$p$ & $q$ & $p \leftrightarrow q$ \\ % 表头横线
\midrule
0 & 0 & 1 \\ % 第一行
0 & 1 & 0 \\ % 第二行
1 & 0 & 0 \\ % 第三行
1 & 1 & 1 \\ % 第四行
\bottomrule
\end{tabular}
\end{center}

\section{Propositional Calculus}
Propositional calculus is a formal system \(\mathcal{L} = \left( \mathrm{A}, \mathrm{\Omega}, \mathrm{Z}, \mathrm{I} \right) \),
whose formulas are constructed as follows:
\begin{description}
    \item[\({\mathrm{A}}\)]  The infinite set consisting of propositional variables or constants.
    \item[\({\mathrm{\Omega}}\)]  The infinite set consisting of logical connectives\footnote{
        It is divided into the following mutually disjoint subsets:
        \[
        \Omega = \Omega_{0} \cup \Omega_{1} \cup \Omega_{2} \cup \cdots \cup \Omega_{m},
        \]
        where \(\Omega_{j}\) is the set of \(j\)-ary logical connectives(operators).
        In general, \(\Omega_{0}=\{ 0,1 \}\), \(\Omega_{1}=\{ \neg \}\), 
        \(\Omega_{2}=\{ \land, \lor, \to, \leftrightarrow \}\).
    }.
    \item[\({\mathrm{Z}}\)]  The infinite set consisting of inference rules.
    \item[\({\mathrm{I}}\)]  The infinite set consisting of axioms(start point).
\end{description}

\chapter{First-Order Logic}

\chapter{Second-Order Logic}


\begin{thebibliography}{99} 
\bibitem{en1} 作者, Title1, Journal1, Year1. \emph{ This is an example of a reference.}
\bibitem{en2} Author2, Title2, Journal2, Year2. \emph{ This is another example of a reference.}
\end{thebibliography}

\end{document}