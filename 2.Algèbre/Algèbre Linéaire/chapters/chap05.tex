\chapter{Linear Spaces} % 线性空间
\section{Linear Spaces and Their Bases}

\begin{leftbarTitle}{Basis Transformation and Coordinate Transformation}\end{leftbarTitle}


\section{Subspaces}
\begin{leftbarTitle}{Intersection and Sum of Subspaces}\end{leftbarTitle}
\begin{definition}{Intersection and Sum of Subspaces}
    Let \( V \) be a linear space over field \( F \), 
    and let \( V_1, V_2 \subseteq V \) be two subspaces of \( V \).
    The \textbf{intersection} of \( V_1 \) and \( V_2 \) is defined as:
    \[
    V_1 \cap V_2 = \{ \alpha | \alpha \in V_1 \text{ and } \alpha \in V_2 \}.
    \]
    The \textbf{sum} of \( V_1 \) and \( V_2 \) is defined as:
    \[
    V_1 + V_2 = \{ \alpha | \alpha = \alpha_1 + \alpha_2, 
    \alpha_1 \in V_1, \alpha_2 \in V_2 \}.
    \]
\end{definition}

If \( V_1 \) and \( V_2 \) are two subspaces of \( V \), 
then both their intersection \( V_1 \cap V_2 \) and their sum \( V_1 + V_2 \) are also subspaces of \( V \).
Simultaneously, \( V_{1}+V_{2} \) is the smallest subspace of \( V \) 
that contains both \( V_1 \) and \( V_2 \) (\(V_{1}\cup V_{2}\)).

\begin{leftbarTitle}{Dimension Formula}\end{leftbarTitle}
\begin{proposition}
    In a finite-dimensional linear space \( V \),
    \[
    \langle \alpha_1, \alpha_2, \dots, \alpha_s \rangle + \langle \beta_1, \beta_2, \dots, \beta_t \rangle = 
    \langle \alpha_1, \alpha_2, \dots, \alpha_s, \beta_1, \beta_2, \dots, \beta_t \rangle.
    \]
\end{proposition}

\begin{theorem}{Dimension Formula}\label{thm:dimension_formula}
    Let \( V \) be a finite-dimensional linear space over field \( F \), 
    and let \( V_1, V_2 \subseteq V \) be two subspaces of \( V \). Then:
    \[
    \dim(V_1) + \dim(V_2) = \dim(V_1 + V_2) + \dim(V_1 \cap V_2).
    \]
    
\end{theorem}

\begin{leftbarTitle}{Direct Sum of Subspaces}\end{leftbarTitle}
\begin{definition}{Direct Sum of Subspaces}
    Let \( V \) be a linear space over field \( F \), 
    and let \( V_1, V_2 \subseteq V \) be two subspaces of \( V \).
    If any vector \( \alpha \in V_1 + V_2 \) can be uniquely expressed as:
    \[
    \alpha = \alpha_1 + \alpha_2, \quad \alpha_1 \in V_1, \alpha_2 \in V_2,
    \]
    then \( V_1 + V_2 \) is called the \textbf{direct sum} of \( V_1 \) and \( V_2 \),
    denoted as \( V_1 \oplus V_2 \).
\end{definition}

\begin{proposition}
    The necessary and sufficient condition for \( V_1 + V_2 \) to be a direct sum is:
    \begin{enumerate}
        \item \(0 = 0 + 0\) (the zero vector can only be expressed as the sum of two zero vectors); 
        \item \( V_1 \cap V_2 = \{0\} \) (the intersection of the two subspaces is only the zero vector);
        \item \(\operatorname{dim}(V_{1}+V_{2})=\operatorname{dim}(V_{1})+\operatorname{dim}(V_{2})\),
            or equivalently, \(\operatorname{dim}(V_{1} \cap V_{2})=0\).
        \item Any bases of \( V_1 \) and \( V_2 \) together form a basis of \( V_1 + V_2 \).
    \end{enumerate}
\end{proposition}




\section{Isomorphisms}

\section{Quotient Spaces}

\section{Special Linear Spaces}
\begin{leftbarTitle}{Commutative Spaces}\end{leftbarTitle}
Let \( A \) is an \( n \)-order matrix over number field \( P \).
It is obvious that the set of all matrices satisfying \( AX = XA \) is a subspace of \( M_{n \times n}(P) \),
denoted as \( C(A) \).

\begin{property}
    \begin{enumerate}
        \item For any \( n \)-order matrix \( A, B \), 
            if \( A \) is similar to \( B \) (\(A \sim B\)), 
            then \( C(A) \) is isomorphic to \( C(B) \) (\(C(A) \cong C(B)\)),
            further \( \dim(C(A)) = \dim(C(B)) \).
    \end{enumerate}    
\end{property}

\begin{example}
    Find the dimension and a basis of the \( C(A) \) where:
    \begin{enumerate}[label=(\roman*)]
        \item \(A\) is \(3\)-order matrix over number field \(P\), and \(A\) has three distinct eigenvalues.
        \item 
            \[
            A = 
            \begin{pmatrix} 1 & 0 & 4 \\ 0 & 1 & 2 \\ 0 & 1 & 2 \\\end{pmatrix}
            \]
    \end{enumerate}
\end{example}