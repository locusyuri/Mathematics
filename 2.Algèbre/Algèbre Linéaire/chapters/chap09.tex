\chapter{Quadratic Forms} % 二次型
\section{Quadratic Forms and Their Standard Forms}
\begin{definition}{Quadratic Form}
    Let \(P\) be a number field, a quadratic homogeneous polynomial in \( n \) variables over \( P \)\footnote{
        That is, the coefficients of the polynomial belong to the field \( P \).
    }:
    \begin{align*}
        f( x_{1}, x_{2}, \cdots, x_{n}) &= \sum_{i=1}^{n} \sum_{j=1}^{n} a_{ij} x_{i} x_{j} \\
    &= a_{11}x_{1}^{2} + 2a_{12}x_{1}x_{2} + \cdots + 2a_{1n}x_{1}x_{n} + a_{22}x_{2}^{2} + \cdots + 2a_{2n}x_{2}x_{n} + \cdots + a_{nn}x_{n}^{2},
    \end{align*}
    is called a \textbf{quadratic form} in \( n \) variables over field \( P \).

    It can be expressed in matrix form as:
    \[
    f( x_{1}, x_{2}, \cdots, x_{n}) = X^{\mathrm{T}} A X,
    \]
    where 
    \[
    X = \begin{pmatrix}
        x_{1} \\
        x_{2} \\
        \vdots \\
        x_{n}
    \end{pmatrix}, \quad
    A = (a_{ij})_{n \times n}, \quad a_{ij} = a_{ji} \quad (1 \leqslant i, j \leqslant n).
    \]
\end{definition}
It is easy to verify that the matrix \( A \) of a quadratic form is symmetric.
\begin{note}
    In fact, for any square matrix \( B \)\footnote{
        For a skew-symmetric matrix \( S \) (\( S^{T} = -S \)),
        \[
        X^{T} S X = - (X^{T} S X)^{T} = - X^{T} S^{T} X = - X^{T} S X \implies X^{T} S X = 0.
        \]
    }, we have:
    \begin{align*}
        X^{T}BX &= X^{T}\left( \frac{B + B^{T}}{2} \right)X + X^{T}\left( \frac{B - B^{T}}{2} \right)X \\
        &= X^{T}\left( \frac{B + B^{T}}{2} \right)X + 0 \\
        &= X^{T}\left( \frac{B + B^{T}}{2} \right)X.
    \end{align*}
    It shows that any quadratic form can be represented by a symmetric matrix.
\end{note}




\section{Canonical Forms}
\section{Definite Quadratic Forms}
\begin{definition}{Positive Definite Quadratic Form}
    A real quadratic form \( f( x_{1}, x_{2}, \cdots, x_{n})=X^{\mathrm{T}}AX \) is called \textbf{positive definite} if:
    \[
    f( x_{1}, x_{2}, \cdots, x_{n}) > 0, \quad \forall X \neq 0.
    \]
    And \( A \) is called a \textbf{positive definite matrix}.
\end{definition}

\begin{theorem}{Sufficient and Necessary Condition for Positive Definiteness}
    A real quadratic form \( f( x_{1}, x_{2}, \cdots, x_{n})=X^{\mathrm{T}}AX \) is positive definite if and only if:
    \begin{enumerate}
        \item The positive inertia index of \( f \) is \( n \);
        \item \(A\) is congruent to the identity matrix \( E \);
        \item All eigenvalues of \( A \) are positive;
        \item All leading principal minors\footnote{
            The leading principal minors of a matrix are the determinants of 
            the top-left \( k \times k \) submatrices for \( k = 1, 2, \ldots, n \).
        } of \( A \) are positive.
    \end{enumerate}
    
\end{theorem}