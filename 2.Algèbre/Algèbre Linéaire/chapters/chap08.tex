\chapter{Jordan Forms} % Jordan 标准型
\section{Polynomial Matrices}
\begin{definition}{Polynomial Matrix}
    A matrix in the form:
    \[
    A(\lambda) = 
    \begin{pmatrix}
        a_{11}(\lambda) & a_{12}(\lambda) & \cdots & a_{1n}(\lambda) \\
        a_{21}(\lambda) & a_{22}(\lambda) & \cdots & a_{2n}(\lambda) \\
        \vdots & \vdots & \ddots & \vdots \\
        a_{m1}(\lambda) & a_{m2}(\lambda) & \cdots & a_{mn}(\lambda)
    \end{pmatrix},
    \]
    is called a \textbf{polynomial matrix} over \( F \),
    where each \( a_{ij}(\lambda) \in F[\lambda] \) is a polynomial in variable \( \lambda \) over field \( F \).
\end{definition}
Similar to ordinary matrices,
polynomial matrices can also perform addition, multiplication, scalar multiplication, transposition, and taking determinants.
\begin{caution}
    \(A (\lambda)\) is invertible if and only if \( |A(\lambda)| \) is a \emph{non-zero constant},
    it is worth noting that this is non-zero constant, not just any non-zero polynomial.
\end{caution}

\begin{definition}{Equivalent Polynomial Matrices}
    Let \( A(\lambda) \in F[\lambda]^{m \times n} \) and \( B(\lambda) \in F[\lambda]^{m \times n} \) be two polynomial matrices.
    If they can be transformed into each other through a series of elementary row and column operations,
    then they are called \textbf{equivalent polynomial matrices}, denoted as \( A(\lambda) \cong B(\lambda) \).
\end{definition}
\begin{note}
    Obviously, \( A(\lambda) \cong B(\lambda) \) if and only if 
    their ranks are equal: \( \operatorname{rank} A(\lambda) = \operatorname{rank} B(\lambda) \).
\end{note}

\begin{definition}{Smith Normal Form (Canonical Form of Polynomial Matrices)}
    Let \( A(\lambda) \in F[\lambda]^{m \times n} \) be a polynomial matrix with rank \( r \geqslant 1 \).
    If \( A(\lambda) \) is equivalent to the following polynomial matrix:
    \[
    \begin{pmatrix}
        d_1(\lambda) & 0 & \cdots & 0 & 0 & \cdots & 0 \\
        0 & d_2(\lambda) & \cdots & 0 & 0 & \cdots & 0 \\
        \vdots & \vdots & \ddots & \vdots & \vdots & \ddots & \vdots \\
        0 & 0 & \cdots & d_r(\lambda) & 0 & \cdots & 0 \\
        0 & 0 & \cdots & 0 & 0 & \cdots & 0 \\
        \vdots & \vdots & \ddots & \vdots & \vdots & \ddots & \vdots \\
        0 & 0 & \cdots & 0 & 0 & \cdots & 0
    \end{pmatrix},
    \]
    where \( d_1(\lambda), d_2(\lambda), \dots, d_r(\lambda) \) are monic polynomials over field \( F \)
    such that \( d_1(\lambda) | d_2(\lambda) | \cdots | d_r(\lambda) \),
    then this polynomial matrix is called the \textbf{Smith normal form} of \( A(\lambda) \).
\end{definition}
It is easy to verify that the Smith normal form of polynomial matrices exists and 
is unique up to the order of the invariant divisors.


\section{Invariant Divisors and Rational Canonical Form}
\begin{leftbarTitle}{Invariant Divisors}\end{leftbarTitle}
\begin{definition}{\(k\) Order Determinant Divisor}
    Let \( A(\lambda) \in F[\lambda]^{m \times n} \) be a polynomial matrix.
    The monic greatest common divisor of all \( k \)-order minors of \( A(\lambda) \) 
    is called the \textbf{\( k \) order determinant divisor} of \( A(\lambda) \),
    denoted as \( D_k(\lambda) \).
\end{definition}

\begin{definition}{Invariant Divisors}
    Let \( A(\lambda) \in F[\lambda]^{m \times n} \) be a polynomial matrix,
    and let \( D_1(\lambda), D_2(\lambda), \dots, D_r(\lambda) \) be its \( 1, 2, \dots, r \) order determinant divisors,
    where \( r = \operatorname{rank} A(\lambda) \).
    The polynomials
    \[
    d_1(\lambda) = D_1(\lambda), \quad
    d_2(\lambda) = \frac{D_2(\lambda)}{D_1(\lambda)}, \quad
    \dots, \quad
    d_r(\lambda) = \frac{D_r(\lambda)}{D_{r-1}(\lambda)}
    \]
    are called the \textbf{invariant divisors} of \( A(\lambda) \).
\end{definition}
\begin{note}
    In fact, for the canonical form of equivalent polynomial matrices,
    \[
    \operatorname{diag}\{ d_{1}(\lambda), d_{2}(\lambda), \ldots, d_{r}(\lambda), 0, \ldots, 0 \},
    \]
    the invariant divisors are just the non-zero diagonal elements of the canonical form.
\end{note}


\begin{leftbarTitle}{Rational Canonical Form}\end{leftbarTitle}
\begin{definition}{Companion Matrix}
    The Frobenius companion matrix of the monic polynomial
    \[
    p(x) = x^n + a_{n-1}x^{n-1} + \cdots + a_1 x + a_0
    \]
    is the square matrix defined as follows:
    \[
    C(p) = \begin{pmatrix}
    0 & 0 & \cdots & 0 & -a_0 \\
    1 & 0 & \cdots & 0 & -a_1 \\
    0 & 1 & \cdots & 0 & -a_2 \\
    \vdots & \vdots & \ddots & \vdots & \vdots \\
    0 & 0 & \cdots & 1 & -a_{n-1}
    \end{pmatrix}.
    \]
\end{definition}

Some authors use the transpose of this matrix, \(C(p)^{T}\), 
which is more convenient for some purposes such as linear recurrence relations.

\(C(p)\) is defined from the coefficients of \(p(x)\), 
while the characteristic polynomial as well as the minimal polynomial of \(C(p)\) are equal to \(p(x)\). 
In this sense, the matrix \(C(p)\) and the polynomial \(p(x)\) are "companions".

\begin{property}
    \begin{enumerate}
        \item The characteristic polynomial and minimal polynomial of \( C(p) \) is \( p(x) \) (monic).
        \item  
        \item 
    \end{enumerate}
\end{property}

\begin{definition}{Frobenius Normal Form (Rational Canonical Form)}
    Let \( V \) be an \( n \)-dimensional linear space over field \( F \),
    and let \( \mathcal{A}\in \operatorname{Hom}(V) \).
    If there exists a basis of \( V \) such that the matrix representation of \( \mathcal{A} \) under this basis is:
    \[
    F = 
    \begin{pmatrix}
        C(p_1) & 0 & \cdots & 0 \\
        0 & C(p_2) & \cdots & 0 \\
        \vdots & \vdots & \ddots & \vdots \\
        0 & 0 & \cdots & C(p_s)
    \end{pmatrix},
    \]
    where \( p_1, p_2, \dots, p_s \) are monic polynomials over field \( F \) such that
    \( p_1 | p_2 | \cdots | p_s \),
    then \( F \) is called the \textbf{Frobenius normal form} (or rational canonical form) of \( \mathcal{A} \).
\end{definition}

\begin{leftbarTitle}{Geometric Interpretation}\end{leftbarTitle} % Rational 标准型的几何解释
\begin{definition}{Cyclic Subspace}
    Let \( V \) be a linear space over field \( F \),
    and let \( \mathcal{A}\in \operatorname{Hom}(V) \).
    For a vector \(0 \neq \xi \in V \), the subspace
    \[
    C(\mathcal{A}, \xi) = \langle \xi, \mathcal{A}(\xi), \mathcal{A}^{2}(\xi), \ldots \rangle
    = \{ f(\mathcal{A})(\xi) | f(x) \in F[x] \}
    \]
    is called the \textbf{cyclic subspace} generated by \( \xi \) under the linear transformation \( \mathcal{A} \).

    If \(C(\mathcal{A}, \xi) = V\), then \( C \) is called a \textbf{cyclic space}.
\end{definition}
Obviously, \( C(\mathcal{A}, \xi) \) is the smallest \( \mathcal{A} \)-subspace containing \( \xi \).

\begin{property}
    If \(\operatorname{dim} C(\mathcal{A}, \xi) = r\), 
    then \(\{ \xi, \mathcal{A}(\xi), \ldots, \mathcal{A}^{r-1}(\xi) \}\) is a basis of \( C(\mathcal{A}, \xi) \).
\end{property}
\begin{proof}

    Let \(k_{0}=\max \{ k \in \mathbb{N}^{*}|\xi, \mathcal{A}(\xi), \ldots, \mathcal{A}^{k-1}(\xi) 
    \text{ is linearly independent} \}  \),
    then according to properties of linear independence (\ref{proposition:properties_of_linear_independence}-6)
    and mathematical induction, we can prove that
    for any \( k \geqslant k_{0} \), \( \mathcal{A}^{k}\) is a linear combination of
    \( \xi, \mathcal{A}(\xi), \ldots, \mathcal{A}^{k_{0}-1}(\xi) \).
    Therefore, \( \{ \xi, \mathcal{A}(\xi), \ldots, \mathcal{A}^{k_{0}-1}(\xi) \} \) is a basis of \( C(\mathcal{A}, \xi) \)
    and \( \operatorname{dim} C(\mathcal{A}, \xi) = k_{0} \).
\end{proof}


\vspace{0.7cm}
Next, we will show that the sufficient and necessary conditions for cyclic subspace and cyclic space.
\begin{proposition}
    Let \(V_{1}\) be \(\mathcal{A}\)-subspace of linear space \(V\) over field \(F\),
    then \(V_{1}\) is a cyclic subspace if and only if
    the matrix representation of \(\mathcal{A}|_{V_{1}}\) under some basis of \(V_{1}\) is a companion matrix.
\end{proposition}

Generally, let the set of invariant divisors of \(\mathcal{A}\) be denoted as
\[
\{ 1, \cdots, 1, d_{1}(x), d_{2}(x), \ldots, d_{k}(x) \},
\]
where \( d_{i}(x) \) are monic polynomials in \( F[x] \) such that
\[
d_{1}(x) | d_{2}(x) | \cdots | d_{k}(x).
\]
Then according to theory of rational canonical form,
there exists a basis of \( V \) such that the matrix representation of \( \mathcal{A} \) under this basis is:
\[
C=\operatorname{diag}\{ C(d_{1}), C(d_{2}), \ldots, C(d_{k}) \}.
\]
Combining with the above proposition,
we can conclude that there exists a decomposition of \( V \) into a direct sum of cyclic subspaces:
\[
V = \bigoplus_{i=1}^{k} C(\mathcal{A}, \xi_{i}),
\]
such that the matrix representation of \( \mathcal{A}|_{C(\mathcal{A}, \xi_{i})} \)
under the basis \( \{ \xi_{i}, \mathcal{A}(\xi_{i}), \ldots, \mathcal{A}^{r_{i}-1}(\xi_{i}) \} \)
is the companion matrix \( C(d_{i}) \),
where \( r_{i} = \operatorname{dim} C(\mathcal{A}, \xi_{i}) \).

This is the geometric interpretation of rational canonical form.

\begin{proposition}
    Let \( V \) be an \( n \)-dimensional linear space over field \( F \),
    and let \( \mathcal{A}\in \operatorname{Hom}(V) \).
    The following statements are equivalent:
    \begin{enumerate}
        \item There exists a vector \( \xi \in V \) such that \(\{ \xi, \mathcal{A}(\xi), \ldots, \mathcal{A}^{n-1}(\xi) \}\) 
            is a basis of \( V \), i.e., \( V \) is a cyclic space generated by \( \xi \);
        \item For any eigenvalue \( \lambda \) of \( \mathcal{A} \),
            the geometric multiplicity is \( 1 \);
        \item The characteristic polynomial of \( \mathcal{A} \) is equal to the minimal polynomial of \( \mathcal{A} \).
    \end{enumerate}
\end{proposition}
\begin{proof}

{\color{violet!80}\textbf{1}\(\Rightarrow\)\textbf{2}}
There exists a vector \( \xi \in V \) such that \(\{ \xi, \mathcal{A}(\xi), \ldots, \mathcal{A}^{n-1}(\xi) \}\) 
is a basis of \( V \). 
The matrix representation of \( \mathcal{A} \) under this basis is a companion matrix,
\[
C(p) = \begin{pmatrix}
0 & 0 & \cdots & 0 & -a_0 \\
1 & 0 & \cdots & 0 & -a_1 \\
0 & 1 & \cdots & 0 & -a_2 \\
\vdots & \vdots & \ddots & \vdots & \vdots \\
0 & 0 & \cdots & 1 & -a_{n-1}
\end{pmatrix}.
\]
For any \(c \in F\), we have
\[
cE - C(p) = \begin{pmatrix}
c & 0 & \cdots & 0 & a_0 \\
-1 & c & \cdots & 0 & a_1 \\
0 & -1 & \cdots & 0 & a_2 \\
\vdots & \vdots & \ddots & \vdots & \vdots \\
0 & 0 & \cdots & c & a_{n-1}
\end{pmatrix}.
\]
The minor of order \( n-1 \) in the bottom left corner is non-zero,
thus the rank of \( cE - C(p) \) is at least \( n-1 \).
\newline Therefore, the solution space of the equation
\[
(cE - C(p))X = 0
\]
is \(n-\operatorname{rank}(cE - C(p))\leqslant 1\).
For any eigenvalue \( \lambda \) of \( \mathcal{A} \),
\[
(\lambda E - C(p))X = 0
\]
has non-zero solutions definitely, thus its solution space is \(\geqslant 1\).
\newline Hence, the dimension of solution space of the equation
\[
(\lambda E - C(p))X = 0
\]
is exactly \( 1 \),i.e., the geometric multiplicity of eigenvalue \( \lambda \) is \( 1 \).

{\color{violet!80}\textbf{2}\(\Rightarrow\)\textbf{1}}
Let the distinct eigenvalues of \( \mathcal{A} \) be \( \lambda_1, \lambda_2, \dots, \lambda_k \),
and let the algebraic multiplicities of these eigenvalues be \( r_1, r_2, \dots, r_k \), respectively.
Obviously, \( r_1 + r_2 + \cdots + r_k = n \) and the characteristic polynomial of \( \mathcal{A} \) is
\[
f(x) = (x - \lambda_1)^{r_1} (x - \lambda_2)^{r_2} \cdots (x - \lambda_k)^{r_k}.
\]
According to the root space decomposition theorem (\ref{theorem:space_decomposition}),
\[
V = V_{\lambda_1}^{(r_1)} \oplus V_{\lambda_2}^{(r_2)} \oplus \cdots \oplus V_{\lambda_k}^{(r_k)},
\]
where \( V_{\lambda_i}^{(r_i)}\) is the root subspace of order \( r_i \) corresponding to eigenvalue \( \lambda_i \),
which is an \( \mathcal{A} \)-subspace of dimension \( r_i \) apparently.
\newline Since the geometric multiplicity of eigenvalue \( \lambda_i \) is \( 1 \),
the Jordan canonical form of \( \mathcal{A}|_{V_{\lambda_i}^{(r_i)}} \) is
\[
J_{r_i}(\lambda_i) = \begin{pmatrix}
\lambda_i & 1 & 0 & \cdots & 0 \\
0 & \lambda_i & 1 & \cdots & 0 \\
0 & 0 & \lambda_i & \cdots & 0 \\
\vdots & \vdots & \vdots & \ddots & \vdots\\
0 & 0 & 0 & \cdots & \lambda_i
\end{pmatrix}_{r_i \times r_i}.
\]
Since the equivalence between Jordan block and cyclic space,
there exists \(\alpha_{i}\in V_{\lambda_i}^{(r_i)}\) such that
\[
J_{r_i}(\lambda_i) = C(\mathcal{A}, \alpha_{i}) 
= \langle \alpha_{i}, (\mathcal{A} - \lambda_i \mathcal{E})(\alpha_{i}), \ldots, 
(\mathcal{A} - \lambda_i \mathcal{E})^{r_i - 1}(\alpha_{i}) \rangle,
\]
that is, \((\mathcal{A} - \lambda_i \mathcal{E})^{r_i - 1}(\alpha_{i}) \neq 0\).
\newline Let \(\alpha = \alpha_{1} + \alpha_{2} + \cdots + \alpha_{k}\),
then we will prove that \(\{ \alpha, \mathcal{A}(\alpha), \ldots, \mathcal{A}^{n-1}(\alpha) \}\)
is a basis of \( V \).
\newline If not, they are linearly dependent.
Then there exist \( p(x) \in F[x] \), \(\operatorname{deg} p(x) \leqslant  n-1\), such that
\[
p(\mathcal{A})(\alpha) = 0.
\]
Since \(V_{\lambda_i}^{(r_i)}\) is an \( \mathcal{A} \)-subspace, we have 
\(p(\mathcal{A})(\alpha) \in V_{\lambda_i}^{(r_i)}\).
\newline Since 
\[
0 = p(\mathcal{A})(\alpha) = p(\mathcal{A})(\alpha_{1}) + p(\mathcal{A})(\alpha_{2}) + \cdots + p(\mathcal{A})(\alpha_{k}),
\]
we have \(p(\mathcal{A})(\alpha_{i}) = 0\) for all \(i\).
\newline Since \((\mathcal{A} - \lambda_i \mathcal{E})^{r_i - 1}(\alpha_{i}) \neq 0\),
\((x - \lambda_i)^{r_i}\) is the minimal degree monic polynomial of \(\mathcal{A}\) annihilating \(\alpha_{i}\),
thus 
\[
(x - \lambda_i)^{r_i} | p(x).
\] 
Since when \(i\neq j\), \((x - \lambda_i)\) and \((x - \lambda_j)\) are coprime,
we have
\[
(x - \lambda_1)^{r_1} (x - \lambda_2)^{r_2} \cdots (x - \lambda_k)^{r_k} | p(x),
\]
i.e., \( f(x) | p(x) \),
which contradicts \(\operatorname{deg} p(x) <  n = \deg f(x)\) and \(p(x)\neq 0\).
\end{proof}


\section{Elementary Divisors and Jordan Canonical Form}
\begin{leftbarTitle}{Elementary Divisors}\end{leftbarTitle}
\begin{definition}{Elementary Divisors}
    Let \( A(\lambda) \in F[\lambda]^{m \times n} \) be a polynomial matrix,
    and let \( d_1(\lambda), d_2(\lambda), \dots, d_r(\lambda) \) be its invariant divisors,
    where \( r = \operatorname{rank} A(\lambda) \).
    If the factorization of each invariant divisor into irreducible polynomials over field \( F \) is:
    \[
    d_i(\lambda) = p_{i1}(\lambda)^{k_{i1}} p_{i2}(\lambda)^{k_{i2}} \cdots p_{is_i}(\lambda)^{k_{is_i}},
    \]
    then the polynomials
    \[
    p_{ij}(\lambda)^{k_{ij}}, \quad i = 1, 2, \dots, r; \quad j = 1, 2, \dots, s_i
    \]
    are called the \textbf{elementary divisors} of \( A(\lambda) \).
\end{definition}

\begin{leftbarTitle}{Jordan Canonical Form}\end{leftbarTitle}


\begin{lemma}\label{lemma:nilpotent_jordan_form}
    Let \( V \) be a linear space over complex number field \( F \), 
    and let \( \mathcal{A}\in \operatorname{Hom}(V) \) and \(\mathcal{A}\) is nilpotent.
    Then there exists a basis of \( V \) such that 
    the matrix representation of \( \mathcal{A} \) under this basis is
    \[
    \operatorname{diag}\{ J_{r_{1}}(0), J_{r_{2}}(0), \ldots, J_{r_{s}}(0) \},
    \]
    where \( J_{r_{i}}(0) \) is the Jordan block of size \( r_i \) corresponding to eigenvalue \( 0 \).
\end{lemma}
Combine with Theorem \ref{theorem:space_decomposition}, we have the following theorem:
\begin{theorem}{Existence of Jordan Canonical Form}
    Let \( V \) be an \( n \)-dimensional linear space over complex number field \( F \),
    and let \( \mathcal{A}\in \operatorname{Hom}(V) \).
    Then there exists a basis of \( V \): \(\{ e_{1}, e_{2}, \ldots, e_{n} \}\),
    such that the matrix representation of \( \mathcal{A} \) under this basis is Jordan canonical form:
    \[
    J = \operatorname{diag}\{ J_{r_{1}}(\lambda_{1}), J_{r_{2}}(\lambda_{2}), \ldots, J_{r_{s}}(\lambda_{s}) \},
    \]
\end{theorem}

\begin{leftbarTitle}{Geometric Interpretation}\end{leftbarTitle} % Jordan 标准型的几何解释
From perspective of space decomposition, a linear space \( V \) can be decomposed into a direct sum of root subspaces,
and according to the proof of Lemma~\ref{lemma:nilpotent_jordan_form},
each root subspace can be further decomposed into a direct sum of cyclic subspaces generated by generalized eigenvectors.
Therefore, \(V\) can be decomposed into a direct sum of cyclic subspaces generated by generalized eigenvectors,
\[
V = C(\mathcal{A}-\lambda_{1}\mathcal{E}, \xi_{1}) \oplus C(\mathcal{A}-\lambda_{2}\mathcal{E}, \xi_{2}) \oplus \cdots \oplus 
C(\mathcal{A}-\lambda_{s}\mathcal{E}, \xi_{s}),
\]
where cyclic subspace \( C(\mathcal{A}-\lambda_{i}\mathcal{E}, \xi_{i}) \) one-to-one corresponds to 
a Jordan block \( J_{r_{i}}(\lambda_{i}) \), which is geometric interpretation of Jordan canonical form.


\section{Finding Jordan Canonical Form and Transition Matrices}
\begin{leftbarTitle}{Finding Jordan Canonical Form}\end{leftbarTitle}


\begin{leftbarTitle}{Finding Transition Matrices}\end{leftbarTitle}

\section{Complete Set of Invariant Quantities for Matrix Similarity}

