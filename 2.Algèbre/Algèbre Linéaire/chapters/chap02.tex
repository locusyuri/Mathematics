\chapter{Determinants} % 行列式
\section{Permutations} % 排列

\section{Determinant and Its Properties} % 行列式及其性质

\section{Expanding by Rows (Columns)} % 按行(列)展开
\begin{leftbarTitle}{Expanding by One Row}\end{leftbarTitle}

\begin{leftbarTitle}{Cramer's Rule}\end{leftbarTitle}

\begin{leftbarTitle}{Expanding by \(k\) Rows}\end{leftbarTitle}

\section{Special Determinants} % 这里放置小节标题
\begin{definition}{Vandermonde Determinant}
    The Vandermonde determinant is defined as
    \[
    V_n = \begin{vmatrix}
    1 & 1 & 1 & \cdots & 1 \\
    x_1 & x_2 & x_3 & \cdots & x_n \\
    x_1^2 & x_2^2 & x_3^2 & \cdots & x_n^2 \\
    \vdots & \vdots & \vdots & \ddots & \vdots \\
    x_1^{n-1} & x_2^{n-1} & x_3^{n-1} & \cdots & x_n^{n-1}
    \end{vmatrix}
    \]
    where \( x_1, x_2, \ldots, x_n \) are distinct variables.
\end{definition}

The value of the Vandermonde determinant is given by
\[
V_n = \prod_{1 \leq i < j \leq n} (x_j - x_i).
\]  

\begin{definition}{Arrow Determinant}
    The Arrow determinant (\(\nwarrow\)) is defined as
    \[
    A_n = \begin{vmatrix}
    a_{11} & a_{12} & a_{13} & \cdots & a_{1n} \\
    a_{21} & a_{22} & 0 & \cdots & 0 \\
    a_{31} & 0 & a_{33} & \cdots & 0 \\
    \vdots & \vdots & \vdots & \ddots & \vdots \\
    a_{n1} & 0 & 0 & \cdots & a_{nn}
    \end{vmatrix}.
    \]

    The value of the Arrow determinant is given by
    \[
    A_n = \left( a_{11}- \sum_{k=2}^{n} \frac{a_{1k}a_{k1}}{a_{kk}}  \right)\prod_{k=2}^{n} a_{kk}.
    \]
\end{definition}

From the first column sequentially, subtract \( \frac{a_{21}}{a_{22}} \) times the second column, \(\cdots\), 
\( \frac{a_{n1}}{a_{nn}} \) times the \( n \)-th column, so that the first column becomes:
\[
\begin{bmatrix}
a_{11} - \sum\limits_{k=2}^n \frac{a_{1k}a_{k1}}{a_{kk}} &
0 &
0 &
\vdots &
0
\end{bmatrix}^{\mathrm{T}}.
\]
Then expand along the first column.

\begin{definition}{Two-Triangular Determinant}
    If the determinant satisfies 
    \[ 
    a_{ij} = \begin{cases} 
    a, & i < j, \\ 
    x_{i}, & i = j, \\ 
    b, & i > j,
    \end{cases} 
    \]
    then \( D_{n} \) is called a two-triangular determinant.
\end{definition}

The value of the two-triangular determinant is given by
\[
\begin{vmatrix}
x_{1} & a & a & \dots & a \\
b & x_{2} & a & \dots & a \\
b & b & x_{3} & \dots & a \\
\vdots & \vdots & \vdots & & \vdots \\
b & b & b & \dots & x_{n}
\end{vmatrix}
=
\begin{cases}
\left[ x_{1} + a \sum\limits^{n}_{k=2} \frac{x_{1}-a}{x_{k}-a} \right] \cdot \prod\limits^{n}_{k=2} (x_{k}-a), & a = b \\
(x_{n}-b) D_{n-1} + \prod\limits^{n-1}_{k=1} (x_{k}-a), & a \neq b 
\end{cases}
\]