\documentclass[11pt]{elegantbook}

\title{Algèbre Linéaire} % 这里放置书名
% \subtitle{Subtitle} % 这里放置副标题

\author{CatMono} % 这里放置作者名
\date{July, 2025} % 这里放置日期
\version{0.1} % 这里放置版本号
% \institute{Elegant\LaTeX{} Program} % 这里放置机构名
% \bioinfo{Custom Key}{Custom Value} % 这里放置自定义信息

% \extrainfo{extra information} % 这里放置额外信息,将显示在最下方中央

\setcounter{tocdepth}{2} % 设置目录深度
\setcounter{secnumdepth}{2} % 设置章节编号深度


% \logo{logo-blue.png} % 这里放置封面logo,默认从figure目录下寻找
% \cover{LogiqueMathematique.png} % 这里放置封面图片,默认从figure目录下寻找

% modify the color in the middle of titlepage
\definecolor{customcolor}{RGB}{32,178,170} % 自定义颜色
\colorlet{coverlinecolor}{customcolor}
\usepackage{cprotect} % 保护命令参数不被 LaTeX 解析器过早处理,允许在某些特殊环境中使用脆弱命令(fragile commands)。
\usepackage{xeCJK} % 使用 xeCJK 包支持中文


% ===== 开始文档 =====
\begin{document}

\maketitle %生成文档的标题页,根据之前定义的标题信息(如标题、作者、日期等)自动创建一个格式化的标题页

% === 前言部分 ===
\frontmatter        % 开始前言,页码为 i, ii, iii...
\tableofcontents    % 目录 (页码: i, ii)
% \listoffigures      % 图表目录 (页码: iii)
% \listoftables       % 表格目录 (页码: iv)

\chapter{Preface}   % 前言章节(无编号,页码: v, vi...)
This is the preface of the book...

% \chapter{Acknowledgments}  % 致谢(无编号)
% I would like to thank...
% === 正文部分 ===
\mainmatter         % 开始正文,页码从 1 重新开始

\chapter{Determinants} % 这里放置章节标题
\section{Section Title} % 这里放置小节标题
\subsection{Subsection Title} % 这里放置子小节标题

\chapter{Systems of Linear Equations}

\chapter{Matrices}

\chapter{Linear Spaces}
\section{Linear Spaces over the Field \(\mathbb{F}\)}

\subsection{Linear Spaces}

\subsection{Dimension, Basis, and Coordinates}

\subsection{Basis Transformation and Coordinate Transformation}

\section{Subspaces}

\subsection{Linear Subspaces}

\subsection{Intersection and Sum of Subspaces}

\subsection{Dimension Formula}

\subsection{Direct Sum of Subspaces}

\section{Isomorphisms}

\section{Quotient Spaces}

\chapter{Linear Mappings}
\section{Linear Mappings and Their Computation}
\subsection{Definition of Linear Mappings}

\subsection{Existence and Uniqueness of Linear Mappings}

\subsection{Operations of Linear Mappings}

\subsection{Special Linear Transformations}

\section{Kernel and Image of Linear Mappings}

\section{Matrix Representation of Linear Mappings}

\section{Linear Functions and Dual Spaces }



\chapter{Diagonalization}
\section{Similarity of Matrices}
\section{Eigenvectors and Diagonalization}
\subsection{Eigenvalues and Eigenvectors}

\subsection{Necessary and Sufficient Conditions for Diagonalization}
\begin{leftbarTitle}{Geometric Multiplicity of Eigenvectors}\end{leftbarTitle}
\begin{leftbarTitle}{Algebraic Multiplicity}\end{leftbarTitle}


\section{Space Decomposition and Diagonalization}

\subsection{Invariant Subspace}

\subsection{Hamilton-Cayley Theorem}

\section{Least Squares and Diagonalization}

\chapter{Jordan Forms}
\section{Polynomial Matrices}

\section{Invariant Factors}

\section{Rational Canonical Form}

\section{Elementary Divisors}

\section{Jordan Canonical Form}

\chapter{Quadratic Forms}
\section{Quadratic Forms and Their Standard Forms}
\section{Canonical Forms}
\section{Definite Quadratic Forms}


\chapter{Inner Product Spaces}
\section{Bilinear Forms}

\section{Real Inner Product Spaces}

\section{Metric Matrices and Standard Orthonormal Bases}

\section{Isomorphism of Real Inner Product Spaces}

\section{Orthogonal Completion and Orthogonal Projection}
\subsection{Orthogonal Completion}

\subsection{Least Squares Method}

\section{Orthogonal Transformations and Symmetric Transformations}
\subsection{Orthogonal Transformations}

\subsection{Symmetric Transformations}

\section{Unitary Spaces and Unitary Transformations}

\section{Symplectic Spaces}

\begin{thebibliography}{99} 
\bibitem{ch1} 丘维声, \emph{ 高等代数 (2nd edition) }, 清华大学出版社, 2019. 
\bibitem{ch2} 谢启鸿, 姚慕生, 吴泉水, \emph{ 高等代数学 (4th edition) }, 复旦大学出版社, 2022.
\bibitem{ch3} 王萼芳, 石生明 \emph{ 高等代数 (5th edition) }, 高等教育出版社, 2019.
\bibitem{ch4} 樊启斌, \emph{ 高等代数典型问题与方法 (1st edition) }, 高等教育出版社, 2021.
\end{thebibliography}

\end{document}