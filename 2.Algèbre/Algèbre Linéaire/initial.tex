\documentclass[11pt]{../../TexTemplate/elegantbook} % 这里是文档类,默认使用 elegantbook

\title{Algèbre Linéaire} % 这里放置书名
% \subtitle{Subtitle} % 这里放置副标题

\author{CatMono} % 这里放置作者名
\date{July, 2025} % 这里放置日期
\version{0.1} % 这里放置版本号
% \institute{Elegant\LaTeX{} Program} % 这里放置机构名
% \bioinfo{Custom Key}{Custom Value} % 这里放置自定义信息

% \extrainfo{extra information} % 这里放置额外信息,将显示在最下方中央

\setcounter{tocdepth}{2} % 设置目录深度
\setcounter{secnumdepth}{2} % 设置章节编号深度


% \logo{logo-blue.png} % 这里放置封面logo,默认从figure目录下寻找
% \cover{LogiqueMathematique.png} % 这里放置封面图片,默认从figure目录下寻找

% modify the color in the middle of titlepage
\definecolor{customcolor}{RGB}{32,178,170} % 自定义颜色
\colorlet{coverlinecolor}{customcolor}
\usepackage{cprotect} % 保护命令参数不被 LaTeX 解析器过早处理,允许在某些特殊环境中使用脆弱命令(fragile commands)。
\usepackage{xeCJK} % 使用 xeCJK 包支持中文


% ===== 开始文档 =====
\begin{document}

\maketitle %生成文档的标题页,根据之前定义的标题信息(如标题、作者、日期等)自动创建一个格式化的标题页

% === 前言部分 ===
\frontmatter        % 开始前言,页码为 i, ii, iii...
\tableofcontents    % 目录 (页码: i, ii)
% \listoffigures      % 图表目录 (页码: iii)
% \listoftables       % 表格目录 (页码: iv)

\chapter{Preface}   % 前言章节(无编号,页码: v, vi...)
This is the preface of the book...

% \chapter{Acknowledgments}  % 致谢(无编号)
% I would like to thank...
% === 正文部分 ===
\mainmatter         % 开始正文,页码从 1 重新开始

\chapter{Determinants} % 这里放置章节标题
\section{Special Determinants} % 这里放置小节标题
\begin{definition}{Vandermonde Determinant}
    The Vandermonde determinant is defined as
    \[
    V_n = \begin{vmatrix}
    1 & 1 & 1 & \cdots & 1 \\
    x_1 & x_2 & x_3 & \cdots & x_n \\
    x_1^2 & x_2^2 & x_3^2 & \cdots & x_n^2 \\
    \vdots & \vdots & \vdots & \ddots & \vdots \\
    x_1^{n-1} & x_2^{n-1} & x_3^{n-1} & \cdots & x_n^{n-1}
    \end{vmatrix}
    \]
    where \( x_1, x_2, \ldots, x_n \) are distinct variables.
\end{definition}

The value of the Vandermonde determinant is given by
\[
V_n = \prod_{1 \leq i < j \leq n} (x_j - x_i).
\]  

\begin{definition}{Arrow Determinant}
    The Arrow determinant (\(\nwarrow\)) is defined as
    \[
    A_n = \begin{vmatrix}
    a_{11} & a_{12} & a_{13} & \cdots & a_{1n} \\
    a_{21} & a_{22} & 0 & \cdots & 0 \\
    a_{31} & 0 & a_{33} & \cdots & 0 \\
    \vdots & \vdots & \vdots & \ddots & \vdots \\
    a_{n1} & 0 & 0 & \cdots & a_{nn}
    \end{vmatrix}.
    \]

    The value of the Arrow determinant is given by
    \[
    A_n = \left( a_{11}- \sum_{k=2}^{n} \frac{a_{1k}a_{k1}}{a_{kk}}  \right)\prod_{k=2}^{n} a_{kk}.
    \]
\end{definition}

From the first column sequentially, subtract \( \frac{a_{21}}{a_{22}} \) times the second column, \(\cdots\), 
\( \frac{a_{n1}}{a_{nn}} \) times the \( n \)-th column, so that the first column becomes:
\[
\begin{bmatrix}
a_{11} - \sum\limits_{k=2}^n \frac{a_{1k}a_{k1}}{a_{kk}} &
0 &
0 &
\vdots &
0
\end{bmatrix}^{\mathrm{T}}.
\]
Then expand along the first column.

\begin{definition}{Two-Triangular Determinant}
    If the determinant satisfies 
    \[ 
    a_{ij} = \begin{cases} 
    a, & i < j, \\ 
    x_{i}, & i = j, \\ 
    b, & i > j,
    \end{cases} 
    \], 
    then \( D_{n} \) is called a two-triangular determinant.
\end{definition}

The value of the two-triangular determinant is given by
\[
\begin{vmatrix}
x_{1} & a & a & \dots & a \\
b & x_{2} & a & \dots & a \\
b & b & x_{3} & \dots & a \\
\vdots & \vdots & \vdots & & \vdots \\
b & b & b & \dots & x_{n}
\end{vmatrix}
=
\begin{cases}
\left[ x_{1} + a \sum\limits^{n}_{k=2} \frac{x_{1}-a}{x_{k}-a} \right] \cdot \prod\limits^{n}_{k=2} (x_{k}-a), & a = b \\
(x_{n}-b) D_{n-1} + \prod\limits^{n-1}_{k=1} (x_{k}-a), & a \neq b 
\end{cases}
\]


\chapter{Systems of Linear Equations}

\chapter{Matrices}
\section{Basic Operations}
\begin{leftbarTitle}{Addition}\end{leftbarTitle}
\begin{leftbarTitle}{Scalar Multiplication}\end{leftbarTitle}
\begin{leftbarTitle}{Transpose}\end{leftbarTitle}
\begin{leftbarTitle}{Matrix Multiplication}\end{leftbarTitle}

\begin{theorem}{Cauchy-Binet Formula}
    Let \( A = (a_{ij})_{m \times n} \) and \( B = (b_{ij})_{n \times m} \):  
    \begin{enumerate}
        \item If \( m > n \), then \( |AB| = 0 \);
        \item If \( m \leq n \), then \( |AB| \) is equal to 
        the sum of products of all \( m \)-step minors of \( A \) and 
        the corresponding \( m \)-step minors of \( B \), that is:
        \[
        |AB| = \sum_{1 \leq v_1 < v_2 < \dots < v_m \leq n} 
        \begin{vmatrix}
        A\begin{pmatrix}
        1, 2, \dots, m \\
        v_1, v_2, \dots, v_m
        \end{pmatrix}
        \end{vmatrix}
        \cdot 
        \begin{vmatrix}
        B\begin{pmatrix}
        v_1, v_2, \dots, v_m \\
        1, 2, \dots, m
        \end{pmatrix}
        \end{vmatrix}.
        \]
    \end{enumerate}
\end{theorem}





\section{Matrix Equivalence}

\section{Special Matrices}

\section{Inverse Matrix}
\begin{leftbarTitle}{Inverse Matrix and Its Operations}\end{leftbarTitle}

\begin{leftbarTitle}{Equivalent Propositions and Method of Inversion}\end{leftbarTitle}

\begin{leftbarTitle}{Generalized Inverse}\end{leftbarTitle}

\section{Block Matrix}

\begin{theorem}{Determinant Reduction Formula}
    Let \(A_{m\times m}, B_{m\times n}, C_{n\times m}, D_{n\times n}\) be matrices. Then:
    \begin{enumerate}
        \item If \( A \) is invertible, then:
            \[
            \begin{vmatrix}
            A & B \\
            C & D
            \end{vmatrix}
            = |A| \cdot |D - CA^{-1}B|.
            \]
        \item If \( D \) is invertible, then:
            \[
            \begin{vmatrix}
            A & B \\
            C & D
            \end{vmatrix}
            = |D| \cdot |A - BD^{-1}C|.
            \]
        \item If both \( A \) and \( D \) are invertible, then:
            \[
            |D| \cdot |A - BD^{-1}C| = |A| \cdot |D - CA^{-1}B|.
            \]
    \end{enumerate}
\end{theorem}

\begin{remark}
    The mnemonic is: 
    For \( \begin{vmatrix} A & B \\ C & D \end{vmatrix} \), for example, if \( A \) is invertible, 
    one factor is \( |A| \), and the other factor is \( D \) (the diagonal element of \( A \)) minus 
    the product of the other three terms arranged clockwise, where the middle one is the inverse matrix.
\end{remark}




\section{Operations of Rank}
\begin{proposition}
    The matrices \( A \) and \( B \) in the following operations do not need to be square matrices; 
    they only need to be compatible for multiplication or addition.
    \begin{description}
        \item [1. Addition]
        \[
        \operatorname{rank}(A + B) \leqslant \operatorname{rank}(A) + \operatorname{rank}(B).
        \]
        \item [2. Multiplication]
        \[
        \operatorname{rank}(AB) \leqslant \operatorname{rank}(A), \quad \operatorname{rank}(AB) \leqslant \operatorname{rank}(B).
        \]
        \item[2.1. Sylvester's Inequality]
        \[
        \operatorname{rank}(AB) \geqslant \operatorname{rank}(A) + \operatorname{rank}(B) - n \quad (A_{s \times n}, B_{n \times m}).
        \]
        Specially, if \( AB = O \), then:
        \[
        \operatorname{rank}(A) + \operatorname{rank}(B) \leqslant n.
        \]

        \item[2.2. Frobenius Inequality]
        \[
        \operatorname{rank}(ABC) \geqslant \operatorname{rank}(AB) + \operatorname{rank}(BC) - \operatorname{rank}(B).
        \]

        \item [3. Transpose]
        \[
        \operatorname{rank}(AA^T) = \operatorname{rank}(A^TA) = \operatorname{rank}(A) = \operatorname{rank}(A^T).
        \]

        \item [4. Inverse]
        \[
        \operatorname{rank}(A) = \operatorname{rank}(A^{-1}) = n.
        \]
    \end{description}
\end{proposition}

\section{Low-Rank Update}
Due to all the row and column vectors of a rank-\(1\) matrix are linearly dependent,
it can be expressed as the outer product of two non-zero vectors;
in other words, a rank-\(1\) matrix can be expressed as \( \alpha \beta^T \),
where \( \alpha \) and \( \beta \) are non-zero column vectors.

Based on the decomposition \( A = \alpha \beta^T \), the matrix of rank-\(1\) has simplified calculation rules:
\begin{property}
    \begin{description}
        \item [Exponentiation]  
        For any positive integer \( k \geq 1 \),
        \[
        A^k = (\beta^T\alpha)^{k-1} \cdot A,
        \]
        where \( \beta^T\alpha \) is a constant (the inner product of vectors).
        \item [Rank Transmission]  
        If \( B \) is any matrix, then:
        \[
        \operatorname{rank}(AB) \leq 1 \quad \text{and} \quad \operatorname{rank}(BA) \leq 1,
        \]
        (rank 1 matrices multiplied by arbitrary matrices result in ranks not exceeding 1).
    \end{description}
\end{property}



\begin{theorem}{Sherman-Morrison Formula}
    If \( A\in \mathbb{R}^{n\times n} \) is an invertible matrix, 
    and \( \alpha, \beta\in \mathbb{R}^n \) are column vectors, 
    then \( A + \alpha \beta^T \) is invertible if and only if \( 1 + \beta^T A^{-1} \alpha \neq 0 \). 
    In this case, the inverse of \( A + \alpha \beta^T \) is given by:
    \[
    \left(A + \alpha \beta^T\right)^{-1} = A^{-1} - \frac{A^{-1} \alpha \beta^T A^{-1}}{1 + \beta^T A^{-1} \alpha},
    \]
    where \(\alpha\beta^{\mathrm{T}}\) is the outer product of \(\alpha\) and \(\beta\).
\end{theorem}

\begin{note}
    Combining the properties of determinants, 
    we can derive the determinant version of the Sherman-Morrison formula:
    \[
    \left|A + \alpha \beta^T\right| = |A| \cdot \left(1 + \beta^T A^{-1} \alpha\right),
    \]
    which is known as the \textbf{matrix determinant lemma}.

    The theorem can also be stated in terms of the adjugate matrix of \( A \):
    \[
    \det(A + uv^T) = \det(A) + v^T \operatorname{adj}(A) u,
    \]
    in which case it applies whether or not the matrix \( A \) is invertible.
\end{note}



\chapter{Linear Spaces}
\section{Linear Spaces over the Field \(\mathbb{F}\)}

\subsection{Linear Spaces}

\subsection{Dimension, Basis, and Coordinates}

\subsection{Basis Transformation and Coordinate Transformation}

\section{Subspaces}

\subsection{Linear Subspaces}

\subsection{Intersection and Sum of Subspaces}

\subsection{Dimension Formula}

\subsection{Direct Sum of Subspaces}

\section{Isomorphisms}

\section{Quotient Spaces}

\chapter{Linear Mappings}
\section{Linear Mappings and Their Computation}
\subsection{Definition of Linear Mappings}

\subsection{Existence and Uniqueness of Linear Mappings}

\subsection{Operations of Linear Mappings}

\subsection{Special Linear Transformations}

\section{Kernel and Image of Linear Mappings}

\section{Matrix Representation of Linear Mappings}

\section{Linear Functions and Dual Spaces }



\chapter{Diagonalization}
\section{Similarity of Matrices}
\section{Eigenvectors and Diagonalization}
\subsection{Eigenvalues and Eigenvectors}

\subsection{Necessary and Sufficient Conditions for Diagonalization}
\begin{leftbarTitle}{Geometric Multiplicity of Eigenvectors}\end{leftbarTitle}
\begin{leftbarTitle}{Algebraic Multiplicity}\end{leftbarTitle}


\section{Space Decomposition and Diagonalization}

\subsection{Invariant Subspace}

\subsection{Hamilton-Cayley Theorem}

\section{Least Squares and Diagonalization}

\chapter{Jordan Forms}
\section{Polynomial Matrices}

\section{Invariant Factors}

\section{Rational Canonical Form}

\section{Elementary Divisors}

\section{Jordan Canonical Form}

\chapter{Quadratic Forms}
\section{Quadratic Forms and Their Standard Forms}
\section{Canonical Forms}
\section{Definite Quadratic Forms}


\chapter{Inner Product Spaces}
\section{Bilinear Forms}

\section{Real Inner Product Spaces}

\section{Metric Matrices and Standard Orthonormal Bases}

\section{Isomorphism of Real Inner Product Spaces}

\section{Orthogonal Completion and Orthogonal Projection}
\subsection{Orthogonal Completion}

\subsection{Least Squares Method}

\section{Orthogonal Transformations and Symmetric Transformations}
\subsection{Orthogonal Transformations}

\subsection{Symmetric Transformations}

\section{Unitary Spaces and Unitary Transformations}

\section{Symplectic Spaces}

\begin{thebibliography}{99} 
\bibitem{ch1} 丘维声, \emph{ 高等代数 (2nd edition) }, 清华大学出版社, 2019. 
\bibitem{ch2} 谢启鸿, 姚慕生, 吴泉水, \emph{ 高等代数学 (4th edition) }, 复旦大学出版社, 2022.
\bibitem{ch3} 王萼芳, 石生明 \emph{ 高等代数 (5th edition) }, 高等教育出版社, 2019.
\bibitem{ch4} 樊启斌, \emph{ 高等代数典型问题与方法 (1st edition) }, 高等教育出版社, 2021.
\bibitem{9} Wikipedia. \url{https://en.wikipedia.org/wiki/}.
\end{thebibliography}

\end{document}