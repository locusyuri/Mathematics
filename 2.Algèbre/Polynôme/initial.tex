\documentclass[11pt]{../../TexTemplate/elegantbook}

\title{Polynôme} % 这里放置书名
% \subtitle{Subtitle} % 这里放置副标题

\author{CatMono} % 这里放置作者名
\date{September, 2025} % 这里放置日期
\version{0.1} % 这里放置版本号
% \institute{Elegant\LaTeX{} Program} % 这里放置机构名
% \bioinfo{Custom Key}{Custom Value} % 这里放置自定义信息

% \extrainfo{extra information} % 这里放置额外信息,将显示在最下方中央

\setcounter{tocdepth}{2} % 设置目录深度
\setcounter{secnumdepth}{2} % 设置章节编号深度


% \logo{logo-blue.png} % 这里放置封面logo,默认从figure目录下寻找
% \cover{LogiqueMathematique.png} % 这里放置封面图片,默认从figure目录下寻找

% modify the color in the middle of titlepage
\definecolor{customcolor}{RGB}{32,178,170} % 自定义颜色
\colorlet{coverlinecolor}{customcolor}
\usepackage{cprotect} % 保护命令参数不被 LaTeX 解析器过早处理,允许在某些特殊环境中使用脆弱命令(fragile commands)。
\usepackage{xeCJK} % 使用 xeCJK 包支持中文
\usepackage{biblatex}


% ===== 开始文档 =====
\begin{document}

\maketitle %生成文档的标题页,根据之前定义的标题信息(如标题、作者、日期等)自动创建一个格式化的标题页

% === 前言部分 ===
\frontmatter        % 开始前言,页码为 i, ii, iii...
\tableofcontents    % 目录 (页码: i, ii)
% \listoffigures      % 图表目录 (页码: iii)
% \listoftables       % 表格目录 (页码: iv)

\chapter{Preface}   % 前言章节(无编号,页码: v, vi...)
This is the preface of the book...

% \chapter{Acknowledgments}  % 致谢(无编号)
% I would like to thank...
% === 正文部分 ===
\mainmatter         % 开始正文,页码从 1 重新开始

\chapter{Preliminaries} % 这里放置章节标题


\chapter{Univariate Polynomial Ring}
\section{Univariate Polynomials}

\section{Division}

\begin{theorem}{Euclidean Division (Division with Remainder)}
    Let \( f(x), g(x) \in P[x] \) with \( g(x) \neq 0 \). 
    Then there exist unique polynomials \( q(x), r(x) \in P[x] \) such that
    \[
    f(x) = g(x) \cdot q(x) + r(x)
    \]
    where \( r(x) = 0 \) or \( \deg(r) < \deg(g) \).
\end{theorem}

\begin{definition}{Exact Division}
    If there exists \( h(x)\in P[x] \) such that \( f(x) = g(x) \cdot h(x) \), 
    we say that \( g(x) \) divides \( f(x) \) and write \( g(x) \mid f(x) \).
    (In other words, the remainder \( r(x) = 0 \).)
\end{definition}

\begin{property}
    
\end{property}

\begin{caution}
    In Euclidean division, \( g(x) \neq 0 \) is required. 
    However, in the case of \( g(x) \mid f(x) \), \( g(x) \) can equal \( 0 \). 
    In this situation, \( f(x) = g(x)h(x) = 0 \cdot g(x) = 0 \), 
    meaning that the \textbf{zero polynomial can only divide the zero polynomial}.
\end{caution}

\section{Greatest Common Divisor and Relatively Prime}
\begin{leftbarTitle}{Greatest Common Divisor}\end{leftbarTitle}
\begin{definition}{Greatest Common Divisor (GCD)}
    Let \( f(x), g(x) \in P[x] \). 
    A polynomial \( d(x) \in P[x] \) is called a greatest common divisor of \( f(x) \) and \( g(x) \) if:
    \begin{enumerate}
        \item \( d(x) \mid f(x) \) and \( d(x) \mid g(x) \);
        \item For any polynomial \( h(x) \in P[x] \), 
            if \( h(x) \mid f(x) \) and \( h(x) \mid g(x) \), then \( h(x) \mid d(x) \).
    \end{enumerate}
    The greatest common divisor of \( f(x) \) and \( g(x) \), 
    whose leading coefficient is \(1\) (also called \textbf{monic}), is denoted as \( \left(f(x), g(x)\right) \).
\end{definition}

\begin{property}
    
\end{property}

\begin{theorem}{Euclidean Algorithm}
    For all \( f(x), g(x) \in P[x] \), there exists \( d(x) \in P[x] \), 
    where \( d(x) \) is a greatest common divisor of \( f(x) \) and \( g(x) \), 
    and \( d(x) \) can be expressed as a linear combination of \( f(x) \) and \( g(x) \), 
    i.e., there exist \( u(x), v(x) \in P[x] \) such that
    \[
    d(x) = u(x)f(x) + v(x)g(x).
    \]
    The converse proposition does not hold in general.
\end{theorem}



\begin{leftbarTitle}{Relatively Prime}\end{leftbarTitle}

\begin{definition}{Relatively Prime}
    Two polynomials \( f(x) \) and \( g(x) \) in \( P[x] \) are called relatively prime 
    if \( (f(x), g(x)) = 1 \), 
    meaning they have no common divisor other than the zero-degree polynomial (nonzero constant).
\end{definition}

\section{Least Common Multiple}

\section{Opposite Polynomials}

\chapter{Factorization and Roots}

\section{Irreducible Polynomials}
\begin{definition}{Irreducible Polynomial}
    A polynomial \( p(x) \) of degree \( \geq 1 \) over a field \( P \) 
    is called an irreducible polynomial over the field \( P \) 
    if it cannot be expressed as the product of two polynomials of 
    lower degree than \( p(x) \) over the field \( P \).
\end{definition}

\begin{proposition}
    For all \(f(x), g(x)\in P[x]\), \(p(x)\) is an irreducible polynomial in \(P[x]\), 
    which is equivalent to the following two propositions:
    \begin{enumerate}
        \item Either \( p(x) \mid f(x) \) or \( \left(p(x), f(x)\right) = 1 \);
        \item If \( p(x) \mid f(x)g(x) \), then either \( p(x) \mid f(x) \) or \( p(x) \mid g(x) \).
    \end{enumerate}

    Similarly, monic polynomial \( p(x) \), with degree greater than \(0\), 
    is a power of an irreducible polynomial over the field \( P \)
    if and only if for all \( f(x), g(x) \in P[x] \),
    \begin{enumerate}
        \item Either \( p(x) \mid f^{m}(x) \) (\(m\in \mathbb{N}^{*}\)) or \( \left(p(x), f(x)\right) = 1 \);
        \item If \( p(x) \mid f(x)g(x) \), then either \( p(x) \mid f^{m}(x) \) (\(m\in \mathbb{N}^{*}\)) or \( p(x) \mid g(x) \).
    \end{enumerate}
\end{proposition}

\section{Polynomials with Rational Coefficients}
\begin{definition}{Primitive Polynomial}
    A polynomial \( f(x) = a_n x^n + a_{n-1} x^{n-1} + \cdots + a_1 x + a_0 \) 
    with integer coefficients is called a \textbf{primitive polynomial}
    if the greatest common divisor of its coefficients is \(\pm 1\), 
    i.e., \( (a_n, a_{n-1}, \ldots, a_1, a_0) = \pm 1 \).
\end{definition}

\begin{lemma}{Gauß's Lemma}
    The product of two primitive polynomials is also a primitive polynomial.
\end{lemma}
With the help of Gauß's lemma, we can establish the following important theorem:
\begin{theorem}
    If a polynomial \( f(x) \) with integer coefficients is reducible over the field of rational numbers \( \mathbb{Q} \), 
    then it is also reducible over the ring of integers \( \mathbb{Z} \).
\end{theorem}
A corollary can be derived from this theorem:
\begin{corollary}
    Let \(f(x), g(x) \in \mathbb{Z}[x]\) be two polynomials, and \(g(x)\) is primitive. 
    If \(f(x) = g(x)h(x)\), where \(h(x) \in \mathbb{Q}[x]\), 
    then \(h(x) \in \mathbb{Z}[x]\).
\end{corollary}

\begin{leftbarTitle}{Searching and Judging of Rational Roots}\end{leftbarTitle}
Now we can use the following theorem to search for rational roots of polynomials with integer coefficients:
\begin{theorem}{Rational Root Theorem}
    Let \( f(x) = a_n x^n + a_{n-1} x^{n-1} + \cdots + a_1 x + a_0 \) 
    be a polynomial with integer coefficients. 
    If \( \frac{r}{s} \) (in lowest terms) is a rational root of \( f(x) \), 
    then \( r\mid a_{0} \) and \( s \mid a_n \).

    Obviously, if \( f(x) \) is monic, then any rational root must be an integer divisor of \( a_0 \).
\end{theorem}

Next, we can use the following theorem to judge whether a polynomial with integer coefficients is 
irreducible over the field of rational numbers:
\begin{theorem}{Eisenstein's Criterion}
    Let \( f(x) = a_n x^n + a_{n-1} x^{n-1} + \cdots + a_1 x + a_0 \) 
    be a polynomial with integer coefficients. 
    If there exists a prime number \( p \) such that:
    \begin{enumerate}
        \item \( p \nmid a_n \);
        \item \( p \mid a_i \) for all \( i = 0, 1, \ldots, n-1 \);
        \item \( p^2 \nmid a_0 \);
    \end{enumerate}
    then \( f(x) \) is \emph{irreducible} over the field of rational numbers \( \mathbb{Q} \).
\end{theorem}



\section{Relation between Roots and Coefficients}
\begin{theorem}{Vièta's Formulas}
    Let \( f(x) = a_n x^n + a_{n-1} x^{n-1} + \cdots + a_1 x + a_0 \) 
    be a polynomial of degree \( n \) over field \( P \), 
    and let its \( n \) roots (counting multiplicities) be \( r_1, r_2, \ldots, r_n \) in an extension field of \( P \). 
    Then the following relations hold:
    \[
    \begin{aligned}
    &r_1 + r_2 + \cdots + r_n = -\frac{a_{n-1}}{a_n}, \\
    &r_1 r_2 + r_1 r_3 + \cdots + r_{n-1} r_n = \frac{a_{n-2}}{a_n}, \\
    &\vdots \\
    &r_1 r_2 \cdots r_n = (-1)^n \frac{a_0}{a_n}.
    \end{aligned}
    \]
\end{theorem}
Using symmetric polynomial notation (\ref{def:symmetric_polynomial}), Vièta's formulas can be expressed as:
\begin{align*}
    \sigma_1(r_1, r_2, \ldots, r_n) &= -\frac{a_{n-1}}{a_n}, \\
    \sigma_2(r_1, r_2, \ldots, r_n) &= \frac{a_{n-2}}{a_n}, \\
    &\vdots \\
    \sigma_n(r_1, r_2, \ldots, r_n) &= (-1)^n \frac{a_0}{a_n},
\end{align*}
that is,
\[
    \sigma_i(r_1, r_2, \ldots, r_n) = (-1)^{i} \frac{a_{n-i}}{a_n}, \quad i = 1, 2, \ldots, n.
\]



\section{Root of Unity}
\begin{definition}{Root of Unity}
    Let \( P \) be a number field and \( n \in \mathbb{N}^{*} \). 
    An element \( \omega \in P \) is called an \( n \)-th root of unity 
    if it satisfies the equation \( x^n - 1 = 0 \), i.e., \( \omega^n = 1 \).
\end{definition}

Unless otherwise specified, the roots of unity may be taken to be complex numbers, 
and in this case, the \( n \)-th roots of unity are
\[
\omega_k = \exp{\frac{2k\pi i}{n}} = \cos\left(\frac{2k\pi}{n}\right) + i\sin\left(\frac{2k\pi}{n}\right), 
\quad k = 0, 1, \ldots, n-1.
\]

Obviously, the modulus of each \( n \)-th root of unity is 1, i.e., \( |\omega_k| = 1 \),
and they are evenly distributed on the unit circle in the complex plane,
with an angle of \( \frac{2\pi}{n} \) between adjacent roots.

\begin{figure}[h]
    \centering
    \includegraphics[width=0.5\textwidth]{img/Visualisation_complex_number_roots.png}
\end{figure}

\begin{property}
    \begin{enumerate}
        \item The \( n \)-th roots of unity form a \emph{cyclic group} under multiplication, 
            with \( \omega = \exp{\frac{2\pi i}{n}} \) as a generator.
    \end{enumerate}
\end{property}

\begin{proposition}{Formulas for Sums and Differences of Powers}
    For \(n \in \mathbb{N^+}\) and \(n\) being odd:
    \[
    a^n + b^n = (a+b)\big(a^{n-1}b^0 - a^{n-2}b^1 + a^{n-3}b^2 - \cdots - a^1b^{n-2} + a^0b^{n-1}\big).
    \]
    When \(n\) is even, there is no general formula for the \(n\)-th power sum.

    For \(n \in \mathbb{N^+}\):
    \[
    a^n - b^n = (a-b)\big(a^{n-1}b^0 + a^{n-2}b^1 + a^{n-3}b^2 + \cdots + a^0b^{n-1}\big).
    \]

    Commonly used special cases:
    \[
    a^2 - b^2 = (a+b)(a-b).
    \]
    \[
    a^3 + b^3 = (a+b)(a^2 - ab + b^2), \quad a^3 - b^3 = (a-b)(a^2 + ab + b^2).
    \]
    \[
    \begin{aligned}
    a^4 - b^4 &= (a^2 + b^2)(a^2 - b^2) = (a^2 + b^2)(a+b)(a-b), \\
    &= (a-b)(a^3 + a^2b + ab^2 + b^3).
    \end{aligned}
    \]
    When \(b=1\),
    \[
    x^n + 1 = (x+1)\big(x^{n-1} - x^{n-2} + x^{n-3} - \cdots + x - 1\big), \quad n \in \mathbb{N^+}, \, n \text{ is odd}.
    \]
    \[
    x^n - 1 = (x-1)\big(x^{n-1} + x^{n-2} + x^{n-3} + \cdots + x + 1\big), \quad n \in \mathbb{N^+}.
    \]
\end{proposition}






\chapter{Integral Valued Polynomials}



\section{Lagrange Interpolation Polynomial}

\chapter{Multivariate Polynomial}
\section{Symmetric Polynomial}
\begin{definition}{Symmetric Polynomial}\label{def:symmetric_polynomial}
    A polynomial \( f(x_1, x_2, \ldots, x_n) \) in \( n \) variables is called a \textbf{symmetric polynomial} 
    if it remains unchanged under any permutation of its variables. 
    In other words, for any permutation \( \sigma \) of the set \( \{1, 2, \ldots, n\} \),
    the following holds:
    \[
    f(x_{\sigma(1)}, x_{\sigma(2)}, \ldots, x_{\sigma(n)}) = f(x_1, x_2, \ldots, x_n).
    \]
\end{definition}
Some common symmetric polynomials include:
\begin{description}
    \item[Elementary Symmetric Polynomials] 
        \[
        \sigma_k(x_1, x_2, \ldots, x_n) = \sum_{1 \leq i_1 < i_2 < \cdots < i_k \leq n} x_{i_1} x_{i_2} \cdots x_{i_k}, 
        \quad k = 1, 2, \ldots, n.
        \]
        That is,
        \begin{align*}
            &\sigma_{0} = 1, \\
            &\sigma_{1} = x_{1} + x_{2} + \cdots + x_{n}, \\
            &\sigma_{2} = \sum_{1 \leq i < j \leq n} x_{i} x_{j}, \\
            &\vdots \\
            &\sigma_{n} = x_{1} x_{2} \cdots x_{n}, \\
            &\sigma_{k} = 0, \quad k > n.
        \end{align*}
        Any symmetric polynomial can be expressed as a polynomial in elementary symmetric polynomials.
    \item[Power Sum Symmetric Polynomials] 
        \[
        p_k(x_1, x_2, \ldots, x_n) = x_1^k + x_2^k + \cdots + x_n^k, 
        \quad k = 1, 2, \ldots.
        \]
    \item[Complete Homogeneous Symmetric Polynomials] 
        \[
        h_k(x_1, x_2, \ldots, x_n) = \sum_{i_1 + i_2 + \cdots + i_n = k} x_1^{i_1} x_2^{i_2} \cdots x_n^{i_n}, 
        \quad k = 1, 2, \ldots.
        \]
\end{description}



\begin{theorem}{Newton's Identities}
    For \( k \geq 1 \), the following relations hold between the elementary symmetric polynomials \( \sigma_k \) 
    and the power sum symmetric polynomials \( p_k \):
    \[
    k \sigma_k = \sum_{i=1}^{k} (-1)^{i-1} \sigma_{k-i} p_i.
    \]
\end{theorem}

We introduce some notations for convenience, where \(f\) can be any function of \(n\) variables,
not necessarily be polynomials:
\begin{description}
    \item[Cyclic Sum] Perform a cyclic shift on all variables in an expression, then sum the resulting terms:
        \[
        \sum_{\text{cyc}} f(x_1, x_2, \ldots, x_n) = \sum_{i=1}^{n} f(x_i, x_{i+1}, \ldots, x_{i+n-1}),
        \]
        where the indices are taken modulo \( n \).
        
        For example, 
        \[
        \sum_{\text{cyc}}\frac{a}{b+c}  = \frac{a}{b+c} + \frac{b}{c+a} + \frac{c}{a+b} \geq \frac{3}{2},
        \]
        which is called Nesbitt's inequality, to be proved later.
    \item[Symmetric Sum] Sum over all distinct permutations of the variables in an expression:
        \[
        \sum_{\text{sym}} f(x_1, x_2, \ldots, x_n) = \sum_{\sigma \in S_n} f(x_{\sigma(1)}, x_{\sigma(2)}, \ldots, x_{\sigma(n)}),
        \]
        where \( S_n \) is the set of all permutations of \( n \) elements.
        
        For example,
        \[
        \sum_{\text{sym}} a^{3} = a^{3} + b^{3} + c^{3}, \quad 
        \sum_{\text{sym}} a^{2}b = a^{2}b + a^{2}c + b^{2}a + b^{2}c + c^{2}a + c^{2}b.
        \]
\end{description}

\section{Symmetric Inequalities}
\begin{definition}{Symmetric Inequality}
    An inequality \( f(x_1, x_2, \ldots, x_n) \geq g(x_1, x_2, \ldots, x_n) \) is called a \textbf{symmetric inequality} 
    if the polynomial \( f(x_1, x_2, \ldots, x_n) \) and \( g(x_1, x_2, \ldots, x_n) \) are symmetric polynomials.
\end{definition}

\begin{leftbarTitle}{Power Mean Inequality}\end{leftbarTitle}
\begin{theorem}{Power Mean Inequality}
    For positive real numbers \(a_1, a_2, \dots, a_n > 0\), define the power mean of order \(p\) as:
    \[
    M_{p}(a_1, a_2, \dots, a_n) = 
    \begin{cases} 
        \left( \frac{a_1^p + a_2^p + \cdots + a_n^p}{n} \right)^{\frac{1}{p}}, & p \neq 0 \\ 
        \lim_{p \to 0} M_{p}(a_1, a_2, \dots, a_n) = \sqrt[n]{a_1 a_2 \cdots a_n}, & p = 0 .
    \end{cases}
    \]
    Specially, when \(p \to 0\), it is the \textbf{geometric mean} (G)
    \[
    G = \sqrt[n]{a_1 a_2 \cdots a_n};
    \]
    when \(p = 1\), it is the \textbf{arithmetic mean} (A)
    \[
    A = \frac{a_1 + a_2 + \cdots + a_n}{n};
    \]
    when \(p = 2\), it is the \textbf{quadratic mean} (Q)
    \[
    Q = \sqrt{\frac{a_1^2 + a_2^2 + \cdots + a_n^2}{n}};
    \]
    when \(p = -1\), it is the \textbf{harmonic mean} (H)
    \[
    H = \frac{n}{\frac{1}{a_1} + \frac{1}{a_2} + \cdots + \frac{1}{a_n}}.
    \]
    The following inequalities hold:
    \[
    \cdots \leqslant M_{-2} \leqslant M_{-1} \leqslant M_{0} \leqslant M_{1} \leqslant M_{2} \leqslant \cdots.
    \]
    Thus, we have:
    \[
    H \leqslant G \leqslant A \leqslant Q.
    \]
\end{theorem}
% n=2时, 幂均不等式链中可以插入对数均值
When \(n=2\), in the chain of power mean inequalities, we can insert the \textbf{logarithmic mean}:
logarithmic mean of \(a\) and \(b\) is defined as:
\[
L(a, b) = \frac{a-b}{\ln a - \ln b} \quad (a \neq b, a, b > 0),
\]
then we have:
\[
G(a, b) \leqslant L(a, b) \leqslant A(a, b).
\]



\begin{leftbarTitle}{Muirhead Inequality}\end{leftbarTitle}
This part mainly references~\cite{3}.

\begin{definition}{Convex Hull}
    Let \(V\) be a linear space over the field \( \mathbb{R} \), for a set \(X\),
    % 所有包含X的凸集的交集被称为X的凸包
    the \textbf{convex hull} of \(X\) is defined as the intersection of all convex sets containing \(X\):
    \[
    S := \bigcap_{X \subseteq K \subseteq V} K, \quad \text{where } K \text{ is a convex set}.
    \]
\end{definition}
For an \(n\)-dimensional vector \( \alpha = (a_1, a_2, \ldots, a_n) \),
define \(\alpha_{[j]},1\leqslant j\leqslant n\) is the \(j\)-th item of \(\alpha\) 
after sorting \(a_1, a_2, \ldots, a_n\) in descending order, i.e.,
\[
a_{[1]} \geqslant a_{[2]} \geqslant \cdots \geqslant a_{[n]}.
\]
Then we can obtain 
\[
a_{\downarrow} = (a_{[1]}, a_{[2]}, \cdots, a_{[n]}).
\]

% S_n为{1,2,...n}中元素的全排列(共n!种)组成的集合
Define \(S_{n}\) be the set of all permutations of the set \(\{1, 2, \ldots, n\}\),
then we define the convex hull of \(\alpha\) as:
\[
H(\alpha) = \{ b_{\tau(1)}, b_{\tau(2)}, \ldots, b_{\tau(n)} | \tau \in S_{n} \}.
\]
\begin{figure}[h]
    \centering
    \includegraphics[width=0.4\textwidth]{img/convex_hull.png}
    \caption{Relation graph of convex hull in \(\mathbb{R}^3\).}
    \label{fig:convex_hull}
\end{figure}

\begin{theorem}{Muirhead Inequality}
    For \(\alpha = (a_{1}, a_{2}, \ldots, a_{n}), \beta = (b_{1}, b_{2}, \ldots, b_{n}) \in \mathbb{R}^{n}\),
    and \(\alpha \in H(\beta)\)\footnote{
        It is often called \textbf{Muirhead's condition} that \(\alpha \in H(\beta)\).
    }, then for positive numbers \(x_{1}, x_{2}, \ldots, x_{n} > 0\), the following inequality holds:
    \[
    \sum_{\sigma \in S_{n}} x^{a_{1}}_{\sigma(1)}x^{a_{2}}_{\sigma(2)}\cdots x^{a_{n}}_{\sigma(n)} \leqslant  
    \sum_{\sigma \in S_{n}} x^{b_{1}}_{\sigma(1)}x^{b_{2}}_{\sigma(2)}\cdots x^{b_{n}}_{\sigma(n)},
    \]
    where \(\sum_{\sigma \in S_{n}}\) denotes summation over all permutations \(\sigma\) in \(S_{n}\).

    The equality holds if and only if \(x_{1} = x_{2} = \cdots = x_{n}\) or \(\alpha_{\downarrow} = \beta_{\downarrow}\).
\end{theorem}

Since Muirhead's condition is difficult to verify directly, we derive the following conditions.
\begin{definition}{Majorization} % 盖
    For \(\alpha = (a_{1}, a_{2}, \ldots, a_{n}), \beta = (b_{1}, b_{2}, \ldots, b_{n}) \in \mathbb{R}^{n}\),
    if the following conditions hold:
    \begin{enumerate}
        \item \(a_{[1]} + a_{[2]} + \cdots + a_{[n]} = b_{[1]} + b_{[2]} + \cdots + b_{[n]}\);
        \item For all \(k = 1, 2, \ldots, n-1\),
            \[
            a_{[1]} + a_{[2]} + \cdots + a_{[k]} \leqslant b_{[1]} + b_{[2]} + \cdots + b_{[k]};
            \]
    \end{enumerate}
    then we say that \(\beta\) majorizes \(\alpha\), denoted as \(\alpha \prec \beta\).
\end{definition}

\begin{definition}{Doubly Stochastic Matrix} % 双重随机矩阵
    An \(n \times n\) matrix \(P = (p_{ij})\) is called a \textbf{doubly stochastic matrix} 
    if the sum of each row and the sum of each column both equal \(1\).
\end{definition}

Now we can give two equivalent conditions:
\begin{theorem}
    In the same conditions as Muirhead's inequality, the following two statements are equivalent to \(\alpha \prec \beta\):
    \begin{enumerate}
        \item There exists a doubly stochastic matrix \(D\) such that \(\alpha = D\beta\).
        \item \(\alpha \in H(\beta)\).
    \end{enumerate}
\end{theorem}

\vspace{0.7cm}
With Muirhead's inequality, we can prove many symmetric inequalities easily.
\begin{proposition}{Schur Inequality}
    For non-negative real numbers \( a, b, c \geq 0 \) and a real number \( r \geq 0 \), the following inequality holds:
    \[
    a^{r}(a - b)(a - c) + b^{r}(b - c)(b - a) + c^{r}(c - a)(c - b) \geq 0.
    \]

    When \( r = 1 \), the following well-known special case can be derived:
    \[
    a^{3} + b^{3} + c^{3} + 3abc \geq ab(a + b) + bc(b + c) + ca(c + a).
    \]
\end{proposition}

\begin{proposition}{Nesbitt's Inequality}
    For positive real numbers \(a, b, c > 0\), the following inequality holds:
    \[
    \sum_{\text{cyc}}\frac{a}{b+c}  = \frac{a}{b+c} + \frac{b}{c+a} + \frac{c}{a+b} \geq \frac{3}{2}.
    \]
\end{proposition}

\begin{proof}
    % 通分并交叉相乘
    By finding a common denominator and cross-multiplying, we have:
    \[
    2\sum_{\text{cyc}} a(a+b)(a+c) \geqslant  3(a+b)(b+c)(c+a),
    \]
    which is equivalent to
    \[
    \sum_{\text{sym}} a^{3} \geqslant \sum_{\text{sym}} a^{2}b.
    \]
    Note that \((3,0,0) \succ (2,1,0)\),
    thus by Muirhead's inequality, the above inequality holds.
\end{proof}












\begin{thebibliography}{99}
\bibitem{en1} 南秀全, 黄振国. \emph{多项式理论}. 哈尔滨工业大学出版社, 2016.
\bibitem{ch3} 王萼芳, 石生明. \emph{ 高等代数 (5th edition) }, 高等教育出版社, 2019.
\bibitem{3} 陈柏宇, 张福春. \emph{ Muirhead 不等式 }, 数学传播-38卷2期, pp.41-58, 2019.
\end{thebibliography}

\end{document}