\chapter{Integers}  % 整数
\section{Divisibility and Prime Numbers} % 整除性与素数
Let\footnote{
    Sometimes, natural numbers refer to the set of positive integers excluding zero, i.e., \(\mathbb{N}^{+}=\{1,2,3,\ldots\}\).
} 
\[
\mathbb{N}=\{0,1,2,3,\ldots\},\quad \mathbb{N}^{+}=\{1,2,3,\ldots\},\quad \mathbb{Z}=\{\ldots,-2,-1,0,1,2,\ldots\}.
\]
% 对于一个实数, 引入高斯符号

\begin{definition}{Gauß Symbols}
    For a real number \(x\), the floor function (greatest integer function) is defined as:
    \[
    \lfloor x \rfloor = \max\{n \in \mathbb{Z} \mid n \leq x\}.
    \]
    Similarly, the ceiling function (least integer function) is defined as:
    \[
    \lceil x \rceil = \min\{n \in \mathbb{Z} \mid n \geq x\}.
    \]
\end{definition}
\begin{property}
    \begin{enumerate}
        \item For any \(m \in \mathbb{N}^{+}\), there is \textbf{Hermite's identity}:
            \[
            \left\lfloor mx \right\rfloor = \left\lfloor x \right\rfloor + \left\lfloor x+ \frac{1}{m} \right\rfloor 
            + \cdots + \left\lfloor x+ \frac{m-1}{m} \right\rfloor.
            \]
            \[
            \left\lceil mx \right\rceil = \left\lceil x \right\rceil + \left\lceil x - \frac{1}{m} \right\rceil 
            + \cdots + \left\lceil x - \frac{m-1}{m} \right\rceil.
            \]
        \item 
    \end{enumerate}
\end{property}


\begin{theorem}{Euclidean Algorithm}
    For any integers \(a\) and \(b\) with \(b > 0\), there exist unique integers \(q\) and \(r\) such that
    \[
    a = bq + r, \quad 0 \leq r < b.
    \]
    \(r\) is called the remainder of \(a\) divided by \(b\), denoted as \(r = a \bmod b\).

    If \(r = 0\), then \(b\) divides \(a\), denoted as \(b \mid a\); 
    otherwise, \(b\) does not divide \(a\), denoted as \(b \nmid a\).
    In other words, \(b \mid a\) if and only if there exists an integer \(k\) such that \(a = bk\).

    If \(a=kb\) and \(b\neq a, b\neq 1\), then \(b\) is called a proper divisor of \(a\).
\end{theorem}

\begin{property}
    If \(b\neq 0, c\neq 0\), then
    \begin{enumerate}
        \item If \(b\mid a, c\mid b\), then \(c\mid a\).
        \item If \(b\mid a\), then \(bc \mid ac\).
        \item If \(c \mid d, c\mid e\), then \(c \mid (md + ne)\), for any integers \(m, n\).
    \end{enumerate}
\end{property}

\section{Carry System} % 进位制
Carry system (or positional numeral system) is a method of representing numbers using a radix (or base) \(r\) (\(r \geq 2\)).
In base \(r\), any non-negative integer \(N\) can be expressed as:
\[
N = a_k r^k + a_{k-1} r^{k-1} + \cdots + a_1 r + a_0 = \sum_{i=0}^{k} a_i r^i =: (a_k a_{k-1} \cdots a_1 a_0)_r,
\]
where \(a_i\) are the digits satisfying \(0 \leq a_i < r\) and \(a_k \neq 0\).

% 扩展到小数
This can be extended to decimal fractions as:
\[
N = a_k r^k + a_{k-1} r^{k-1} + \cdots + a_1 r + a_0 + a_{-1} r^{-1} + a_{-2} r^{-2} + \cdots = \sum_{i=-m}^{k} a_i r^i =: 
(a_k a_{k-1} \cdots a_1 a_0 . a_{-1} a_{-2} \cdots a_{-m})_r,
\]
where \(m\) is a positive integer.

\begin{leftbarTitle}{Radix Conversion}\end{leftbarTitle} % 进制转换
% 十进制与 r 进制的转换
Here are \emph{methods for converting between decimal and base \(r\)}:
\begin{itemize}
    \item Decimal to base \(r\):
        \begin{enumerate}
            \item For the integer part, repeatedly divide by \(r\) and record the remainders.
            \item For the fractional part, repeatedly multiply by \(r\) and record the integer parts.
            \item Combine the results to form the base \(r\) representation.
        \end{enumerate}
    \item Base \(r\) to decimal:
        \begin{enumerate}
            \item For the integer part, multiply each digit by \(r\) raised to its position power and sum them.
            \item For the fractional part, multiply each digit by \(r\) raised to its negative position power and sum them.
            \item Combine both sums to get the decimal representation.
        \end{enumerate}
\end{itemize}

\begin{example}
    \begin{enumerate}
        \item Convert decimal \(45.625\) to binary.
        \item Convert binary \((1101.101)_2\) to decimal.
    \end{enumerate}
\end{example}

\begin{solution}
\begin{enumerate}
    \item For integer part \(45\):
        \begin{align*}
            &45 \div 2 = 22 \text{ remainder } 1 \\
            &22 \div 2 = 11 \text{ remainder } 0 \\
            &11 \div 2 = 5 \text{ remainder } 1 \\
            &5 \div 2 = 2 \text{ remainder } 1 \\
            &2 \div 2 = 1 \text{ remainder } 0 \\
            &1 \div 2 = 0 \text{ remainder } 1 \\
        \end{align*}
        Reading remainders from bottom to top gives \(101101\).
        \newline For fractional part \(0.625\):
        \begin{align*}
            &0.625 \times 2 = 1.25 \quad (\text{integer part } 1) \\
            &0.25 \times 2 = 0.5 \quad (\text{integer part } 0) \\
            &0.5 \times 2 = 1.0 \quad (\text{integer part } 1)
        \end{align*}
        Reading integer parts gives \(101\).
        \newline Combining both parts, we get \(45.625_{10} = (101101.101)_2\).
    \item Since  
        \(
        1 \times 2^3 + 1 \times 2^2 + 0 \times 2^1 + 1 \times 2^0 + 1 \times 2^{-1} + 0 \times 2^{-2} + 1 \times 2^{-3} 
        = 8 + 4 + 0 + 1 + 0.5 + 0 + 0.125 = 13.625;
        \)
        thus, \((1101.101)_2 = 13.625_{10}\).
\end{enumerate}
\end{solution}

\begin{leftbarTitle}{Generalized Carry System}\end{leftbarTitle} % 广义进制系统

\begin{leftbarTitle}{Balanced Ternary}\end{leftbarTitle} % 平衡三进制
\textbf{Balanced ternary} (symmetric ternary) is a non-standard positional numeral system 
that uses three digits: \(-1\), \(0\), and \(1\).
Since \(-1\) is not a standard digit, it is often represented by the symbol \(Z\) or \(\bar{1}\).
The weight calculation is the same as standard ternary, with the weight of the \(i\)-th digit being \(3^{i}\).

\begin{theorem}{Uniqueness of Balanced Ternary Representation}
    Every integer can be uniquely represented in balanced ternary.
\end{theorem}

\begin{proposition}
    % 对于负数, 只需将对应整数的每一位取反即可
    For negative numbers, simply negate each digit of the corresponding positive integer's balanced ternary representation.
\end{proposition}

\vspace{0.7cm}
Here are \emph{methods for converting between decimal and balanced ternary}:
\begin{itemize}
    \item Decimal to balanced ternary:
        \begin{enumerate}
            \item Repeatedly divide the number by \(3\), recording the remainders.
            \item If a remainder is \(2\), replace it with \(-1\) (or \(Z\)) and increment the quotient by \(1\).
            \item Continue until the quotient is \(0\).
            \item Read the remainders from bottom to top to form the balanced ternary representation.
        \end{enumerate}
    \item Balanced ternary to decimal:
        \begin{enumerate}
            \item Multiply each digit by \(3\) raised to its position power and sum them.
            \item For digits equal to \(-1\) (or \(Z\)), treat them as \(-1\) in the calculation.
        \end{enumerate}
\end{itemize}


\begin{example}
    \begin{enumerate}
        \item Convert decimal \(64\) to balanced ternary. 
        \item Convert balanced ternary \(1Z0Z1\) to decimal.
    \end{enumerate}
\end{example}

\begin{solution}
\begin{enumerate}
    \item For integer part \(64\):
        \begin{align*}
            &64 \div 3 = 21 \text{ remainder } 1 \\
            &21 \div 3 = 7 \text{ remainder } 0 \\
            &7 \div 3 = 2 \text{ remainder } 1 \\
            &2 \div 3 = 0 \text{ remainder } 2 \quad (\text{replace } 2 \text{ with } Z, \text{ increment quotient to } 1) \\
            &1 \div 3 = 0 \text{ remainder } 1
        \end{align*}
        Reading remainders from bottom to top gives \(1Z0Z1\).
        Thus, \(64_{10} = (1Z0Z1)_{3b}\).
    \item Since  
        \(
        1 \times 3^4 + (-1) \times 3^3 + 0 \times 3^2 + (-1) \times 3^1 + 1 \times 3^0 
        = 81 - 27 + 0 - 3 + 1 = 52;
        \)
        thus, \((1Z0Z1)_{3b} = 52_{10}\).
\end{enumerate}
\end{solution}

\section{Greatest Common Divisor and Least Common Multiple} % 最大公约数与最小公倍数

\section{Fundamental Theorem of Arithmetic} % 算数基本定理